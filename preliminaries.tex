\chapter{Type theory}
\label{cha:typetheory}

\section{Type theory versus set theory}
\label{sec:types-vs-sets}
\label{sec:axioms}

Homotopy type theory is (among other things) a foundational language for mathematics, i.e., an alternative to Zermelo--Fraenkel\index{set theory!Zermelo--Fraenkel} set theory.
However, it behaves differently from set theory in several important ways, and that can take some getting used to.
Explaining these differences carefully requires us to be more formal here than we will be in the rest of the book.
As stated in the introduction, our goal is to write type theory \emph{informally}; but for a mathematician accustomed to set theory, more precision at the beginning can help avoid some common misconceptions and mistakes.

We note that a set-theoretic foundation has two ``layers'': the deductive system of first-order logic, and, formulated inside this system, the axioms of a particular theory, such as ZFC.
Thus, set theory is not only about sets, but rather about the interplay between sets (the objects of the second layer) and propositions (the objects of the first layer).

By contrast, type theory is its own deductive system: it need not be formulated inside any superstructure, such as first-order logic.
Instead of the two basic notions of set theory, sets and propositions, type theory has one basic notion: \emph{types}.
Propositions (statements which we can prove, disprove, assume, negate, and so on\footnote{Confusingly, it is also a common practice (dating 
back to Euclid) to use the word ``proposition'' synonymously with ``theorem''.
  We will confine ourselves to the logician's usage, according to which a \emph{proposition} is a statement \emph{susceptible to} proof, whereas a \emph{theorem} (or ``lemma'' or ``corollary'') is such a statement that \emph{has been} proven.
Thus ``$0=1$'' and its negation ``$\neg(0=1)$'' are both propositions, but only the latter is a theorem.}) are identified with particular types, via the correspondence shown in \autoref{tab:pov} on page~\pageref{tab:pov}.
Thus, the mathematical activity of \emph{proving a theorem} is identified with a special case of the mathematical activity of \emph{constructing an object}---in this case, an inhabitant of a type that represents a proposition.

This leads us to another difference between type theory and set theory, but to explain it we must say a little about deductive systems in general.
Informally, a deductive system is a collection of \define{rules}
\indexdef{rule}%
for deriving things called \define{judgments}.
\indexdef{judgment}%
If we think of a deductive system as a formal game,
\index{game!deductive system as}%
then the judgments are the ``positions'' in the game which we reach by following the game rules.
We can also think of a deductive system as a sort of algebraic theory, in which case the judgments are the elements (like the elements of a group) and the deductive rules are the operations (like the group multiplication).
From a logical point of view, the judgments can be considered to be the ``external'' statements, living in the metatheory, as opposed to the ``internal'' statements of the theory itself.

In the deductive system of first-order logic (on which set theory is based), there is only one kind of judgment: that a given proposition has a proof.
That is, each proposition $A$ gives rise to a judgment ``$A$ has a proof'', and all judgments are of this form.
A rule of first-order logic such as ``from $A$ and $B$ infer $A\wedge B$'' is actually a rule of ``proof construction'' which says that given the judgments ``$A$ has a proof'' and ``$B$ has a proof'', we may deduce that ``$A\wedge B$ has a proof''.
Note that the judgment ``$A$ has a proof'' exists at a different level from the \emph{proposition} $A$ itself, which is an internal statement of the theory.
% In particular, we cannot manipulate it to construct propositions such as ``if $A$ has a proof, then $B$ does not have a proof''---unless we are using our set-theoretic foundation as a meta-theory with which to talk about some other axiomatic system.

The basic judgment of type theory, analogous to ``$A$ has a proof'', is written ``$a:A$'' and pronounced as ``the term $a$ has type $A$'', or more loosely ``$a$ is an element of $A$'' (or, in homotopy type theory, ``$a$ is a point of $A$'').
\indexdef{term}%
\indexdef{element}%
\indexdef{point!of a type}%
When $A$ is a type representing a proposition, then $a$ may be called a \emph{witness}\index{witness!to the truth of a proposition} to the provability of $A$, or \emph{evidence}\index{evidence!of the truth of a proposition} of the truth of $A$ (or even a \emph{proof}\index{proof} of $A$, but we will try to avoid this confusing terminology).
In this case, the judgment $a:A$ is derivable in type theory (for some $a$) precisely when the analogous judgment ``$A$ has a proof'' is derivable in first-order logic (modulo differences in the axioms assumed and in the encoding of mathematics, as we will discuss throughout the book).
% (Already here type theory has an advantage in that the term $a$ records the essence of the proof of $A$, whereas in first-order logic that information is only available by backtracking through the derivation of ``$A$ has a proof''.)

On the other hand, if the type $A$ is being treated more like a set than like a proposition (although as we will see, the distinction can become blurry), then ``$a:A$'' may be regarded as analogous to the set-theoretic statement ``$a\in A$''.
However, there is an essential difference in that ``$a:A$'' is a \emph{judgment} whereas ``$a\in A$'' is a \emph{proposition}.
In particular, when working internally in type theory, we cannot make statements such as ``if $a:A$ then it is not the case that $b:B$'', nor can we ``disprove'' the judgment ``$a:A$''.

A good way to think about this is that in set theory, ``membership'' is a relation which may or may not hold between two pre-existing objects ``$a$'' and ``$A$'', while in type theory we cannot talk about an element ``$a$'' in isolation: every element \emph{by its very nature} is an element of some type, and that type is (generally speaking) uniquely determined.
Thus, when we say informally ``let $x$ be a natural number'', in set theory this is shorthand for ``let $x$ be a thing and assume that $x\in\nat$'', whereas in type theory ``let $x:\nat$'' is an atomic statement: we cannot introduce a variable without specifying its type.

At first glance, this may seem an uncomfortable restriction, but it is arguably closer to the intuitive mathematical meaning of ``let $x$ be a natural number''.
In practice, it seems that whenever we actually \emph{need} ``$a\in A$'' to be a proposition rather than a judgment, there is always an ambient set $B$ of which $a$ is known to be an element and $A$ is known to be a subset.
This situation is also easy to represent in type theory, by taking $a$ to be an element of the type $B$, and $A$ to be a predicate on $B$; see \autoref{subsec:prop-subsets}.

A last difference between type theory and set theory is the treatment of equality.
The familiar notion of equality in mathematics is a proposition: e.g.\ we can disprove an equality or assume an equality as a hypothesis.
Since in type theory, propositions are types, this means that equality is a type: for elements $a,b:A$ (that is, both $a:A$ and $b:A$) we have a type ``$\id[A]ab$''.
(In \emph{homotopy} type theory, of course, this equality proposition can behave in unfamiliar ways: see \autoref{sec:identity-types,cha:basics}, and the rest of the book).

However, in type theory there is also a need for an equality \emph{judgment}, existing at the same level as the judgment ``$x:A$''.
\symlabel{defn:judgmental-equality}%
This is called \define{judgmental equality}
\indexdef{equality!judgmental}%
\indexdef{judgmental equality}%
or \define{definitional equality},
\indexdef{equality!definitional}%
\indexdef{definitional equality}%
and we write it as $a\jdeq b$ or $a \jdeq_A b$.
It is helpful to think of this as meaning ``equal by definition''.
For instance, if we define a function $f:\nat\to\nat$ by the equation $f(x)=x^2$, then the expression $f(3)$ is equal to $3^2$ \emph{by definition}.
It does not make sense to negate or assume an equality-by-definition; we cannot say ``if $x$ is equal to $y$ by definition, then $z$ is not equal to $w$ by definition''.
Whether or not two expressions are equal by definition is just a matter of expanding out the definitions; in particular, it is algorithmically\index{algorithm} decidable.\index{decidable!definitional equality}

As type theory becomes more complicated, judgmental equality can get more subtle than this, but it is a good intuition to start from.
Alternatively, if we regard a deductive system as an algebraic theory, then judgmental equality is simply the equality in that theory, analogous to the equality between elements of a group---the only potential for confusion is that there is \emph{also} an object \emph{inside} the deductive system of type theory (namely the type ``$a=b$'') which behaves internally as a notion of ``equality''.

The reason we \emph{want} a judgmental notion of equality is so that it can control the other form of judgment, ``$a:A$''.
For instance, suppose we have given a proof that $3^2=9$, i.e.\ we have derived the judgment $p:(3^2=9)$ for some $p$.
Then the same witness $p$ ought to count as a proof that $f(3)=9$, since $f(3)$ is $3^2$ \emph{by definition}.
The best way to represent this is with a rule saying that given the judgments $a:A$ and $A\jdeq B$, we can derive the judgment $a:B$.

Thus, for us, type theory will be a deductive system based on two forms of judgment:
\begin{center}
\medskip
\begin{tabular}{cl}
  \toprule
  Judgment & Meaning\\
  \midrule
  $a : A$       & ``$a$ is an object of type $A$''\\
  $a \jdeq_A b$ & ``$a$ and $b$ are definitionally equal objects of type $A$''\\
  \bottomrule
\end{tabular}
\medskip
\end{center}
%
\symlabel{defn:defeq}%
When introducing a definitional equality, i.e., defining one thing to be equal to another, we will use the symbol ``$\defeq$''.
Thus, the above definition of the function $f$ would be written as $f(x)\defeq x^2$.

Because judgments cannot be put together into more complicated statements, the symbols ``$:$'' and ``$\jdeq$'' bind more loosely than anything else.%
\footnote{In formalized\indexfoot{mathematics!formalized} type theory, commas and turnstiles can bind even more loosely.
  For instance, $x:A,y:B\vdash c:C$ is parsed as $((x:A),(y:B))\vdash (c:C)$.
  However, in this book we refrain from such notation until \autoref{cha:rules}.}
Thus, for instance, ``$p:\id{x}{y}$'' should be parsed as ``$p:(\id{x}{y})$'', which makes sense since ``$\id{x}{y}$'' is a type, and not as ``$\id{(p:x)}{y}$'', which is senseless since ``$p:x$'' is a judgment and cannot be equal to anything.
Similarly, ``$A\jdeq \id{x}{y}$'' can only be parsed as ``$A\jdeq(\id{x}{y})$'', although in extreme cases such as this, one ought to add parentheses anyway to aid reading comprehension.
This is perhaps also an appropriate place to mention that the common mathematical notation ``$f:A\to B$'', expressing the fact that $f$ is a function from $A$ to $B$, can be regarded as a typing judgment, since we use ``$A\to B$'' as notation for the type of functions from $A$ to $B$ (as is standard practice in type theory; see \autoref{sec:pi-types}).

Judgments may depend on assumptions of the form $x:A$, where $x$ is a variable
\indexdef{variable}%
and $A$ is a type.
For example, we may construct an object $m + n : \nat$ under the assumptions that $m,n : \nat$.
Another example is that assuming $A$ is a type, $x,y : A$, and $p : \id[A]{x}{y}$, we may construct an element $p^{-1} : \id[A]{y}{x}$.
The collection of all such assumptions is called the \define{context};%
\index{context}
from a topological point of view it may be thought of as a ``parameter space''.
In fact, technically the context must be an ordered list of assumptions, since later assumptions may depend on previous ones: the assumption\index{assumption} $x:A$ can only be made \emph{after} the assumptions of any variables appearing in the type $A$.

If the type $A$ in an assumption $x:A$ represents a proposition, then the assumption is a type-theoretic version of a \emph{hypothesis}:
\indexdef{hypothesis}%
we assume that the proposition $A$ holds.
When types are regarded as propositions, we may omit the names of their proofs.
Thus, in the second example above we may instead say that assuming $\id[A]{x}{y}$, we can prove $\id[A]{y}{x}$.
However, since we are doing ``proof-relevant'' mathematics,
\index{mathematics!proof-relevant}%
we will frequently refer back to proofs as objects.
In the example above, for instance, we may want to establish that $p^{-1}$ together with the proofs of transitivity and reflexivity behave like a groupoid; see \autoref{cha:basics}.

In the rest of this chapter, we attempt to give an informal presentation of type theory, sufficient for the purposes of this book; we will give a more formal account in \autoref{cha:rules}.
Aside from some fairly obvious rules (such as the fact that judgmentally equal things can always be substituted\index{substitution} for each other), the rules of type theory can be grouped into \emph{type formers}.
Each type former consists of a way to construct types (possibly making use of previously constructed types), together with rules for the construction and behavior of elements of that type.
In most cases, these rules follow a fairly predictable pattern, but we will not attempt to make this precise.

\index{axiom!versus rules}%
\index{rule!versus axioms}%
An important aspect of the type theory presented in this chapter is that it consists entirely of \emph{rules}, without any \emph{axioms}.
In the description of deductive systems in terms of judgments, the \emph{rules} are what allow us to conclude one judgment from collection of others, while the \emph{axioms} are the judgments we are given at the outset.
If we think of a deductive system as a formal game, then the rules are the rules of the game, while the axioms are the starting position.
And if we think of a deductive system as an algebraic theory, then the rules are the operations of the theory, while the axioms are the \emph{generators} for some particular free model of that theory.

In set theory, the only rules are the rules of first-order logic (such as the rule allowing us to deduce ``$A\wedge B$ has a proof'' from ``$A$ has a proof'' and ``$B$ has a proof''): all the information about the behavior of sets is contained in the axioms.
By contrast, in type theory, it is usually the \emph{rules} which contain all the information, with no axioms being necessary.
For instance, in \autoref{sec:finite-product-types} we will see that there is a rule allowing us to deduce the judgment ``$(a,b):A\times B$'' from ``$a:A$'' and ``$b:B$'', whereas in set theory the analogous statement would be (a consequence of) the pairing axiom.

The advantage of formulating type theory using only rules is that rules are ``procedural''.
In particular, this property is what makes possible (though it does not automatically ensure) the good computational properties of type theory, such as ``canonicity''.
However, while this style works for traditional type theories, we do not yet understand how to formulate everything we need for \emph{homotopy} type theory in this way.
In particular, in \autoref{sec:compute-pi,sec:compute-universe,cha:hits} we will have to augment the rules of type theory presented in this chapter by introducing additional axioms, notably the \emph{univalence axiom}.
In this chapter, however, we confine ourselves to a traditional rule-based type theory.


\section{Function types}
\label{sec:function-types}

\index{type!function|(defstyle}%
\indexsee{function type}{type, function}%
Given types $A$ and $B$, we can construct the type $A \to B$ of \define{functions}
\index{function|(defstyle}%
\indexsee{map}{function}%
with domain $A$ and codomain $B$.
We also sometimes refer to functions as \define{maps}.
\index{domain!of a function}%
\index{codomain!of a function}%
\index{function!domain of}%
\index{function!codomain of}%
Unlike in set theory, functions are not defined as
functional relations; rather they are a primitive concept in type theory.
We explain the function type by prescribing what we can do with functions, 
how to construct them and what equalities they induce.

Given a function $f : A \to B$ and an element of the domain $a : A$, we
can \define{apply}
\indexdef{application!of function}%
\indexdef{function!application}%
the function to obtain an element of the codomain $B$,
denoted $f(a)$ and called the \define{value} of $f$ at $a$.
\indexdef{value!of a function}%
It is common in type theory to omit the parentheses and denote $f(a)$ simply by $f\,a$, and we will sometimes do this as well.

But how can we construct elements of $A \to B$? There are two equivalent ways:
either by direct definition or by using
$\lambda$-abstraction. Introducing a function by definition
\indexdef{definition!of function, direct}%
means that
we introduce a function by giving it a name --- let's say, $f$ --- and saying
we define $f : A \to B$ by giving an equation
\begin{equation}
  \label{eq:expldef}
  f(x) \defeq b
\end{equation}
where $x$ is a variable
\index{variable}%
and $b$ is an expression which may use $x$.
In order for this to be valid, we have to check that $b : B$ assuming $x:A$.

Now we can compute $f(a)$ by replacing the variable $x$ in $b$ with
$a$. As an example, consider the function $f : \nat \to \nat$ which is
defined by $f(x) \defeq x+x$.  (We will define $\nat$ and $+$ in \autoref{sec:inductive-types}.)
Then $f(2)$ is judgmentally equal to $2+2$.

If we don't want to introduce a name for the function, we can use
\define{$\lambda$-abstraction}.%
\index{lambda abstraction@$\lambda$-abstraction|defstyle}%
\indexsee{function!lambda abstraction@$\lambda$-abstraction}{$\lambda$-abstraction}%
\indexsee{abstraction!lambda-@$\lambda$-}{$\lambda$-abstraction}%
Given an expression $b$ of type $B$ which may use $x:A$, as above, we write $\lam{x:A} b$ to indicate the same function defined by~\eqref{eq:expldef}.
Thus, we have
\[ (\lamt{x:A}b) : A \to B. \]
For the example in the previous paragraph, we have the typing judgment
\[ (\lam{x:\nat}x+x) : \nat \to \nat. \]
As another example, for any types $A$ and $B$ and any element $y:B$, we have a \define{constant function}
\indexdef{constant!function}%
\indexdef{function!constant}%
$(\lam{x:A} y): A\to B$.

We generally omit the type of the variable $x$ and write $\lam{x}b$, since the typing $x:A$ is inferrable from the judgment that the function $\lam x b$ has type $A\to B$.
By convention, the ``scope''
\indexdef{variable!scope of}%
\indexdef{scope}%
of the variable binding ``$\lam{x}$'' is the entire rest of the expression, unless delimited with parentheses.
Thus, for instance, $\lam{x} x+x$ should be parsed as $\lam{x} (x+x)$, not as $(\lam{x}x)+x$ (which would, in this case, be ill-typed anyway).

Another equivalent notation is
\symlabel{mapsto}
\[ (x \mapsto b) : A \to B. \]
\symlabel{blank}
We may also sometimes use a blank ``$\blank$'' in the expression $b$ in place of a variable, to denote an implicit $\lambda$-abstraction.
For instance, $g(x,\blank)$ is another way to write $\lam{y} g(x,y)$.
Multiple blanks denote abstraction over multiple variables, e.g.\ $g(\blank,\blank)$ means $\lam{x}{y} g(x,y)$.

Now a $\lambda$-abstraction is a function, so we can apply it to an argument $a:A$.
We then have the definitional equality\footnote{Use of this equality is often referred to as \define{$\beta$-conversion}
\index{beta-conversion@$\beta $-conversion|footstyle}%
\index{conversion!beta@$\beta $-|footstyle}%
or \define{$\beta$-reduction}.%
\index{beta-reduction@$\beta $-reduction|footstyle}%
\index{reduction!beta@$\beta $-|footstyle}%
}
\[(\lamu{x:A}b)(a) \jdeq b'\]
where $b'$ is the
expression $b$ in which all occurrences of $x$ have been replaced by $a$.
Continuing the above example, we have
%
\[ (\lamu{x:\nat}x+x)(2) \jdeq 2+2. \]
%
Note that from any function $f:A\to B$, we can construct a lambda abstraction function $\lam{x} f(x)$.
Since this is by definition ``the function which applies $f$ to its argument'' we consider it to be definitionally equal to $f$:\footnote{Use of this equality is often referred to as \define{$\eta$-conversion}
\index{eta-conversion@$\eta $-conversion|footstyle}%
\index{conversion!eta@$\eta $-|footstyle}%
or \define{$\eta$-expansion.
\index{eta-expansion@$\eta $-expansion|footstyle}%
\index{expansion!eta@$\eta $-|footstyle}%
}}
\[ f \jdeq (\lam{x} f(x)). \]

The introduction of functions by definitions with explicit parameters can be reduced
to simple definitions by using $\lambda$-abstraction: i.e., we can read 
a definition of $f: A\to B$ by
\[ f(x) \defeq b \]
as 
\[ f \defeq \lamu{x:A}b.\]
%
When doing calculations involving variables, we have to be 
careful when replacing a variable with an expression that also involves
variables, because we want to preserve the binding structure of
expressions. By the \emph{binding structure}\indexdef{binding structure} we mean the
invisible link generated by binders such as $\lambda$, $\Pi$ and
$\Sigma$ (the latter we are going to meet soon) between the place where the variable is introduced and where it is used. As an example, consider $f : \nat \to (\nat \to \nat)$
defined as 
\[ f(x) \defeq \lamu{y:\nat} x + y \] 
Now if we have assumed somewhere that $y : \nat$, then what is $f(y)$? It would be wrong to just naively replace $x$ by $y$ everywhere in the expression ``$\lam{y}x+y$'' defining $f(x)$, obtaining $\lamu{y:\nat} y + y$, because this means that $y$ gets \define{captured}.
\indexdef{variable!captured}%
Previously, the substituted\index{substitution} $y$ was referring to our assumption, but now it is referring to the argument of the $\lambda$-abstraction. Hence, this naive substitution would destroy the binding structure, allowing us to perform calculations which are semantically unsound.

But what \emph{is} $f(y)$ in this example? Note that bound (or ``dummy'')
variables
\indexdef{variable!bound}%
\indexdef{variable!dummy}%
such as $y$ in the expression $\lamu{y:\nat} x + y$
have only a local meaning, and can be consistently replaced by any
other variable, preserving the binding structure. Indeed, $\lamu{y:\nat} x + y$ is declared to be judgmentally equal\footnote{Use of this equality is often referred to as \define{$\alpha$-conversion.
\indexfoot{alpha conversion@$\alpha$-conversion}
\indexfoot{conversion!alpha@$\alpha$-}
}} to
$\lamu{z:\nat} x + z$.  It follows that 
$f(y)$ is judgmentally equal to  $\lamu{z:\nat} y + z$, and that answers our question.  (Instead of $z$,
any variable distinct from $y$ could have been used, yielding an equal result.)

Of course, this should all be familiar to any mathematician: it is the same phenomenon as the fact that if $f(x) \defeq \int_1^2 \frac{dt}{x-t}$, then $f(t)$ is not $\int_1^2 \frac{dt}{t-t}$ but rather $\int_1^2 \frac{ds}{t-s}$.
A $\lambda$-abstraction binds a dummy variable in exactly the same way that an integral does.

We have seen how to define functions in one variable. One
way to define functions in several variables would be to use the
cartesian product, which will be introduced later; a function with
parameters $A$ and $B$ and results in $C$ would be given the type 
$f : A \times B \to C$. However, there is another choice that avoids
using product types, which is called \define{currying}
\indexdef{currying}%
\indexdef{function!currying of}%
in functional programming.\footnote{The technique is named after the mathematician Haskell Curry}
\index{programming}%

The idea of currying is to represent a function of two inputs $a:A$ and $b:B$ as a function which takes \emph{one} input $a:A$ and returns \emph{another function}, which then takes a second input $b:B$ and returns the result.
That is, we consider two-variable functions to belong to an iterated function type, $f : A \to (B \to C)$.
We may also write this without the parentheses, as $f : A \to B \to C$, with
associativity\index{associativity!of function types} to the right as the default convention.  Then given $a : A$ and $b : B$,
we can apply $f$ to $a$ and then apply the result to $b$, obtaining
$f(a)(b) : C$. To avoid the proliferation of parentheses, we allow ourselves to
write $f(a)(b)$ as $f(a,b)$ even though there are no products
involved. Our notation for definitions with explicit parameters extends to
this situation: we can define a named function $f : A \to B \to C$ by
giving an equation
\[ f(x,y) \defeq c\]
where $c:C$ assuming $x:A$ and $y:B$. Using $\lambda$-abstraction\index{lambda abstraction@$\lambda$-abstraction} this
corresponds to
\[ f \defeq \lamu{x:A}{y:B} c, \]
which may also be written as 
\[ f \defeq x \mapsto y \mapsto c. \] 
Currying a function of three or more arguments is a straightforward extension of what we have just described.
 
\index{type!function|)}%
\index{function|)}%


\section{Universes and families}
\label{sec:universes}

So far, we have been using the expression ``$A$ is a type'' informally. We
are going to make this more precise by introducing \define{universes}.
\index{type!universe|(defstyle}%
\indexsee{universe}{type, universe}%
A universe is a type whose elements are types. As in naive set theory,
we might wish for a universe of all types $\UU_\infty$ including itself
(that is, with $\UU_\infty : \UU_\infty$).
However, as in set
theory, this is unsound, i.e.\ we can deduce from it that every type,
including the empty type representing the proposition False, is inhabited. Indeed, using a
representation of sets as trees, we can directly encode Russell's
paradox \cite{coquand:paradox}.
%  or alternatively, in order to avoid the use of
% inductive types to define trees, we can follow Girard \cite{girard:paradox} and encode the Burali-Forti paradox,
% which shows that the collection of all ordinals cannot be an ordinal.

To avoid the paradox we introduce a hierarchy of universes
\[ \UU_0 : \UU_1 : \UU_2 : \cdots \]
where every universe $\UU_i$ is an element of the next universe
$\UU_{i+1}$. Moreover we assume that our universes are
\define{cumulative},
\indexdef{universe!cumulative}%
\indexdef{cumulative!universes}%
that is that all the elements of the $i$th
universe are also elements of the $(i+1)^{\mathrm{st}}$ universe, i.e.\ if
$A:\UU_i$ then also $A:\UU_{i+1}$.

When we say that $A$ is a type, we mean that it inhabits some universe
$\UU_i$. We usually want to avoid mentioning the level $i$ explicitly,
and just assume that levels can be assigned in a consistent way; thus we
may write $A:\UU$ omitting the level. This way we can even write
$\UU:\UU$, which can be read as $\UU_i:\UU_{i+1}$, having left the
indices implicit.  Writing universes in this style is referred to as
\define{typical ambiguity}.
\indexdef{typical ambiguity}%
When some universe \UU is assumed, we may refer to types belonging to \UU as \define{small types}.
\indexdef{small!type}%
\indexdef{type!small}%

To model a collection of types varying over a given type $A$, we use functions $B : A \to \UU$  whose
codomain is a universe. These functions are called
\define{families of types} (or sometimes \emph{dependent types});
\indexsee{family!of types}{type, family of}%
\indexdef{type!family of}%
\indexsee{type!dependent}{type, family of}%
\indexsee{dependent!type}{type, family of}%
they correspond to families of sets as used in
set theory. An example of a family is the family of finite sets $\Fin
: \nat \to \UU$, where $\Fin(n)$ is a type with exactly $n$ elements, denoted $0_n,1_n,\dots,(n-1)_n$.
(We will be able to define this family soon; see \autoref{ex:fin}.)
A more trivial (but very important) example is the \define{constant} type family
\indexdef{constant!type family}%
\indexdef{type!family of!constant}%
at a type $B:\UU$, which is of course the constant function $(\lam{x:A} B):A\to\UU$.

\index{type!universe|)}%

\section{Dependent function types (\texorpdfstring{$\Pi$}{Π}-types)}
\label{sec:pi-types}

\index{type!dependent function|(defstyle}%
\index{function!dependent|(defstyle}%
\indexsee{dependent!function}{function, dependent}%
\indexsee{type!Pi-@$\Pi$-}{type, dependent function}%
\indexsee{Pi-type@$\Pi$-type}{type, dependent function}%
In type theory we often use a more general version of function
types, called a $\Pi$-type or dependent function type. The elements of
a $\Pi$-type are functions
whose codomain type can vary depending on the
element of the domain to which the function is applied. The name ``$\Pi$-type''
is used because this type can also be regarded as the infinite cartesian
product over a given type.

Given a type $A:\UU$ and a family $B:A \to \UU$, we may construct
the type of dependent functions $\prd{x:A}B(x) : \UU$.
There are many alternative notations for this type, such as
\[ \tprd{x:A} B(x) \qquad \dprd{x:A}B(x) \qquad \lprd{x:A} B(x). \]
If $B$ is a constant family, then the dependent product type is the ordinary function type:
\[\tprd{x:A} B \jdeq (A \to B).\]
Indeed, all the constructions of $\Pi$-types are generalisations of the corresponding constructions on ordinary function types.

\indexdef{definition!of function, direct}%
We can introduce dependent functions by explicit definitions: to
define $f : \prd{x:A}B(x)$, where $f$ is the name of a dependent function to be
defined, we need an expression $b : B(x)$ involving the variable $x:A$,
\index{variable}%
and we write
\[ f(x) \defeq b \qquad \mbox{for $x:A$}\]
Alternatively, we can use \define{$\lambda$-abstraction}%
\index{lambda abstraction@$\lambda$-abstraction|defstyle}%
\begin{equation}
  \label{eq:lambda-abstraction}
  \lamu{x:A} b \ :\ \prd{x:A} B(x).
\end{equation}
\indexdef{application!of dependent function}%
\indexdef{function!dependent!application}%
As with non-dependent functions, we can \define{apply} a dependent function $f : \prd{x:A}B(x)$ to an argument $a:A$ to obtain an element $f(a):B(a)$.
The equalities are the same as for the ordinary function type, i.e.\
given $a:A$ we have $f(a) \jdeq b'$ and  
$(\lamu{x:A} b)(a) \jdeq b'$, where $b' $ is obtained by replacing all
occurrences of $x$ in $b$ by $a$ (avoiding variable capture, as always).
Similarly, we have $f\jdeq (\lam{x} f(x))$ for any $f:\prd{x:A} B(x)$.

An example of a dependent function is $\fmax : \prd{n:\nat} \Fin(n+1)$
which returns the largest element in a nonempty finite type, $\fmax(n) \defeq
n_{n+1}$. 
Another important class of dependent function types are functions which are \define{polymorphic}
\indexdef{function!polymorphic}%
\indexdef{polymorphic function}%
over a given universe.
A polymorphic function is one which takes a type as one of its arguments, and then acts on elements of that type (or other types constructed from it).
\symlabel{idfunc}
\indexdef{function!identity}%
\indexdef{identity!function}%
An example is the polymorphic identity function $\idfunc : \prd{A:U} A \to A$, which we define by $\idfunc{} \defeq \lam{A:\type}{x:A} x$, or more succinctly $\idfunc[A] \defeq \lamu{x:A}x$.

As for ordinary functions, we use currying to define dependent functions with
several arguments. However, in the dependent case the second domain may
depend on the first one, and the codomain may depend on both. That is,
given $A:U$ and type families $B : A \to U$ and $C : \prd{x:A}B(x) \to \UU$, we may construct
the type $\prd{x:A}{y : B(x)} C(x,y)$ of functions with two
arguments.
(Like $\lambda$-abstractions, $\Pi$s automatically scope\index{scope} over the rest of the expression unless delimited; thus $C : \prd{x:A}B(x) \to \UU$ means $C : \prd{x:A}(B(x) \to \UU)$.)
Likewise, given $f:\prd{x:A}{y : B(x)} C(x,y)$ and arguments $a:A$ and $b:B(a)$, we have $f(a)(b) : C(a,b)$, which,
as before, we write as $f(a,b) : C(a,b)$.

\index{type!dependent function|)}%
\index{function!dependent|)}%


\section{Product types}
\label{sec:finite-product-types}

Given types $A,B:\UU$ we introduce the type $A\times B:\UU$, which we call their \define{cartesian product}.
\indexsee{cartesian product}{type, product}%
\indexsee{type!cartesian product}{type, product}%
\index{type!product|(defstyle}%
\indexsee{product!of types}{type, product}%
We also introduce a nullary product type, called the \define{unit type} $\unit : \UU$
\indexsee{unit!type}{type, unit}%
\index{type!unit|(defstyle}%
We intend the elements of $A\times B$ to be pairs $\tup{a}{b} : A \times B$, where $a:A$ and $b:B$, and the only element of $\unit$ to be some particular object $\ttt : \unit$.
\indexdef{pair!ordered}%
However, unlike in set theory, where we define ordered pairs to be particular sets and then collect them all together into the cartesian product, in type theory, ordered pairs are a primitive concept, as are functions.   

There is a general pattern to how one introduces a new primitive concept in type theory, and because products are our second example following this pattern,\footnote{The description of universes above is an   exception.} it is worth emphasizing the general form:
To specify a type, we say\index{rule}
\begin{enumerate}
\item how to construct elements of that type.  These are called the   type's \emph{constructors}.  For example, the function type has one   constructor, $\lambda$-abstraction\index{lambda abstraction@$\lambda$-abstraction}.
  \indexdef{constructor}%
\item how to use elements of that type.  These are called the type's   \emph{eliminators}.
  \indexdef{eliminator}%
  For example, the function type has one   eliminator, function application.
\item equalities relating the constructors and eliminators.  For   example, for functions, we have the rule that $(\lamu{x:A}b)(a)$ is   equal to the substitution of $a$ for $x$ in $b$.
  These are sometimes called the type's \emph{computation rules}.
  \indexdef{computation rule}%
\end{enumerate}

The way to construct pairs is obvious: given $a:A$ and $b:B$, we may form $(a,b):A\times B$.
Similarly, there is a unique way to construct elements of $\unit$, namely we have $\ttt:\unit$.
However, we do not assert as a rule of type theory that ``every element of $A\times B$ is a pair'' --- it turns out that we can \emph{prove} that using the principles to be introduced below.

Now, how can we \emph{use} pairs, i.e.\ how can we define functions out of a product type?
Let us first consider the definition of a non-dependent function $f : A\times B \to C$.
Since we intend the only elements of $A\times B$ to be pairs, in order to define such a function, it should be enough to prescribe the result
when $f$ is applied to a pair $\tup{a}{b}$.  We prescribe these results by
providing a function $g : A \to B \to C$ and defining
\[ f(\tup{a}{b}) \defeq g(a)(b). \]
We avoid writing $g(a,b)$ here, in order to emphasize that $g$ is not a
function on a product. 

In particular, we can derive the \define{projection}
\indexsee{function!projection}{projection}%
\indexdef{projection!from cartesian product type}%
functions
\symlabel{defn:proj}
\begin{align*}
  \fst & :  A \times B \to A \\
  \snd & :  A \times B \to B
\end{align*}
with the defining equations 
\begin{align*}
  \fst(\tup{a}{b}) & \defeq  a \\
  \snd(\tup{a}{b}) & \defeq  b
\end{align*}
%
\symlabel{defn:recursor-times}%
Rather than invoking this principle of function definition every time we want to define a function, an alternative approach is to invoke it once, in a universal case, and then simply apply the resulting function in all other cases.
That is, we may define a function of type
\begin{equation}
  \rec{A\times B} : \prd{C:\UU}(A \to B \to C) \to A \times B \to C
\end{equation}
with the defining equation
\[\rec{A\times B}(C,g,\tup{a}{b}) \defeq g(a)(b). \]
Then instead of defining functions such as $\fst$ and $\snd$ directly by a defining equation, we could instead define
\begin{align*}
  \fst &\defeq \rec{A\times B}(A, \lam{a}{b} a)\\
  \snd &\defeq \rec{A\times B}(B, \lam{a}{b} b).
\end{align*}
We refer to the function $\rec{A\times B}$ as the \define{recursor}
\indexsee{recursor}{recursion principle}%
for product types.
We may also speak of the \define{recursion principle}
\index{recursion principle!for cartesian product}%
\index{cartesian product!recursion principle for}%
for cartesian products, meaning the fact that we can define a function $f:A\times B\to C$ as above by giving its value on pairs.
The name ``recursor'' is a bit unfortunate here, since no recursion is taking place.
It comes the fact that product types are a degenerate example of a general framework for inductive types, and for types such as
the natural numbers, the recursor will actually be recursive.  

We leave it as a simple exercise to show that the recursor can be
derived from the projections and vice versa.
% Ex: Derive from projections

\symlabel{defn:recursor-unit}
We also have a recursor for the unit type:
\[\rec{\unit} : \prd{C:\UU}C \to \unit \to C\]
with the defining equation
\[ \rec{\unit}(C,c,\ttt) \defeq c \]
However, this is completely useless,
because we could have defined such a function directly
by simply ignoring the argument of type $\unit$.

To be able to define \emph{dependent} functions over the product type, we have
to generalize the recursor. Given $C: A \times B \to \UU$, we may
define a function $f : \prd{x : A \times B} C(x)$ by providing a
function
\narrowequation{
 g : \prd{x:A}\prd{y:B} C(\tup{x}{y})
}
with defining equation
\[ f(\tup x y) \defeq g(x)(y). \] 
For example, in this way we can prove that every element of $A\times B$ is a pair; this is called the principle of \define{surjective pairing}.
\index{surjective!pairing, principle of}%
Specifically, we can construct a function
\[ \spr : \prd{x:A \times B} (\id[A\times B]{\tup{\fst {(x)}}{\snd {(x)}}}{x}). \]
Here we are using the identity type, which we are going to introduce below in \autoref{sec:identity-types}.
However, all we need here is to know that there is a reflexivity element $\refl{x} : \id[A]{x}{x}$ for any $x:A$.
Given this, we can define
\[ \spr(\tup{a}{b}) \defeq \refl{\tup{a}{b}} \]
This construction works, because in the case that $x \defeq \tup{a}{b}$ we can 
calculate 
\[ \tup{\fst(\tup{a}{b})}{\snd{(\tup{a}{b})}} \jdeq \tup{a}{b} \]
using the defining equations for the projections. Therefore,
\[ \refl{\tup{a}{b}} : \id{\tup{\fst(\tup{a}{b})}{\snd{(\tup{a}{b})}}}{\tup{a}{b}} \]
is well-typed, since both sides of the equality are judgmentally equal.

More generally, the ability to define dependent functions in this way means that to prove a property for all elements of a product, it is enough 
to prove it for its canonical elements, the tuples.
When we come to inductive types such as the natural numbers, the analogous property will be the ability to write proofs by induction.
Thus, if we do as we did above and apply this principle once in the universal case, we call the resulting function \define{induction} for product types: given $A,B : \UU$ we have
\symlabel{defn:induction-times}
\[ \ind{A\times B} : \prd{C:A \times B \to \UU}
\Parens{\prd{x:A}{y:B} C(\tup{x}{y})} \to \prd{x:A \times B} C(x) \]
with the defining equation 
\[ \ind{A\times B}(C,g,\tup{a}{b}) \defeq g(a)(b). \]
Similarly, we may speak of a dependent function defined on pairs being obtained from the \define{induction principle}
\index{induction principle}%
\index{cartesian product!induction principle for}%
\index{induction principle!for cartesian product}%
of the cartesian product.
It is easy to see that the recursor is just the special case of induction
in the case that the family $C$ is constant.
Because induction describes how to use an element of the product type, induction is also called the \define{(dependent) eliminator},
\indexsee{eliminator!of inductive type!dependent}{induction principle}%
and recursion the \define{non-dependent eliminator}.
\indexsee{eliminator!of inductive type!non-dependent}{recursion principle}%

% We can read induction propositionally as saying that a property which
% is true for all pairs holds for all elements of the product type.

Induction for the unit type turns out to be more useful than the
recursor: 
\symlabel{defn:induction-unit}
\[ \ind{\unit} : \prd{C:\unit \to \UU} C(\ttt) \to \prd{x:\unit}C(x)\]
with the defining equation
\[ \ind{\unit}(C,c,\ttt) \defeq c \]
Induction enables us to \emph{prove} that the only inhabitant of the
unit type is $\ttt$, that is we can construct
\[\un : \prd{x:\unit} \id{x}{\ttt} \]
by using the defining equations
\[\un(\ttt) \defeq \refl{\ttt} \]
or equivalently by using induction:
\[\un \defeq \ind{\unit}(\lamu{x:\unit} \id{x}{\ttt},\refl{\ttt}). \]

\index{type!product|)}%
\index{type!unit|)}%

\section{Dependent pair types (\texorpdfstring{$\Sigma$}{Σ}-types)}
\label{sec:sigma-types}

Just as we generalized function types (\autoref{sec:function-types}) to dependent function types (\autoref{sec:pi-types}), it is often useful to generalize the product types from \autoref{sec:finite-product-types} to allow the type of
the second component of a pair to vary depending on the choice of the first
component. This is called a $\Sigma$-type, because in set theory it
corresponds to an indexed sum (in the sense of coproduct or
disjoint union) over a given type.

\index{type!dependent pair|(defstyle}%
\indexsee{type!dependent sum}{type, dependent pair}%
\indexsee{type!Sigma-@$\Sigma$-}{type, dependent pair}%
\indexsee{Sigma-type@$\Sigma$-type}{type, dependent pair}%
\indexsee{sum!dependent}{type, dependent pair}%
Given a type $A:\UU$ and a family $B : A \to \UU$, the dependent
pair type is written as $\sm{x:A} B(x) : \UU$.
Alternative notations are 
\[ \tsm{x:A} B(x) \hspace{2cm} \dsm{x:A}B(x) \hspace{2cm} \lsm{x:A} B(x). \]
Like other binding constructs such as $\lambda$-abstractions and $\Pi$s, $\Sigma$s automatically scope\index{scope} over the rest of the expression unless delimited, so e.g.\ $\sm{x:A} B(x) \times C(x)$ means $\sm{x:A} (B(x) \times C(x))$.

\symlabel{defn:dependent-pair}%
\indexdef{pair!dependent}%
The way to construct elements of a dependent pair type is by pairing: we have
$\tup{a}{b} : \sm{x:A} B(x)$ given $a:A$ and $b:B(a)$.
If $B$ is constant, then the dependent pair type is the
ordinary cartesian product type:
\[ (\sm{x:A} B) \jdeq (A \times B).\]
All the constructions on $\Sigma$-types arise as straightforward generalisations of the ones for product types, with dependent functions often replacing non-dependent ones.

For instance, the recursion principle%
\index{recursion principle!for dependent pair type}
says that to define a non-dependent function out of a $\Sigma$-type
$f : (\sm{x:A} B(x)) \to C$, we provide a function 
$g : \prd{x:A} B(x) \to C$, and then we can define $f$ via the defining
equation
\[ f(\tup{a}{b}) \defeq g(a)(b) \]
\indexdef{projection!from dependent pair type}%
For instance, we can derive the first projection from a $\Sigma$-type:
\symlabel{defn:dependent-proj1}
\begin{eqnarray*}
  \fst & : & (\sm{x : A}B(x)) \to A.
  % \snd & : & \prd{p:\sm{x : A}B(x)}B(\fst(p)).
\end{eqnarray*}
by the defining equation
\begin{eqnarray*}
  \fst(\tup{a}{b}) & \defeq & a.
  % \snd(\tup{a}{b}) & \defeq & b
\end{eqnarray*}
However, since the type of the second component of a pair
\narrowequation{
  (a,b):\sm{x:A} B(x)
}
is $B(a)$, the second projection must be a \emph{dependent} function, whose type involves the first projection function:
\symlabel{defn:dependent-proj2}
\[ \snd : \prd{p:\sm{x : A}B(x)}B(\fst(p)). \]
Thus we need the \emph{induction} principle%
\index{induction principle!for dependent pair type}
for $\Sigma$-types (the ``dependent eliminator'').
This says that to construct a dependent function out of a $\Sigma$-type into a family $C : (\sm{x:A} B(x)) \to \UU$, we need a function
\[ g : \prd{a:A}{b:B(x)} C(\tup{a}{b}). \]
We can then derive a function 
\[ f : \prd{p : \sm{x:A}B(x)} C(p) \]
with  defining equation
\[ f(\tup{a}{b}) \defeq g(a)(b).\]
Applying this with $C(p)\defeq B(\fst(p))$, we can define
\narrowequation{
\snd : \prd{p:\sm{x : A}B(x)}B(\fst(p))
}
with the obvious equation
\[ \snd(\tup{a}{b})  \defeq  b. \]
To convince ourselves that this is correct, we note that $B (\fst(\tup{a}{b})) \jdeq B(a)$, using the defining equation for $\fst$, and
indeed $b : B(a)$.

We can package the recursion and induction principles into the recursor for $\Sigma$:
\symlabel{defn:recursor-sm}
\[ \rec{\sm{x:A}B(x)} : \dprd{C:\UU}\Parens{\tprd{x:A} B(x) \to C} \to
\big(\tsm{x:A}B(x)\big) \to C \]
with the defining equation
\[ \rec{\sm{x:A}B(x)}(C,g,\tup{a}{b}) \defeq g(a)(b) \]
and the corresponding induction operator:
\symlabel{defn:induction-sm}
\begin{narrowmultline*}
  \ind{\sm{x:A}B(x)} : \narrowbreak
    \dprd{C:(\sm{x:A} B(x)) \to \UU}
    \Big(\tprd{a:A}{b:B(x)} C(\tup{a}{b})\Big)
    \to \dprd{p : \sm{x:A}B(x)} C(p)
\end{narrowmultline*}
with the defining equation 
\[ \ind{\sm{x:A}B(x)}(C,g,\tup{a}{b}) \defeq g(a)(b). \]
As before, the recursor is the special case of induction
when the family $C$ is constant.

As a further example, consider the following principle, where $A$ and $B$ are types and $R:A\to B\to \UU$.
\[ \ac : \Big(\tprd{x:A} \tsm{y :B} R(x,y)\Big) \to
\Big(\tsm{f:A\to B} \tprd{x:A} R(x,f(x))\Big)
\]
We may regard $R$ as a ``proof-relevant relation''
\index{mathematics!proof-relevant}%
between $A$ and $B$, with $R(a,b)$ the type of witnesses for relatedness of $a:A$ and $b:B$.
Then $\ac$ says intuitively that if we have a dependent function $g$ assigning to every $a:A$ a dependent pair $(b,r)$ where $b:B$ and $r:R(a,b)$, then we have a function $f:A\to B$ and a dependent function assigning to every $a:A$ a witness that $R(a,f(a))$.
Our intuition tells us that we can just split up the values of $g$ into their components.
Indeed, using the projections we have just defined, we can define:
\[ \ac(g) \defeq \Big(\lamu{x:A} \fst(g(x)),\, \lamu{x:A} \snd(g(x))\Big). \]
To verify that this is well-typed, note that if $g:\prd{x:A} \sm{y :B} R(x,y)$, we have
\begin{eqnarray*}
\lamu{x:A} \fst(g(x)) & : &  A \to  B \\
\lamu{x:A} \snd(g(x)) & : &  \tprd{x:A} R(a,\fst(g(x))).
\end{eqnarray*}
Moreover, the type $\prd{x:A} R(a,\fst(g(x)))$ is the result of substituting the function $\lamu{x:A} \fst(g(x))$ for $f$ in the family being summed over in the co\-do\-main of \ac:
\[ \tprd{x:A} R(x,\fst(f(x))) \jdeq
\Big(\lamu{f:A\to B} \tprd{x:A} R(x,f(x))\Big)\big(\lamu{x:A} \fst(g(x))\big). \]
Thus, we have
\[ \Big(\lamu{x:A} \fst(g(x)),\, \lamu{x:A} \snd(g(x))\Big) : \tsm{f:A\to B} \tprd{x:A} R(x,f(x))\]
as required.

If we read $\Pi$ as ``for all'' and $\Sigma$ as ``there exists'', then the type of the function $\ac$ expresses:
\emph{if for all $x:A$ there is a $y:B$ such that $R(x,y)$, then there is a function $f : A \to B$ such that for all $x:A$ we have $R(x,f(x))$}.
Since this sounds like a version of the axiom of choice, the function \ac has traditionally been called the \define{type-theoretic axiom of choice}.
\index{axiom!of choice!type-theoretic}%
However, no choice is actually involved, since the choices have already been given to us in the premise: all we have to do is take it apart into two functions: one representing the choice and the other its correctness.
In \autoref{sec:axiom-choice} we will give another formulation of an ``axiom of choice'' which is closer to the usual one.

\index{type!dependent pair|)}%

\section{Coproduct types}
\label{sec:coproduct-types}

Given $A,B:\UU$, we introduce their \define{coproduct} type $A+B:\UU$.
\indexsee{coproduct}{type, coproduct}%
\index{type!coproduct|(defstyle}%
\indexsee{disjoint sum}{type, coproduct}%
\indexsee{disjoint union}{type, coproduct}%
\indexsee{sum!disjoint}{type, coproduct}%
\indexsee{union!disjoint}{type, coproduct}%
This corresponds to the \emph{disjoint union} in set theory, and we may also use that name for it.
In type theory, as was the case with functions and products, the coproduct must be a fundamental construction, since there is no previously given notion of ``union of types''.
We also introduce a
nullary version: the \define{empty type $\emptyt:\UU$}.
\indexsee{empty type}{type, empty}%
\index{type!empty|(defstyle}%

There are two ways to construct elements of $A+B$, either as $\inl(a) : A+B$ for $a:A$, or as
$\inr(b):A+B$ for $b:B$. There are no ways to construct elements of the empty type. 

\index{recursion principle!for coproduct}
To construct a non-dependent function $f : A+B \to C$, we need 
functions $g_0 : A \to C$ and $g_1 : B \to C$. Then $f$ is defined
via the defining equations
\begin{eqnarray*}
  f(\inl(a)) & \defeq & g_0(a) \\
  f(\inl(b)) & \defeq & g_1(b).
\end{eqnarray*}
That is, the function $f$ is defined by \define{case analysis}.
\indexdef{case analysis}%
As before, we can derive the recursor:
\symlabel{defn:recursor-plus}%
\[ \rec{A+B} : \dprd{C:\UU}(A \to C) \to (B\to C) \to A+B \to C\]
with the defining equations
\begin{eqnarray*}
\rec{A+B}(C,g_0,g_1,\inl(a)) & \defeq & g_0(a) \\
\rec{A+B}(C,g_0,g_1,\inr(b)) & \defeq & g_1(b)
\end{eqnarray*}

\index{recursion principle!for empty type}
We can always construct a function $f : \emptyt \to C$ without
having to give any defining equations, because there are no elements of \emptyt on which to define $f$.
Thus, the recursor for $\emptyt$ is
\symlabel{defn:recursor-emptyt}
\[\rec{\emptyt} : \tprd{C:\UU} \emptyt \to C,\]
which constructs the canonical function from the empty type to any other type.
Logically, it corresponds to the principle \emph{ex falso quod libet}. 

\index{induction principle!for coproduct}
To construct a dependent function $f:\prd{x:A+B}C(x)$ out of a coproduct, we assume as given the family 
$C: (A + B) \to \UU$, and 
require 
\begin{eqnarray*}
  g_0 & : & \prd{a:A} C(\inl(a)) \\
  g_1 & : & \prd{b:B} C(\inr(b)).
\end{eqnarray*}
This yields $f$ with the defining equations:
\begin{eqnarray*}
  f(\inl(a)) & \defeq & g_0(a) \\
  f(\inl(b)) & \defeq & g_1(b) .
\end{eqnarray*}
We package this scheme into the induction principle for coproducts:
\symlabel{defn:induction-plus}
\begin{narrowmultline*}
  \ind{A+B} :
  \dprd{C: (A + B) \to \UU}
  \Big(\tprd{a:A} C(\inl(a))\Big) \to \narrowbreak
  \Big(\tprd{b:B} C(\inr(b))\Big) \to \tprd{x:A+B}C(x) 
\end{narrowmultline*}
As before, the recursor arises in the case that the family $C$ is
constant. 

\index{induction principle!for empty type}
The induction principle for the empty type
\symlabel{defn:induction-emptyt}
\[ \ind{\emptyt} : \prd{C:\emptyt \to \UU}{z:\emptyt} C(z) \]
gives us a way to define a trivial dependent function out of the
empty type. % In the presence of $\eta$-equality it is derivable
% from the recursor.
% ex

\index{type!coproduct|)}%
\index{type!empty|)}%


\section{The type of booleans}
\label{sec:type-booleans}

\index{type!of booleans|(defstyle}%
The type of booleans $\bool:\UU$ is intended to have exactly two elements 
$\btrue,\bfalse : \bool$. It is clear that we could construct this
type out of coproduct\index{type!coproduct} and unit\index{type!unit} types as $\unit + \unit$. However,
since it is used frequently, we give the explicit rules here.
Indeed, we are going to observe that we can also go the other way
and derive binary coproducts from $\Sigma$-types and $\bool$.

\index{recursion principle!for type of booleans}
To derive a function $f : \bool \to C$ we need $c_0,c_1 : C$ and
add the defining equations
\begin{eqnarray*}
  f(\bfalse) & \defeq & c_0 \\
  f(\btrue) & \defeq & c_1
\end{eqnarray*}
The recursor corresponds to the if-then-else construct in
functional programming:
\symlabel{defn:recursor-bool}
\[ \rec{\bool} : \prd{C:\UU}  C \to C \to \bool \to C \]
with the defining equations
\begin{eqnarray*}
  \rec{\bool}(C,c_0,c_1,\bfalse) & \defeq & c_0 \\
  \rec{\bool}(C,c_0,c_1,\btrue) & \defeq & c_1
\end{eqnarray*}

\index{induction principle!for type of booleans}
Given $C : \bool \to \UU$, to derive a dependent function 
$f : \prd{x:\bool}C(x)$ we need $c_0:C(\bfalse)$ and $c_1 : C(\btrue)$, in which case we can give the defining equations
\begin{eqnarray*}
  f(\bfalse) & \defeq & c_0 \\
  f(\btrue) & \defeq & c_1.
\end{eqnarray*}
We package this up into the induction principle
\symlabel{defn:induction-bool}
\[ \ind{\bool} : \dprd{C:\bool \to \UU}  C(\bfalse) \to C(\btrue)
\to \tprd{x:\bool} C(x) \]
with the defining equations
\begin{eqnarray*}
  \ind{\bool}(C,c_0,c_1,\bfalse) & \defeq & c_0 \\
  \ind{\bool}(C,c_0,c_1,\btrue) & \defeq & c_1
\end{eqnarray*}

As an example, using the induction principle we can prove that, as we expect, every element of $\bool$ is either $\btrue$ or $\bfalse$.
As before, we use the equality types which we have not yet introduced, but we need only the fact that everything is equal to itself by $\refl{x}:x=x$.

\begin{thm}\label{thm:allbool-trueorfalse}
  We have
  \[ \prd{x:\bool}(x=\bfalse)+(x=\btrue). \]
\end{thm}
\begin{proof}
  We use the induction principle with $C(x) \defeq (x=\bfalse)+(x=\btrue)$.
  The two inputs are $\inl(\refl{\bfalse}) : C(\bfalse)$ and $\inr(\refl{\btrue}):C(\btrue)$.
\end{proof}

We have remarked that $\Sigma$-types can be regarded as analogous to indexed disjoint unions, while coproducts are binary disjoint unions.
It is natural to expect that a binary disjoint union $A+B$ could be constructed as an indexed one over the two-element type \bool.
For this we need a type family $P:\bool\to\type$ such that $P(\bfalse)\jdeq A$ and $P(\btrue)\jdeq B$.
Indeed, we can obtain such a family precisely by the recursion principle for $\bool$.
\index{type!family of}%
(The ability to define \emph{type families} by induction and recursion, using the fact that the universe $\UU$ is itself a type, is one of the most subtle and important aspects of type theory.)
Thus, we could have defined
\index{type!coproduct}%
\[ A + B \defeq \sm{x:\bool} \rec{\bool}(\UU,A,B,x) \]
with
\begin{eqnarray*}
  \inl(a) & \defeq & \tup{\bfalse}{a} \\
  \inr(b) & \defeq & \tup{\btrue}{b}.
\end{eqnarray*}
We leave it as an exercise to derive the induction principle of a coproduct type from this definition.
(See also \autoref{ex:sum-via-bool,sec:appetizer-univalence}.)

We can apply the same idea to products and $\Pi$-types: we could have defined
\[ A \times B \defeq \prd{x:\bool}\rec{\bool}(\UU,A,B,x) \]
Pairs could then be constructed using induction for \bool:
\[ \tup{a}{b} \defeq \ind{\bool}(\rec{\bool}(\UU,A,B),a,b) \]
while the projections are straightforward applications
\begin{eqnarray*}
  \fst(p) & \defeq & p(\btrue) \\
  \snd(p) & \defeq & p(\bfalse).
\end{eqnarray*}
The derivation of the induction principle for binary products defined in this way is a bit more involved, and requires function extensionality, which we will introduce in \autoref{sec:compute-pi}.
Moreover, we do not get the same judgmental equalities; see \autoref{ex:prod-via-bool}.
This is a recurrent issue when encoding one type as another; we will return to it in \autoref{sec:htpy-inductive}. 

We may occasionally refer to the elements $\btrue$ and $\bfalse$ of $\bool$ as ``true'' and ``false'' respectively.
However, note that unlike in classical\index{mathematics!classical} mathematics, we do not use elements of $\bool$ as truth values
\index{value!truth}%
or as propositions.
(Instead we identify propositions with types; see \autoref{sec:pat}.)
In particular, the type $A \to \bool$ is not generally the power set\index{power set} of $A$; it represents only the ``decidable'' subsets of $A$ (see \autoref{cha:logic}).
\index{decidable!subset}%

\index{type!of booleans|)}%


\section{The natural numbers}
\label{sec:inductive-types}

\index{type!of natural numbers|(defstyle}%
\index{natural numbers|(defstyle}%
\indexsee{number!natural}{natural numbers}%
The rules we have introduced so far do not allow us to construct any infinite types.
The simplest infinite type we can think of (and which is also, of course, extremely useful) is the type $\nat : \UU$ of natural numbers.
The elements of $\nat$ are constructed using $0 : \nat$\indexdef{zero} and the successor\indexdef{successor} operation $\suc : \nat \to \nat$.
When denoting natural numbers, we adopt the usual decimal notation $1 \defeq \suc(0)$, $2 \defeq \suc(1)$, $3 \defeq \suc(2)$, \dots.

The essential property of the natural numbers is that we can define functions by recursion and perform proofs by induction --- where now the words ``recursion'' and ``induction'' have a more familiar meaning.
\index{recursion principle!for natural numbers}%
To construct a non-dependent function $f : \nat \to C$ out of the natural numbers by recursion, it is enough to provide a starting point $c_0 : C$ and a ``next step'' function $c_s : \nat \to C \to C$.
This gives rise to $f$ with the defining equations
\begin{eqnarray*}
  f(0) & \defeq & c_0 \\
  f(\suc(n)) & \defeq & c_s(n,f(n)).
\end{eqnarray*}
We say that $f$ is defined by \define{primitive recursion}.
\indexdef{primitive!recursion}%
\indexdef{recursion!primitive}%

As an example, we look at how to define a function on natural numbers which doubles its argument.
In this case we have $C\defeq \nat$.
We first need to supply the value of $\dbl(0)$, which is easy: we put $c_0 \defeq 0$.
Next, to compute the value of $\dbl(\suc(n))$ for a natural number $n$, we first compute the value of $\dbl(n)$ and then perform the successor operation twice.
This is captured by the recurrence $c_s(n,y) \defeq \suc(\suc(y))$.
Note that the second argument $y$ of $c_s$ stands for the result of the \emph{recursive call} $\dbl(n)$.

Defining $\dbl:\nat\to\nat$ by primitive recursion in this way, therefore, we obtain the defining equations:
\begin{align*}
  \dbl(0) &\defeq 0\\
  \dbl(\suc(n)) &\defeq \suc(\suc(\dbl(n))).
\end{align*}
This indeed has the correct computational behavior: for example, we have 
\begin{align*}
  \dbl(2) &\jdeq \dbl(\suc(\suc(0)))\\
  & \jdeq c_s(\suc(0), \dbl(\suc(0))) \\
                 & \jdeq \suc(\suc(\dbl(\suc(0)))) \\
                 & \jdeq \suc(\suc(c_s(0,\dbl(0)))) \\
                 & \jdeq \suc(\suc(\suc(\suc(\dbl(0))))) \\
                 & \jdeq \suc(\suc(\suc(\suc(c_0)))) \\
                 & \jdeq \suc(\suc(\suc(\suc(0))))\\
                 &\jdeq 4.
\end{align*}
We can define multi-variable functions by primitive recursion as well, by currying and allowing $C$ to be a function type.
\indexdef{addition!of natural numbers}
For example, we define addition $\add : \nat \to \nat \to \nat$ with $C \defeq \nat \to \nat$ and the following data:
\begin{align*}
  \add_0 & : \nat \to \nat \\
  \add_0 (n) & \defeq n \\
  \add_s & : \nat \to (\nat \to \nat) \to (\nat \to \nat) \\
  \add_s(n,g,m) & \defeq \suc(g(m))
\end{align*}
We thus obtain $\add : \nat \to \nat \to \nat$ satisfying the definitional equalities
\begin{eqnarray*}
  \add(0,n) & \jdeq & n \\
  \add(\suc(m),n) & \jdeq & \suc(\add(m,n)) 
\end{eqnarray*}
As usual, we write $\add(m,n)$ as $m+n$.
The reader is invited to verify that $2+2\jdeq 4$.
% ex: define multiplication and exponentiation.

As in previous cases, we can package the principle of primitive recursion into a recursor:
\[\rec{\nat}  : \dprd{C:\UU} C \to (\nat \to C \to C) \to \nat \to C \]
with the defining equations
\symlabel{defn:recursor-nat}
\begin{eqnarray*}
\rec{\nat}(C,c_0,c_s,0)  & \defeq & c_0 \\
\rec{\nat}(C,c_0,c_s,\suc(n)) & \defeq & c_s(n,\rec{\nat}(C,c_0,c_s,n))
\end{eqnarray*}
%ex derive rec from it
Using $\rec{\nat}$ we can present $\dbl$ and $\add$ as follows:
\begin{align}
\dbl &\defeq \rec\nat\big(\nat,\, 0,\, \lamu{n:\nat}{y:\nat} \suc(\suc(y))\big) \label{eq:dbl-as-rec}\\
\add &\defeq \rec{\nat}\big(\nat \to \nat,\, \lamu{n:\nat} n,\, \lamu{n:\nat}{g:\nat \to \nat}{m :\nat} \suc(g(m))\big)
\end{align}
Of course, all functions definable only using the primitive recursion principle will be \emph{computable}.
(The presence of higher function types does, however, mean we can define more than the usual primitive recursive functions; see e.g.~\autoref{ex:ackermann}.)
This is appropriate in constructive mathematics;
\index{mathematics!constructive}%
in \autoref{sec:intuitionism,sec:axiom-choice} we will see how to augment type theory so that we can define more general mathematical functions.

\index{induction principle!for natural numbers}
We now follow the same approach as for other types, generalizing primitive recursion to dependent functions to obtain an \emph{induction principle}.
Thus, assume as given a family $C : \nat \to \UU$, an element $c_0 : C(0)$, and a function $c_s : \prd{n:\nat} C(n) \to C(\suc(n))$; then we can construct $f : \prd{n:\nat} C(n)$ with the defining equations:
\begin{eqnarray*}
  f(0) & \defeq & c_0 \\
  f(\suc(n)) & \defeq & c_s(n,f(n))
\end{eqnarray*}
We can also package this into a single function
\symlabel{defn:induction-nat}
\[\ind{\nat}  : \dprd{C:\nat\to \UU} C(0) \to \big(\tprd{n : \nat} C(n) \to C(\suc(n))\big) \to \tprd{n : \nat} C(n) \]
with the defining equations
\begin{eqnarray*}
\ind{\nat}(C,c_0,c_s,0)  & \defeq & c_0 \\
\ind{\nat}(C,c_0,c_s,\suc(n)) & \defeq & c_s(n,\ind{\nat}(C,c_0,c_s,n))  
\end{eqnarray*}
Here we finally see the connection to the classical notion of proof by induction.
Recall that in type theory we represent propositions by types, and proving a proposition by inhabiting the corresponding type.
In particular, a \emph{property} of natural numbers is represented by a family of types $P:\nat\to\type$.
From this point of view, the above induction principle says that if we can prove $P(0)$, and if for any $n$ we can prove $P(\suc(n))$ assuming $P(n)$, then we have $P(n)$ for all $n$.
This is, of course, exactly the usual principle of proof by induction on natural numbers.

\index{associativity!of addition!of natural numbers}
As an example consider how we might represent an explicit proof that $+$ is associative.
(We will not actually write out proofs in this style, but it serves as a useful example for understanding how induction is represented formally in type theory.)
To derive
\[\ass : \prd{i,j,k:\nat} \id{i + (j + k)}{(i + j) + k}, \]
it is sufficient to supply
\[ \ass_0 :  \prd{j,k:\nat} \id{0 + (j + k)}{(0+ j) + k} \]
and
\begin{narrowmultline*}
  \ass_s  : \prd{i:\nat} \left(\prd{j,k:\nat} \id{i + (j + k)}{(i + j) + k}\right)
   \narrowbreak
   \to \prd{j,k:\nat} \id{\suc(i) + (j + k)}{(\suc(i) + j) + k}.
\end{narrowmultline*}
To derive $\ass_0$, recall that $0+n \jdeq n$, and hence  $0 + (j + k) \jdeq j+k \jdeq (0+ j) + k$.
Hence we can just set
\[ ass_0(j,k) \defeq \refl{j+k} \]
For $\ass_s$, recall that the definition of $+$ gives $\suc(m)+n \jdeq \suc(m+n)$, and hence 
\begin{eqnarray*}
   \suc(i) + (j + k)  & \jdeq & \suc(i+(j+k)) \qquad\text{and}\\
   (\suc(i)+j)+k & \jdeq & \suc((i+j)+k).
\end{eqnarray*}
Thus, the output type of $\ass_s$ is equivalently $\id{\suc(i+(j+k))}{\suc((i+j)+k)}$.
But its input (the ``inductive hypothesis'')
\index{hypothesis!inductive}%
\index{inductive hypothesis}%
yields $\id{i+(j+k)}{(i+j)+k}$, so it suffices to invoke the fact that if two natural numbers are equal, then so are their successors.
(We will prove this obvious fact in \autoref{lem:map}, using the induction principle of identity types.)
We call this latter fact
$\apfunc{\suc} : %\prd{m,n:\nat}
(\id[\nat]{m}{n}) \to (\id[\nat]{\suc(m)}{\suc(n)})$, so we can define
\[\ass_s(i,h,j,k) \defeq \apfunc{\suc}( %n+(j+k),(n+j)+k,
h(j,k)). \]
Putting these together with $\ind{\nat}$, we obtain a proof of associativity.

\index{type!of natural numbers|)}%
\index{natural numbers|)}%


\section{Pattern matching and recursion}
\label{sec:pattern-matching}

\index{pattern matching|(defstyle}%
\index{definition!by pattern matching|(}%
The natural numbers introduce an additional subtlety over the types considered up until now.
In the case of coproducts, for instance, we could define a function $f:A+B\to C$ either with the recursor:
\[ f \defeq \rec{A+B}(C, g_0, g_1) \]
or by giving the defining equations:
\begin{align*}
  f(\inl(a)) &\defeq g_0(a)\\
  f(\inr(b)) &\defeq g_1(b).
\end{align*}
To go from the former expression of $f$ to the latter, we simply use the computation rules for the recursor.
Conversely, given any defining equations
\begin{align*}
  f(\inl(a)) &\defeq c_0\\
  f(\inr(b)) &\defeq c_1
\end{align*}
where $c_0$ and $c_1$ are expressions that may involve the variables
\index{variable}%
$a$ and $b$, we can express these equations equivalently in terms of the recursor by using $\lambda$-abstraction\index{lambda abstraction@$\lambda$-abstraction}:
\[ f\defeq \rec{A+B}(C, \lam{a} c_0, \lam{b} c_1).\]
In the case of the natural numbers, however, the ``defining equations'' of a function such as $\dbl$:
\begin{align}
  \dbl(0) &\defeq 0 \label{eq:dbl0}\\
  \dbl(\suc(n)) &\defeq \suc(\suc(\dbl(n)))\label{eq:dblsuc}
\end{align}
involve \emph{the function $\dbl$ itself} on the right-hand side.
However, we would still like to be able to give these equations, rather than~\eqref{eq:dbl-as-rec}, as the definition of \dbl, since they are much more convenient and readable.
The solution is to read the expression ``$\dbl(n)$'' on the right-hand side of~\eqref{eq:dblsuc} as standing in for the result of the recursive call, which in a definition of the form $\dbl\defeq \rec{\nat}(\nat,c_0,c_s)$ would be the second argument of $c_s$.

More generally, if we have a ``definition'' of a function $f:\nat\to C$ such as
\begin{align*}
  f(0) &\defeq c_0\\
  f(\suc(n)) &\defeq c_s
\end{align*}
where $c_0$ is an expression of type $C$, and $c_s$ is an expression of type $C$ which may involve the variable $n$ and also the symbol ``$f(n)$'', we may translate it to a definition
\[ f \defeq \rec{\nat}(C,\,c_0,\,\lam{n}{r} c_s') \]
where $c_s'$ is obtained from $c_s$ by replacing all occurrences of ``$f(n)$'' by the new variable $r$.

This style of defining functions by recursion (or, more generally, dependent functions by induction) is so convenient that we frequently adopt it.
It is called definition by \define{pattern matching}.
Of course, it is very similar to how a computer programmer may define a recursive function with a body that literally contains recursive calls to itself.
However, unlike the programmer, we are restricted in what sort of recursive calls we can make: in order for such a definition to be re-expressible using the recursion principle, the function $f$ being defined can only appear in the body of $f(\suc(n))$ as part of the composite symbol ``$f(n)$''.
Otherwise, we could write nonsense functions such as
\begin{align*}
  f(0)&\defeq 0\\
  f(\suc(n)) &\defeq f(\suc(\suc(n))).
\end{align*}
If a programmer wrote such a function, it would simply call itself forever on any positive input, going into an infinite loop and never returning a value.
In mathematics, however, to be worthy of the name, a \emph{function} must always associate a unique output value to every input value, so this would be unacceptable.

This point will be even more important when we introduce more complicated inductive types in \autoref{cha:induction,cha:hits,cha:real-numbers}.
Whenever we introduce a new kind of inductive definition, we always begin by deriving its induction principle.
Only then do we introduce an appropriate sort of ``pattern matching'' which can be justified as a shorthand for the induction principle.

\index{pattern matching|)}%
\index{definition!by pattern matching|)}%

\section{Propositions as types}
\label{sec:pat}

\index{proposition!as types|(defstyle}%
\index{logic!propositions as types|(}%
As mentioned in the introduction, to show that a proposition is true in type theory corresponds to exhibiting an element of the type corresponding to that proposition.
\index{evidence!of the truth of a proposition}%
\index{witness!to the truth of a proposition}%
\index{proof|(}
We regard the elements of this type as \emph{evidence} or \emph{witnesses} that the proposition is true. (They are sometimes even called \emph{proofs}, but this terminology can be misleading, so we generally avoid it.)
In general, however, we will not construct witnesses explicitly; instead we present the proofs in ordinary mathematical prose, in such a way that \emph{in principle} they could be translated into an element of a type.
This is no different from reasoning in classical set theory, where we don't expect to see an explicit derivation using the rules of predicate logic and the axioms of set theory.

However, the type-theoretic perspective on proofs is nevertheless different in important ways.
The basic principle of the logic of type theory is that a proposition is not merely true or false, but rather can be seen as the collection of all possible witnesses of its truth.
Under this conception, proofs are not just the means by which mathematics is communicated, but rather are mathematical objects in their own right, on a par with more familiar objects such as numbers, mappings, groups, and so on.
Thus, since types classify the available mathematical objects and govern how they interact, propositions are nothing but special  types --- namely, types whose elements are proofs.

The basic observation which makes this identification feasible is that we have the following natural correspondence between \emph{logical} operations on propositions, expressed in English, and \emph{type-theoretic} operations on their corresponding types of witnesses.
\index{false}%
\index{true}%
\begin{center}
\medskip
\begin{tabular}{ll}
  \toprule
  English & Type Theory\\
  \midrule
  True & $\unit$ \\
  False & $\emptyt$ \\
  $A$ and $B$ & $A \times B$ \\
  $A$ or $B$ & $A + B$ \\
  If $A$ then $B$ & $A \to B$ \\
  $A$ if and only if $B$ & $(A \to B) \times (B \to A)$ \\
  Not $A$ &  $A \to \emptyt$ \\
  \bottomrule
\end{tabular}
\medskip
\end{center}

The point of the correspondence is that in each case, the rules for constructing and using elements of the type on the right correspond to the rules for reasoning about the proposition on the left.
For instance, the basic way to prove a statement of the form ``$A$ and $B$'' is to prove $A$ and also prove $B$, while the basic way to construct an element of $A\times B$ is as a pair $(a,b)$, where $a$ is an element (or witness) of $A$ and $b$ is an element (or witness) of $B$.
And if we want to use ``$A$ and $B$'' to prove something else, we are free to use both $A$ and $B$ in doing so, analogously to how the induction principle for $A\times B$ allows us to construct a function out of it by using elements of $A$ and of $B$.

Similarly, the basic way to prove an implication ``if $A$ then $B$'' is to assume $A$ and prove $B$, while the basic way to construct an element of $A\to B$ is to give an expression which denotes an element (witness) of $B$ which may involve an unspecified variable element (witness) of type $A$.
And the basic way to use an implication ``if $A$ then $B$'' is deduce $B$ if we know $A$, analogously to how we can apply a function $f:A\to B$ to an element of $A$ to produce an element of $B$.
We strongly encourage the reader to do the exercise of verifying that the rules governing the other type constructors translate sensibly into logic.

Of special note is that the empty type $\emptyt$ corresponds to falsity.
When speaking logically, we refer to an inhabitant of $\emptyt$ as a \define{contradiction}:
\indexdef{contradiction}%
thus there is no way to prove a contradiction, while from a contradiction anything can be derived.
We also define the \define{negation}
\indexdef{negation}%
of a type $A$ to be
%
\begin{equation*}
  \neg A \ \defeq\ A \to \emptyt.
\end{equation*}
%
Thus, a witness of $\neg A$ is a function $A \to \emptyt$, which we may construct by assuming $x : A$ and deriving an element of~$\emptyt$.
\index{proof!by contradiction}%
\index{logic!constructive vs classical}
Note that although the logic we obtain is ``constructive'', as discussed in the introduction, this sort of ``proof by contradiction'' (assume $A$ and derive a contradiction, concluding $\neg A$) is perfectly valid constructively: it is simply invoking the \emph{meaning} of ``negation''.
The sort of ``proof by contradiction'' which is disallowed is to assume $\neg A$ and derive a contradiction as a way of proving $A$.
Constructively, such an argument would only allow us to conclude $\neg\neg A$, and the reader can verify that there is no obvious way to get from $\neg\neg A$ (that is, from $(A\to \emptyt)\to\emptyt$) to $A$.

\mentalpause

The above translation of logical connectives into type-forming operations is referred to as \define{propositions as types}: it gives us a way to translate propositions and their proofs, written in English, into types and their elements.
For example, suppose we want to prove the tautology (one of ``de Morgan's laws''):
\index{law!de Morgan's|(}%
\index{de Morgan's laws|(}%
\begin{equation}\label{eq:tautology1}
  \text{\emph{``If not $A$ and not $B$, then not ($A$ or $B$)''}.}
\end{equation}
An ordinary English proof of this fact might go as follows.
\begin{quote}
  Suppose not $A$ and not $B$, and also suppose $A$ or $B$; we will derive a contradiction.
  There are two cases.
  If $A$ holds, then since not $A$, we have a contradiction.
  Similarly, if $B$ holds, then since not $B$, we also have a contradiction.
  Thus we have a contradiction in either case, so not ($A$ or $B$).
\end{quote}
Now, the type corresponding to our tautology~\eqref{eq:tautology1}, according to the rules given above, is
\begin{equation}\label{eq:tautology2}
  (A\to \emptyt) \times (B\to\emptyt) \to (A+B\to\emptyt)
\end{equation}
so we should be able to translate the above proof into an element of this type.

As an example of how such a translation works, let us describe how a mathematician reading the above English proof might simultaneously construct, in his or her head, an element of~\eqref{eq:tautology2}.
The introductory phrase ``Suppose not $A$ and not $B$'' translates into defining a function, with an implicit application of the recursion principle for the cartesian product in its domain $(A\to\emptyt)\times (B\to\emptyt)$.
This introduces unnamed variables
\index{variable}%
(hypotheses)
\index{hypothesis}%
$x$ and $y$ of types $A\to\emptyt$ and $B\to\emptyt$. At this point our partial definition of an element of~\eqref{eq:tautology2} can be written as
\[ f((x,y)) \defeq\; \Box\;:A+B\to\emptyt \]
with a ``hole'' $\Box$ of type $A+B\to\emptyt$ indicating what remains to be done.
(We could equivalently write $f \defeq \rec{(A\to\emptyt)\times (B\to\emptyt)}(A+B\to\emptyt,\lam{x}{y} \Box)$, using the recursor instead of pattern matching.)
The next phrase ``also suppose $A$ or $B$; we will derive a contradiction'' indicates filling this hole by a function definition, introducing another unnamed hypothesis $z:A+B$, leading to the proof state:
\[ f((x,y))(z) \defeq \;\Box\; :\emptyt \]
Now saying ``there are two cases'' indicates a case split, i.e.\ an application of the recursion principle for the coproduct $A+B$.
If we write this using the recursor, it would be
\[ f((x,y))(z) \defeq \rec{A+B}(\emptyt,\lam{a} \Box,\lam{b}\Box,z) \]
while if we write it using pattern matching, it would be
\begin{align*}
  f((x,y))(\inl(a)) &\defeq \;\Box\;:\emptyt\\
  f((x,y))(\inr(b)) &\defeq \;\Box\;:\emptyt
\end{align*}
Note that in both cases we now have two ``holes'' of type $\emptyt$ to fill in, corresponding to the two cases where we have to derive a contradiction.
Finally, the conclusion of a contradiction from $a:A$ and $x:A\to\emptyt$ is simply application of the function $x$ to $a$, and similarly in the other case.
\index{application!of hypothesis or theorem}%
(Note the convenient coincidence of the phrase ``applying a function'' with that of ``applying a hypothesis'' or theorem.)
Thus our eventual definition is
\begin{align*}
  f((x,y))(\inl(a)) &\defeq x(a)\\
  f((x,y))(\inr(b)) &\defeq y(b).
\end{align*}

As an exercise, you should verify 
the converse tautology \emph{``If not ($A$ or $B$), then  (not $A$) and (not $B$)}'' by exhibiting an element of 
\[ ((A + B) \to \emptyt) \to (A \to \emptyt) \times (B \to \emptyt), \]
for any types $A$ and $B$, using the rules we have just introduced.

\index{logic!classical vs constructive|(}
However, not all classical\index{mathematics!classical} tautologies hold under this interpretation.
For example, the rule 
\emph{``If not ($A$ and $B$), then (not $A$) or (not $B$)''} is not valid: we cannot, in general, construct an element of the corresponding type
\[ ((A \times B) \to \emptyt) \to (A \to \emptyt) + (B \to \emptyt).\]
This reflects the fact that the ``natural'' propositions-as-types logic of type theory is \emph{constructive}.
This means that it does not include certain classical principles, such as the law of excluded middle (\LEM{})\index{excluded middle}
or proof by contradiction,\index{proof!by contradiction}
and others which depend on them, such as this instance of de Morgan's law.
\index{law!de Morgan's|)}%
\index{de Morgan's laws|)}%

Philosophically, constructive logic is so-called because it confines itself to constructions that can be carried out \emph{effectively}, which is to say those with a computational meaning.
Without being too precise, this means there is some sort of algorithm\index{algorithm} specifying, step-by-step, how to build an object (and, as a special case, how to see that a theorem is true).
This requires omission of \LEM{}, since there is no \emph{effective}\index{effective!procedure} procedure for deciding whether a proposition is true or false.

The constructivity of type-theoretic logic means it has an intrinsic computational meaning, which is of interest to computer scientists.
It also means that type theory provides \emph{axiomatic freedom}.\index{axiomatic freedom}
For example, while by default there is no construction witnessing \LEM{}, the logic is still compatible with the existence of one (see \autoref{sec:intuitionism}).
Thus, because type theory does not \emph{deny} \LEM{}, we may consistently add it as an assumption, and work conventionally without restriction.
In this respect, type theory enriches, rather than constrains, conventional mathematical practice.

We encourage the reader who is unfamiliar with constructive logic to work through some more examples as a means of getting familiar with it.
See \autoref{ex:tautologies,ex:not-not-lem} for some suggestions.
\index{logic!classical vs constructive|)}

\mentalpause

So far we have discussed only propositional logic.
\index{quantifier}%
\index{existential quantifier}%
\index{universal quantifier}%
Now we consider \emph{predicate} logic, where in addition to logical connectives like ``and'' and ``or'' we have quantifiers ``there exists'' and ``for all''.
In this case, types play a dual role: they serve as propositions and also as types in the conventional sense, i.e., domains we quantify over.
% Indeed, we are going to exploit this duality meaning that we can also quantify over the type of proofs of a proposition.
% (Does this add anything here?)
A predicate over a type $A$ is represented as a family $P : A \to \UU$, assigning to every element $a : A$ a type $P(a)$ corresponding to the proposition that $P$ holds for $a$. We now extend the above translation with an explanation of the quantifiers:
\begin{center}
  \medskip
  \begin{tabular}{ll}
    \toprule
    English & Type Theory\\
    \midrule
    For all $x:A$, $P(x)$ holds & $\prd{x:A} P(x)$ \\
    There exists $x:A$ such that $P(x)$ & $\sm{x:A}$ $P(x)$ \\
    \bottomrule
  \end{tabular}
  \medskip
\end{center}
As before, we can show that tautologies of (constructive) predicate logic translate into inhabited types.
For example, \emph{If for all $x:A$, $P(x)$ and $Q(x)$ then (for all $x:A$, $P(x)$) and (for all $x:A$, $Q(x)$)} translates to
\[ (\tprd{x:A} P(x) \times Q(x)) \to (\tprd{x:A} P(x)) \times (\tprd{x:A} Q(x)). \]
An informal proof of this tautology might go as follows:
\begin{quote}
  Suppose for all $x$, $P(x)$ and $Q(x)$.
  First, we suppose given $x$ and prove $P(x)$.
  By assumption, we have $P(x)$ and $Q(x)$, and hence we have $P(x)$.
  Second, we suppose given $x$ and prove $Q(x)$.
  Again by assumption, we have $P(x)$ and $Q(x)$, and hence we have $Q(x)$.
\end{quote}
The first sentence begins defining an implication as a function, by introducing a witness for its hypothesis:\index{hypothesis}
\[ f(p) \defeq \;\Box\; : (\tprd{x:A} P(x)) \times (\tprd{x:A} Q(x)) \]
At this point there is an implicit use of the pairing constructor to produce an element of a product type, which is somewhat signposted in this example by the words ``first'' and ``second'':
\[ f(p) \defeq \Big( \;\Box\; : \tprd{x:A} P(x) \;,\; \Box\; : \tprd{x:A}Q(x) \;\Big). \]
The phrase ``we suppose given $x$ and prove $P(x)$'' now indicates defining a \emph{dependent} function in the usual way, introducing a variable
\index{variable}%
for its input.
Since this is inside a pairing constructor, it is natural to write it as a $\lambda$-abstraction\index{lambda abstraction@$\lambda$-abstraction}:
\[ f(p) \defeq \Big( \; \lam{x} \;\big(\Box\; : P(x)\big) \;,\; \Box\; : \tprd{x:A}Q(x) \;\Big). \]
Now ``we have $P(x)$ and $Q(x)$'' invokes the hypothesis, obtaining $p(x) : P(x)\times Q(x)$, and ``hence we have $P(x)$'' implicitly applies the appropriate projection:
\[ f(p) \defeq \Big( \; \lam{x} \proj1(p(x))  \;,\; \Box\; : \tprd{x:A}Q(x) \;\Big). \]
The next two sentences fill the other hole in the obvious way:
\[ f(p) \defeq \Big( \; \lam{x} \proj1(p(x))  \;,\; \lam{x} \proj2(p(x)) \; \Big). \]
Of course, the English proofs we have been using as examples are much more verbose than those that mathematicians usually use in practice; they are more like the sort of language one uses in an ``introduction to proofs'' class.
The practicing mathematician has learned to fill in the gaps, so in practice we can omit plenty of details; and we will generally do so.
The criterion of validity for proofs, however, is always that they can be translated back into the construction of an element of the corresponding type.

\symlabel{leq-nat}
As a more concrete example, consider how to define inequalities of natural numbers.
One natural definition is that $n\le m$ if there exists a $k:\nat$ such that $n+k=m$.
(This uses again the identity types that we will introduce in the next section, but we will not need very much about them.)
Under the propositions-as-types translation, this would yield:
\[ (n\le m) \defeq \sm{k:\nat} (\id{n+k}{m}) .\]
The reader is invited to prove the familiar properties of $\le$ from this definition.
For strict inequality, there are a couple of natural choices, such as
\[ (n<m) \defeq \sm{k:\nat} (\id{n+\suc(k)}{m}) \]
or
\[ (n<m) \defeq (n\le m) \times \neg(\id{n}{m}). \]
The former is more natural in constructive mathematics, but in this case it is actually equivalent to the latter, since $\nat$ has ``decidable equality'' (see \autoref{sec:intuitionism,prop:nat-is-set}).
\index{decidable!equality}%

Note that we can use the universes in type theory to represent ``higher order logic'' --- that is, we can quantify over all propositions or over all predicates.
For example, we can represent the proposition \emph{for all properties $P : A \to \UU$, if $P(a)$ then $P(b)$} as
\[ \prd{P : A \to \UU} P(a) \to P(b) \]
where $A : \UU$ and $a,b : A$.
However, \textit{a priori} this proposition lives in a different, higher, universe than the
propositions we are quantifying over; that is
\[ \Big(\prd{P : A \to \UU_i} P(a) \to P(b)\Big) : \UU_{i+1}. \]
We will return to this question in \autoref{subsec:prop-subsets}.

\mentalpause

We have descibed here a ``proof-relevant''
\index{mathematics!proof-relevant}%
translation of propositions, where the proofs of disjunctions and existential statements carry some information.
For instance, if we have an inhabitant of $A+B$, regarded as a witness of ``$A$ or $B$'', then we know whether it came from $A$ or from $B$.
Similarly, if we have an inhabitant of $\sm{x:A} P(x)$, regarded as a witness of ``there exists $x:A$ such that $P(x)$'', then we know what the element $x$ is (it is the first component of the dependent pair).

As a consequence of the proof-relevant nature of this logic, we may have ``$A$ if and only if $B$'' (which, recall, means $(A\to B)\times (B\to A)$), and yet the types $A$ and $B$ exhibit different behavior.
For instance, it is easy to verify that ``$\mathbb{N}$ if and only if $\unit$'', and yet clearly $\mathbb{N}$ and $\unit$ differ in important ways.
The statement ``$\mathbb{N}$ if and only if $\unit$'' tells us only that \emph{when regarded as a proposition}, the type $\mathbb{N}$ represents the same proposition as $\unit$ (in this case, the true proposition).
We sometimes express ``$A$ if and only if $B$'' by saying that $A$ and $B$ are \define{logically equivalent}.
\indexdef{logical equivalence}%
\indexdef{equivalence!logical}%
This is to be distinguished from the stronger notion of \emph{equivalence of types} to be introduced in \autoref{sec:basics-equivalences,cha:equivalences}:
although $\mathbb{N}$ and $\unit$ are logically equivalent, they are not equivalent.

In \autoref{cha:logic} we will introduce a class of types called ``mere propositions'' for which equivalence and logical equivalence coincide.
Using these types, we will introduce a modification to the above-described logic that is sometimes appropriate, in which the additional information contained in disjunctions and existentials is discarded.

Finally, we note that the propositions-as-types correspondence can be viewed in reverse, allowing us to regard any type $A$ as a proposition, which we prove by exhibiting an element of $A$.
Sometimes we will state this proposition as ``$A$ is \define{inhabited}.''
\indexdef{inhabited type}%
\indexsee{type!inhabited}{inhabited type}%
That is, when we say that $A$ is inhabited, we mean that we have given a (particular) element of $A$, but that we are choosing not to give a name to that element.
Similarly, to say that $A$ is \emph{not inhabited} is the same as to give an element of $\neg A$.
In particular, the empty type $\emptyt$ is obviously not inhabited, since $\neg \emptyt \jdeq (\emptyt \to \emptyt)$ is inhabited by $\idfunc[\emptyt]$.\footnote{This should not be confused with the statement that type theory is consistent, which is the \emph{meta-theoretic} claim that it is not possible to obtain an element of $\emptyt$ by following the rules of type theory.\indexfoot{consistency}}

\index{proof|)}%
\index{proposition!as types|)}%
\index{logic!propositions as types|)}%

\section{Identity types}
\label{sec:identity-types}

\index{type!identity|(defstyle}%
\indexsee{identity!type}{type, identity}%
\indexsee{type!equality}{type, identity}%
\indexsee{equality!type}{type, identity}%
While the previous constructions can be seen as generalisations of
standard set theoretic constructions, our way of handling identity  seems to be
a particular feature of type theory.
According to the propositions-as-types conception, the \emph{proposition} that two elements of the same type $a,b:A$ are equal must correspond to some \emph{type}.
Since this proposition depends on what $a$ and $b$ are, these \define{equality types} or \define{identity types} must be type families dependent on two copies of $A$.

We may write the family as $\idtypevar{A}:A\to A\to\type$, so that $\idtype[A]ab$ is the type representing the proposition of equality between $a$ and $b$.
Once we are familiar with propositions-as-types, however, it is convenient to also use the standard equality symbol for this; thus ``$\id{a}{b}$'' will also be a notation for the \emph{type} $\idtype[A]ab$ corresponding to the proposition that $a$ equals $b$.
For clarity, we may also write ``$\id[A]{a}{b}$'' to specify the type $A$.
If we have an element of $\id[A]{a}{b}$, we may say that $a$ and $b$ are \define{equal}, or sometimes \define{propositionally equal} if we want to emphasize that this is different from judgmental equality as discussed in \autoref{sec:types-vs-sets}.
\index{equality!propositional}%

Just as we remarked in \autoref{sec:pat} that the propositions-as-types versions of ``or'' and ``there exists'' can include more information than just the fact that the proposition is true, nothing prevents an element of the type $\id{a}{b}$ from also including more information.
Indeed, this is the cornerstone of the homotopical interpretation, where we regard witnesses of $\id{a}{b}$ as \emph{paths}\indexdef{path} or \emph{equivalences} between $a$ and $b$ in the space $A$.  Just as there can be more than one path between two points of a space, there can be more than one witness that two objects are equal.  Put differently, we may regard $\id{a}{b}$ as the type of \emph{identifications}\indexdef{identification} of $a$ and $b$, and there may be many different ways in which $a$ and $b$ can be identified.
We will return to the interpretation in \autoref{cha:basics}; for now we focus on the basic rules for the identity type.

Given a type $A:\UU$ and two elements $a,b:A$, we can form the type $\id[A]{a}{b}:\UU$ in the same universe.
The basic way to construct an element of $\id{a}{b}$ is to know that $a$ and $b$ are the same.
Thus, we have a dependent function
\[\refl{} : \prd{a:A} (\id[A]{a}{a})\]
called \define{reflexivity},
\indexdef{reflexivity!of equality}%
which says that every element of $A$ is equal to itself (in a specified way).

In particular, this means that if $a$ and $b$ are \emph{judgmentally} equal, $a\jdeq b$, then we also have an element $\refl{a} : \id[A]{a}{b}$.
This is well-typed because $a\jdeq b$ means that also the type $\id[A]{a}{b}$ is judgmentally equal to $\id{a}{a}$, which is the type of $\refl{a}$.

The induction principle for the identity types is one of the most subtle parts of type theory, and crucial to the homotopy interpretation.
We begin by considering an important consequence of it, the principle that ``equals may be substituted for equals,'' as expressed by the following:
\index{indiscenability of identicals}%
\index{equals may be substituted for equals}%
\begin{description}
\item[Indiscernability of identicals:]
For every family 
\[
C : A \to \UU
\]
there is a function
\[
f : \prd{x,y:A}{p:\id[A] x y} C(x) \to C(y)
\]
such that
\[
f(x,x,\refl{x}) \defeq \idfunc[C(x)].
\]
\end{description}
This says that every family of types $C$ respects equality, in the sense that applying $C$ to \emph{equal} elements of $A$ also results in a function between the resulting types. The displayed equality states that the function associated to reflexivity is the identity function (and we shall see that, in general, the function $f(x,y,p): C(x) \to C(y)$ is always an equivalence of types).

Indiscernability of identicals can be regarded as a recursion principle for the identity type, analogous to those given for booleans and natural numbers above.  It gives a mapping property of $\id[A] x y$ with respect to certain other reflexive, binary relations on $A$, namely those of the form $C(x) \to C(y)$.  We could also formulate a more general recursion principle with respect to reflexive relations of the more general form $C(x,y)$.  However, 
in order to fully characterize the identity type, we must generalize it to an induction principle, which not only considers maps out of $\id[A] x y$ but also families over it.   Put differently, we consider not only allowing equals to be substituted for equals, but also taking into account the evidence $p$ for the equality.  
    
\subsection{Path induction}

The induction principle for the identity type is called \define{path induction}
\index{path induction}%
\index{induction principle!for identity type}%
in view of the homotopical interpretation to be explained in  the introduction to \autoref{cha:basics}.  It can be seen as stating that the family of identity types is freely generated by the elements of the form $\refl{x}: \id{x}{x}$.

\begin{description}
\item[Path induction:] 
  Given a family 
  \[ C : \prd{x,y:A} (\id[A]{x}{y}) \to \UU \]
  and a function
  \[ c :  \prd{x:A} C(x,x,\refl{x}),\]
  there is a function
  \[ f : \prd{x,y:A}{p:\id[A]{x}{y}} C(x,y,p) \]
  such that 
  \[ f(x,x,\refl{x}) \defeq c(x). \]
\end{description}

To understand this principle, consider first the simpler case when $C$
does not depend on $p$.  Then we have $C:A\to A\to \UU$, which we may
regard as a predicate depending on two elements of $A$.  We are
interested in knowing when the proposition $C(x,y)$ holds for some pair
of elements $x,y:A$.  In this case, the hypothesis of path induction
says that we know $C(x,x)$ holds for all $x:A$, i.e.\ that if we
evaluate $C$ at the pair $x, x$, we get a true proposition --- so $C$ is
a reflexive relation.  The conclusion then tells us that $C(x,y)$ holds
whenever $\id{x}{y}$.  This is exactly the more general recursion principle
for reflexive relations mentioned above.

The general, inductive form of the rule allows $C$ to also depend on the witness $p:\id{x}{y}$ to the identity between $x$ and $y$.  In the premise, we not only replace $x, y$ by $x,x$, but also simultaneously replace $p$ by reflexivity: to prove a property for all elements $x,y$ and paths $p : \id{x}{y}$ between them, it suffices to consider all the cases where the elements are $x,x$ and the path is $\refl{x}: \id{x}{x}$.  If we were viewing types just as sets, it would be unclear what this buys us, but since there may be many different identifications $p : \id{x}{y}$ between $x$ and $y$, it makes sense to keep track of them in considering families over the type $\id[A]{x}{y}$.
In \autoref{cha:basics} we will see that this is very important to the homotopy interpretation.

If we package up path induction into a single function, it takes the form:
\symlabel{defn:induction-ML-id}
\begin{narrowmultline*}
  \indid{A} :  \dprd{C : \prd{x,y:A} (\id[A]{x}{y}) \to \UU}
  \Big(\tprd{x:A} C(x,x,\refl{x})\Big) \to 
  \narrowbreak
  \dprd{x,y:A}{p:\id[A]{x}{y}}   C(x,y,p)
\end{narrowmultline*}
with the equality
\[ \indid{A}(C,c,x,x,\refl{x}) \defeq c(x) \]
The function $ \indid{A}$ is traditionally called $J$.  We leave it as an easy exercise to show that indiscernability of identicals follows from path induction.  

\mentalpause

Given a proof $p : \id{a}{b}$,
path induction requires you to replace \emph{both} $a$ and $b$ with the same unknown element $x$; thus in order to define an element of a family
$C$, for all pairs of elements of $A$, it suffices to define it on the diagonal.
In some proofs, however, it is simpler to make use of an equation $p : \id{a}{b}$ by replacing all occurrences of $b$ with $a$ (or vice versa), because it is sometimes easier to do the remainder of the proof for the specific element $a$ mentioned in the equality than for a general unknown $x$.  This motivates a second induction principle for identity types, which says that the family of types $\id[A]{a}{x}$ is generated by the element $\refl{a} : \id{a}{a}$.  As we show below, this second principle is equivalent to the first; it is just sometimes a more convenient formulation.

\index{path induction!based}%
\index{induction principle!for identity type!based}%
\begin{description}
\item[Based path induction:] 
  Fix an element $a:A$, and suppose given a family
  \[ C : \prd{x:A} (\id[A]{a}{x}) \to \UU \]
  and an element
  \[ c : C(a,\refl{a}). \]
  Then we obtain a function
  \[ f : \prd{x:A}{p:\id{a}{x}} C(x,p) \]
  such that
  \[ f(a,\refl{a}) \defeq c .\]
\end{description}

Here, $C(x,p)$ is a family of types, where $x$ is an element of $A$ and $p$ is an element of the identity type $\id[A]{a}{x}$, for fixed $a$ in $A$. The based path induction principle says that to define an element of this family for all $x$ and $p$, it suffices to consider
just the case where $x$ is $a$ and $p$ is $\refl{a} : \id{a}{a}$.

Packaged as a function, based path induction becomes:
\symlabel{defn:induction-PM-id}
\begin{align*}
  \indidb{A} :  \dprd{a:A}{C : \prd{x:A} (\id[A]{a}{x}) \to \UU}
  C(a,\refl{a}) \to \dprd{x:A}{p : \id[A]{a}{x}} C(x,p) 
\end{align*}
with the equality
\[ \indidb{A}(a,C,c,a,\refl{a}) \defeq c. \]
%\[ g(x)(x,\refl{x}) \defeq d(x) \]

Below, we show that path induction and based path induction are equivalent.  Because of this, we will sometimes be sloppy and also refer to based path induction simply as ``path induction,'' relying on the reader to infer which principle is meant from the form of the proof.  

\begin{rmk}
Intuitively, the induction principle for the natural numbers expresses the fact that the only natural numbers are $0$ and $\suc(n)$, so if we prove a property for these cases, then we have proved it for all natural numbers.  Applying this same reading to path induction, we might loosely say that path induction expresses the fact that the only path is \refl{}, so if we prove a property for reflexivity, then we have proved it for all paths.  However, this reading is quite confusing in the context of the homotopy interpretation of paths, where there may be many different ways in which two elements $a$ and $b$ can be identified, and therefore many different elements of the identity type!  How can there be many different paths, but at the same time we have an induction principle asserting that the only path is reflexivity?

The key observation is that it is not the identity \emph{type} that is inductively defined, but the identity \emph{family}.
In particular, path induction says that the \emph{family} of types $(\id[A]{x}{y})$, as $x,y$ vary over all elements of $A$, is inductively defined by the elements of the form $\refl{x}$.
This means that to give an element of any other family $C(x,y,p)$ dependent on a \emph{generic} element $(x,y,p)$ of the identity family, it suffices to consider the cases of the form $(x,x,\refl{x})$.
In the homotopy interpretation, this says that the type of triples $(x,y,p)$, where $x$ and $y$ are the endpoints of the path $p$ (in other words, the $\Sigma$-type $\sm{x,y:A}(\id{x}{y})$), is inductively generated by the constant loops at each point $x$.
In homotopy theory, the space corresponding to $\sm{x,y:A}(\id{x}{y})$ is the \emph{free path space} --- the space of paths in $A$ whose endpoints may vary --- and it is in fact the case that any point of this space is homotopic to the constant loop at some point, since we can simply retract one of its endpoints along the given path.

Similarly, based path induction says that for a fixed $a:A$, the \emph{family} of types $(\id[A]{a}{y})$, as $y$ varies over all elements of $A$, is inductively defined by the element $\refl{a}$.
Thus, to give an element of any other family $C(y,p)$ dependent on a generic element $(y,p)$ of this family, it suffices to consider the case $(a,\refl{a})$.
Homotopically, this expresses the fact that the space of paths starting at some chosen point (the \emph{based path space} at that point, which type-theoretically is $\sm{y:A} (\id{a}{y})$) is contractible to the constant loop on the chosen point.
Note that according to propositions-as-types, the type $\sm{y:A}(\id{a}{y})$ can be regarded as ``the type of all elements of $A$ which are equal to $a$'', a type-theoretic version of the ``singleton subset'' $\{a\}$.

Neither of these two principles provides a way to give an element of a family $C(p)$ where $p$ has \emph{two fixed endpoints} $a$ and $b$.  In particular, for a family $C(p : \id[A]{a}{a})$ dependent on a loop, we \emph{cannot} apply path induction and consider only the case for $C(\refl{a})$, and consequently, we cannot prove
that all loops are reflexivity.  Thus, inductively defining the identity family does not prohibit non-reflexivity paths in specific instances of the identity type.
In other words, a path $p:\id{x}{x}$ may not be equal to reflexivity as an element of $(\id{x}{x})$, but the pair $(x,p)$ will nevertheless be equal to the pair $(x,\refl{x})$ as elements of $\sm{y:A}(\id{x}{y})$.
\end{rmk}

\subsection{Equivalence of path induction and based path induction}

The two induction principles for the identity type introduced above are equivalent.
It is easy to see that path induction follows from based path induction principle.
Indeed, let us assume the premises of path induction:
\begin{eqnarray*}
C & : & \prd{x,y:A}(\id[A]{x}{y}) \to \UU  \\
c & :  & \prd{x:A} C(x,x,\refl{x}).
\end{eqnarray*}
Now, given an element $x:A$, we can instantiate both of the above, obtaining
\begin{eqnarray*}
C' & : & \prd{y:A} (\id[A]{x}{y}) \to \UU  \\
C' & \defeq & C(x) \\
c' & : & C'(x,\refl{x}) \\
c' & \defeq & c(x).
\end{eqnarray*}
Clearly, $C'$ and $c'$ match the premises of based path induction and hence we can construct 
\begin{eqnarray*}
g : \prd{y:A}{p : \id{x}{y}} C'(y,p)
\end{eqnarray*}
with the defining equality
\[ g(x,\refl{x}) \defeq c' .\]
Now we observe that $g$'s codomain is equal to $C(x,y,p)$.
Thus, discharging our assumption $x:A$, we can derive a function 
\[ f : \prd{x,y:A}{p : \id[A]{x}{y}} C(x,y,p) \]
with the required judgmental equality $f(x,x,\refl{x}) \judgeq g(x,\refl{x}) \defeq c' \defeq c(x)$.

Another proof of this fact is to observe that any such $f$ can be obtained as an instance of $\indidb{A}$
so it suffices to define $\indid{A}$ in terms of $\indidb{A}$ as
\[ \indid{A}(C,c,x,y,p) \defeq \indidb{A}(x,C(x),c(x),y,p). \]

The other direction is a bit trickier; it is not clear how we can use a particular instance of path induction to derive a particular instance of
based path induction. What we can do instead is to construct one instance of path induction which shows 
all possible instantiations of based path induction at once.
Define
\begin{eqnarray*}
D & : & \prd{x,y:A} (\id[A]{x}{y}) \to \UU \\
D(x,y,p) & \defeq & \prd{C : \prd{z:A}{p : \id[A]{x}{z}} \UU} C(x,\refl{x}) \to C(y,p).
\end{eqnarray*}
Then we can construct the function
\begin{eqnarray*}
d & : & \prd{x : A} D(x,x,\refl{x}) \\
d & \defeq & \lamu{x:A}\lamu{C:\prd{z:A}{p : \id[A]{x}{z}} \UU}\lam{c:C(x,\refl{x})} c
\end{eqnarray*}
and hence using path induction obtain
\[ f : \prd{x,y:A}{p:\id[A]{x}{y}} D(x,y,p) \]
with $f(x,x,\refl{x}) \defeq d(x)$. Unfolding the definition of $D$, we can expand the type of $f$:
\[ f : \prd{x,y:A}{p:\id[A]{x}{y}}{C : \prd{z:A}{p : \id[A]{x}{z}} \UU} C(x,\refl{x}) \to C(y,p). \]
Now given $x:A$ and $p:\id[A]{a}{x}$, we can derive the conclusion of based path induction:
\[ f(a,x,p,C,c) : C(x,p) \]
Notice that we also obtain the correct definitional equality.

Another proof is to observe that any use of based path induction is an instance of $\indidb{A}$  and to define
\begin{narrowmultline*}
\indidb{A}(a,C,c,x,p) \defeq \narrowbreak
\indid{A}
  \begin{aligned}[t]
    \big(
    &\big(\lamu{x,y:A}{p:\id[A]{x}{y}} \tprd{C : \prd{z:A} (\id[A]{x}{z}) \to \UU} C(x,\refl{x}) \to C(y,p)\big),\\
    &(\lamu{x:A}{C:\prd{z:A} (\id[A]{x}{z}) \to \UU}{d:C(x,\refl{x})} d),
     a, x, p, C, c\big) 
   \end{aligned}
\end{narrowmultline*}


Note that the construction given above uses universes. That is, if we want to
model $\indidb{A}$ with $C : \prd{x:A} (\id[A]{a}{x}) \to \UU_i$, we need
to use $\indid{A}$ with 
%
\[ D:\prd{x,y:A} (\id[A]{x}{y}) \to \UU_{i+1} \]
%
since $D$ quantifies over all $C$ of the given type. While this is
compatible with our definition of universes, it is also possible to
derive $\indidb{A}$ without using universes: we can show that $\indid{A}$ entails \autoref{lem:transport,thm:contr-paths}, and that these two principles imply $\indidb{A}$ directly.
We leave the details to the reader as \autoref{ex:pm-to-ml}.

We can use either of the foregoing formulations of identity types
to establish that equality is an equivalence relation, that every function preserves equality and that every family respects equality. We leave the details to the next chapter, where this will be derived  and explained in the context of homotopy type theory.

\subsection{Disequality}
\label{sec:disequality}

Let us also say something about \define{disequality},
\indexdef{disequality}%
which is negation of equality:%
\footnote{We use ``inequality''
  \indexfoot{inequality}%
  to refer to $<$ and $\leq$. Also, note that this is negation of the \emph{propositional} identity type.
Of course, it makes no sense to negate judgmental equality $\jdeq$, because judgments are not subject to logical operations.}
%
\begin{equation*}
  (x \neq_A y) \ \defeq\ \lnot (\id[A]{x}{y}).
\end{equation*}
If $x\neq y$, we say that $x$ and $y$ are \define{unequal}
\indexdef{unequal}%
or \define{not equal}.
%
Just like negation, disequality plays a less important role here than it does in classical\index{mathematics!classical}
mathematics. For example, we cannot prove that two things are equal by proving that they
are not unequal: that would be an application the classical law of double negation, see \autoref{sec:intuitionism}.
Sometimes it is useful to phrase disequality in a positive way. For example,
in~\autoref{RD-inverse-apart-0} we shall prove that a real number $x$ has an inverse if,
and only if, its distance from~$0$ is positive, which is a stronger requirement than $x
\neq 0$.

\index{type!identity|)}%

\sectionNotes

The type theory presented here is a version of Martin-L\"{o}f's intuitionistic type 
theory~\cite{martinlof71itt,Martin-Lof-1972,Martin-Lof-1973,Martin-Lof-1979,martin-lof:bibliopolis}, which itself is based on and influenced 
by the foundational work of Brouwer~\cite{beeson}, Heyting~\cite{heyting1966intuitionism}, Scott~\cite{scott70}, de 
Bruijn~\cite{deBruijn-1973}, Howard~\cite{howard:pat}, Tait~\cite{Tait-1966,Tait-1968}, and Lawvere~\cite{lawvere:adjinfound}.
\index{proof assistant}%
Three principal variants of Martin-L\"{o}f's type theory underlie the NuPRL~\cite{constable+86nuprl-book}, \Coq~\cite{Coq}, and 
\Agda \cite{norell2007towards} computer implementations of type theory.  The theory given here differs from these formulations in a number 
of respects, some of which are critical to the homotopy interpretation, while others are technical conveniences or involve concepts that 
have not yet been studied in the homotopical setting.

Most significantly, the type theory described here is derived from the \emph{intensional} version of Martin-L\"{o}f's type 
theory~\cite{Martin-Lof-1973}, rather than the \emph{extensional} version~\cite{Martin-Lof-1979}.  Whereas the extensional theory makes no 
distinction between judgmental and propositional equality, the intensional theory regards judgmental equality as purely definitional, and 
admits a much broader, proof-relevant interpretation of the identity type that is central to the homotopy interpretation.  From the 
homotopical perspective, extensional type theory confines itself to homotopically discrete sets (see \autoref{sec:basics-sets}), whereas the 
intensional theory admits types with higher-dimensional structure.  The NuPRL system~\cite{constable+86nuprl-book} is extensional, whereas 
both \Coq~\cite{Coq} and \Agda~\cite{norell2007towards} are intensional.  Among intensional type theories, there are a number of variants 
that differ in the structure of identity proofs.  The most liberal interpretation, on which we rely here, admits a \emph{proof-relevant} 
interpretation of equality, whereas more restricted variants impose restrictions such as \emph{uniqueness of identity proofs 
  (UIP)}~\cite{StreicherK}, stating that any two proofs of equality are judgmentally equal, and \emph{Axiom K}~\cite{StreicherK},
\index{axiom!Streicher's Axiom K}
stating that 
the only proof of equality is reflexivity (up to judgmental equality).  These additional requirements may be selectively imposed in the \Coq 
and \Agda\ systems.

%(In the terminology of \autoref{cha:hlevels} such a type theory is about $0$-truncated types.)

Another point of variation among intensional theories is the strength of judgmental equality, particularly as regards objects of function type.  Here we include the equation $f \jdeq \lam{x} f(x)$, which is called \emph{$\eta$-conversion}, as a principle of judgmental equality.  This principle is used, for example, in \autoref{sec:univalence-implies-funext}, to show that univalence implies propositional function extensionality.  Other forms of ``$\eta$ equivalences'' are sometimes considered for other types: for instance, the $\eta$-rule for cartesian products would be a judgmental version of surjective pairing, $u \jdeq (\proj1(u),\proj2(u))$.  This and the corresponding version for dependent pairs would be reasonable choices (which we did not make), but we cannot include all such rules because the corresponding ``$\eta$-rule'' for identity types would trivialize all the higher homotopical structure.  So we are \emph{forced} to leave it out, and the question then becomes where to draw the line. With regards to inductive types, we discuss these points further in~\autoref{sec:htpy-inductive}.

It is important for our purposes that (propositional) equality of functions is taken to be \emph{extensional}.
This is not a consequence of the rules in this chapter; it will be expressed by \autoref{axiom:funext}.
\index{function extensionality}%
This decision is significant for our purposes, because it specifies that equality of functions is as expected in mathematics.  Although we include \autoref{axiom:funext} as an axiom, it may be derived from the univalence axiom and $\eta$-equivalence (see \autoref{sec:univalence-implies-funext}), as well as from the existence of an interval type (see \autoref{thm:interval-funext}).

Regarding inductive types such as products, $\Sigma$-types, coproducts, natural numbers, and so on (see \autoref{cha:induction}), there are additional choices regarding how to precisely formulate induction and recursion.
\index{pattern matching}%
Formally, one may describe type theory by taking either \emph{pattern matching} or \emph{induction principles} as basic and deriving the other; see \autoref{cha:rules}.
However, pattern matching in general is not yet well understood from the homotopical perspective (in particular, ``nested'' or ``deep'' pattern-matching is difficult to make general sense of for higher inductive types).
Moreover, it can be dangerous unless sufficient care is taken: for instance, the form of pattern-matching implemented by default in \Agda
\index{proof assistant!Agda@\textsc{Agda}}%
allows proving Axiom K.
\index{axiom!Streicher's Axiom K}%
For these reasons, we have chosen to regard the induction principle as the basic property of an inductive definition, with pattern matching justified in terms of induction.

\index{rule}%
In general, a kind of type in type theory (such as function types, product types, identity types, etc.)\ is associated with four kinds of rules: \emph{formation rules},
\index{rule!formation}%
\index{formation rule}%
which specify how to form types of this sort; \emph{introduction rules}
\index{rule!introduction}%
\index{introduction rule}%
or \emph{constructors},
\index{constructor}%
which specify how to construct elements of such types; \emph{elimination rules}
\index{rule!elimination}%
\index{elimination rule}%
or \emph{eliminators},
\index{eliminator}%
which specify how to use elements of such types to construct elements of other types; and \emph{computation rules},
\index{rule!computation}%
\index{computation rule}%
which specify what happens when the introduction and elimination rules are combined. The computation rules are also called \emph{reduction rules} or \emph{conversion rules}.
\indexsee{reduction!rule}{computation rule}%
\indexsee{conversion!rule}{computation rule}%
For inductive types such as products, $\Sigma$s, coproducts, and so on, the introduction rules are the constructors (like pairing or injections), and the elimination rule is the induction principle, for which we have already mentioned the alternative terminology ``eliminator''.
For function types and $\Pi$-types, the introduction rule is $\lambda$-abstraction\index{lambda abstraction@$\lambda$-abstraction}, while the elimination rule is application of a function to an argument.

\index{proof assistant!Coq@\textsc{Coq}}%
Unlike the type theory of \Coq, we do not include a primitive type of propositions.  Instead, as discussed in \autoref{sec:pat}, we embrace 
the \emph{propositions-as-types (PAT)} principle, identifying propositions with types.
This was suggested originally by de Bruijn~\cite{deBruijn-1973}, Howard~\cite{howard:pat}, Tait~\cite{Tait-1968}, and Martin-L\"{o}f~\cite{Martin-Lof-1972}.
(Our decision is explained more fully in \autoref{subsec:pat?,subsec:hprops}.)

We do, however, include a full cumulative hierarchy of universes, so that the type formation and equality judgments become instances of the membership and equality judgments for a universe.
As a convenience, we regard objects of a universe as types, rather than as codes for types; in the terminology of \cite{martin-lof:bibliopolis}, this means we use ``Russell-style universes'' rather than ``Tarski-style universes''.
An alternative would be to use Tarski-style universes, with an explicit coercion function required to make an element $A:\UU$ of a universe into a type $\mathsf{El}(A)$, and just say that the coercion is omitted when working informally.

We also treat the universe hierarchy as cumulative, in that every type in $\UU_i$ is also in $\UU_j$ for each $j\geq i$.
There are different ways to implement cumulativity formally: the simplest is just to include a rule that if $A:\UU_i$ then $A:\UU_j$.
However, this has the annoying consequence that for a type family $B:A\to \UU_i$ we cannot conclude $B:A\to\UU_j$, although we can conclude $\lam{a} B(a) : A\to\UU_j$.
A more sophisticated approach that solves this problem is to introduce a judgmental subtyping relation $<:$ generated by $\UU_i<:\UU_j$, but this makes the type theory more complicated to study.
Another alternative would be to include an explicit coercion function $\uparrow : \UU_i \to \UU_j$, which could be omitted when working informally.

The path induction principle for identity types was formulated by Martin-L\"{o}f~\cite{Martin-Lof-1972}.
The based path induction rule in the setting of Martin-L\"of type theory is due to Paulin-Mohring~\cite{Moh93}; it can be seen as an intensional generalization of the concept of ``pointwise functionality'' for hypothetical judgments from NuPRL~\cite[Section~8.1]{constable+86nuprl-book}
The fact that Martin-L\"of's rule implies Paulin-Mohring's was proved by Streicher using Axiom K (see~\autoref{sec:hedberg}), by Altenkirch and Goguen as in \autoref{sec:identity-types}, and finally by Hofmann without universes (as in \autoref{ex:pm-to-ml}); see~\cite[\S1.3 and Addendum]{Streicher93}.

\sectionExercises

\begin{ex}\label{ex:composition}
  Given functions $f:A\to B$ and $g:B\to C$, define their \define{composite}
  \indexdef{composition!of functions}%
  \indexdef{function!composition}%
 $g\circ f:A\to C$.
  \index{associativity!of function composition}%
  Show that we have $h \circ (g\circ f) \jdeq (h\circ g)\circ f$.
\end{ex}

\begin{ex}
  Derive the recursion principle for products $\rec{A\times B} $ using only the projections, and verify that the definitional equalities are valid.
  Do the same for $\Sigma$-types.
\end{ex}

\begin{ex}
  Derive the induction principle for products $\ind{A\times B}$, using only the projections and the principle $\spr$ of surjective pairing.
  Verify that the definitional equalities are valid.
  Generalize $\spr$ to $\Sigma$-types, and do the same for $\Sigma$-types.
  \emph{(This requires concepts from \autoref{cha:basics}.)}
\end{ex}

\begin{ex}
\index{iterator!for natural numbers}
Assuming as given only the \emph{iterator} for natural numbers
\[\ite : \prd{C:\UU} C \to (C \to C) \to \nat \to C \]
with the defining equations
\begin{eqnarray*}
\ite(C,c_0,c_s,0)  & \defeq & c_0 \\
\ite(C,c_0,c_s,\suc(n)) & \defeq & c_s(\ite(C,C_0,c_s,n)  
\end{eqnarray*}
derive the recursor $\rec{\nat}$.
\end{ex}

\begin{ex}\label{ex:sum-via-bool}
\index{type!coproduct}%
Show that if we define $A + B \defeq \sm{x:\bool} \rec{\bool}(\UU,A,B,x)$, then we can give a definition of $\ind{A+B}$ for which the definitional equalities stated in \autoref{sec:coproduct-types} hold.
\end{ex}

\begin{ex}\label{ex:prod-via-bool}
\index{type!product}%
Show that if we define $A \times B \defeq \prd{x:\bool}\rec{\bool}(\UU,A,B,x)$, then we can give a definition of  $\ind{A\times B}$ for which the definitional equalities stated in \autoref{sec:finite-product-types} hold propositionally (i.e.\ using equality types).
\end{ex}

\begin{ex}\label{ex:pm-to-ml}
Give an alternative derivation of $\indidb{A}$ from $\indid{A}$ which avoids the use of universes.
  \emph{(This is easiest using concepts from later chapters.)}
\end{ex}

\begin{ex}
  \index{multiplication!of natural numbers}%
  Define multiplication and exponentiation using $\rec{\nat}$.
  Verify that $(\nat,+,0,\times,1)$ is a semiring using only $\ind{\nat}$.  
\end{ex}

\begin{ex}\label{ex:fin}
  Define the type family $\Fin : \nat \to \UU$ mentioned at the end of \autoref{sec:universes}, and the dependent function $\fmax : \prd{n:\nat} \Fin(n+1)$ mentioned in \autoref{sec:pi-types}.
\end{ex}

\begin{ex}\label{ex:ackermann}
  \indexdef{function!Ackermann}%
  \indexdef{Ackermann function}%
  Show that the Ackermann function $\ack : \nat \to \nat \to \nat$ is definable using only $\rec{\nat}$ satisfying the following equations:
  \begin{eqnarray*}
    \ack(0,n) & \jdeq & \suc(n) \\
    \ack(\suc(m),0) & \jdeq & 1 \\
    \ack(\suc(m),\suc(n)) & \jdeq & \ack(m,\ack(\suc(m),n))
  \end{eqnarray*}
\end{ex}

\begin{ex}\label{ex:neg-ldn}
  Show that for any type $A$, we have $\neg\neg\neg A \to \neg A$.
\end{ex}

\begin{ex}\label{ex:tautologies}
  Using the propositions as types interpretation, derive the following tautologies.
  \begin{enumerate}
  \item If $A$, then (if $B$ then $A$).
  \item If $A$, then not (not $A$).
  \item If (not $A$ or not $B$), then not ($A$ and $B$).
  \end{enumerate}
\end{ex}

\begin{ex}\label{ex:not-not-lem}
  Using propositions-as-types, derive the double negation of the principle of excluded middle, i.e.\ prove \emph{not (not ($P$ or not $P$))}.
\end{ex}

\begin{ex}\label{ex:without-K}
  Why do the induction principles for identity types not allow us to construct a function $f: \prd{x:A}{p:\id{x}{x}} (\id{p}{\refl{x}})$ with the defining equation
  \[ f(x,\refl{x}) \defeq \refl{\refl{x}} \quad ?\]
\end{ex}

\begin{ex}\label{ex:subtFromPathInd}
  Show that indiscernability of identicals follows from path induction.  
\end{ex}


% Local Variables:
% TeX-master: "hott-online"
% End:
