\newcommand{\vset}{\mathsf{set}}  % point constructor for cummulative hierarchy V

\chapter{Set Theory}
\label{cha:set-math}

Our conception of sets as types with particularly simple homotopical character, cf.\
\autoref{sec:basics-sets}, is quite different from the sets of Zermelo-Frankel set theory, which form a
cumulative hierarchy. For many mathematical purposes the homotopy-theoretic sets are just as good as
the Zermelo-Fraenkel ones, but there are also important differences. We consider these in
the present chapter. 

This chapter has three parts. First, we show that the category $\set$ of all $0$-types and maps between them has (most of) the usual properties of the category of sets; i.e.\ technically speaking, it is a ``$\Pi\mathsf{W}$-pretopos".  We then  consider the possibility of constructing a cumulative hierarchy of sets, equipped with a binary membership relation. Algebraic set theory is a convenient framework for this construction. Finally, we show that the theory of ordinals and cardinals from classical set theory can be developed in a natural way by considering sets with isomorphisms as equalities. This equality is, of course, precisely the one we obtain from the univalence axiom.

%\section{\texorpdfstring{$\set$}{Set} is a \texorpdfstring{$\Pi$}{Π}W-pretopos}
\section{The category of sets}
\label{sec:piw-pretopos}

%We will show that $\set$ is a $\Pi$W-pretopos. 
We begin by proving that $\set$ is a \emph{regular} category, i.e.:
%
\begin{enumerate}
\item $\set$ is finitely complete.
\item The kernel pair $\pi_1,\pi_2: (\sm{x,y:A} f(x)= f(y)) \to A$ of any
      function $f : A \to B$ has a coequalizer.
\item Pullbacks of regular epimorphisms are again regular epimorphisms.
\end{enumerate}
%
Recall that an epi is called \emph{regular} if it is the coequalizer of \emph{some} pair of maps; so in (iii) the pullback of a coequalizer is required to again be a coequalizer, but not necessarily of the pulled-back pair. 

The obvious candidate for the coequalizer of the kernel pair of a function is the \emph{image} of the function, to be defined below.
%
\begin{equation*}
\begin{tikzpicture}
\matrix (m) [std] {\sm{x,y:A} f(x)= f(y) & A & \mathsf{im}(f) \\ & & B \\};
\draw[ar] ([yshift=.5ex]m-1-1.east) -- node[above] {$\pi_1$} ([yshift=.5ex]m-1-2.west);
\draw[ar] ([yshift=-.5ex]m-1-1.east) -- node[below] {$\pi_2$} ([yshift=-.5ex]m-1-2.west);
\draw[ar] (m-1-2) -- node[above] {$\tilde{f}$} (m-1-3);
\draw[ar] (m-1-2) -- node[auto,swap] {$f$} (m-2-3);
\draw[ar,densely dotted] (m-1-3) -- (m-2-3);
\end{tikzpicture}
\end{equation*}

We will show that the image of a function $f$ is indeed the coequalizer of its kernel pair, that the resulting map from the image into the codomain $B$ is a monic, and that the resulting (regular) epi-mono factorization is stable under pullback.

\subsection{The image of a function}
\label{sec:image}

The truncations introduced in~\autoref{subsec:reflective-subuniverses} allow us to define the stable factorization
system we seek.
%\footnote{\url{http://uf-ias-2012.wikispaces.com/file/view/images.pdf/401765624/images.pdf}}
 Specifically, we use the $(-1)$-truncation $\|X\|$ to define a stable orthogonal factorization system  on $\set$: every map factors as an ``surjection" followed by an ``injection", and this factorization is stable under pullback.
We first define these notions:

We can use the logical notation from \autoref{defn:logical-notation}; in particular, below we can also write $\fall{b:B} \exis{a:A} f(a)=b$ for $\prd{b:B} \|\hfiber{f}b\|$.

\begin{defn}
A function $f:A\to B$  between sets is \emph{surjective} if
\begin{align*}
\mathsf{surj}(f) &\defeq\prd{b:B} \|\hfiber{f}b\| \\
&= \fall{b:B} \exis{a:A}f(a)=b
\end{align*}

%In general, the monomorphisms of types are functions with contractible homotopy fibers over images of points in $A$; thus we define
%\note{Move to \autoref{sec:mono-epi}?}

A function $f:A\to B$  between sets is \emph{injective} if
\begin{align*}
\mathsf{inj}(f) &\defeq\prd{a:A} \mathsf{isContr}(\hfiber{f}{f(a)})\\
	&= \fall{a,a':A} f(a) = f(a') \Rightarrow a=a'
\end{align*}
\end{defn}

\noindent The injective functions can be seen to be the monomorphisms of $\set$, which can also be characterized as those $f:A\to B$ such that the function
\begin{equation*}
  \lam{a} \pairr{a,a,\refl{f(a)}}:A\to A\times_B A
\end{equation*}
is an equivalence.

\begin{defn}
Let $f:A\to B$ be a function between sets. Define the \emph{image} of $f$ by: 
\begin{align*}
\im(f) &\defeq\sm{b:B} \|\hfiber{f}b\|\\
	&= \sm{b:B} \|\sm{a:A}f(a)=b\|
\end{align*}
Moreover, define the functions $\tilde{f}:A\to\im(f)$ and $i_f:\im(f)\to B$ by
\begin{align*}
  \tilde{f} & \defeq \lam{a} \pairr{f(a),|\pairr{a,\refl{f(a)}}|}\\
i_f & \defeq \pi_1.
\end{align*}
\end{defn}

\begin{lem}
As just defined, $\tilde{f}$ is surjective and $i_f$ is injective.
\end{lem}

Next, we use the insights from~\autoref{subsec:reflective-subuniverses} and consider coequalizers
in $\set$. The coequalizer of a pair of maps in $\set$ is defined as the 0-truncation of a general coequalizer.  Given a pair of maps $f,g:A\to B$, the general, non-truncated, pushout does not have the quotient-like properties expected in the category $\set$.  For instance, the untruncated coequalizer of $\unit\rightrightarrows\unit$ is the circle. Nevertheless, the $0$-truncation of the circle is again $\unit$, which is the expected coequalizer in $\set$. Thus we set

\begin{defn}
Let $A$ and $B$ be sets and let $f,g:A\to B$.
We define the \emph{set-coequalizer} of $f$ and $g$ to be the $0$-truncation of their general coequalizer, as defined in \autoref{sec:colimits},
\[
c_{f,g}:B\to B\sslash(f=g) \defeq \|B/(f=g)\|_0.
\]
\end{defn}

Recall from \autoref{defn:homotopy} the notation $f \sim g$ for homotopy between functions.

\begin{lem}
Let $f,g:A\to B$ be functions between sets $A$ and $B$. The 
{set-co}equalizer $c_{f,g}:B\to B\sslash(f=g)$ has the property that, for any set $C$, any $h:B\to C$ with $h\circ f\sim h\circ g$, the type\begin{equation*}
\sm{k:B\sslash(f=g)\to C} k\circ c_{f,g}\sim h
\end{equation*}
is contractible.
\end{lem}
%
\noindent As explained in~\autoref{subsec:reflective-subuniverses}, this is a general phenomenon, the 0-truncated colimits behave as expected.

We next connect regular epis and surjective functions; for this, we use the principle of unique choice; see~\autoref{cor:UC}. 

\begin{defn}
Let $f:A\to B$ be a function between sets. Define
\begin{align*}
\mathsf{epi}(f) & \defeq \prd{X:\set}{g,h:B\to X}
(g\circ f\sim h\circ f)\to (g\sim h)\\
\mathsf{epi}'(f) & \defeq\prd{X:\set}{g:B\to X}
\mathsf{isContr}\big(\sm{h:B\to X} g\circ f\sim h\circ f\big).
\end{align*}
\end{defn}

Note that $\mathsf{epi}(f)$ is the usual notion of epimorphism while the
notion $\mathsf{epi}'(f)$ seems to be stronger. The latter seems to be the right notion for general types.

\begin{lem}\label{epis-surj}
For any function $f:A\to B$ between sets, the following are equivalent:
\begin{enumerate}
\item $f$ is an epimorphism, i.e.\ we have $\mathsf{epi}(f)$.
\item We have $\mathsf{epi}'(f)$.
\item Consider the pushout diagram
\begin{equation*}
\begin{tikzpicture}
\matrix (m) [std] {A & B \\ \mathsf{unit} & C_f \\};
\draw[ar] (m-1-1) -- node[above] {$f$} (m-1-2);
\draw[ar] (m-1-2) -- node[right] {$\iota$} (m-2-2);
\draw[ar] (m-1-1) -- (m-2-1);
\draw[ar] (m-2-1) -- node[below] {$t$} (m-2-2);
\end{tikzpicture}
\end{equation*}
in $\set$ defining the mapping cone. The type $C_f$ is contractible.
\item $f$ is surjective.
\end{enumerate}
\end{lem}

\begin{proof}
Let $f:A\to B$ be a function between sets.
To show that $\mathsf{epi}(f)\to\mathsf{epi}'(f)$, assume $f$ is an epimorphism,
let $X$ be a set and let $g:B\to X$ be a function. Then we find
\begin{equation*}
\pairr{g,\refl{g\circ f}}:\sm{h:B\to X} g\circ f\sim h\circ f.
\end{equation*}
Thus, to show that $\sm{h:B\to X} g\circ f\sim h\circ f$ 
is contractible we need to show that
\begin{equation*}
\prd{h:B\to X}{K:g\circ f\sim h\circ f}\sm{L:g\sim h} K\sim L\circ f.
\end{equation*}
Note that $K\sim L\circ f$ is contractible since $X$ is assumed to be a set,
hence it suffices to show that
\begin{equation*}
\prd{h:B\to X} (g\circ f\sim h\circ f)\to (g\sim h).
\end{equation*}
This follows at once from the assumption that $f$ is an epimorphism.

To show that $\mathsf{epi}'(f)\to\mathsf{isContr}(C_f)$, suppose that
$\mathsf{epi}'(f)$. The basic constructor of $C_f$ corresponding to
$\mathsf{unit}\to C_f$ gives us an element $t:C_f$. We have to show that
\begin{equation*}
\prd{x:C_f} x= t.
\end{equation*}
Note that $x= t$ is a proposition, hence we can use induction on
$C_f$: it suffices to find
\begin{align*}
I_0 & : \prd{b:B} \iota(b)= t\\
I_1 & : \prd{a:A} \opp{\alpha_1(a)} \ct I_0(f(a))=\refl{t}.
\end{align*}
where $\alpha_1:\prd{a:A} \iota(f(a))= t$ is a basic constructor
of $C_f$. Note that $\alpha_1$ is a homotopy from $\iota\circ f$ to
$\mathsf{const}_t\circ f$, so we find the terms
\begin{equation*}
\pairr{\iota,\refl{\iota\circ f}},\pairr{\mathsf{const}_t,\alpha_1}:
\sm{h:B\to C_f} \iota\circ f\sim h\circ f.
\end{equation*}
By the assumption of $\mathsf{epi}'(f)$, it follows that there is a path
\begin{equation*}
\gamma:\pairr{\iota,\refl{\iota\circ f}}=\pairr{\mathsf{const}_t,\alpha_1}
\end{equation*}
Hence, we may define $I_0(b)\defeq \happly(\projpath1(\gamma),b):\iota(b)=t$.
We also have
\[\projpath2(\gamma) : \trans{\projpath1(\gamma)}{\refl{\iota\circ f}} = \alpha_1. \]
This transport involves precomposition with $f$, which commutes with $\happly$.
Thus, from transport in path types we obtain $I_0(f(a)) = \alpha_1(a)$ for any $a:A$, which gives us $I_1$.

To show that $\mathsf{isContr}(C_f)\to\mathsf{surj}(f)$, assume $\mathsf{isContr}(C_f)$.
We first construct a dependent type $P:C_f\to\prop$ by recursion on $C_f$, which is valid since \prop is a set.
On the point constructors, we define
\begin{align*}
P(t) & \defeq \mathsf{unit}\\
P(\iota(b)) & \defeq \brck{\hfiber{f}b}.
\end{align*}
To complete the construction of $P$, it remains to give a path $\brck{\hfiber{f}{f(a)}} =_\prop \unit$ for all $a:A$.
However, $\brck{\hfiber{f}{f(a)}}$ is inhabited by $(f(a),\refl{f(a)})$.
Since it is a mere proposition, this means it is contractible --- and thus equivalent, hence equal, to \unit.
This completes the definition of $P$.
Now, since $C_f$ is assumed to be contractible, it follows that $P(x)$ is equivalent to $P(t)$ for any $x:C_f$.
In particular, $P(\iota(b))\jdeq \brck{\hfiber{f}b}$ is equivalent to $P(t)\jdeq \unit$ for each $b:B$, and hence contractible.
Thus, $f$ is surjective.

To show that $\mathsf{surj}(f)\to\mathsf{epi}(f)$,
let $f:A\to B$ be a surjective function and consider a set $C$ and two functions
$g,h:B\to C$ with the property that $g\circ f\sim h\circ f$. Since $f$ 
is assumed to be surjective,
we have an equivalence $B\simeq\mathsf{im}(f)$. We have the following equivalences
\begin{align*}
\prd{b:B} g(b)= h(b) 
& \simeq \prd{w:\mathsf{im}(f)} g(\pi_1 w)= h(\pi_1(w))\\
& \simeq \prd{b:B}{a:A}{p:f(a)= b} g(b)= h(b)\\
& \simeq \prd{a:A} g(f(a))= h(f(a)).
\end{align*}
By assumption, there is an element of the latter type.
\end{proof}
The proof that epis are surjective in~\cite{Mines/R/R:1988} uses the power set operation. 
This proof can be made predicative by using a large power set and typical ambiguity.
A predicative proof for setoids was given by Wilander~\cite{Wilander2010}. 
The proof above is similar, but avoids setoids by using the pushout and the
univalence axiom. 
% \begin{rem}
% The above theorem is not true when we replace $\set$ by $\type$
% (replacing it also in the definition of $\mathsf{epi}$ and $\mathsf{epi}'$). 
% However, we do
% get the implications $\textit{ii.}\Rightarrow\textit{iii.}\Rightarrow
% \textit{iv.}$
% \end{rem}

\begin{lem}\label{lem:images_are_coequalizers}
Surjective functions between sets are regular epimorphisms.
\end{lem}

\begin{proof}
Note that it suffices to show that for any function $f:A\to B$, the diagram
\begin{equation*}
\begin{tikzpicture}
\matrix (m) [std] {\sm{x,y:A} f(x)= f(y) & A & \mathsf{im}(f) \\};
\draw[ar] ([yshift=.5ex]m-1-1.east) -- node[above] {$\pi_1$} ([yshift=.5ex]m-1-2.west);
\draw[ar] ([yshift=-.5ex]m-1-1.east) -- node[below] {$\pi_2$} ([yshift=-.5ex]m-1-2.west);
\draw[ar] (m-1-2) -- node[above] {$\tilde{f}$} (m-1-3);
\end{tikzpicture}
\end{equation*}
is a coequalizer diagram.

The first thing we have to verify is that there is a homotopy 
$H:\tilde{f}\circ\pi_1\sim\tilde{f}\circ\pi_2$. Let $\pairr{x,y,p}$ be an element
of $\sm{x,y:A} f(x)= f(y)$. 
Then we have $\tilde{f}(\pi_1(\pairr{x,y,p}))= \pairr{f(x),u}$, 
where $u$ belongs to the contractible type $\|\hfiber{f}{f(x)}\|$. 
Similarly, we have a path 
$\tilde{f}(\pi_2(\pairr{x,y,p}))= \pairr{f(y),v}$,
where $v$ belongs to the contractible type $\|\hfiber{f}{f(y)}\|$.
Since we have $p:f(x)= f(y)$ and since
$\|\hfiber{f}{f(y)}\|$ is contractible, 
it follows that we get a path from $\tilde{f}(\pi_1(\pairr{x,y,p}))$ to
$\tilde{f}(\pi_2(\pairr{x,y,p}))$, which gives us our homotopy.

Now suppose that $g:A\to X$ is a function for which there is a homotopy 
$K:g\circ\pi_1\sim g\circ\pi_2$. We have to show that the space
\begin{equation*}
\sm{h:\mathsf{im}(f)\to X}{L:h\circ\tilde{f}\sim g} h\circ H\sim K
\end{equation*}
is contractible. We will apply unique choice to define a 
function from $\mathsf{im}(f)$ to $X$. Let $R',R:\mathsf{im}(f)\to
X\to\type$ be the relations defined by 
\begin{align*}
R'(w,x) & \defeq\prd{a:A} (\tilde{f}(a)= w)\to (g(a)= x)\\
R(w,x) & \defeq \|R'(w,x)\|.
\end{align*}
Then $\mathsf{atMostOne}(R(w))$ is inhabited for every $w:\mathsf{im}(f)$. 
To see this, note that the type $\mathsf{atMostOne}(R(w))$ is a
proposition. Therefore, there is an equivalence
\begin{equation*}
\big(\prd{w:\mathsf{im}(f)} \mathsf{atMostOne}(R(w))\big)
\simeq\prd{a:A} \mathsf{atMostOne}(R(\tilde{f}(a))).
\end{equation*}
We can simplify this even further to
\begin{equation*}
\prd{a:A} \mathsf{atMostOne}(R^\prime(\tilde{f}(a))).
\end{equation*}
This follows from the assumption that $g(a)= g(a^\prime)$ 
whenever $f(a)= f(a^\prime)$. Let $a:A$, $x,x^\prime:X$,
$u:R^\prime(\tilde{f}(a),x)$ and $u^\prime:R^\prime(\tilde{f}(a),x^\prime)$. 
Then there are the paths $u(a,\refl{\tilde{f}(a)}):g(a)=
x$ and $u^\prime(a,\refl{\tilde{f}(a)}):g(a)= x^\prime$, 
showing that $x= x^\prime$. 

Also, $\|\sm{x:X} R(w,x)\|$ is inhabited for every $w:\mathsf{im}(f)$. Indeed, the space
\begin{equation*}
\prd{w:\mathsf{im}(f)} \big\| \sm{x:X} R(w,x)\big\|
\end{equation*}
is equivalent to the space
\begin{equation*}
\prd{a:A} \big\| \sm{x:X} R(\tilde{f}(a),x)\big\|.
\end{equation*}
The space $\prd{a:A}\sm{x:X} R^\prime(\tilde{f}(a),x)$ is inhabited by
\begin{equation*}
  \lam{a} \pairr{g(a),\lam{a^\prime}{p} K(\pairr{a^\prime,a,p^{-1}})}
\end{equation*}
This shows that the hypotheses of the principle of unique choice are satisfied, so we get an element of
\begin{equation*}
\sm{h:\mathsf{im}(f)\to X}\prd{w:\mathsf{im}(f)} R(w,h(w)).
\end{equation*}
An immediate consequence of the way we constructed our function $h:\mathsf{im}(f)\to X$ is that $h\circ\tilde{f}\sim g$. The result follows
now from the observation that the space
\begin{equation*}
\sm{h^\prime:\mathsf{im}(f)\to X} h^\prime\circ\tilde{f}\sim h\circ\tilde{f}
\end{equation*}
is contractible because $\tilde{f}$ is an epimorphism. 
\end{proof}

\begin{lem}\label{lem:pb_of_coeq_is_coeq}
Pullbacks of surjective functions are surjective. Consequently,
pullbacks of regular epis are regular epis.
\end{lem}

\begin{proof}
Consider a pullback diagram
\begin{equation*}
\begin{tikzpicture}
\matrix (m) [std] {A & B \\ C & D \\};
\draw[ar] (m-1-1) -- (m-1-2);
\draw[ar] (m-1-2) -- node[right] {$g$} (m-2-2);
\draw[ar] (m-1-1) -- node[left]  {$f$} (m-2-1);
\draw[ar] (m-2-1) -- node[below] {$h$} (m-2-2);
\end{tikzpicture}
\end{equation*}
and assume that $g$ is surjective. Applying the pasting lemma of pullbacks
with the morphism $c:\mathsf{unit}\to C$, we obtain an
equivalence $\hfiber{f}c\simeq\hfiber{g}{h(c)}$ for any
$c:C$. This equivalence gives that $f$ is surjective.
\end{proof}

\begin{thm}\label{thm:set_regular}
The category $\set$ is regular.
\end{thm}

\begin{proof}
$\set$ has all limits, so it is finitely complete. 
\autoref{lem:images_are_coequalizers} gives
that the kernel pair of each function has a coequalizer.
\autoref{lem:pb_of_coeq_is_coeq} shows that
regular epis are stable under pullback.
\end{proof}


\subsection{Quotients}\label{subsec:quotients}
Now that we know that $\set$ is regular, to show that $\set$ is exact, we need to show that every
equivalence relation is effective. In other words, given an equivalence
relation $R:A\to A\to\prop$, there is a coequalizer $c_R$ of the pair
$\pi_1,\pi_2:\sm{x,y:A} R(x,y)\to A$ and, moreover, the $\pi_1$ and $\pi_2$
for the kernel pair of $c_R$.

We have already seen, in \autoref{sec:set-quotients}, two general ways to construct the quotient of a set by an equivalence relation $R:A\to A\to\prop$.
The first can be described as the set-coequalizer of the two projections
\[\pi_1,\pi_2:\sm{x,y:A} R(x,y)\to A.\]
The important property of such a quotient is the following.

\begin{defn}
A relation $R:A\to A\to\prop$ is said to be \emph{effective} if the square
\begin{equation*}
\begin{tikzpicture}
\matrix (m) [std] {\sm{x,y:A} R (x,y) & A \\ A & A/R \\};
\draw[ar] (m-1-1) -- node[above] {$\pi_1$} (m-1-2);
\draw[ar] (m-1-2) -- node[right] {$c_{R }$} (m-2-2);
\draw[ar] (m-1-1) -- node[left]  {$\pi_2$} (m-2-1);
\draw[ar] (m-2-1) -- node[below] {$c_R $} (m-2-2);
\end{tikzpicture}
\end{equation*}
is a pullback. 
\end{defn}

\begin{lem}\label{lem:sets_exact}
Suppose $\pairr{A,R}$ is an equivalence relation. Then there is an equivalence
\begin{equation*}
(c_R(x)= c_R(y))\simeq R(x,y)
\end{equation*}
for any $x,y:A$. In other words, equivalence relations are effective.
\end{lem}

\begin{proof}
We begin by extending $R$ to a relation $\tilde{R}:A/R\to A/R\to\prop$. After
the construction of $\tilde{R}$ we will show that $\tilde{R}$ is equivalent
to the identity type on $A/R$. We define $\tilde{R}$ by double induction on
$A/R$ (note that $\prop$ is a set by univalence for propositions). We
define $\tilde{R}(c_R(x),c_R(y)) \defeq R(x,y)$. For $r:R(x,x')$ and $s:R(y,y')$,
the transitivity and symmetry 
of $R$ gives an equivalence from $R(x,y)$ to $R(x',y')$. This completes the
definition of $\tilde{R}$. To finish the proof of the proposition, we need
to show that $\tilde{R}(w,w')\simeq w= w'$ for every $w,w':A/R$. We can
do this by showing that the type $\sm{w':A/R} \tilde{R}(w,w')$ is contractible for
each $w:A/R$. We do this by induction. Let $x:A$. We have the element
$\pair{c_R(x),\rho(x)}:\sm{w':A/R} \tilde{R}(c_R(x),w')$, where $\rho$ is
the reflexivity proof of $R$, hence we only
have to show that
\begin{equation*}
\prd{w':A/R}{r:\tilde{R}(c_R(x),w')} \pairr{w',r}=\pairr{c_R(x),\rho(x)},
\end{equation*}
which we do by induction on $w'$. Let $y:A$ and let $r:R(x,y)$. Then we have
the path $p_R(r)^{-1}:c_R(y)= c_R(x)$. We automatically get a path from
$p_R(r)^{-1}\cdot r=\rho(x)$, finishing the proof.
\end{proof}

The second construction of quotients, which uses impredicativity of mere propositions, was as the set of equivalence classes of $R$ (a subset of its power set):
\[ A\sslash R \defeq \setof{ P:A\to\prop | P \text{ is an equivalence class of } R} \]
Note that if we regard $R$ as a function from $A$ to $A\to \prop$, then $A\sslash R$ is equivalent to $\im(R)$, as constructed in \autoref{sec:image}.
Now in \ref{lem:images_are_coequalizers} we have shown that images are
coequalizers. In particular, we immediately get the coequalizer diagram
\begin{equation*}
\begin{tikzpicture}
\matrix (m) [std] {\sm{x,y:A} R (x)= R (y) & A & A\sslash  R.  \\};
\draw[ar] ([yshift=.5ex]m-1-1.east) -- node[above] {$\pi_1$} ([yshift=.5ex]m-1-2.west);
\draw[ar] ([yshift=-.5ex]m-1-1.east) -- node[below] {$\pi_2$} ([yshift=-.5ex]m-1-2.west);
\draw[ar] (m-1-2) -- (m-1-3);
\end{tikzpicture}
\end{equation*}
We can use this to give an alternative proof that any equivalence relation is effective and that the two definitions of quotients agree.

\begin{thm}\label{prop:kernels_are_effective}
For any function $f:A\to B$ between any two sets, 
the relation $\ker(f):A\to A\to\type$ given by 
$\ker(f,x,y)\defeq f(x)= f(y)$ is effective. 
\end{thm}

\begin{proof}
We will use that $\im(f)$ is the coequalizer of $\pi_1,\pi_2:
\sm{x,y:A} f(x)= f(y)\to A$; 
%we get this equivalence from proposition~\ref{prop:images_are_coequalizers}
. Note that the canonical kernel pair of the function 
$c_f\defeq\lam{a} \pairr{f(a),\tau_1(\pairr{a,\refl{f(a)}})}$ consists
of the two projections
\begin{equation*}
\pi_1,\pi_2:\big(\sm{x,y:A} c_f(x)= c_f(y)\big)\to A.
\end{equation*}
For any $x,y:A$, we have equivalences
\begin{align*}
c_f(x)= c_f(y) & \simeq \sm{p:f(x)= f(y)} p\cdot\tau_1(\pairr{x,\refl{f(x)}})
=\tau_1(\pairr{y,\refl{f(x)}})\\ & \simeq f(x)= f(y),
\end{align*}
where the last equivalence holds because 
$\|\hfiber{f}b\|$ is a proposition for any $b:B$. 
Therefore, we get that
\begin{equation*}
\sm{x,y:A} c_f(x)= c_f(y)\simeq\sm{x,y:A} f(x)= f(y)
\end{equation*}
and hence we may conclude that $\ker f$ is an effective relation 
for any function $f$.
\end{proof}

\begin{thm}
Equivalence relations are effective and there is an equivalence $A/R \simeq A\sslash  R $. 
\end{thm}

\begin{proof}
We need to analyze the coequalizer diagram
\begin{equation*}
\begin{tikzpicture}
\matrix (m) [std] {\sm{x,y:A} R (x)= R (y) & A & A\sslash  R. \\};
\draw[ar] ([yshift=.5ex]m-1-1.east) -- node[above] {$\pi_1$} ([yshift=.5ex]m-1-2.west);
\draw[ar] ([yshift=-.5ex]m-1-1.east) -- node[below] {$\pi_2$} ([yshift=-.5ex]m-1-2.west);
\draw[ar] (m-1-2) -- (m-1-3);
\end{tikzpicture}
\end{equation*}
By the univalence axiom, the space $R (x)= R (y)$ is equivalent to the space of homotopies from $R (x)$ to $R (y)$, which is
equivalent to $\prd{z:A} R (x,z)\simeq R (y,z)$. Since $R $ is an equivalence relation, the latter space is equivalent to $R (x,y)$. To
summarize, we get that $(R (x)= R (y))\simeq R (x,y)$, so $R $ is effective since it is equivalent to an effective relation. Also,
the diagram
\begin{equation*}
\begin{tikzpicture}
\matrix (m) [std] {\sm{x,y:A} R (x,y) & A & A\sslash  R. \\};
\draw[ar] ([yshift=.5ex]m-1-1.east) -- node[above] {$\pi_1$} ([yshift=.5ex]m-1-2.west);
\draw[ar] ([yshift=-.5ex]m-1-1.east) -- node[below] {$\pi_2$} ([yshift=-.5ex]m-1-2.west);
\draw[ar] (m-1-2) -- (m-1-3);
\end{tikzpicture}
\end{equation*}
is a coequalizer diagram. Since coequalizers are unique up to equivalence, it follows that $A/R \simeq A\sslash  R $.
\end{proof}

%\subsection{Predicativity}\label{sec:resizing}
%One may wonder about the predicative interpretation of the quotient constructions above.
%One could argue that the HIT construction is predicative by considering the interpretation of this
%quotient in the setoid model~\cite{Altenkirch1999,coquand2012constructive}. 
%In this model the quotient does not raise the universe level. 
%A similar observation holds for constructions that can be carried out in the groupoid 
%model~\cite{hofmann1998groupoid}. These observations should suffice for the set-level higher inductive 
%types we use in the present paper.
%
%Voevodsky~\cite{Universe-poly} has proposed to allow some impredicativity by adding the following resizing rule:
%\[\text{if } a:\UU_n \text{ and } p:\isprop(a), \text{ then } \mathsf{rr}(a,p): \UU_0,\]
%where $\UU_n$ is any universe and $\UU_0$ is the lowest universe.
%Moreover, when lifted to $\UU_n$, $\mathsf{rr}(a,p)= a$.
%Hence, $\mathsf{rr}$ is an equivalence between $\prop_n$ and $\prop_0$. 
%Here we assume that $\UU_0$ is a subuniverse of $\UU_n$. 
%Another axiom resizes each $\prop_n$ to the lowest universe.
%
%The following replacement axiom is derivable from the resizing axiom; see~\cite{Universe-poly}.
%\note{This doesn't require resizing, only quotients.  HITs work as well.}
%\begin{lem}\label{lem:replacement}
%  Let $\UU$ be a universe and $X:\UU$.
%  If $f:X\twoheadrightarrow Y$ is a surjection and $Y$ is a set, then there exists a $Z:\UU$ which is equivalent to $Y$.
%\end{lem}
%\begin{proof}
%Define $Z\defeq X\sslash \ker f$ using the type of equivalence classes; see~\ref{def:VVquotient}. 
%Then $Z:\UU$, since $\prop_\UU$ is in the lowest universe. Finally, we observe that $Z\defeq X/\ker f\simeq\im f\simeq Y$.
%\end{proof}
%
%%
% I have commented this section out, because it doesn't say anything -- SA
%%

%%
% I have commented the following section out because it doesn't belong in a chapter on Sets -- and doesn't fit anywhere else.
%%
%%%%%%%%%%%%%%%%%%%%%%%%%%%%%%%%%%%%%%%%%%%%%%%%%%%%%%%
%\subsection{The object classifier}\label{subsec:object-classification}
%In topos theory we have a small subobject classifier classifying the monomorphisms.
%In HoTT without resizing rules, $\prop$ is large. It is a reflective subuniverse see \autoref{subsec:reflective-subuniverses}.
%In higher topos theory one considers not only subobject classifiers, but also \emph{object classifiers} 
%which classify more general classes of maps.
%In this section we will establish the existence of an internal analogue of such object classifiers, 
%they will always sit a higher universe. Moreover, we will find an $n$-object classifier for
%every $n:\mathbb{N}_{\geq-2}$, where the $n$-object classifier will classify the
%functions with $n$-truncated homotopy fibers. 
%
%In addition to the size issue of the object classifers, we will see that the
%$n$-object classifier will generally not be of (homotopy) level $n$, but of level
%$n+1$. This observation should be regarded in contrast to the theory of 
%predicative toposes, where a universal small
%map is required to exist. Such a universal small map is suggested to be a map
%between sets, 
%but it seems that within homotopy type theory we cannot expect
%such a map to exist. The main reason is that the universal small map of sets
%will in general be a map of groupoids; a universal small map of groupoids will
%in general be a map of $2$-groupoids, etc.
%
%\begin{thm}\label{thm:nobject_classifier_appetizer}
%For any type $B$ there is an equivalence
%\begin{equation*}
%\chi:\big(\sm{A:\type} A\to B\big)\simeq B\to\type.
%\end{equation*}
%Likewise, there is an equivalence
%\begin{equation*}
%\chi_n:\big(\sm{A:\type}{f:A\to B} \mathsf{isTrunc}(n,f)\big)\simeq B\to \typele{n}
%\end{equation*}
%for every $n:\mathbb{N}_{\geq-2}$. Here $\mathsf{isTrunc}(n,f)$ denotes that all fibers of $f$ are $n$-truncated
%\end{thm}
%
%\begin{proof}
%We begin by constructing the first equivalence, i.e.\ we have to construct functions
%\begin{align*}
%\chi & : \big(\sm{A:\type} A\to B\big)\to B\to\type\\
%\psi & : (B\to\type)\to\big(\sm{A:\type} A\to B\big).
%\end{align*}
%The function $\chi$ is defined by $\chi(f,b)\defeq\hfiber{f}b$. The
%function $\psi$ is defined by $\psi(P)\defeq\pairr{(\sm{b:B} P(b)),\pi_1}$. Now
%we have to verify that $\chi\circ\psi\sim\idfunc{}$ and that $\psi\circ\chi
%\sim\idfunc{}$:
%\begin{enumerate}
%\item Let $P$ be a dependent type over $B$. It is a basic fact that 
%$\hfiber{\pi_1}{b}\simeq P(b)$ and therefore it follows immediately
%that $P\sim\chi(\psi(P))$.
%\item Let $f:A\to B$ be a function. We have to find a path
%\begin{equation*}
%\pairr{(\sm{b:B} \hfiber{f}b),\pi_1}=\pairr{A,f}
%\end{equation*}
%First note that we have the basic equivalence
%$e:\sm{b:B} \hfiber{f}b\simeq A$ with $e(b,a,p)\defeq a$ and $e^{-1}(a)
%\defeq(f(a),a,\refl{f(a)})$. It also follows that
%$e\cdot\pi_1=\pi_1\circ e^{-1}$. From this, we immediately read off
%that $(e\cdot\pi_1)(a)= f(a)$ for each $a:A$. This completes the proof
%of the first of the asserted equivalences.
%\end{enumerate}
%To find the second set of equivalences, note that if we restrict $\chi$ to
%functions with $n$-truncated homotopy fibers we get an $n$-truncated dependent
%type. Likewise, if we restrict $\psi$ to $n$-truncated dependent types we get
%a function with $n$-truncated homotopy fibers.
%\end{proof}
%
%\begin{defn}
%Define
%\begin{equation*}
%\pointed{\type}\defeq\sm{A:\type} A\qquad\text{and}\qquad \pointed{\typele{n}}\defeq
%\sm{A:\typele{n}} A.
%\end{equation*}
%Thus, $\pointed{\type}$ stands for the \emph{pointed types} and $\pointed{\typele{n}}$ stands for
%the pointed $n$-types.
%\end{defn}
%
%The following theorem states that we have an object classifier.
%\begin{thm}\label{thm:nobject_classifier}
%Let $f:A\to B$ be a function. Then the diagram
%\begin{equation*}
%\begin{tikzpicture}
%\matrix (m) [std] {A & \pointed{\type} \\ B & \type \\};
%\draw[ar] (m-1-1) -- node[above] {$\vartheta_f$} (m-1-2);
%\draw[ar] (m-1-2) -- node[right] {$\pi_1$} (m-2-2);
%\draw[ar] (m-1-1) -- node[left]  {$f$} (m-2-1);
%\draw[ar] (m-2-1) -- node[below] {$\chi_f$} (m-2-2);
%\end{tikzpicture}
%\end{equation*}
%is a pullback diagram. Here, the function $\vartheta_f$ is defined by
%\begin{equation*}
%  \lam{a} \pairr{\hfiber{f}{f(a)},\pairr{a,\refl{f(a)}}}.
%\end{equation*}
%A similar statement holds when we replace $\type$ by $\typele{n}$.
%\end{thm}
%\begin{proof}
%Note that we have the equivalences
%\begin{align*}
%A & \simeq \sm{b:B} \hfiber{f}b\\
% & \simeq \sm{b:B}{X:\type}{p:\hfiber{f}b= X} X\\
% & \simeq \sm{b:B}{X:\type}{x:X} \hfiber{f}b= X\\
% & \equiv B\times_{\type}\pointed{\type}.
%\end{align*}
%which gives us a composite equivalence $e:A\simeq B\times_\type\pointed{\type}$. 
%We may display the action of this composite equivalence step by step by
%\begin{align*}
%a & \mapsto \pairr{f(a),\pairr{a,\refl{f(a)}}}\\
% & \mapsto \pairr{f(a),\hfiber{f}{f(a)},\refl{\hfiber{f}{f(a)}},\pairr{a,\refl{f(a)}}}\\
% & \mapsto \pairr{f(a),\hfiber{f}{f(a)},\pairr{a,\refl{f(a)}},\refl{\hfiber{f}{f(a)}}}
%\end{align*}
%Therefore, we get homotopies $f\sim\pi_1\circ e$ and $\vartheta_f\sim \pi_2\circ e$. 
%\end{proof}
%
%\begin{lem}\label{lem:subobject}
%The type $\pointed{\prop}$ is contractible.
%\end{lem}
%\begin{proof}
%Suppose that $\pairr{P,u}$ is an element of $\sm{P:\prop} P$. Then we have $u:P$ and hence there is a term of type $\mathsf{isContr}(P)$. It
%follows that $P\simeq\unit$ and therefore we get from the univalence axiom that there is a path
%$\pairr{P,u}=\pairr{\unit,\mathsf{tt}}$.
%\end{proof}


\subsection{\texorpdfstring{$\set$}{Set} is a \texorpdfstring{$\Pi$}{Π}$\mathsf{W}$-pretopos}
\label{subsec:piw}

The notion of a \emph{$\Pi\mathsf{W}$-pretopos} -- that is, a locally cartesian closed category with finite coproducts, effective equivalence relations, and $\mathsf{W}$-types -- is intended as a ``predicative" notion of topos, i.e.\ a category of ``predicative sets", which can serve the purpose for constructive mathematics that the usual category of sets does for classical mathematics.  

Typically, in constructive type theory, one resorts to an external construction of ``setoids" -- an exact completion -- to obtain a category with such closure properties.  In particular, the well-behaved quotients are required for many constructions in mathematics that usually involve (non-constructive) power sets.  It is noteworthy that HoTT  provides these constructions \emph{internally} (via HITs), without requiring such external constructions.  This represents a powerful advantage of our approach, as we shall see in subsequent examples.

\begin{thm} The category $\set$ is a $\Pi\mathsf{W}$-pretopos.
\end{thm}
\begin{proof}
We have an initial object $0$ and finite, disjoint sums $A+B$.  These are stable under pullback, simply because pullback has a right adjoint.  Indeed, $\set$ is locally cartesian closed, since for any map $f:A\to B$ between sets, the ``fibrant replacement" $\sm{a:A}f(a)=b$ is equivalent to $A$ (over $B$), and we have $\prod$-types for the replacement.
We've just shown that $\set$ is regular (\autoref{thm:set_regular}) and that quotients are effective (\autoref{lem:sets_exact}). We thus have an LCC pretopos. Finally, since the $n$-types are closed under the formation of $\mathsf{W}$-types
(\autoref{sec:w-types}), we see that $\set$ is a $\Pi\mathsf{W}$-pretopos. 
\end{proof}

One naturally wonders what, if anything, prevents $\set$ from being a topos? In addition to the structure already mentioned, a topos has a subobject classifier, i.e.\ a pointed object classifying (equivalence classes of) monomorphisms.  (In fact, in the presence of a subobject classifier, things become somewhat simpler: one merely needs cartesian closure in order to get the colimits).  In HoTT,  univalence implies that the type $\prop$ does classify monos, but in general it is as large as the universe $\UU$ over which it is formed by the definition $\prop \defeq \sm{X:U}\isprop(X)$. It is a ``set" in the sense of being a $0$-type, but it is not ``small" in the sense of being an object of $\UU$.

Now consider the following ``resizing rule", with respect to a sequence $\UU_0, \UU_1, \ldots$ of universes:
\[\text{if } a:\UU_n \text{ and\ } \isprop(a), \text{ then } \mathsf{r}_{n}(a): \UU_0\]
where $\UU_n$ is any universe and $\UU_0$ is the lowest universe.
It may also be assumed that $\mathsf{r}_{n}(a) = a$; hence, that $\mathsf{r}_n$ is an equivalence between $\prop_n$ and $\prop_0$. 
This is a version of the ``axiom of impredicativity for mere propositions" mentioned in \autoref{subsec:prop-subsets}.  A related axiom resizes, not just mere propositions $a:\UU$, but each type $\prop_n$ itself, down to the lowest universe -- observe that this can be justified from the previous version just by considering the next higher universe $\UU_1$, in which $\prop_0$ already \emph{is} small.
If we assume some such a resizing rule allowing us to replace the large type $\prop$ with a small type $\Omega:U$ containing all mere propositions, we thereby obtain the usual notion of a subobject classifier.
 
 \begin{thm}\label{thm:settopos} Assuming a resizing rule for mere propositions, together with a type $\Omega:\UU$ of all mere propositions, the category $\set$ is then an elementary  topos.
 \end{thm}

Another condition sufficient for making $\set$ into a topos is the law of excluded middle, in the ``mere-propositional" form that we have called LEM; for then we have $\prop = \bool$, which \emph{is} small, and which then also classifies all mere propositions.
In topos theory a well-known sufficient condition for LEM is the axiom of choice, which is of course often assumed as an axiom in classical set theory. We next briefly investigate the relation between these conditions in our setting.

%%%%%%%%%%%%%%%%%%%%%%%%%%%%%%%%%%%%%%%%%%%%%%%%
\subsection{Axioms of choice and excluded middle for sets}
\label{subsec:emacinsets}

Recall from \autoref{sec:logic}  the following definitions of the variations on the axiom of choice and the law of excluded middle:

\begin{align*}
 \mathsf{LEM}_\infty\;\defeq\;&\ 
  \prd{A:\UU} (A + \neg A)\\
  %
  \mathsf{LEM}\;\defeq\;&\ 
  \prd{A:\UU} \Big(\isprop(A) \to (A + \neg A)\Big)\\
  %
  \mathsf{AC}_\infty\;\defeq\;&\ 
    \Big(\prd{x:X} \sm{a:A(x)} P(x,a)\Big) \\
    & \to\sm{g:\prd{x:X} A(x)} \prd{x:X} P(x,g(x))\\
   %
   \mathsf{AC}\;\defeq\;&\ 
    \prd{X:\set}  \prd{A:X\to\set} \prd{P:\prd{x:X}A(x)\to\prop} \\
    & \Big(\prd{x:X} \Brck{\sm{a:A(x)} P(x,a)}\Big) \\
    & \to\Brck{\sm{g:\prd{x:X} A(x)} \prd{x:X} P(x,g(x))}
\end{align*}

The subscript $\infty$ denotes the absence of any assumptions that the types involved are $n$-types for any $n$, and also the absence of any truncation operations.
The unsubscripted versions, which match most closely the versions assumed in classical set theory, involve a judicious choice of places to apply truncation assumptions and operations.
Of course, there are many other possible variations, involving truncation assumptions and operations into $n$-types for any $n$, which might be placed in some places but not others.

Even if we restrict ourselves to sets (rather than $n$-types for $n>0$), there are other possible variations.
For instance, consider the following generalizations of $\mathsf{LEM}$:
%
\begin{align*}
  \mathsf{LEM}_{0,0}\;\defeq\;&\ \prd{A:\set} (A + \neg A) \\
  \mathsf{LEM}_{0,-1}\;\defeq\;&\ \prd{A:\set} \brck{A + \neg A}.
\end{align*}
%
More generally, we might consider $\mathsf{LEM}_{n,m}\defeq \prd{A:\ntype{n}} \trunc m{A+\neg A}$, with $\mathsf{LEM}$ being $\mathsf{LEM}_{-1,\infty}$.
(If $A$ is a mere proposition, then so is $A+\neg A$; thus $\mathsf{LEM}$ is also the same as $\mathsf{LEM}_{-1,-1}$.)
Similarly, in $\mathsf{AC}$ we might include various homotopy level assumptions on $X$, $A$, and $P$, and/or truncations into various levels.
The implications between these principles have yet to be fully explored, but in this section we will prove the following implications that are most relevant for the study of sets.
\[
\mathsf{LEM}_\infty \Rightarrow \mathsf{LEM}_{0,0} \Rightarrow  \mathsf{AC} \Rightarrow \mathsf{LEM}_{0,-1} \Rightarrow \mathsf{LEM}\Rightarrow \mathsf{AC}_\infty
\]
At one extreme, in \autoref{subsec:pat?}, we showed that $\mathsf{LEM}_\infty$ is inconsistent with univalence, and an inspection of the proof shows that it applies just as well to $\mathsf{LEM}_{0,0}$.
At the other extreme, we have seen that $\mathsf{AC}_\infty$ is provable in pure type theory, without any further assumptions (not even function extensionality).
In this section we will show the implication $\mathsf{AC} \Rightarrow \mathsf{LEM}_{0,-1}$.

First recall that in \autoref{thm:ac-epis-split} we showed that $\mathsf{AC}$ is equivalent to
\[ \big(\prd{x:X} \brck{Y(x)}\big) \to \bbrck{\prd{x:X} Y(x)} \]
for any $X:\set$ and $Y:X\to\set$.
It is easy to see that this is equivalent to asking that any surjection between sets merely has a section, i.e.\ if $X,Y:\set$ and $f:Y\to X$ then
\[ \big(\fall{x:X}\exis{y:Y} (f(x)=y)\big) \Rightarrow \exis{g:X\to Y}\fall{x:X} f(g(x))=x. \]

In this subsection, we will first show that if every surjective function between sets merely has
a section, then for all propositions $A$ we can decide whether $A$ or $\neg A$
holds, i.e.\ the implication $\mathsf{AC} \Rightarrow \mathsf{LEM}$.  The implication $\mathsf{AC} \Rightarrow \mathsf{LEM}_{0,-1}$ then also follows.

Recall the notion of \emph{suspension} from \autoref{sec:suspension}.
In the case where $A$ is a proposition, it turns out that the
suspension $\susp(A)$ is equivalent to the quotient $\bool/R$,
where $R$ is the equivalence relation we get by setting
$R(\mathsf{true},\mathsf{false})\defeq A$. This will become clear in the development
below.

\begin{lem}\label{prop:trunc_of_prop_is_set}
The suspension of any proposition $A$ is a set and the path space 
$\north=\south$ in $\susp(A)$ is equivalent to $A$. 
\end{lem}

\begin{proof}
Let $A$ be a proposition. Using the univalence axiom, we will define a 
dependent type $P:\susp(A)\to\susp(A)\to\type$ with the 
property that $P(x,y)$ is a proposition for each $x,y:\susp(A)$ 
and which turns out to be equivalent to the dependent type 
$\mathsf{Id}_{\susp(A)}$.

We make the following definitions:
\begin{align*}
P(\north,\north) & \defeq \unit & P(\south,\north) & \defeq A\\
P(\north,\south) & \defeq A & P(\south,\south) & \defeq \unit.
\end{align*}
To show that this gives a dependent type we need to verify that there 
is an equivalence $P(\north,\north)\simeq P(\north,\south)$ and an eqivalence 
$P(\south,\north)\simeq P(\south,\south)$ for every $a:A$. Since $A$ is assumed to 
be a proposition, this is indeed the case.

Note that $P$ is a reflexive relation. Therefore we may 
apply \autoref{thm:h-set-refrel-in-paths-sets}, so it suffices to
show that $\tau:\prd{x,y:\susp(A)}P(x,y)\to(x=y)$.
We make the definitions
\begin{align*}
  \tau(\north,\north) & \defeq \lam{t} \refl{\north} & \tau(\south,\north) &\defeq \lam{a} \opp{\merid(a)}\\
  \tau(\north,\south) & \defeq \merid & \tau(\south,\south) &\defeq \lam{t} \refl{\south}.
\end{align*}
To find a path $\trans{\merid(a)}{\tau(\north,\north)}=\tau(\north,\south)$ for $a:A$, 
note that because $A$ is a proposition there are paths
\begin{equation*}
(\trans{\merid(a)}{\tau(\north,\north)})(x)=\tau(\north,\north,x)\ct\opp{\merid(a)}
=\opp{\merid(a)}= \opp{\merid(x)}. 
\end{equation*}
Thus we get the requested path from function extensionality. Likewise, 
we obtain a path $\tau(\north)=\tau(\south)$ for every $a:A$, 
which finishes the construction of $\tau$.
\end{proof}

\begin{defn}
For any  type  $A$, we define
\begin{equation*}
\mathsf{isDecidable}(A)\defeq  (A+\neg A).
\end{equation*}
\end{defn}

\begin{thm}[Bishop/Diaconescu]\label{thm:1surj_to_surj_to_pem}
If all surjections between sets purely have a section, then for all $A:\prop$,
$\mathsf{isDecidable}(A)$. Consequently, if all surjections between sets merely have a section, then for all $A:\prop,
\brck{\mathsf{isDecidable}(A)}$.
\end{thm}

\begin{proof}
Consider the function $\pi:\mathsf{bool}\to\susp(A)$ defined by 
$\pi(\mathsf{true})\equiv \south$ and $\pi(\mathsf{false})\equiv \north$. 
First  we  show that $\pi$ is surjective, 
by showing that there is a function of type
\begin{equation*}
\prd{x:\susp(A)} \|\hfiber\pi{x}\|.
\end{equation*}
We have the terms 
$\langle{\mathsf{false},\refl{\north}}\rangle:\hfib\pi\north$ 
and $\langle{\mathsf{true},\refl{\south}}\rangle:\hfib\pi\south$. 
Since $\|\hfiber\pi{x}\|$ is a proposition for each 
$x:\susp(A)$, by induction this gives us the claimed surjectivity.

By proposition~\ref{prop:trunc_of_prop_is_set} the suspension of 
a proposition is always a set, so our assumption gives us a 
section $g:\susp(A)\to\mathsf{bool}$ of $\pi$. 
Since $\mathsf{bool}$ has decidable equality, there is an element of
\begin{equation*}
\mathsf{isDecidable}(g(\pi(\mathsf{true}))= g(\pi(\mathsf{false})))
\end{equation*}
and since $g$ is a section of $\pi$ it follows that there is an element of
\begin{equation*}
\mathsf{isDecidable}(\pi(\mathsf{true})=\pi(\mathsf{false}))
\end{equation*}
Now we see that it is enough to show that 
$\pi(\mathsf{true})= \pi(\mathsf{false})$ is equivalent to $A$. 
This we also obtained in proposition~\ref{prop:trunc_of_prop_is_set}.
\end{proof}

\begin{cor}\label{cor:ACtoLEM0}
Suppose that all surjections between sets merely have a section. Then for all sets $A$, we have $\brck{A + \neg A}$.
\end{cor}

\begin{proof}
  We have a surjection
\[
A + \neg A \epi \brck{A} + \brck{\neg A} \epi \brck{(\brck{A} + \brck{\neg A})} = \brck{A} \vee \brck{\neg A} = \brck{A} \vee \neg \brck A = \unit
\]
which therefore merely has a section.
But a section of this is exactly an inhabitant of $A+\neg A$, so we have $\brck{A+\neg A}$.
\end{proof}

%%%%%%%%%%%%%%%%%%%%%%%%%%%%%%%%%%%%%%%%%%%%%%%%%
%\section{Algebraic set theory}\label{sec:ast}
%
%We should obtain a model of CST following Awodey, Forsell, Warren. 
%We may want to use the replacement axiom~\ref{lem:replacement}.
%We should consider as classes only those large sets with small diagonal.
%
%The universe is naturally a groupoid, but it may be possible to distort it to be a set.
%
%It seems unlikely that the axiom of collection and the axiom of multiple choice are derivable in HoTT,
%simply because none of its axioms seem applicable.  In the constructive set theory CZF~\cite{aczel2001notes}, unlike in classical Zermelo
%set theory, the collection axiom is \emph{stronger} than the replacement axiom. 
%The replacement axiom \emph{is} derivable from the resizing rules; see section~\ref{sec:resizing}.
%In line with Voevodsky's proposal to add resizing rules to homotopy type theory, one could also consider its extension with the
%collection axiom.
%
%\subsection{Class of stable maps}
%\begin{defn}\label{defn:small_maps}
%A class $\smal:\prd{X,Y:\type} (X\to Y)\to\prop$ is \emph{stable} of if it satisfies:
%\begin{description}
%\item[pullback stability] If we have a pullback diagram
%\begin{equation*}
%\begin{tikzpicture}
%\matrix (m) [std] {X & A \\ Y & B \\};
%\draw[ar] (m-1-1) -- (m-1-2);
%\draw[ar] (m-1-2) -- node[right] {$f$} (m-2-2);
%\draw[ar] (m-1-1) -- node[left]  {$g$} (m-2-1);
%\draw[ar] (m-2-1) -- (m-2-2);
%\end{tikzpicture}
%\end{equation*}
%then $\smal(f)\to\smal(g)$.
%\item[Descent] If the bottom arrow in the above diagram is surjective, then $\smal(g)\to\smal(f)$. 
%\item[Sum] If $\smal(f),\smal(g)$, then $\smal(f+g)$.
%\end{description}
%A class of stable class of maps is \emph{locally full}~\citep[3.2]{MoerdijkPalmgren2002} if for all $g:X\to Y$ and $f:Y\to Z$ such
%that $\smal(f)$: $\smal(g)$ iff $\smal(fg)$.
%
%A class of maps $\smal$ is called a \emph{class of small maps} if it is stable, locally full and for each $X$, 
%$\smal_X$ --- the small over maps over $X$ --- forms a $\Pi\mathsf{W}$-pretopos; see~\citep[3.3]{MoerdijkPalmgren2002}.
%\end{defn}
%
%\begin{lem}
%The class of set-fibered maps is a class of small maps.
%\end{lem}
%\begin{proof}
%The class of maps with set-fibers is stable. It even has dependent sums.
%
%We claim that it is locally full: If $g$ has set-fibers, then $fg$ has set-fibers, as sets are
%closed under $\Sigma$-types. Conversely, fix $y\in Y$, then $\hfiber g y$ is the pullback
%\[\sm {x:\hfiber{fg}{f(y)}} g(x)=y.\]
%
%By the use of the object classifier, \autoref{thm:nobject_classifier}, we see that the type $\smal_X$ is equivalent
%to the sets in context $X$. Now, sets in any context form a $\Pi\mathsf{W}$-pretopos; see \autoref{subsec:piw}.
%\end{proof}
%
%%\section{Excluded middle and the axiom of choice}
%%\label{sec:excl-middle-axiom}
%%
%%\textbf{TODO:} Discuss excluded middle and choice, state that the appropriate versions are
%%consistent, so me may assume them. Prove that choice implies excluded middle?
%%See~\ref{thm:1surj_to_surj_to_pem}
%%
%
%\subsection{Sets and classes}
%\label{subsec:sets-classes}
%
%\textbf{TODO:} Discuss the role of universes as classes-of-classes-of-classes, or
%alternatively as inaccessible sets.
%This could be part of AST(?)
%
%\subsection{Powersets}
%\label{subsec:powersets}
%
%\textbf{TODO:} Point out that we get \textbf{powerclasses} when we consider $X \to \prop$, that the
%powerset axiom is about powerclasses being sets, that this is equivalent to $\prop$ being
%a set, so resizing comes into play.
%
%For a set $X$, we speak of predicates $Y:X\to \prop$ equivalently as \textbf{subsets} of $X$, and sometimes write $x\in Y$ to mean $Y(x)$.
%We will also use the set-builder notation for such subsets:
%\[ \setof{x:X | P } \defeq \lam{x} P \]
%Univalence for \prop, plus function extensionality, implies that such subsets are \emph{extensional} in the usual sense of set theory:
%\[ (Y_1 = Y_2) \leftrightarrow \Big(\prd{x:X} (x\in Y_1) \leftrightarrow (x\in Y_2)\Big) \]
%Note that both sides of this equivalence are mere propositions.
%
%We define the \textbf{power set} of $X$ to be
%\begin{align*}
%  \mathcal{P} X &\defeq X\to \prop
%\end{align*}
%Likewise, we define the subset of merely inhabited subsets in $X$ to be
%\begin{align*}
%  \mathcal{P}_+ X &\defeq \setof{P:X\to \prop | \brck{(\exists x:X), P(x)}}
%\end{align*}
%Assuming excluded middle, we have $\mathcal{P}_+ X \cong \setof{P:X\to \prop | P \neq (\lam{x} \bot)}$ and also $\mathcal{P} X \cong \mathcal{P}_+ X + \unit$.
%
%For $Y:\mathcal{P}X$, we write $X\setminus Y \defeq \setof{x:X | x\notin Y}$.
%Similarly, we have unions $Y_1 \cup Y_2$, intersections $Y_1 \cap Y_2$, and so on.

\section{The cumulative hierarchy}
\label{sec:cumulative-hierarchy}

We can define a cumulative hierarchy $V$ of all sets in a given universe $\UU$ as a higher inductive type, in such a way that $V$ is again a set (in a larger universe $\UU'$), equipped with a binary ``membership" relation $x\in y$ which satisfies the usual laws of set theory.

\begin{defn}
  The \emph{cumulative hierarchy $V$} relative to a type universe $\UU$ is the
  higher inductive type generated by the following constructors.
  %
  \begin{enumerate}
  \item For every $A : \UU$ and $f : A \to V$, there is an element $\vset(A, f)$ : V.
  \item For all $A, B : \UU$, $f : A \to V$ and $g : B \to V$ such that
    %
    \begin{equation} \label{eq:V-path}
      \big(\fall{a:A} \exis{b:B} \id[V]{f(a)}{g(b)}\big) \land \big(\fall{b:B} \exis{a:A} \id[V]{f(a)}{g(b)}\big)
    \end{equation}
    %
    there is a path $\id[V]{\vset(A,f)}{\vset(B,g)}$.
  \item The 0-truncation constructor: for all $x,y:V$ and $p,q:x=y$, we have $p=q$.
  \end{enumerate}
\end{defn}

In set-theoretic language, $\vset(A,f)$ can be understood as the set that is the image of $A$ under $f$, i.e.\ $\{ f(a) \mid a
\in A \}$, but of course we cannot write this in type theory. The hierarchy $V$ is
bootstrapped from the empty map $! : \emptyt \to V$, which gives the empty set as $\varnothing = \vset(\emptyt,!)$.
Then the singleton $\{\varnothing\}$ enters $V$ through $\unit \to V$, defined as $\ttt \mapsto \varnothing$, and so
on. The type $V$ lives in the same universe as the base universe $\UU$.

The induction principle for $V$ says that given $P:V\to \set$, in order to construct $h:\prd{x:V} P(x)$, it suffices to give the following.
\begin{enumerate}
\item For any $f:A\to V$, construct $h(\vset(A,f))$, assuming as given $h(f(a))$ for all $a:A$.
\item Verify that if $f : A \to V$ and $g : B \to V$ satisfy~\eqref{eq:V-path}, then $h(\vset(A,f)) = h(\vset(B,g))$, assuming inductively that $h(f(a)) = h(g(b))$ whenever $f(a)=g(b)$.
\end{enumerate}
The second clause may seem tautologous, but is not.
The point is that in the first clause, ``$h(f(a))$'' is essentially a formal symbol which we cannot peek inside of, which $h(\vset(A,f))$ must be defined in terms of. Thus, in the second clause, we assume equality of these formal symbols when appropriate, and verify that the elements resulting from the construction of the first clause are also equal.

Observe that, by induction, every $v:V$ is of the form $v=\vset(A,f)$ for some $A:\UU$ and $f:A\to V$, and so we can define the \define{membership relation} $x\in v$ on $V$ by setting:
%
% Note: "membership" rather than "elementhood", because "element" is taken.
%
\begin{equation*}
  (x \in \vset(A,f)) \defeq \exis{a : A} x = f(a).
\end{equation*}
%
To see that the definition is valid, suppose we have a path $\vset(A, f) = \vset(B, g)$
constructed through~\eqref{eq:V-path}. If $x \in \vset(A,f)$ then there is $a : A$ such
that $x = f(a)$, but by~\eqref{eq:V-path} there is $b : B$ such that $f(a) = g(b)$, hence
$x = g(b)$ and $x \in \vset(B,f)$. The converse is symmetric.

The \define{subset relation} $x\subseteq y$ is defined on $V$ as usual by
%
\begin{equation*}
  (x \subseteq y) \defeq \fall{z : V} z \in x \Rightarrow z \in y.
\end{equation*}

A \define{class} may be taken to be a mere predicate on~$V$. We can say that a class $C : V \to \prop$ is a
  \emph{$V$-set} if there merely exists $v\in V$ such that
%
\begin{equation*}
  \fall{x : V} C(x) \Leftrightarrow x \in v.
\end{equation*}

We may also use the conventional notation for classes:
\[
\setof{ x | C(x) } \defeq \lam{x}C(x)
\]

A class $C: V\to \prop$ will be called \define{$\UU$-small} if all of its values $C(x)$ lie in $\UU$, specifically $C: V\to \prop_{\UU}$.
Since $V$ lives in the same universe $\UU'$ as does the base universe $\UU$ from which it is built, the same is true for the identity types $v=_V w$ for any $v,w:V$. It will be convenient to have a $\UU$-small ``resizing" of the identity relation, which we can define by induction as follows.

\begin{defn}\label{def:bisimulation}
Define the \define{bisimulation} relation $$\sim\ : V\times V \longrightarrow \prop_{\UU}$$ by double induction over $V$, where for $\set(A,f)$ and $\set(B,g)$ we let:
\begin{enumerate}
\item[] $\set(A,f) \sim \set(B,g) \defeq \big(\fall{a:A}\exis{b:B}fa \sim gb\big) \land \big(\fall{b:B}\exis{a:A}fa \sim gb\big)$
\end{enumerate}
\end{defn}
%
To verify that the definition is correct, we just need to check that it respects paths $\vset(A, f) = \vset(B, g)$ constructed through~\eqref{eq:V-path}, but this is obvious, and that $\prop_{\UU}$ is a set, which it is.  Note that $u\sim v$ is in $\prop_\UU$ by construction.

\begin{lem}\label{lem:BisimEqualsId}
For any $u,v:V$ we have $(u=_V v) = (u\sim v)$.
\end{lem}

\begin{proof}
It suffices to show that $(u\sim v)\to (u=_V v)$.
So assume that $\set(A,f)\sim\set(B,g)$. By definition, we have $(\fall{a:A}\exis{b:B}fa \sim gb)$ and conversely.
But the induction hypothesis then tells us that $(\fall{a:A}\exis{b:B}fa = gb)$ and conversely.
So by the path-constructor for $V$ we have $\set(A,f) =_V \set(B,g)$.
\end{proof}

Now we can use the resized identity relation to get the following useful principle.

\begin{lem}\label{lem:MonicSetPresent}
For every $u:V$ there is a given $A_u:\UU$ and monic $m_u: A \mono V$ such that $u = \vset(A_u, m_u)$.
\end{lem}

\begin{proof}
Take any presentation $u = \vset(A,f)$ and factor $f:A\to V$ as $$f = m_u\circ e_u : A \epi A_u \mono V.$$  Clearly $u = \vset(A_u, m_u)$ if only $A_u$ is still in $\UU$, which holds if the kernel of $e_u : A \epi A_u$ is in $\UU$.  But the kernel of $e_u : A \epi A_u$ is the pullback along $f : A\to V$ of the identity on $V$, which we just showed to be $\UU$-small, up to equivalence.  Now, this construction of the pair $(A_u, m_u)$ with $m_u :A_u \mono V$ and $u = \vset(A_u, m_u)$ from $u:V$ is unique up to equivalence over $V$, and hence up to identity by univalence.  Thus by the principle of unique choice \eqref{cor:UC} there is a map $c : V\to\sm{A:\UU}A\to V$ such that $c(u) = (A_u, m_u)$, with $m_u :A_u \mono V$ and $u = \vset(c(u))$, as claimed.
\end{proof}

\begin{defn}\label{def:TypeOfElements}
For $u:V$, the just constructed monic presentation $m_u: A_u \mono V$ such that $u = \vset(A_u, m_u)$ may be called the \define{type of elements} of $u$ and denoted $m_u : [u] \mono V$, or even $[u] \mono V$.  We can think of $[u]$ as the ``subclass of $V$ consisting of members of $u$".
\end{defn}

\begin{thm}\label{thm:VisCST}
  The following hold for $(V, {\in})$:
  %
  \begin{enumerate}
  \item \emph{extensionality:}
    %
    \begin{equation*}
      \fall{x, y : V} x \subseteq y \land y \subseteq x \Leftrightarrow x = y.
    \end{equation*}
    %
     \item \emph{empty set:} for all $x:V$, we have $\neg (x\in \varnothing)$.
    %
    \item \emph{pairing:} for all $u, v:V$, the class $\setof{ x | x = u \vee x = v}$ is a $V$-set.
      %
    \item \emph{infinity:} there is a $v:V$ with $\varnothing\in v$ and $x\in v$ implies $x\cup \{x\}\in v$.
    %
  \item \emph{union:} for all $v:V$, the class $\setof{ x | \exis{u:V} x \in u \in v}$ is a $V$-set.
    %
    \item \emph{function set:} for all $u, v:V$, the class $\setof{ x | x : u\to v}$ is a $V$-set.
    %
   \item \emph{$\in$-induction:} if $C : V \to \prop$ is a class such that $C(a)$ holds whenever $C(x)$ for all $x\in a$, then $C(v)$ for all $v:V$.
   %
     \item \emph{replacement:} given any $r : V \to V$ and $a : V$, the following class  is a $V$-set. 
  $$\setof{ x | \exis{y : V} y \in a \land x = r(y)}$$ 

  %
   \item \emph{separation:} given any $a : V$ and $\UU$-small $C : V \to \prop_\UU$, the following class is a $V$-set.
   $$\setof{ x | x \in a \land C(x)}$$
  \end{enumerate}
\end{thm}


\begin{proof}[Sketch of proof]
  \mbox{}
  %
  \begin{enumerate}
  \item Extensionality: if $\vset(A,f) \subseteq \vset(B, g)$ then $f(a) \in \vset(B, g)$
    for every $a : A$, therefore for every $a : A$ there merely exists $b : B$ such that
    $f(a) = g(b)$. The assumption $\vset(B, g) \subseteq \vset(A, f)$ gives the other half
    of~\eqref{eq:V-path}, therefore $\vset(A,f) = \vset(B,g)$.
    
  \item Empty set: suppose $x\in \varnothing = \vset(\emptyt,!)$.  Then $\exis{a:0}x=\, !a$, which is absurd.
  
  \item Pairing: given $u = \vset(A,f)$ and $v=\vset(B,g)$, let $w=\vset(A+B,[f,g])$.

  \item Infinity: take $w = \vset(\nat,I)$, where $I: \nat \to V$ is given by the recursion $I(0) = \varnothing$ and $I(n+1) = I(n)\cup \{I(n)\}$.
 
  \item Union: Take any $v:V$ and any presentation $f :A\to V$ with $v=\vset(A,f)$.  Then let $\tilde{A} \defeq \sm{a:A}[fa]$, where $[fa] \mono V$ is the type of elements from \autoref{def:TypeOfElements}.  $\tilde{A}$ is plainly $\UU$-small and has a projection map $\pi:\tilde{A}\to A$.  We then let $w \defeq \vset(\tilde{A}, f\circ\pi)$.
  
  \item Function set: given $u, v:V$, take the types of elements $[u] \mono V$ and $[u] \mono V$, and the function type $[u]\to [v]$.  We want to define a map
  \[
 r: ([u]\to [v])\ \longrightarrow\ V
  \] 
   with ``$r(f) = \{ \langle x, f(x)\rangle\ |\ x:[u] \}$", but in order for this to make sense we must first define the ordered pair $\langle x, y\rangle$, and then we take the map $r': x \mapsto \langle x, f(x)\rangle$, and then we can put $r(f)\defeq \vset([u], r')$.  But the ordered pair can be defined in terms of unordered pairing as usual.
   
  \item $\in$-induction: let $C : V \to \prop$ be a class such that $C(a)$ holds whenever $C(x)$ for all $x\in a$, and take any $v=\vset(B,g)$.  To show that $C(v)$ by induction, assume that $C(gb)$ for all $b:B$.  For every $x\in v$ there merely exists some $b:B$ with $x = gb$, and so $C(x)$.  Thus $C(v)$.

  \item Replacement: the statement ``$C$ is a $V$-set'' is a mere proposition, so we may
    proceed by induction as follows. Supposing $x$ is $\vset(A, f)$, we claim that $w
    \defeq \vset(A, r \circ f)$ is the set we are looking for.  If $C(y)$ then there merely exists
    $z : V$ and $a : A$ such that $z = f(a)$ and $y = r(z)$, therefore $y \in w$.
    Conversely, if $y \in w$ then there merely exists $a : A$ such that $y = r(f(a))$, so
    if we take $z \defeq f(a)$ we see that $C(y)$ holds.

\item Let us say that a class $C: V\to\prop$ is \define{separable} if for any $a:V$ the class $$a\cap C \defeq\setof{x | x\in a \wedge C(x)}$$ is a $V$-set.
We need to show that any $\UU$-small  $C: V \to \prop_\UU$ is separable. Indeed, given $a=\vset(A,f)$, let $A' = \exis{x:A}C(fx)$, and take $f' = f\circ i$, where $i : A' \to A$ is the obvious inclusion.  Then we can take $a' = \vset(A',f')$ and we have $x\in a\wedge C(x) \Leftrightarrow x\in a'$ as claimed.  We needed the assumption that $C$ lands in $\UU$ in order for $A' = \exis{x:A}C(fx)$ to be in $\UU$.
\end{enumerate}
\end{proof}

It is also convenient to have a strictly syntactic criterion of separability, so that one can read off from the expression for a class that it produces a $V$-set.  One such familiar condition is being ``$\Delta_0$", which means that the expression is built up from equality $x=_V y$ and membership $x\in y$, using only propositional logical operations $\neg, \land, \lor, \Rightarrow$ and ((-1)-truncated) quantifiers over particular sets, i.e.\ of the form $\exists(x\in a)$ and $\forall(y\in b)$.  

\begin{cor}\label{cor:Delta0sep}
If the class $C: V \to \prop$ is $\Delta_0$ in the above sense, then it is separable.
\end{cor}

\begin{proof}
Recall that we have a $\UU$-small resizing $x\sim y$ of identity $x = y$. Since $x\in y$ is defined in terms of $x=y$, we also have a $\UU$-small resizing of membership $$x\bin\vset(A,f) \defeq \exis{a:A} x\sim f(a).$$ Now, let $\Phi$ be the $\Delta_0$ expression for $C$, so $\Phi = C$, and let $\widetilde{\Phi}$ be the result of replacing all occurrences of $=$ and $\in$ by their resized equivalents $\sim$ and $\bin$.  Clearly then $\widetilde{\Phi}$ also expresses $C$, in the sense that for all $x:V$, $\Phi(x) \Leftrightarrow C(x)$, and hence $\widetilde{\Phi}=C$ by univalence.  It now suffices to show that $\widetilde{\Phi}$ is $\UU$-small, for then it will be separable by the theorem.  

We show that  $\widetilde{\Phi}$ is $\UU$-small by induction on the construction of the expression.  The base cases are $x\sim y$ and $x\bin y$, which have already been resized into $\UU$.  Let $p(v)$ be a $\UU$-small proposition for every $v:V$, and consider the proposition $$\exists(x\bin a)p(x) \defeq \exis{x:V}(x\bin a \land p(x)).$$
Duh.
\end{proof}

We have shown that in type theory with a universe $\UU$, the cumulative hierarchy $V$ is a model of a ``constructive set theory" with many of the standard axioms, but (as far as we know) without the usual \emph{strong collection} or \emph{subset collection} axioms of the set theory CZF.  We do not expect these to hold, since they seem to be related to the interaction between the type-theoretic axiom of choice $\mathsf{AC}_\infty$ and the (setoid-like) definition of equality in the usual interpretation of CZF into type theory.  Indeed, our model $(V,\in)$ is based on a cumulative hierarchy $V$ which is a higher inductive type \emph{inside} the system, rather than being an \emph{external} construction, and so it is not surprising that it differs in some ways from prior interpretations.

Finally, consider the result of adding the axiom of choice for sets to our type theory, in the form  $\mathsf{AC}$ from \autoref{subsec:emacinsets} above.  This has the consequence that $\mathsf{LEM}$ then also holds, by \autoref{thm:1surj_to_surj_to_pem}, and so $\set$ is a topos with subobject classifier $\bool$, by \autoref{thm:settopos}.  In this case, we have $\prop = \bool:\UU$, and so \emph{all classes are separable}.  Thus we have shown:

\begin{lem}\label{lem:fullsep}
In type theory with $\mathsf{AC}$, the law of \emph{(full) separation} holds: given \emph{any} class $C : V \to \prop$ and $a : V$, the class $a \cap C$ is a $V$-set.
\end{lem}

\begin{thm}\label{thm:zfc}
In type theory with $\mathsf{AC}$ and a universe $\UU$, the cumulative hierarchy $V$ is a model of Zermelo-Frankel set theory with choice, ZFC.
\end{thm}

\begin{proof}
We have all the axioms listed in \autoref{thm:VisCST}, plus full separation, so we just need to show that there are power sets $\mathcal{P}(a):V$ for all $a:V$.  But since we have $\mathsf{LEM}$ these are simply function types $\mathcal{P}(a) = (a\to\bool)$.
\end{proof}

%It follows from the foregoing, of course, that HoTT with UA, HITs, and AC is proof-theoretically stronger than ZFC.  The converse is also true for ZFC plus one regular cardinal, by Voevodsky's Kan simplicial set model.

\newcommand{\cd}[1]{\left|#1\right|}
\newcommand{\inj}{\ensuremath{\mathsf{inj}}}

\section{Cardinal numbers}
\label{sec:cardinals}

\begin{defn}
  The \textbf{type of cardinal numbers} is the 0-truncation of \set:
  \[ \card \defeq \pizero{\set} \]
  Thus, a \textbf{cardinal number}, or \textbf{cardinal}, is an inhabitant of $\card\jdeq \pizero\set$.
\end{defn}

\begin{rmk}
  As in some previous sections, here we are being universe polymorphic and typically ambiguous.
  Technically, there is a separate type ``\card'' associated to each universe ``\type'', but with these conventions we can state theorems beginning with ``for all cardinal numbers\dots''\ and give them exactly the same sort of meaning as those beginning ``for all types\dots''.
\end{rmk}

If $A$ is a set, we write $\cd{A}$ for its image under the canonical projection $\set \to \card$.
Of course, by definition, \card is a set.
It also inherits the structure of a semiring from \set.

\begin{defn}
  The operation of \textbf{cardinal addition}
  \[ \blank+\blank : \card \to \card \to \card \]
  is defined by induction on truncation:
  \[ \cd{A} + \cd{B} \defeq \cd{A+B} .\]
\end{defn}
\begin{proof}
  Since $\card\to\card$ is a set, to define $\alpha+\blank:\card\to\card$ for all $\alpha:\card$, by induction it suffices to assume that $\alpha$ is $\cd{A}$ for some $A:\set$.
  Now we want to define $\cd{A}+\blank :\card\to\card$, i.e.\ we want to define $\cd{A}+\beta :\card$ for all $\beta:\card$.
  However, since $\card$ is a set, by induction it suffices to assume that $\beta$ is $\cd{B}$ for some $B:\set$.
  But now we can define $\cd{A}+\cd{B}$ to be $\cd{A+B}$.
\end{proof}

\begin{defn}
  Similarly, the operation of \textbf{cardinal multiplication}
  \[ \blank\cdot\blank : \card \to \card \to \card \]
  is defined by induction on truncation:
  \[ \cd{A} \cdot \cd{B} \defeq \cd{A\times B} \]
\end{defn}

\begin{lem}\label{card:semiring}
  \card is a commutative semiring, i.e.\ for $\alpha,\beta,\gamma:\card$ we have the following.
  \begin{align*}
    (\alpha+\beta)+\gamma &= \alpha+(\beta+\gamma)\\
    \alpha+0 &= \alpha\\
    \alpha + \beta &= \beta + \alpha\\
    (\alpha \cdot \beta) \cdot \gamma &= \alpha \cdot (\beta\cdot\gamma)\\
    \alpha \cdot 1 &= \alpha\\
    \alpha\cdot\beta &= \beta\cdot\alpha\\
    \alpha\cdot(\beta+\gamma) &= \alpha\cdot\beta + \alpha\cdot\gamma
  \end{align*}
  where $0 \defeq \cd{\emptyset}$ and $1\defeq\cd{\unit}$.
\end{lem}
\begin{proof}
  We prove the commutativity of multiplication, $\alpha\cdot\beta = \beta\cdot\alpha$; the others are exactly analogous.
  Since \card is a set, the type $\alpha\cdot\beta = \beta\cdot\alpha$ is a mere proposition, and in particular a set.
  Thus, by induction it suffices to assume $\alpha$ and $\beta$ are of the form $\cd{A}$ and $\cd{B}$ respectively, for some $A,B:\set$.
  Now $\cd{A}\cdot \cd{B} \jdeq \cd{A\times B}$ and $\cd{B}\times\cd{A} \jdeq \cd{B\times A}$, so it suffices to show $A\times B = B\times A$.
  Finally, by univalence, it suffices to give an equivalence $A\times B \simeq B\times A$.
  But this is easy: take $(a,b) \mapsto (b,a)$ and its obvious inverse.
\end{proof}

\begin{defn}
  The operation of \textbf{cardinal exponentiation} is also defined by induction on truncation:
  \[ \cd{A}^{\cd{B}} \defeq \cd{B\to A}. \]
\end{defn}

\begin{lem}\label{card:exp}
  For $\alpha,\beta,\gamma:\card$ we have
  \begin{align*}
    \alpha^0 &= 1\\
    1^\alpha &= 1\\
    \alpha^1 &= \alpha\\
    \alpha^{\beta+\gamma} &= \alpha^\beta \cdot \alpha^\gamma\\
    \alpha^{\beta\cdot \gamma} &= (\alpha^{\beta})^\gamma\\
    (\alpha\cdot\beta)^\gamma &= \alpha^\gamma \cdot \beta^\gamma
  \end{align*}
\end{lem}
\begin{proof}
  Exactly like \autoref{card:semiring}.
\end{proof}

\begin{defn}
  The relation of \textbf{cardinal inequality}
  \[ \blank\le\blank : \card\to\card\to\prop \]
  is defined by induction on truncation:
  \[ \cd{A} \le \cd{B} \defeq \brck{\inj(A,B)} \]
  where $\inj(A,B)$ is the type of injections from $A$ to $B$.
  In other words, $\cd{A} \le \cd{B}$ means that there merely exists an injection from $A$ to $B$.
\end{defn}

\begin{lem}
  Cardinal inequality is a preorder, i.e.\ for $\alpha,\beta:\card$ we have
  \begin{gather*}
    \alpha \le \alpha\\
    (\alpha \le \beta) \to (\beta\le\gamma) \to (\alpha\le\gamma)
  \end{gather*}
\end{lem}
\begin{proof}
  As before, by induction on truncation.
  For instance, since $(\alpha \le \beta) \to (\beta\le\gamma) \to (\alpha\le\gamma)$ is a mere proposition, by induction on 0-truncation we may assume $\alpha$, $\beta$, and $\gamma$ are $\cd{A}$, $\cd{B}$, and $\cd{C}$ respectively.
  Now since $\cd{A} \le \cd{C}$ is a mere proposition, by induction on $(-1)$-truncation we may assume given injections $f:A\to B$ and $g:B\to C$.
  But then $g\circ f$ is an injection from $A$ to $C$, so $\cd{A} \le \cd{C}$ holds.
  Reflexivity is even easier.
\end{proof}

We may likewise show that cardinal inequality is compatible with the semiring operations.

\begin{lem}\label{thm:injsurj}
  Consider the following statements:
  \begin{enumerate}
  \item There is an injection $A\to B$.\label{item:cle-inj}
  \item There is a surjection $B\to A$.\label{item:cle-surj}
  \end{enumerate}
  Then, assuming excluded middle:
  \begin{itemize}
  \item Given $a_0:A$, we have~\ref{item:cle-inj}$\to$\ref{item:cle-surj}.
  \item Therefore, if $A$ is merely inhabited, we have~\ref{item:cle-inj} $\to$ merely \ref{item:cle-surj}.
  \item Assuming the axiom of choice, we have~\ref{item:cle-surj} $\to$ merely \ref{item:cle-inj}.
  \end{itemize}
\end{lem}
\begin{proof}
  If $f:A\to B$ is an injection, define $g:B\to A$ at $b:B$ as follows.
  Since $f$ is injective, the fiber of $f$ at $b$ is a mere proposition.
  Therefore, by excluded middle, either there is an $a:A$ with $f(a)=b$, or not.
  In the first case, define $g(b)\defeq a$; otherwise set $g(b)\defeq a_0$.
  Then for any $a:A$, we have $a = g(f(a))$, so $g$ is surjective.

  The second statement follows from this by induction on truncation.
  For the third, if $g:B\to A$ is surjective, then by the axiom of choice, there merely exists a function $f:A\to B$ with $g(f(a)) = a$ for all $a$.
  But then $f$ must be injective.
\end{proof}

\begin{thm}[Schroeder-Bernstein]
  Assuming excluded middle, for sets $A$ and $B$ we have
  \[ \inj(A,B) \to \inj(B,A) \to (A\cong B) \]
\end{thm}
\begin{proof}
  The usual ``back-and-forth'' argument applies without significant changes.
  Note that it actually constructs an isomorphism $A\cong B$ (assuming excluded middle so that we can decide whether a given element belongs to a cycle, an infinite chain, a chain beginning in $A$, or a chain beginning in $B$).
\end{proof}

\begin{cor}
  Assuming excluded middle, cardinal inequality is a partial order, i.e.\ for $\alpha,\beta:\card$ we have
  \[ (\alpha\le\beta) \to (\beta\le\alpha) \to (\alpha=\beta). \]
\end{cor}
\begin{proof}
  Since $\alpha=\beta$ is a mere proposition, by induction on truncation we may assume $\alpha$ and $\beta$ are $\cd{A}$ and $\cd{B}$, respectively, and that we have injections $f:A\to B$ and $g:B\to A$.
  But then the Schroeder-Bernstein theorem gives an isomorphism $A\simeq B$, hence an equality $\cd{A}=\cd{B}$.
\end{proof}

Finally, we can reproduce Cantor's theorem, showing that for every cardinal there is a greater one.

\begin{thm}[Cantor]
  For $A:\set$, there is no surjection $A \to (A\to \mathbf{2})$.
\end{thm}
\begin{proof}
  Suppose $f:A \to (A\to \mathbf{2})$ is any function, and define $g:A\to \mathbf{2}$ by $g(a) \defeq \neg f(a)(a)$.
  If $g = f(a_0)$, then $g(a_0) = f(a_0)(a_0)$ but $g(a_0) = \neg f(a_0)(a_0)$, a contradiction.
  Thus, $f$ is not surjective.
\end{proof}

\begin{cor}
  Assuming excluded middle, for any $\alpha:\card$, there is a cardinal $\beta$ such that $\alpha\le\beta$ and $\alpha\neq\beta$.
\end{cor}
\begin{proof}
  Let $\beta = 2^\alpha$.
  Now we want to show a mere propositon, so by induction we may assume $\alpha$ is $\cd{A}$, so that $\beta\jdeq \cd{A\to \mathbf{2}}$.
  Using excluded middle, we have a function $f:A\to (A\to \mathbf{2})$ defined by
  \[f(a)(a') \defeq
  \begin{cases}
    \top &\quad a=a'\\
    \bot &\quad a\neq a'.
  \end{cases}
  \]
  And if $f(a)=f(a')$, then $f(a')(a) = f(a)(a) = \top$, so $a=a'$; hence $f$ is injective.
  Thus, $\alpha \jdeq \cd{A} \le \cd{A\to \mathbf{2}} \jdeq 2^\alpha$.

  On the other hand, if $2^\alpha \le \alpha$, then we would have an injection $(A\to\mathbf{2})\to A$.
  By \autoref{thm:injsurj}, since we have $(\lam{x} \bot):A\to \mathbf{2}$ and excluded middle, there would then be a surjection $A \to (A\to \mathbf{2})$, contradicting Cantor's theorem.
\end{proof}

\section{Ordinal numbers}
\label{sec:ordinals}

\newcommand{\acc}{\ensuremath{\mathsf{acc}}}

\begin{defn}
  Let $A$ be a set and
  \[\blank<\blank:A\to A\to \prop\]
  a binary relation on $A$.
  We define by induction what it means for an element $a:A$ to be \textbf{accessible} by $<$:
  \begin{itemize}
  \item If $b$ is accessible for every $b<a$, then $a$ is accessible.
  \end{itemize}
  We write $\acc(a)$ to mean that $a$ is accessible.
\end{defn}

It may seem that such an inductive definition can never get off the ground, but of course if $a$ has the property that there are \emph{no} $b$ such that $b<a$, then $a$ is vacuously accessible.

\begin{lem}
  Accessibility is a mere property.
\end{lem}
\begin{proof}
  We claim that for any $a_1,a_2:A$, any $p:a_1=a_2$, and any $s_1:\acc(a_1)$ and $s_2:\acc(a_2)$, we have $\trans{p}{s_1}=s_2$.
  By induction, we may assume that $s_1$ is given by a function assigning to each $b_1<a_1$ a proof $s_1(b_1):\acc(b_1)$, and moreover that for any $b_2:A$, any $q:b_1=b_2$, and $t_2:\acc(b_2)$, we have $\trans{q}{s_1(b_1)}=t_2$.
  Similarly, we may assume that $s_2$ is given by a function assigning to each $b_2<a_2$ a proof $s_2(b_2):\acc(b_2)$, and that for any $b_1:A$, any $q:b_1=b_2$, and $t_1:\acc(b_1)$, we have $\trans{q}{t_1}=s_2(b_2)$.
  
  Now by function extensionality, to show $\trans{p}{s_1}=s_2$ it suffices to show that for any $b_2<a_2$ we have $\trans{p}{s_1}(b_2) = s_2(b_2)$.
  However, we can obtain this from the induction hypothesis for $s_2$ with $q\defeq \refl{b_2}$ and $t_1 \defeq \trans{p}{s_1}(b_2)$.
  This proves the claim.

  Finally, we instantiate the claim with $a_2\defeq a_1\defeq a$ and $p\defeq \refl{a}$.
  Thus, for any $a:A$ and $s_1,s_2:\acc(a)$, we have $s_1=s_2$, as desired.
\end{proof}

\begin{defn}
  A binary relation $<$ on a set $A$ is \textbf{well-founded} if every element of $A$ is accessible.
\end{defn}

\begin{lem}
  Well-foundedness is a mere property.
\end{lem}
\begin{proof}
  Well-foundedness of $<$ is the type $\prd{a:A} \acc(a)$, which is a mere proposition since each $\acc(a)$ is.
\end{proof}

\begin{eg}\label{thm:nat-wf}
  Perhaps the most familiar well-founded relation is the usual strict ordering on \nat.
  To show that this is well-founded, we must show that $n$ is accessible for each $n:\nat$.
  This is just the usual proof of ``strong induction'' from ordinary induction on \nat.

  Specifically, we prove by induction on $n:\nat$ that $k$ is accessible for all $k\le n$.
  The base case is just that $0$ is accessible, which is vacuously true since nothing is strictly less than $0$.
  For the inductive step, we assume that $k$ is accessible for all $k\le n$, which is to say for all $k<n+1$; hence by definition $n+1$ is also accessible.

  A different relation on \nat which is also well-founded is obtained by setting only $n < \suc(n)$ for all $n:\nat$.
  Well-foundedness of this relation is almost exactly the ordinary induction principle of \nat.
\end{eg}

\begin{eg}\label{thm:wtype-wf}
  Let $A:\set$ and $B : A \to \set$ be a family of sets.
  Recall from \autoref{sec:w-types} that the $W$-type $\wtype{a:A} B(a)$ is inductively generated by the single constructor
  \begin{itemize}
  \item $\supp : \prd{a:A} (B(a) \to \wtype{x:A} B(x)) \to \wtype{x:A} B(x)$
  \end{itemize}
  We define the relation $<$ on $\wtype{x:A} B(x)$ by recursion on its second argument:
  \begin{itemize}
  \item For any $a:A$ and $f:B(a) \to \wtype{x:A} B(x)$, we define $w<\supp(a,f)$ to mean that there merely exists a $b:B(a)$ such that $w = f(b)$.
  \end{itemize}
  Now we prove that every $w:\wtype{x:A} B(x)$ is accessible for this relation, using the usual induction principle for $\wtype{x:A}B(x)$.
  This means we assume given $a:A$ and $f:B(a) \to \wtype{x:A} B(x)$, and also a lifting $f' : \prd{b:B(a)} \acc(f(b))$.
  But then by definition of $<$, we have $\acc(w)$ for all $w<\supp(a,f)$; hence $\supp(a,f)$ is accessible.
\end{eg}

Well-foundedness allows us to define functions by recursion and prove statements by induction, such as for instance the following.

\begin{lem}\label{thm:wfrec}
  Suppose $B$ is a set and we have a function
  \[ g : \mathcal{P}B \to B \]
  Then if $<$ is a well-founded relation on $A$, there is a function $f:A\to B$ such that for all $a:A$ we have
  \begin{equation*}
    f(a) = g\Big(\setof{ f(a') | a'<a }\Big).
  \end{equation*}
\end{lem}
\begin{proof}
  We first define, for every $a:A$ and $s:\acc(a)$, an element $\bar f(a,s):B$.
  By induction, it suffices to assume that $s$ is a function assigning to each $a'<a$ a proof $s(a'):\acc(a')$, and that moreover for each such $a'$ we have an element $\bar f(a',s(a')):B$.
  In this case, we define
  \begin{equation*}
    \bar f(a,s) \defeq g\Big(\setof{ \bar f(a',s(a')) | a'<a }\Big).
  \end{equation*}

  Now since $<$ is well-founded, we have a function $w:\prd{a:A} \acc(a)$.
  Thus, we can define $f(a)\defeq \bar f (a,w(a))$.
\end{proof}

In classical logic, well-foundedness has a more well-known reformulation.

\begin{lem}\label{thm:wfmin}
  Assuming excluded middle, $<$ is well-founded if and only if every nonempty subset $B\subseteq A$ merely has a minimal element.
\end{lem}
\begin{proof}
  Suppose first $<$ is well-founded, and suppose $B\subseteq A$ is a subset with no minimal element.
  That is, for any $a:A$ with $a\in B$, there merely exists a $b:A$ with $b<a$ and $b\in B$.

  We claim that for any $a:A$ and $s:\acc(a)$, we have $a\notin B$.
  By induction, we may assume $s$ is a function assigning to each $a'<a$ a proof $s(a'):\acc(a)$, and that moreover for each such $a'$ we have $a'\notin B$.
  If $a\in B$, then by assumption, there would merely exist a $b<a$ with $b\in B$, which contradicts this assumption.
  Thus, $a\notin B$; this completes the induction.
  Since $<$ is well-founded, we have $a\notin B$ for all $a:A$, i.e. $B$ is empty.

  Now suppose each nonempty subset merely has a minimal element.
  Let $B = \setof{ a:A | \neg \acc(a) }$.
  Then if $B$ is nonempty, it merely has a minimal element.
  Thus there merely exists an $a:A$ with $a\in B$ such that for all $b<a$, we have $\acc(b)$.
  But then by definition (and induction on truncation), $a$ is merely accessible, and hence accessible, contradicting $a\in B$.
  Thus, $B$ is empty, so $<$ is well-founded.
\end{proof}

\begin{defn}
  A well-founded relation $<$ on a set $A$ is \textbf{extensional} if for any $a,b:A$, we have
  \[ \Big(\prd{c:A} (c<a) \leftrightarrow (c<b)\Big) \to (a=b). \]
\end{defn}

Note that since $A$ is a set, extensionality is a mere proposition.
This notion of ``extensionality'' is unrelated to function extensionality, and also unrelated to the extensionality of identity types.
Rather, it is a ``local'' counterpart of the axiom of extensionality in classical set theory.

\begin{thm}
  The type of extensional well-founded relations is a set.
\end{thm}
\begin{proof}
  By the univalence axiom, it suffices to show that if $(A,<)$ is extensional and well-founded and $f:(A,<) \cong (A,<)$, then $f=\idfunc[A]$.
  We prove by induction on $<$ that $f(a)=a$ for all $a:A$.
  The inductive hypothesis is that for all $a'<a$, we have $f(a')=a'$.

  Now since $A$ is extensional, to conclude $f(a)=a$ it is sufficient to show
  \[\prd{c:A}(c<f(a)) \leftrightarrow (c<a).\]
  However, since $f$ is an automorphism, we have $(c<a) \leftrightarrow (f(c)<f(a))$.
  But $c<a$ implies $f(c)=c$ by the induction hypothesis, so $(c<a) \to (c<f(a))$.
  On the other hand, if $c<f(a)$, then $f^{-1}(c)<a$, and so $c = f(f^{-1}(c)) = f^{-1}(c)$ by the induction hypothesis again; thus $c<a$.
  Therefore, we have $(c<a) \leftrightarrow (c<f(a))$ for any $c:A$, so $f(a)=a$.
\end{proof}

\begin{defn}\label{def:simulation}
  If $(A,<)$ and $(B,<)$ are extensional and well-founded, a \textbf{simulation} is a function $f:A\to B$ such that
  \begin{enumerate}
  \item if $a<a'$, then $f(a)<f(a')$, and\label{item:sim1}
  \item for all $a:A$ and $b:B$, if $b<f(a)$, then there merely exists an $a'<a$ with $f(a')=b$.\label{item:sim2}
  \end{enumerate}
\end{defn}

\begin{lem}
  Any simulation is injective.
\end{lem}
\begin{proof}
  We prove by double well-founded induction that for any $a,b:A$, if $f(a)=f(b)$ then $a=b$.
  The induction hypothesis for $a:A$ says that for any $a'<a$, and any $b:B$, if $f(a')=f(b)$ then $a=b$.
  The inner induction hypothesis for $b:A$ says that for any $b'<b$, if $f(a')=f(b')$ then $a'=b'$.

  Suppose $f(a)=f(b)$; we must show $a=b$.
  By extensionality, it suffices to show that for any $c:A$ we have $(c<a)\leftrightarrow (c<b)$.
  If $c<a$, then $f(c)<f(a)$ by \autoref{def:simulation}\ref{item:sim1}.
  Hence $f(c)<f(b)$, so by \autoref{def:simulation}\ref{item:sim2} there merely exists $c':A$ with $c'<b$ and $f(c)=f(c')$.
  By the induction hypothesis for $a$, we have $c=c'$, hence $c<b$.
  The dual argument is symmetrical.
\end{proof}

In particular, this implies that the word ``merely'' in \autoref{def:simulation}\ref{item:sim2} could be omitted without change of sense.

\begin{cor}
  If $f:A\to B$ is a simulation, then for all $a:A$ and $b:B$, if $b<f(a)$, there \emph{purely} exists an $a'<a$ with $f(a')=b$.
\end{cor}
\begin{proof}
  Since $f$ is injective, $\sm{a:A} (f(a)=b)$ is a mere proposition.
\end{proof}

We say that a subset $C :\mathcal{P}B$ is an \textbf{initial segment} if $c\in C$ and $b<c$ imply $b\in C$.
The image of a simulation must be an initial segment, while the inclusion of any initial segment is a simulation.
Thus, by univalence, every simulation $A\to B$ is \emph{equal} to the inclusion of some initial segment of $B$.

\begin{thm}
  For a set $A$, let $P(A)$ be the type of extensional well-founded relations on $A$.
  If $\mathord{<_A} : P(A)$ and $\mathord{<_B} : P(B)$ and $f:A\to B$, let $H_{\mathord{<_A}\mathord{<_B}}(f)$ be the mere proposition that $f$ is a simulation.
  Then $(P,H)$ is a standard notion of structure over \uset in the sense of \autoref{sec:sip}.
\end{thm}
\begin{proof}
  We leave it to the reader to verify that identities are simulations, and that composites of simulations are simulations.
  Thus, we have a notion of structure.
  For standardness, we must show that if $<$ and $\prec$ are two extensional well-founded relations on $A$, and $\idfunc[A]$ is a simulation in both directions, then $<$ and $\prec$ are equal.
  Since extensionality and well-foundedness are mere propositions, for this it suffices to have $\prd{a,b:A} (a<b) \leftrightarrow (a\prec b)$.
  But this follows from \autoref{def:simulation}\ref{item:sim1} for $\idfunc[A]$.
\end{proof}

\begin{cor}\label{thm:wfcat}
  There is a category whose objects are sets equipped with extensional well-founded relations, and whose morphisms are simulations.
\end{cor}

In fact, this category is a poset.

\begin{lem}
  For extensional and well-founded $(A,<)$ and $(B,<)$, there is at most one simulation $f:A\to B$.
\end{lem}
\begin{proof}
  Suppose $f,g:A\to B$ are simulations.
  Since being a simulation is a mere property, it suffices to show $\prd{a:A}(f(a)=g(a))$.
  By induction on $<$, we may suppose $f(a')=g(a')$ for all $a'<a$.
  And by extensionality of $B$, to have $f(a)=g(a)$ it suffices to have $\prd{b:B}(b<f(a)) \leftrightarrow (b<g(a))$.

  But since $f$ is a simulation, if $b<f(a)$, then we have $a'<a$ with $f(a')=b$.
  By the inductive hypothesis, we have also $g(a')=b$, hence $b<g(a)$.
  The dual argument is symmetrical.
\end{proof}

Thus, if $A$ and $B$ are equipped with extensional and well-founded relations, we may write $A\le B$ to mean there exists a simulation $f:A\to B$.
\autoref{thm:wfcat} implies that if $A\le B$ and $B\le A$, then $A=B$.

\begin{defn}
  An \textbf{ordinal} is a set $A$ with an extensional well-founded relation which is \emph{transitive}, i.e.\ $\prd{a,b,c:A}(a<b)\to (b<c) \to (a<c)$.
\end{defn}

\begin{eg}
  Of course, the usual strict order on \nat is transitive.
  It is easily seen to be extensional as well; thus it is an ordinal.
  As usual, we denote this ordinal by $\omega$.
\end{eg}

Let \ord denote the type of ordinals.
By the previous results, \ord is a set and has a natural partial order.
We now show that \ord also admits a well-founded relation.

If $A$ is an ordinal and $a:A$, let $\ordsl A a$ denote the initial segment $\setof{ b:A | b<a}$.
Note that if $\ordsl A a = \ordsl A b$ as ordinals, then that isomorphisms must respect their inclusions into $A$ (since simulations form a poset), and hence they are equal as subsets of $A$.
Therefore, since $A$ is extensional, $a=b$.
Thus the function $a\mapsto \ordsl A a$ is an injection $A\to \ord$.

\begin{defn}
  For ordinals $A$ and $B$, a simulation $f:A\to B$ is said to be \textbf{bounded} if there exists $b:B$ such that $A = \ordsl B b$.
\end{defn}

The remarks above imply that such a $b$ is unique when it exists, so that boundedness is a mere property.

We write $A<B$ if there exists a bounded simulation from $A$ to $B$.
Since simulations are unique, $A<B$ is also a mere proposition.

\begin{thm}\label{thm:ordord}
  $(\ord,<)$ is an ordinal.
\end{thm}

\begin{rmk}
  Note the use of universe polymorphism and typical ambiguity.
  If universe levels were made explicit, this theorem would state that the set of ordinals in one universe is an ordinal in the next higher universe.
\end{rmk}

\begin{proof}
  Let $A$ be an ordinal; we first show that $\ordsl A a$ is accessible (in \ord) for all $a:A$.
  By induction, suppose $\ordsl A b$ is accessible for all $b:A$.
  By definition of accessibility, we must show that $B$ is accessible in \ord for all $B<\ordsl A a$.
  However, if $B<\ordsl A a$ then there is some $b<a$ such that $B = \ordsl{(\ordsl A a)}{b} = \ordsl A b$, which is accessible by the inductive hypothesis.
  Thus, $\ordsl A a$ is accessible for all $a:A$.

  Now to show that $A$ is accessible in \ord, by definition we must show $B$ is accessible for all $B<A$.
  But as before, $B<A$ means $B=\ordsl A a$ for some $a:A$, which is accessible as we just proved.
  Thus, \ord is well-founded.

  For extensionality, suppose $A$ and $B$ are ordinals such that $\prd{C:\ord} (C<A) \leftrightarrow (C<B)$.
  Then for every $a:A$, since $\ordsl A a<A$, we have $\ordsl A a<B$, hence there is $b:B$ with $\ordsl A a = \ordsl B b$.
  Define $f:A\to B$ to take each $a$ to the corresponding $b$; it is straightforward to verify that $f$ is an isomorphism.
  Thus $A\cong B$, hence $A=B$ by univalence.

  Finally, it is easy to see that $<$ is transitive.
\end{proof}

Treating \ord as an ordinal is often very convenient, but it has its pitfalls as well.
For instance, consider the following lemma, for whose statement we drop briefly into \emph{explicit} universe polymorphism.

\begin{lem}\label{thm:ordsucc}
  Let \bbU be a universe.
  For any $A:\ord_\bbU$, there is a $B:\ord_\bbU$ such that $A<B$.
\end{lem}
\begin{proof}
  Let $B=A+\unit$, with the element $\star:\unit$ being greater than all elements of $A$.
  Then $B$ is an ordinal and it is easy to see that $A\cong \ordsl B \star$.
\end{proof}

This lemma illustrates a potential pitfall of typical ambiguity.
Consider the following alternative proof of it.

\begin{proof}[Another putative proof of \autoref{thm:ordsucc}]
  Note that $C<A$ if and only if $C=\ordsl A a$ for some $a:A$.
  This gives an isomorphism $A \cong \ordsl \ord A$, so that $A<\ord$.
  Thus we may take $B\defeq\ord$.
\end{proof}

The second proof would be valid if we had stated \autoref{thm:ordsucc} in a typically ambiguous style.
But the resulting lemma would be less useful, because the second proof would constrain the second ``\ord'' in the lemma statement to refer to a higher universe level than the first one.
The first proof allows both universes to be the same.

Similar remarks apply to the next lemma, which could be proved in a less useful way by observing that $A\le \ord$ for any $A:\ord$.

\begin{lem}\label{thm:ordunion}
  Let \bbU be a universe.
  For any $X:\type_\bbU$ and $F:X\to \ord_\bbU$, there exists $B:\ord_\bbU$ such that $Fx\le B$ for all $x:X$.
\end{lem}
\begin{proof}
  Let $B$ be the quotient of the equivalence relation on $\sm{x:X} Fx$ defined as follows:
  \[ (x,y) \sim (x',y')
  \;\defeq\;
  \Big(\ordsl{(Fx)}{y} \cong \ordsl{(Fx')}{y'}\Big).
  \]
  Define $(x,y)<(x',y')$ if $\ordsl{(Fx)}{y} < \ordsl{(Fx')}{y'}$.
  This clearly descends to the quotient, and can be seen to make $B$ into an ordinal.
  Moreover, for each $x:X$ the induced map $Fx\to B$ is a simulation.
\end{proof}



\section{Classical well-orderings}
\label{sec:wellorderings}

We now show the equivalence of our ordinals with the more familiar classical well-orderings.

\begin{lem}
  Assuming excluded middle, every ordinal is trichotomous:
  \[ \prd{a,b:A} (a<b) \vee (a=b) \vee (b<a). \]
\end{lem}
\begin{proof}
  By induction on $a$, we may assume that for every $a'<a$ and every $b':A$, we have $(a'<b') \vee (a'=b') \vee (b'<a')$.
  Now by induction on $b$, we may assume that for every $b'<b$, we have $(a<b') \vee (a=b') \vee (b'<a)$.

  By excluded middle, either there merely exists a $b'<b$ such that $a<b'$, or there merely exists a $b'<b$ such that $a=b'$, or for every $b'<b$ we have $b'<a$.
  In the first case, merely $a<b$ by transitivity, hence $a<b$ as it is a mere proposition.
  Similarly, in the second case, $a<b$ by transport.
  Thus, suppose $\prd{b':A}(b'<b)\to (b'<a)$.

  Now analogously, either there merely exists $a'<a$ such that $b<a'$, or there merely exists $a'<a$ such that $a'=b$, or for every $a'<a$ we have $a'<b$.
  In the first and second cases, $b<a$, so we may suppose $\prd{a':A}(a'<a)\to (a'<b)$.
  However, by extensionality, our two suppositions now imply $a=b$.
\end{proof}

\begin{lem}
  A well-founded relation contains no cycles, i.e.\
  \[ \prd{n:\mathbb{N}}{a:\mathbb{N}_n\to A} \neg\Big((a_0<a_1) \wedge \dots \wedge (a_{n-1}<a_n)\wedge (a_n<a_0)\Big). \]
\end{lem}
\begin{proof}
  We prove by induction on $a:A$ that there is no cycle containing $a$.
  Thus, suppose by induction that for all $a'<a$, there is no cycle containing $a'$.
  But in any cycle containing $a$, there is some element less than $a$ and contained in the same cycle.
\end{proof}

\begin{thm}\label{thm:wellorder}
  Assuming excluded middle, $(A,<)$ is an ordinal if and only if every nonempty subset $B\subseteq A$ has a least element.
\end{thm}
\begin{proof}
  If $A$ is an ordinal, then by \autoref{thm:wfmin} every nonempty subset merely has a minimal element.
  But trichotomy implies that any minimal element is a least element.
  Moreover, least elements are unique when they exist, so merely having one is as good as having one.

  Conversely, if every nonempty subset has a least element, then $A$ is well-founded by \autoref{thm:wfmin}.
  We also have trichotomy, since for any $a,b$ the set $\setof{a,b}$ merely has a least element, which must be either $a$ or $b$.
  This implies transitivity, since if $a<b$ and $a<c$, then either $a=c$ or $c<a$ would produce a cycle.
  Similarly, it implies extensionality, for if $\prd{c:A}(c<a)\leftrightarrow (c<b)$, then $a<b$ implies (letting $c$ be $a$) that $a<a$, which is a cycle, and similarly if $b<a$; hence $a=b$.
\end{proof}

In classical mathematics, the characterization of \autoref{thm:wellorder} is taken as the definition of a \textbf{well-ordering}, with the \emph{ordinals} being a canonical set of representatives of isomorphism classes for well-orderings.
In our context, the Structure Identity Principle means that there is no need to look for such representatives: any well-ordering is as good as any other.

We now move on to consider consequences of the axiom of choice.

\begin{thm}\label{thm:wop}
  Assuming excluded middle, the following are equivalent.
  \begin{enumerate}
  \item For every set $X$, there merely exists a function
    $ f: \mathcal{P}_+X \to X $
    such that $f(Y)\in Y$ for all $Y:\mathcal{P}X$.\label{item:wop1}
  \item Every set merely admits the structure of an ordinal.\label{item:wop2}
  \end{enumerate}
\end{thm}

Of course,~\ref{item:wop1} is a standard classical version of the axiom of choice.

\begin{proof}
  One direction is easy: suppose~\ref{item:wop2}.
  Since we aim to prove the mere proposition~\ref{item:wop1}, we may assume $A$ is an ordinal.
  But then we can define $f(B)$ to be the least element of $B$.

  Now suppose~\ref{item:wop1}.
  As before, since~\ref{item:wop2} is a mere proposition, we may assume given such an $f$.
  We extend $f$ to a function
  \[ \bar f:\mathcal{P}X \cong (\mathcal{P}_+ X) + \unit \longrightarrow X+\unit
  \]
  in the obvious way.
  Now for any ordinal $A$, we can define $g_A:A\to X+\unit$ by well-founded recursion:
  \[ g_A(a) \defeq 
    \bar f\Big(X \setminus \setof{ g_A(b) | \rule{0pt}{1em} (b<a) \wedge (g_A(b) \in X) }\Big)
  \]
  (regarding $X$ as a subset of $X+\unit$ in the obvious way).

  Let $A'$ be the preimage of $X$; then we claim the restriction $g_A':A' \to X$ is injective.
  For if $a,a':A$ with $a\neq a'$, then by trichotomy and without loss of generality, we may assume $a'<a$.
  Thus $g_A(a') \in \setof{ g_A(b) | b<a }$, so since $f(Y)\in Y$ for all $Y$ we have $g_A(a) \neq g_A(a')$.

  Moreover, $A'$ is an initial segment of $A$.
  For $g_A(a)$ lies in \unit if and only if $\setof{g_A(b)|b<a} = X$, and if this holds then it also holds for any $a'>a$.
  Thus, $A'$ is itself an ordinal.

  Finally, since \ord is an ordinal, we can take $A\defeq\ord$.
  Let $X'$ be the image of $g_\ord':\ord' \to X$; then the inverse of $g_\ord'$ yields an injection $H:X'\to \ord$.
  By \autoref{thm:ordunion}, there is an ordinal $C$ such that $Hx\le C$ for all $x:X'$.
  Then by \autoref{thm:ordsucc}, there is a further ordinal $D$ such that $C<D$, hence $Hx<D$ for all $x:X'$.
  Now we have
  \begin{align*}
    g_{\ord}(D) &= \bar f\Big( X \setminus \setof{ g_\ord(B) | \rule{0pt}{1em} B<D \wedge (g_\ord(B) \in X)} \Big)\\
    &=\bar f\Big( X \setminus \setof{ g_\ord(B) | \rule{0pt}{1em} B:\ord \wedge (g_\ord(B) \in X)} \Big)
  \end{align*}
  since if $B:\ord$ and $(g_\ord(B) \in X)$, then $B = Hx$ for some $x:X'$, hence $B<D$.
  Now if
  \[\setof{ g_\ord(B) | \rule{0pt}{1em} B:\ord \wedge (g_\ord(B) \in X)}\]
  is not all of $X$, then $g_\ord(D)$ would lie in $X$ but not in this subset, which would be a contradiction since $D$ is itself a potential value for $B$.
  So this set must be all of $X$, and hence $g_\ord'$ is surjective as well as injective.
  Thus, we can transport the ordinal structure on $\ord'$ to $X$.
\end{proof}

\begin{rmk}
  If we had given the wrong proof of \autoref{thm:ordsucc} or \autoref{thm:ordunion}, then the resulting proof of \autoref{thm:wop} would be invalid: there would be no way to consistently assign universe levels.
\end{rmk}

\begin{cor}
  Assuming the axiom of choice, the function $\ord\to\set$ (which forgets the order structure) is a surjection.
\end{cor}

Note that \ord is a set, while \set is a 1-type.
In general, there is no reason for a 1-type to admit any surjective function from a set.
Even the axiom of choice does not appear to imply that \emph{every} 1-type does so, but it readily implies that this is so for 1-types constructed out of \set, such as the types of objects of categories of structures as in \autoref{sec:sip}.
The following corollary also applies to such categories.

\begin{cor}
  Assuming AC, \uset admits a weak equivalence functor from a strict category.
\end{cor}
\begin{proof}
  Let $X_0\defeq \ord$, and for $A,B:X_0$ let $\hom_X(A,B) \defeq (A\to B)$.
  Then $X$ is a strict category, since \ord is a set, and the above surjection $X_0 \to \set$ extends to a weak equivalence functor $X\to \uset$.
\end{proof}

Now recall from \autoref{sec:cardinals} that we have a further surjection $\cd{-}:\set\to\card$, and hence a composite surjection $\ord\to\card$ which sends each ordinal to its cardinality.

\begin{thm}
  Assuming AC, the surjection $\ord\to\card$ has a section.
\end{thm}
\begin{proof}
  There is an easy and wrong proof of this: since \ord and \card are both sets, AC implies that any surjection between them \emph{merely} has a section.
  However, we actually have a canonical \emph{specified} section: because \ord is an ordinal, every nonempty subset of it has a uniquely specified least element.
  Thus, we can map each cardinal to the least element in the corresponding fiber.
\end{proof}

It is traditional in set theory to identify cardinals with their image in \ord: the least ordinal having that cardinality.

It follows that \card also canonically admits the structure of an ordinal: in fact, one isomorphic to \ord.
Specifically, we define by well-founded recursion a function $\aleph:\ord\to\ord$, such that $\aleph(A)$ is the least ordinal having cardinality greater than $\aleph({\ordsl A a})$ for all $a:A$.
Then (assuming AC) the image of $\aleph$ is exactly the image of \card.

\sectionNotes

The treatment of the category of sets in \autoref{sec:piw-pretopos} roughly follows that in~\cite{RijkeSpitters}.
The notion of a predicative topos, as a $\Pi$W pretopos, was introduced in \cite{MoerdijkPalmgren}.

The implication in \autoref{sec:surj-split} from $\mathsf{AC}$ to $\mathsf{LEM}$ is an adaptation to homotopy type theory of a theorem from topos theory due to Diaconescu~\cite{Diaconescu}; it was posed as a problem already by Bishop~\cite[Problem 2,p58]{Bishop1967}.

Voevodsky~\cite{Universe-poly} has proposed resizing rules of the kind considered in \autoref{subsec:piw}.

The idea of algebraic set theory, which informs our development in \autoref{sec:ast} of the cumulative hierarchy, is due to \cite{JoyalMoerdijk}, but it derives from earlier work by \cite{AczelCZF}.

\dots


\sectionExercises

\begin{ex}
  Show that if every surjection has a section in the category $\set$, then $\mathsf{AC}$ holds.
\end{ex}

\begin{ex}
  Prove that if $(A,<_A)$ and $(B,<_B)$ are well-founded, extensional, or ordinals, then so is $A+B$, with $<$ defined by
  \begin{alignat*}{2}
    (a<a') &\defeq (a<_A a') &\qquad\text{for }& a,a':A\\
    (b<b') &\defeq (b<_B b') &\qquad\text{for }& b,b':B\\
    (a<b) &\defeq \unit &\qquad\text{for }& (a:A),(b:B)\\
    (b<a) &\defeq \emptyset &\qquad\text{for } &(a:A),(b:B)
  \end{alignat*}
\end{ex}
% \begin{proof}
%   We first prove by induction on $<_A$ that every element of $A$ is accessible in $A+B$.
%   This is easy since the only elements less than $a:A$ in $A+B$ are also in $A$.
%   We then prove by induction on $<_B$ that every element of $B$ is accessible in $A+B$.
%   This is easy since we have already proven that every element of $A$ is accessible.
% \end{proof}

\begin{ex}
  Prove that if $(A,<_A)$ and $(B,<_B)$ are well-founded, extensional, or ordinals, then so is $A\times B$, with $<$ defined by
  \[ ((a,b) <(a',b')) \defeq (a<_A a') \vee ((a=a') \wedge (b<_B b')). \]
\end{ex}
% \begin{proof}
%   We prove by induction on $<_A$ that for every $a:A$, every element of the form $(a,b)$ is accessible in $A\times B$.
%   The induction hypothesis is that for all $a'<_A a$, every pair $(a',b)$ is accessible.
%   Inside this induction, we prove by induction on $<_B$ that for every $b:B$, the element $(a,b)$ is accessible.
%   The nested induction hypothesis is that for every $b'<_B b$, the element $(a,b')$ is accessible.
%   But now, if $(a',b')< (a,b)$, then either $a<_A a'$ in which case $(a',b')$ is accessible by the first induction hypothesis, or $a=a'$ and $b'<_B b$, in which case $(a,b')$ is accessible by the second induction hypothesis.
%   Thus, by definition of accessibility, $(a,b)$ is accessible.
%   This completes both inductions.
% \end{proof}

\begin{ex}
  Define the usual algebraic operations on ordinals, and prove that they satisfy the usual properties.
\end{ex}


% Local Variables:
% TeX-master: "main"
% End:


% Local Variables:
% TeX-master: "main"
% End:
