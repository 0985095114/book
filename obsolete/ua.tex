% !Tex root = note.tex

\section{Introduction}
{\it This is an informal presentation of a proof of function extensionality in a univalent universe.  It is fairly self-contained for those readers who are familiar with the propositions as types treatment of logic in Per Martin-L\"{o}f's `intensional' dependent type theory.  In fact I give two proofs of function extensionality, the second one is in an appendix and I believe is essentially a presentation of the original Coq proof by Vladimir Voevodsky.  The first proof seems to me to be a simpler proof and is essentially a proof I found in a note of Carlo Angiuli.  I do not know where it originated.  In each proof there is a main lemma concerning composition of functions that comes in two versions, one for each proof of the theorem.

Because I have tried to give a self-contained presentation there is a lot of preliminary material that might be better placed in separate chapter(s).  There is one key result, needed for the second proof, that I do not prove.
This is $2\ra 1$ of \autoref{prop:6.3}; i.e. every isomorphism is an 
equivalence, which is used in the proof of \autoref{lem:6.7}.

I have chosen to give a low-level presentation, where the reason for most 
inference steps is explicitly given.  So there should be little doubt in the readers mind that the mathematics is rigorous.  Also it should be fairly routine to implement the informal mathematics in Coq.  With more experience of informal HoTT it should be possible to present rigorous HoTT at a higher level that would require more work from the reader.

I have also chosen to minimize the reference to ideas and terminology from homotopy theory so that the mathematics can mostly be taken to be a piece of constructive mathematics done in dependent type theory using propositions as types.  Of course the notion of a univalent universe comes from homotopy theory.  The definition that a universe is univalent states that, for types $A,A'$ in the universe, a certain function $E_{A,A'} :(A=A')\ra (A\simeq A')$ is an equivalence.  As a consequence this function is surjective.  It may be worth noting the following fact. The proof that a univalent universe has function extensionality only uses the surjectivity of the function.
}

\section{The Notion of a Univalent Universe}
\begin{defn} A type $A$ is {\em contractible} if there is $a:A$, called the {\em center of contraction} such that $a=x$ for all $x:A$.
\end{defn}
\begin{defn} A map $f:A\ra B$ is an {\em equivalence}
if, for $y:B$, its {\em fiber}, $\{x:A\mid fx = y\}$, is contractible.
We write $A\simeq B$ if  there is an equivalence $A\ra B$.
\end{defn}

\begin{lem}\label{lem:2.3}  For each type $A$, the identity map 
$1_A:= \lam{x:A}x :A\ra A$ is an equivalence.
\end{lem}
\begin{proof}  Let $a:A$ and let $\{ a\}_A:= \{ x:A\mid x=a\}$ be its fiber with respect to $1_A$.  We show that $\{ a\}_A$ is contractible.  Then, discharging $a:A$ we get that $\forall_{x:A}(\{ x\}_A$ is contractible); i.e. $1_A$ is an equivalence.

Let $\oa =(a,\refl{a}):\{ a\}_A$.  As $(a,\refl{a}) = \oa$ we can use Id-induction \`{a} la Christine-Paulin to get
  \[\forall_{x:A}\forall_{z:(x=a)}\; (x,z)=\oa.\]
Hence, by $\Sigma$-folding, $\forall_{u:\{ a\}_A}\; u=\oa$ so that $\{ a\}_A$ is contractible, as desired.
\end{proof}

\begin{cor}\label{ua:idtoeqv}
If $\bbU$ is a type universe then
\begin{equation}\label{eq:idtoeqv}
  \forall_{X,Y:\bbU}\; [X=Y\;\ra\; X\simeq Y].
\end{equation}
\end{cor}
\begin{proof}
By the lemma, $\forall_{X:\bbU}\; X\simeq X$.  Hence~\eqref{eq:idtoeqv}, by Id-induction on $\bbU$.
\end{proof}

\begin{defn}
A type universe $\bbU$ is {\em univalent} if the function
  \[ E:\forall_{X,Y:\bbU}\; [X=Y\;\ra\; X\simeq Y],\] 
constructed in the proof of~\eqref{eq:idtoeqv}, is such that $E_{X,Y}$ is an an equivalence for all $X,Y:\bbU$.
\end{defn}
%\pagebreak


\begin{notes}[Identity] 
We use the Propositions-as-Types (PaT) interpretation of logic.  So each proposition is a type and the standard logical operations on propositions are the corresonding type forming operations.  In particular we have the following, where we assume that $A$ is a type in some (implicit) context.
\begin{enumerate}
\item For $a,a':A$ the proposition 
  \[ a=a'\] 
is the identity type $Id_A(a,a')$.  It has the introduction rule 
  \[\refl{a}:(a=a) \mbox{ for each $a:A$},\]
which gives us that $=$ is reflexive.  We get the symmetry and transitivity of $=$ by using Id-induction.  So we get, for $a,a':A$
  \[ z^{-1}:a=a'\mbox{ for } z:a=a',\]
and, for $a,a',a'':A$
  \[ z@z':a=a''\mbox{ for $z:a=a',z':a'=a''$}.\]

\item If $P[x,y,z]$ is a proposition for $x,y:A,z:(x=y)$ then proof by Id-induction on
 $x,y:A,z:(x=y)$ is given by the following rule.
\begin{quote}
From $\forall_{x:A}\; P[x,x,\refl{a}x]$ infer $\forall_{x,y:A}\forall_{z:(x=y)}P[x,y,z]$.  
\end{quote}
This is the standard rule.  Sometimes the following version due to Christine Paulin is useful when $a:A$ and $P[x,z]$ is a proposition for $x:A,z:(x=a)$.
\begin{quote}
From $P[a,\refl{a}]$ infer $\forall_{x:A}\forall_{z:(x=a)}P[x,z]$.
\end{quote}

\item The following {\em transport rule} is an easy exercise.
If $P[x]$ is a type for $x:A$ then there is $z^*_P:P[x]\ra P[x']$ for $z:x=x'$ such that
  \[ (\refl{x})^*_P = 1_{P[x]}\] 
for $x:A$.  Moreover $z^*_P$ is an equivalence for $z:x=x'$.

Note that we have left implicit $x,x':A$ as, when writing $z:x=x'$, the type of $x$ and $x'$ can usually be recovered when needed.
\end{enumerate}
\end{notes}
%\pagebreak

\begin{notes}[ $\Sigma$ types, Contractible types and Equivalences]
$\;$
\begin{enumerate}
\item Let $B[x]$ be a type for $x:A$ and let $C:=\Sigma_{x:A}B[x]$.  $C$ is the type of pairs $(x,y)$ such that $x:A$ and $y:B[x]$.
Note that we use $(x,y)$ rather than ${\sf pair}(x,y)$.

If $P[u]$ is a proposition for $u: C$ then $\Sigma$-folding is the following rule, with a similar rule for the existential quantifier.
\begin{quote}
From $\forall_{x:A}\forall_{y:B[x]}\; P[{(x,y)}]$ infer $\forall_{u\in C[u]}P[u]$.
\end{quote}
Note that $\Sigma$-unfolding is the converses of the rules for the two quantifiers.
\item For $y:A$, $\{ y\}_A$ is the type $\Sigma_{x:A}(x=y)$.

\item The proposition 
  \[ A\mbox{ is contractible }\]
is $\exists_{x:A}\forall_{y:A}\; x=y$; i.e. the type $\Sigma_{x:A}\Pi_{y:A}\; x=y$.
\item $\{ x:A\mid fx =y\}$ is the type $\Sigma_{x:A}(fx = y)$.  Note that we use $fx$ rather than ${\sf app}(f,x)$.
\item For $f:A\ra B$, the proposition
  \[ f\mbox{ is an equivalence }\]
is $\forall_{y:A} (\{ x:A\mid fx=y\}\mbox{ is contractible})$; i.e. the type\\ 
  \[ \Pi_{y:A}\; (\{ x:A\mid fx=y\}\mbox{ is contractible}).\]



\end{enumerate}
\end{notes}
%\pagebreak

%%%%%%%%%%%%%%%%%%%%%%%%%%%%%
\section{Two more Equivalence Relations between Types}
%%%%%%%%%%%%%%%%%%%%%%%%%%%%%

We have the notion of an equivalence function $f:A\ra B$ giving rise to the equivalence relation $\simeq$ between types.  We define three further notions of function which give rise to three further equivalence relations between types.  We show that each of the three new notions of function are logically equivalent to the notion of an equivalence function so that all four equivalence relations are logically equivalent.  The first two new notions of function are the notions of bijection and isomorphism and are dealt with next.  later, in the Appendix, we will define yet one more new notion, that of an adjoint isomorphism.

%%%%%%%%%%%%%%%%%%%%%%%%%%%%%%%%%%%%%%%%%%%%
\subsection{Bijections and Isomorphisms}
%%%%%%%%%%%%%%%%%%%%%%%%%%%%%%%%%%%%%%%%%%%%

\begin{defn} 
Let $f:A\ra B$.
\begin{enumerate}
\item $f$ is a {\em bijection} if it is both injective and surjective, where
\begin{itemize}
\item $f$ is {\em injective} if $\forall_{x,x':A}\; \; fx=fx'\;\ra\; x=x'$,
\item $f$ is {\em surjective} if $\forall_{y:B}\exists_{x:A}\;\; fx=y$,
\end{itemize}
\item $f$ is {\em an isomorphism} if there is $g:B\ra A$ such that 
  \[\forall_{x:A}\; g(fx)=x\;\;\wedge\;\;\forall_{y:B}\; f(gy)=y.\]
\end{enumerate}
\end{defn}

\begin{thm} $\;$
\begin{enumerate}
\item $1_A:A\ra A$ is an isomorphism.
\item If $f:A\ra B$ and $g:B\ra C$ are isomorphisms then so is $g\circ f:A\ra C$.
\end{enumerate}
\end{thm}
\begin{proof} Routine
\end{proof}
\begin{thm}\label{prop:3.3}
The following are logically equivalent for $f:A\ra B$.
\begin{enumerate}
\item $f$ is a bijection.
\item $\forall_{y:B}\exists_{x:A}\;\; (fx=y\;\wedge \forall_{x':A}\;\; (fx'=y\;\ra\; x=x'))$.
\item $f$ is an isomorphism.
\end{enumerate}
\end{thm}
\begin{proof}
The proof of this proposition in type theory can be done by proving the implications $1\ra 2\ra 3\ra 1$.  $1\ra 2$ and $3\ra 1$ are the same as in set theory.  The proof that $2\ra 3$ requires, given $2$, the proof of the existence of a function $g$ inverse to $f$.  In set theory this uses the fact that functions are defined to be total single-valued relations.  Instead, in type theory the proof of the existence of $g$
uses the non-dependent version of the type theoretic axiom of choice.  This axiom holds in the propositions as types interpretation of logic because of the strong form of the existential quantifier.
\end{proof}
\begin{quote}
{\bf \large The type theoretic axiom of choice: }
We first state the general dependent form of the axiom.  Let $B[x]$ be a type for $x:A$ and let $C[x,y]$ be a proposition for $x:A,y:B[x]$.  Then 
  \[ \forall_{x:A}\exists_{y:B[x]}\; C[x,y]\;\;\ra\;\; \exists_{f:\Pi_{x:A}B[x]}\forall_{x:A}\; C[x,fx].\]
The non-dependent version is when $B[x]$ does not depend on $x$.

If $C[x,y]$ is a proposition for $x:A, y:B$, where $A,B$ are types then
 \[ \forall_{x:A}\exists_{y:B}\; C[x,y]\;\;\ra\;\; \exists_{f:A\ra B}\forall_{x:A}\; C[x,fx].\]
\end{quote}

\begin{thm}\label{prop:3.4}
Each equivalence is an isomorphism.
\end{thm}
\begin{proof} Let $f:A\ra B$ be an equivalence.  So each type $Cy$ is contractible, where for $y:B$, $Cy := \Sigma_{x:A}\; fx=y$.  This means that
  \[ \forall_{y:B}\exists_{u:Cy}\forall_{u':Cy}\; u'=u.\]
By $\Sigma$-unfolding we get
  \[ \forall_{y:B}\exists_{x:A}\exists_{z:(fx=y)}
                            \forall_{x':A}\forall_{z':(fx'=y)}\; (x,z)=(x',z').
  \]
It follows that
  \[ \forall_{y:B}\exists_{x:A}[fx=y\wedge\forall_{x':A}(fx'=y\ra  x=x')].
  \]
Hence, by $2\ra 3$ of \autoref{prop:3.3} , $f$ is an isomorphism.
\end{proof} 

\begin{cor}\label{cor:3.5}
If $A,A'$ are elements of a univalent universe and $E_{A,A'}$ is the equivalence
$EAA':(A=A')\ra (A\simeq A')$ then $E_{A,A'}$ is a a surjection.
\end{cor}
\begin{proof}
  By \autoref{prop:3.4}, $E_{A,A'}$ is an isomorphism and hence is surjective.
\end{proof}




% Local Variables:
% TeX-master: "main"
% End:
