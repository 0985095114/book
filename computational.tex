%%FIXME: what to say about resp

\chapter{Programming in HoTT}

One of the central strengths of type theory is that it is simultaneously
a formalism for doing math and a programming language.  While we don't
understand the programming interpretation of HoTT in full generality
yet, it's important to realize how it's different than the standard
operational interpretation of type theory.

\newcommand\ua[1]{\ensuremath{\mathsf{ua}} \: #1}

For example, consider the term 
\[
\transport{X \mapsto X}{(\ua{f})} : A \to B
\]
where $f : \eqv{A}{B}$ and 
$\mathsf{ua} : (\eqv A B) \to \id A B$ follows from the univalence axiom.  
Consider closed types $A$ and $B$ and a closed equivalence that is
not the identity; for example, take $A$ and $B$ to be $\mathsf{bool}$ and
$f$ to be the negation function.  Then the above term is \emph{stuck}:
there's no way to run it as a program.  The reason is that the only
computation step for $\mathsf{transport}$ applies when $f$ is the
identity: $\transport{X \mapsto A(X)}{\refl{}} \jdeq \idfunc$.  

While we do not yet understand the operational semantics for all of
HoTT, what we understand about low dimensions provides some useful tools
for doing proofs:

\begin{itemize}

\item Types have the structure of groupoids: for each type $A$, the
  type \id[A]{x}{y} should often be thought of as a \emph{defined notion}: it
  will usually turn out to be equivalent to some other type in theory.  
  
  Moreover, the operations 
  \[
  \begin{array}{l}
    \refl{x} : \id[A]{x}{x} \\
    \opp{-} : \id[A]{x}{y} \to \id[A]{y}{x} \\
    - \ct - : \id[A]{x}{y} \to \id[A]{y}{z} \to \id[A]{x}{z} \\ 
  \end{array}
  \]
  should be thought of as defined operations, that reduce depending on
  the type $A$ and the terms being equated.  

\item Families are functors: given $p : \id[A]{x}{y}$ and a dependent type
  $C : A \to \type$,
  \[
    \transport{C}{p} : C(x) \to C(y)
  \]
  should be thought of as a defined operation, which computes depending
  on the structure of $C$.  

  Similarly, given function $f : A \to B$
  \[
    \mapdep{f}{p} : \id[C y]{\transport{C}{p} \: (f x)}{f y}
  \]
  should be thought of as a defined operation, which computes for each
  $f$.  

\end{itemize}

In full generality, types are $\infty$-groupoids, and similar ideas will
apply to the higher-dimensional structure (e.g., the associator, the
pentagon for the associator, etc.).  But these low-dimensional
operations are already very useful for proofs: expanding these
``definitions'' is often a good way to proceed.  At present, the
relevant equations are not taken as definitional equalities (though this
could change in future versions of HoTT), so we will need to provide
proof terms for them.  

Let's go through some type constructors and see what their groupoid
structure looks like.

\section{Type-Specific Rules}

\subsection{Product Types}

Given types $A$ and $B$, consider the product type $A \times B$.  
A path in $A \times B$ is a pair of componentwise paths:

\[
\eqv{(\id[A \times B] p q)}{(\id[A]{\proj{1} p}{\proj{1} q}) \times (\id[B]{\proj{1} p}{\proj{1} q})}
\]

Read logically, this says that two pairs are equal if they are equal
componentwise.  Read category-theoretically, this says that the
morphisms in a product category are pairs of morphisms.  Read
homotopy-theoretically, this says that the paths in a direct product
space are pairs of paths.  

From a programming perspective, it's useful to unpack the above
equivalence into 

\newcommand{\pairpath}{\mathsf{pair}^{\mathord{=}}}
\newcommand{\projpath}[1]{\proj{#1}^{\mathord{=}}}

\begin{itemize}
\item An introduction rule for \id[A \times B]{p}{q}:
  \[
  \pairpath : (\id{\proj{1} p}{\proj{1} q}) \times (\id{\proj{1} p}{\proj{1} q}) \to {(\id p q)}
  \]
\item Elimination rules:
  \[
  \projpath{i} : \id{p}{q} \to (\id{\proj{i} p}{\proj{i} q})
  \]
\item $\beta$-reduction:
  \[
  \begin{array}{l}
  \id[(\id{\proj{1} p}{\proj{1} q})]
       {\projpath{1}{(\pairpath \: \alpha \: \beta})}
       {\alpha} \\
  \id[(\id{\proj{2} p}{\proj{2} q})]
       {\projpath{2}{(\pairpath \: \alpha \: \beta})}
       {\beta}
  \end{array}
  \]
\item $\eta$-equivalence: For any $\alpha : \id[A \times B] p q$
  \[
  \id{\alpha}{\pairpath \: (\projpath{1} \alpha) \: (\projpath{2} \alpha) }
  \]
\end{itemize}
All of these equations can be derived by using path induction on the given
paths and then returning reflexivity.  

Moreover, reflexivity, inverses, and composition are defined
componentwise:
\[
\begin{array}{l}
\id {\refl{(p : A \times B)}} {\pairpath \: (\refl{\proj{1} p}) \: (\refl{\proj{2} p})} \\
\id {\opp{\alpha}} {\pairpath \: (\opp{\projpath{1} \alpha}) \: (\opp{\projpath{2} \alpha})}  \\
\id {{\alpha \ct \beta}} {\pairpath \: 
  ({\projpath{1} \alpha} \ct
   {\projpath{1} \beta})
   \: 
   ({\projpath{2} \alpha} \ct {\projpath{2} \beta})}
\end{array}
\]

Given $ A, B : A_0 \to \type$ and $\alpha : \id[A_0]{x}{y}$ 
and $p : A(x) \times B(x)$

\[
\id[A(y) \times B(y)]{\transport{x_0:A_0 \mapsto A(x_0) \times B(x_0)}
  \: \alpha \: {p}}
   {(\transport{x_0 \mapsto A(x_0)} \alpha \: (\proj{1} p), 
       \transport{x_0 \mapsto B(x_0)} \alpha \: (\proj{2} p))}
\]

Similarly, 

(equations for map?)

\subsection{$\Sigma$-types}

With a little more work, this generalizes to $\Sigma$-types. 

Material from Egbert?

\subsection{$\Pi$-types}

Given $A$ and $B : A \to \type$, consider the dependent function type $\prd{x:A}B(x)$.
A path in $\prd{x:A} B(x)$ is given by a homotopy:  

\[
\eqv{(\id[\prd{x:A} B(x)]{f}{g})}{\prd{x:A} \id[B(x)]{(f x)}{(g x)}}
\]

Again it is useful to break this into 

\begin{itemize}
\item An introduction rule for {(\id[\prd{x:A} B(x)]{f}{g})}, function extensionality
  \[
  \funext : (\prd{x:A} \id[B(x)]{(f x)}{(g x)}) \to {(\id[\prd{x:A} B(x)]{f}{g})}
  \]
\item An elimination rule: for all $f,g : \prd{x:A} B(x)$
  \[
  \happly{-}{-} : \id{f}{g} \to \prd{\alpha : \id{x}{y}} \id{\transport{B}{\alpha} \: {(f x)}}{(g y)}
  \]
\item $\beta$-reduction: 
  \[
  \begin{array}{l}
  \id{\happly \: {\funext{(x \mapsto \alpha(x))}} \: {\refl{a}}}{\alpha(a)}
  \end{array}
  \]
\item $\eta$-equivalence: For any $\alpha : \id[\prd{x:A} B(x)] f g$
  \[
  \id{\alpha}{\funext (x \mapsto \happly \: \alpha \: x)}
  \]
\end{itemize}

%% FIXME: where do the rules for \alpha[\delta] go in this style?

Identity, inverses, and composition:
\[
\begin{array}{l}
\refl{(f : \prd{x:A} B)} = \funext(x \mapsto \refl{f x}) \\
\opp{\alpha} = \funext (x \mapsto \opp{(\happly \: {\alpha} \: {\refl x})})  \\
{\alpha} \ct \beta = \funext (x \mapsto {(\happly \: {\alpha} \: {\refl x}) \ct (\happly \: {\beta} \: {\refl x})})  \\
\end{array}
\]

\newcommand{\fcomp}{\circ}

Transport, first for non-dependent functions: given $\alpha : \id {a_0} {a_0'}$
and $f : A(a_0) \to B(a_0)$, 
\[
\begin{array}{l}
\transport{x_0:A_0 \mapsto A(x_0) \to B(x_0)}{\alpha} \: f = 
   (\transport{x_0:A_0 \mapsto B(x_0)}{\alpha}) \fcomp f \fcomp (\transport{x:A_0 \mapsto A(x_0)}{\opp \alpha})
\end{array}
\]

Transport for $\Pi$:  
\[
\begin{array}{l}
\transport{x_0:A_0 \mapsto \prd{x : A(x_0)} B(x_0)}{\alpha} \: f =  \\
   x \mapsto 
   \transport{(p : \sm {x_0:A_0} A) \mapsto B(\proj{1} \: p, \proj{2} \: p)}{\alpha^{\mathord{-}}(x) } 
      \: (f (\transport{x:A_0 \mapsto A(x)}{\opp \alpha}\:  x))
\end{array}
\]
where 
\[
\alpha^-(x) : \id[\sm {x_0:A_0} A(x_0)] {(a_0 , \transport{x_0 \mapsto A(x_0)} {\opp \alpha} \: x)} {(a_0' , x)}
\]
can be defined by path induction on $\alpha$.  
%%FIXME: should discuss this with sigma types

\subsection{The Universe}

A path in $\type$ is given by univalence

\[
\eqv{(\id[\type]{A}{B})}{\eqv A B}
\]

Again it is useful to break this into 

\newcommand\isequiv{\mathsf{isEquiv}}

\begin{itemize}
\item An introduction rule for {(\id[\type]{A}{B})}:
  \[
  \ua{} : {\eqv A B} \to (\id[\type]{A}{B})
  \]
\item The elimination rule is transport at $X:\type \mapsto X$:
  \[
  \transport{X \mapsto X}{} : \id{A}{B} \to (A \to B)
  \]
\item $\beta$-reduction: 
  \[
  \begin{array}{l}
  \id{\transport{X \mapsto X} \: (\ua {(f, fIsEquiv)})}{f}
  \end{array}
  \]
\item $\eta$-equivalence: For any $\alpha : \id A B$
  \[
  \id{\alpha}{\ua {(\transport{X \mapsto X}(\alpha) , \beta)}}
  \]
  where $\mathsf{transportIsEquiv} : \isequiv{\transport{X \mapsto X}(\alpha)}$ can be
  defined by doing path induction on $\alpha$, at which point it
  suffices to show that the identity function is an equivalence.  
\end{itemize}

Identity, inverses, and composition: (FIXME: pick a specific definition
of equivalence?)
\[
\begin{array}{l}
\refl{A} = \ua{\idfunc} \\
\opp{\alpha} = \ua {(\transport {X \mapsto X} {\opp \alpha}, \mathsf{transportIsEquiv} \: (\opp \alpha))} \\ 
{\alpha} \ct \beta = ? \\
\end{array}
\]

\subsection{Identity Type}

When we know what \id[A]{}{} is, \id[ {\id[A]{}{}} ]{}{} follows:

Transport, when $A$ is non-dependent:
\[
\begin{array}{l}
\transport{x_0:A_0 \mapsto \id[A] {a_1(x_0)}{a_2(x_0)}} {\alpha_0}{\alpha} = 
\opp{(\map{a_1}{\alpha_0})} \ct \alpha \ct \map{a_2}{\alpha_0}
\end{array}
\]

Useful special cases:
\[
\begin{array}{l}
\transport{x:A \mapsto \id[A] {a}{x}} {\alpha_0} \: {\alpha} = \alpha \ct \alpha_0 \\
\transport{x:A \mapsto \id[A] {x_0}{a}} {\alpha_0} \: {\alpha} = \opp {\alpha_0} \ct \alpha \\
\end{array}
\]

Transport, when $A$ is dependent:
\[
\begin{array}{l}
\transport{x_0:A_0 \mapsto \id[A(x_0)] {a_1(x_0)}{a_2(x_0)}} {\alpha_0} \: {\alpha} = \\
\opp{(\map{a_1}{\alpha_0})} \ct \mapdep{(\transport{A}{\alpha_0})}{\alpha} \ct \map{a_2}{\alpha_0}
\end{array}
\]

\subsection{Higher Inductives}

\newcommand{\sone}{\mathsf{S^1}}

Consider a higher inductive type such as $\sone$.  The definition of the
higher inductive type does not immediately characterize
\id[\sone]{x}{y}---which is good, because the calculation of higher
homotopy groups can be a significant theorem, so we don't want it to be
baked into the definitons.  However, we will often be able to prove a
theorem characterizing the loop space, which follows the above form.
For example, the proof in Chapter~\ref{sec:pi1s1} that the fundamental
group of the circle is the integers plays this role:

\begin{itemize}
\item An introduction rule for \id[\sone]{\mathsf{base}}{\mathsf{base}}:
  \[
  \mathsf{loopToThe} : \mathbb{Z} \to \id{\mathsf{base}}{\mathsf{base}}
  \]
\item An elimination rule:
  \[
  \mathsf{encode} : \id{\mathsf{base}}{\mathsf{base}} \to \mathbb{Z}
  \]
\item With $\beta$ and $\eta$ rules stating that these are mutually inverse.
\end{itemize}

It's less clear that you want to think about identity, inverses, and
composition as being defined through this encoding (rather than thinking
of them as constructors), but you can:

\[
\begin{array}{l}
\refl{\mathsf{base}} = \mathsf{loopToThe} \: 0 \\
\opp{\alpha} = \mathsf{loopToThe} \: (- (\mathsf{encode} \: \alpha)) \\
\alpha \ct \beta = \mathsf{loopToThe} \: ((\mathsf{encode} \: \alpha) + (\mathsf{encode} \: \beta)) \\
\end{array}
\]

This changes the representation of the group structure
from the identity type to an explicit representation, as the free group
on one generator (the additive group on the integers).  

FIXME: say something about map

\section{Examples}

\subsection{Monoid}


