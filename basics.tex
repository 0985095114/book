\chapter{Homotopy type theory}
\label{cha:basics}

\section{Types are higher groupoids}
\label{sec:equality}

Recall that for any type $A$, and any $x,y:A$, we have a identity type $\id[A]{x}{y}$, also written $\idtype[A]{x}{y}$ or just $x=y$.
We can think of inhabitants of $x=y$ as either
\begin{enumerate}
\item proofs that $x$ and $y$ are equal,
\item paths from $x$ to $y$, or
\item isomorphisms/equivalences from $x$ to $y$.
\end{enumerate}
The first is more traditional in type theory; but in homotopy type theory we often take the latter two perspectives as well.
It turns out that the defining rules of identity types, as described in the previous chapter, give them structure which corresponds precisely to that of a space or a higher groupoid.

Recall that the induction principle for the identity types $\id[A]{x}{y}$ (with $A$ a fixed type) says that if
\begin{itemize}
\item for every $x,y:A$ and every $p:\id[A]xy$ we have a type $D(x,y,p)$, and
\item for every $a:A$ we have an element $d(a):D(a,a,\refl a)$, 
\end{itemize}
then
\begin{itemize}
\item there exists an element $J_{D,d}(x,y,p):D(x,y,p)$ for \emph{every} two elements $x,y:A$ and $p:\id[A]xy$, such that $J_{D,d}(a,a,\refl a) \jdeq d(a)$.
\end{itemize}
In other words, given dependently typed functions
\begin{align*}
D & :\prd{x,y:A}{p:\id{x}{y}} \type\\
d & :\prd{a:A} D(a,a,\refl{a})
\end{align*}
there is a dependently typed function
\[J_{D,d}:\prd{x,y:A}{p:\id{x}{y}} D(x,y,p)\]
such that 
\begin{equation}\label{eq:Jconv}
J_{D,d}(a,a,\refl{a})\jdeq d(a)
\end{equation}
for every $a:A$.
The notation $J$ is traditional for this function, but we will not use it very much.
Usually, every time we apply this induction rule we will either not care about the specific function being defined, or we will immediately give it a different name.

Informally, the induction principle for identity types says that if we want to construct an object (or prove a statement) which depends on an inhabitant $p:\id[A]xy$ of an identity type, then it suffices to perform the construction (or the proof) in the special case when $x$ and $y$ are the same (judgmentally) and $p$ is a reflexivity term $\refl{x}$ (judgmentally).
When writing informally, we may express this with a phrase such as ``by induction, it suffices to assume\dots''.
This reduction to the ``reflexivity case'' is analogous to the reduction to the ``base case'' and ``inductive step'' in an ordinary proof by induction on the natural numbers.

The ``conversion rule''~\eqref{eq:Jconv} is less familiar in the context of proof by induction on natural numbers, but there is an analogous notion in the related concept of definition by recursion.
If a sequence $(a_n)_{n\in \mathbb{N}}$ is defined by giving $a_0$ and specifying $a_{n+1}$ in terms of $a_n$, then in fact the $0^{\mathrm{th}}$ term of the resulting sequence \emph{is} the given one, and the given recurrence relation relating $a_{n+1}$ to $a_n$ holds for the resulting sequence.
(This may seem so obvious as to not be worth saying, but if we view a definition by recursion as an algorithm for calculating values of a sequence, then it is precisely the process of executing that algorithm.)
The rule~\eqref{eq:Jconv} is analogous: it says that if we define an object $f(p)$ for all $p:x=y$ by specifying what the value should be when $p$ is $\refl{x}:x=x$, then the value we specified is in fact the value of $f(\refl{x})$.

We now derive from this induction principle all the structure of a higher groupoid.
We begin with symmetry of equality, which, in topological language, means that ``paths can be reversed''.

\begin{lem}\label{lem:opp}
  For every type $A$ and every $x,y:A$ there is a function
  \begin{equation*}
    (x= y)\to(y= x)
  \end{equation*}
  denoted $p\mapsto \opp{p}$, such that $\opp{\refl{x}}\jdeq\refl{x}$ for each $x:A$.
\end{lem}
\begin{proof}[First proof]
  Let $D:\prd{x,y:A}{p:x= y} \type$ be the dependent type defined by $D(x,y,p)\defeq (y= x)$.
  In other words, $D$ is a function assigning to any $x,y:A$ and $p:x=y$ a type, namely the type $y=x$.
  Then we have
  \begin{equation*}
    d\defeq \lambda x.\refl{x}:\prd{x:A} D(x,x,\refl{x}).
  \end{equation*}
  Thus, the eliminator $J$ for identity types gives us a term $J_{D,d}(x,y,p): (y= x)$ for each $p:(x= y)$.
  We can now define the desired function $\opp{(-)}$ to be $\lambda p. J_{D,d}(x,y,p)$, i.e.\ we set $\opp{p} \defeq J_{D,d}(x,y,p)$.
  The conversion rule~\eqref{eq:Jconv} gives $\opp{\refl{x}}\jdeq \refl{x}$.
\end{proof}

We have written out this proof in a very formal style, which may be helpful while the induction rule on identity types is unfamiliar.
However, eventually we prefer to use more natural language, such as in the following equivalent proof.

\begin{proof}[Second proof]
  We want to construct, for each $x,y:A$ and $p:x=y$, an element $\opp{p}:y=x$.
  By induction, it suffices to do this in the case when $y$ is $x$ and $p$ is $\refl{x}$.
  But in this case, the type $x=y$ of $p$ and the type $y=x$ in which we are trying to construct $\opp{p}$ are both simply $x=x$.
  Thus, in the ``reflexivity case'', we can define $\opp{\refl{x}}$ to be simply $\refl{x}$.
  The general case then follows by the induction principle, and the conversion rule $\opp{\refl{x}}\jdeq\refl{x}$ is precisely the proof in the reflexivity case that we gave.
\end{proof}

We will write out the next few proofs in both styles, to help the reader become accustomed to the latter one.
Next we prove the transitivity of equality, or equivalently we ``concatenate paths''.

\begin{lem}\label{lem:concat}
  For every type $A$ and every $x,y,z:A$ there is a function
  \begin{equation*}
    (x= y) \to (y= z)\to (x=  z)
  \end{equation*}
  written $(p,q)\mapsto p\ct q$, such that $\refl{x}\ct \refl{x}\jdeq \refl{x}$ for any $x:A$.
\end{lem}

\begin{proof}[First proof]
  Let $D:\prd{x,y:A}{p:x=y} \type$ be the dependent type
  \begin{equation*}
    D(x,y,p)\defeq \prd{z:A}{q:y=z} (x=z).
  \end{equation*}
  Note that $D(x,x,\refl x) \jdeq \prd{z:A}{q:x=z} (x=z)$.
  Thus, in order to apply the induction principle for identity types to this $D$, we need a function of type
  \begin{equation}\label{eq:concatD}
    \prd{x:A} D(x,x,\refl{x})
    \jdeq \prd{x,z:A}{q:x=z} (x=z).
  \end{equation}
  Now let $E:\prd{x,z:A}{q:x=z}\type$ be the dependent type $E(x,z,q)\defeq (x=z)$.
  Note that $E(x,x,\refl x) \jdeq (x=x)$.
  Thus, we have the function
  \begin{equation*}
    e(x) \defeq \refl{x} : E(x,x,\refl{x}).
  \end{equation*}
  By the induction principle for identity types applied to $E$, we obtain a function
  \begin{equation*}
    d(x,z,q) : \prd{x,z:A}{q:x=z} E(x,z,q) \jdeq \prd{x,z:A}{q:x=z} (x=z)
  \end{equation*}
  which is~\eqref{eq:concatD}.
  Thus, we can use this function $d$ and apply the induction principle for identity types to $D$, to obtain our desired function of type
  \begin{equation*}
    \prd{x,y,z:A}{q:y=z}{p:x=y} (x=z).
  \end{equation*}
  The conversion rules for the two induction principles give us $\refl{x}\ct \refl{x}\jdeq \refl{x}$ for any $x:A$.
\end{proof}

\begin{proof}[Second proof]
  We want to construct, for every $x,y,z:A$ and every $p:x=y$ and $q:y=z$, an element of $x=z$.
  By induction on $p$, it suffices to assume that $y$ is $x$ and $p$ is $\refl{x}$.
  In this case, the type $y=z$ of $q$ is $x=z$.
  Now by induction on $q$, it suffices to assume also that $z$ is $x$ and $q$ is $\refl{x}$.
  But in this case, $x=z$ is $x=x$, and we have $\refl{x}:(x=x)$.
\end{proof}

The reader may well feel that we have given an overly convoluted proof of this lemma.
In fact, we could stop after the induction on $p$, since at that point what we want to produce is an equality $x=z$, and we already have such an equality, namely $q$.
Why do we go on to do another induction on $q$?

The answer is that, as described in the introduction, we are doing \emph{proof-relevant} mathematics.
When we prove a lemma, we are defining an inhabitant of some type, and it can matter what \emph{specific} element we defined in the course of the proof, not merely the type that that element inhabits (that is, the \emph{statement} of the lemma).
\autoref{lem:concat} has three obvious proofs: we could do induction over $p$, induction over $q$, or induction over both of them.
If we proved it three different ways, we would have three different elements of the same type.
It's not hard to show that these three elements would be (provably) \emph{equal} (see \autoref{ex:basics:concat}), but there can still be reasons to prefer a particular definition over a provably equal one.

In the case of \autoref{lem:concat}, the difference hinges on the computation rule.
If we proved the lemma using a single induction over $p$, then we would end up with a computation rule of the form $\refl{y} \ct q \jdeq q$.
If we proved it with a single induction over $q$, we would have instead $p\ct\refl{x}\jdeq p$, while proving it with a double induction (as we did) gives only $\refl{x}\ct\refl{x} \jdeq \refl{x}$.

The asymmetrical computation rules can sometimes be convenient when doing formalized mathematics, as they allow the computer to reduce some things automatically.
However, in informal mathematics, and arguably even in the formalized case, it can be confusing to have a concatenation operation which behaves asymmetrically and to have to remember which side is the ``special'' one.
Treating both sides symmetrically makes for more robust proofs; this is why we have given the proof that we did.
(However, this is admittedly a stylistic choice.)

The table below summarizes the ``equality'' and ``homotopical'' points of view on what we have done so far.
\begin{center}
  \begin{tabular}{c|c}
    Equality & Homotopy \\\hline
    reflexivity & constant path\\
    symmetry & inversion of paths\\
    transitivity & concatenation of paths
  \end{tabular}
\end{center}

However, proof-relevance also means that we can't stop after proving ``symmetry'' and  ``transitivity'' of equality: we need to know that these \emph{operations} on equalities are well-behaved.
(This issue is invisible to set-level mathematics, where symmetry and transitivity are mere \emph{properties} of equality, rather than structure on paths.)
For instance, we need to know that concatenation is \emph{associative}, and that inversion provides \emph{inverses} with respect to concatenation.
This is to be expected from the topological point of view, where these are regarded as \emph{operations} on paths.

\begin{lem}\label{thm:omg}%[The $\omega$-groupoid structure of types]
  Suppose $A:\type$, that $x,y,z,w:A$ and that $p:x= y$ and $q:y = z$ and $r:z=w$.
  We have the following:
  \begin{enumerate}
  \item $p= p\ct \refl{y}$ and $p = \refl{x} \ct p$.\label{item:omg1}
  \item $\opp{p}\ct p=  \refl{y}$ and $p\ct \opp{p}= \refl{x}$.
  \item $\opp{(\opp{p})}= p$.
  \item $p\ct (q\ct r)=  (p\ct q)\ct r$.\label{item:omg4}
  \end{enumerate}
\end{lem}

Note, in particular, that~\ref{item:omg1}--\ref{item:omg4} are themselves propositional equalities, living in the identity types of the identity types $x=y$.
Topologically, they are \emph{paths of paths}, and we are familiar topologically with the idea that concatenating a path with the reversed path only gives a constant path \emph{up to homotopy}, i.e.\ up to a higher path.
The paths~\ref{item:omg1}--\ref{item:omg4} also satisfy their own higher coherence laws, which are themselves higher paths, and so on all the way up.

However, for most purposes it is unnecessary to make the whole infinite-dimensional structure explicit.
One of the nice things about homotopy type theory is that all of this structure can be \emph{proven} starting from only the inductive property of identity types, so we can make explicit as much or as little of it as we need.
In particular, often we don't need the complicated combinatorics involved in making precise notions such as ``coherent structure at all higher levels''.

\begin{proof}[Proof of~\autoref{thm:omg}]
  All the proofs use the induction principle for equalities.
  \begin{enumerate}
  \item \emph{(First proof)} Let $D:\prd{x,y:A}{p:x=y} \type$ be the dependent type given by 
    \begin{equation*}
      D(x,y,p)\defeq (p= p\ct \refl{y}).
    \end{equation*}
    Then $D(x,x,\refl{x})$ is $\refl{x}=\refl{x}\ct\refl{x}$.
    Since $\refl{x}\ct\refl{x}\jdeq\refl{x}$, it follows that $D(x,x,\refl{x})\jdeq (\refl{x}=\refl{x})$.
    Thus, there is a term
    \begin{equation*}
      d\defeq\lambda x.\refl{\refl{x}}:\prd{x:A} D(x,x,\refl{x}).
    \end{equation*}
    Now $J$ gives a term $J(D,d,p):(p= p\ct\refl{y})$ for each $p:x= y$.
    The other equality is proven similarly.

    \noindent
    \emph{(Second proof)} By induction on $p$, it suffices to assume that $y$ is $x$ and that $p$ is $\refl x$.
    But in this case, we have $\refl{x}\ct\refl{x}\jdeq\refl{x}$.
  \item \emph{(First proof)} Let $D:\prd{x,y:A}{p:x=y} \type$ be the dependent type given by 
    \begin{equation*}
      D(x,y,p)\defeq (\opp{p}\ct p=  \refl{y}).
    \end{equation*}
    Then $D(x,x,\refl{x})$ is $\opp{\refl{x}}\ct\refl{x}=\refl{x}$.
    Since $\opp{\refl{x}}\jdeq\refl{x}$ and $\refl{x}\ct\refl{x}\jdeq\refl{x}$, we get that $D(x,x,\refl{x})\jdeq (\refl{x}=\refl{x})$.
    Hence we find the function
    \begin{equation*}
      d\defeq\lambda x.\refl{\refl{x}}:\prd{x:A} D(x,x,\refl{x}).
    \end{equation*}
    Now $J$ gives a term $J(D,d,p):\opp{p}\ct p=\refl{y}$ for each $p:x= y$ in $A$.
    The other equality is similar.

    \noindent \emph{(Second proof)} By induction, it suffices to assume $p$ is $\refl x$.
    But in this case, we have $\opp{p} \ct p \jdeq \opp{\refl x} \ct \refl x \jdeq \refl x$.
  \item \emph{(First proof)} Let $D:\prd{x,y:A}{p:x=y} \type$ be the dependent type given by
    \begin{equation*}
      D(x,y,p)\defeq (\opp{\opp{p}}= p).
    \end{equation*}
    Then $D(x,x,\refl{x})$ is the type $(\opp{\opp{\refl x}}=\refl{x})$.
    But since $\opp{\refl{x}}\jdeq \refl{x}$ for each $x:A$, we have $\opp{\opp{\refl{x}}}\jdeq \opp{\refl{x}} \jdeq\refl{x}$, and thus $D(x,x,\refl{x})\jdeq(\refl{x}=\refl{x})$.
    Hence we find the function
    \begin{equation*}
      d\defeq\lambda x.\refl{\refl{x}}:\prd{x:A} D(x,x,\refl{x}).
    \end{equation*}
    Now $J$ gives a term $J(D,d,p):\opp{\opp{p}}= p$ for each $p:x= y$.

    \noindent \emph{(Second proof)} By induction, it suffices to assume $p$ is $\refl x$.
    But in this case, we have $\opp{\opp{p}}\jdeq \opp{\opp{\refl x}} \jdeq \refl x$.
  \item \emph{(First proof)} Let $D_1:\prd{x,y:A}{p:x=y} \type$ be the dependent type given by
    \begin{equation*}
      D_1(x,y,p)\defeq\prd{z,w:A}{q:y= z}{r:z= w} \big(p\ct (q\ct r)=  (p\ct q)\ct r\big).
    \end{equation*}
    Then $D_1(x,x,\refl{x})$ is
    \begin{equation*}
      \prd{z,w:A}{q:x= z}{r:z= w} \big(\refl{x}\ct(q\ct r)= (\refl{x}\ct q)\ct r\big).
    \end{equation*}
    To construct a term in this type, let $D_2:\prd{x,z:A}{q:x=z} \type$ be the dependent type
    \begin{equation*}
      D_2 (x,z,q) \defeq \prd{w:A}{r:z=w} \big(\refl{x}\ct(q\ct r)= (\refl{x}\ct q)\ct r\big).
    \end{equation*}
    Then $D_2(x,x,\refl{x})$ is
    \begin{equation*}
      \prd{w:A}{r:x=w} \big(\refl{x}\ct(\refl{x}\ct r)= (\refl{x}\ct \refl{x})\ct r\big).
    \end{equation*}
    To construct a term in \emph{this} type, let $D_3:\prd{x,w:A}{r:x=w} \type$ be the dependent type
    \begin{equation*}
      D_3(x,w,r) \defeq \big(\refl{x}\ct(\refl{x}\ct r)= (\refl{x}\ct \refl{x})\ct r\big).
    \end{equation*}
    Then $D_3(x,x,\refl{x})$ is
    \begin{equation*}
      \big(\refl{x}\ct(\refl{x}\ct \refl{x})= (\refl{x}\ct \refl{x})\ct \refl{x}\big)
    \end{equation*}
    which is definitionally equal to the type $(\refl{x} = \refl{x})$, and is therefore inhabited by $\refl{\refl{x}}$.
    Applying the identity elimination rule three times, therefore, we obtain a term of the overall desired type.

    \noindent \emph{(Second proof)} By induction, it suffices to assume $p$, $q$, and $r$ are all $\refl x$.
    But in this case, we have
    \begin{align*}
      p\ct (q\ct r)
      &\jdeq \refl{x}\ct(\refl{x}\ct \refl{x})\\
      &\jdeq \refl{x}
      &\jdeq (\refl{x}\ct \refl x)\ct \refl x\\
      &\jdeq (p\ct q)\ct r.
    \end{align*}
    Thus, we have $\refl{\refl{x}}$ inhabiting this type.\qedhere
  \end{enumerate}
\end{proof}

\begin{rmk}
  There are other ways to define all of these higher paths.
  For instance, in \autoref{thm:omg}\ref{item:omg4} we might do induction only over one or two paths rather than all three.
  All possibilities will produce \emph{definitionally} different proofs, but they will always be propositionally the same.
  Such an equality between any two particular proofs can, again, be proven by induction, reducing all the paths in question to reflexivities and then observing that both proofs reduce themselves to reflexivities.
\end{rmk}

\section{Functions are functors}
\label{sec:functors}

Now we wish to establish that functions $f:A\to B$ behave functorially on paths.
In traditional type theory, this is equivalently the statement that functions respect equality.
Topologically, this corresponds to saying that every function is ``continuous'', i.e.\ preserves paths.

\begin{lem}\label{lem:map}
  Suppose that $f:A\to B$ is a function and that $p:(\id[A]xy)$. Then there is an equality
  \begin{equation*}
    \map f p : f(x)= f(y)
  \end{equation*}
  Moreover, for each $x:A$ we have $\map {f}{\refl{x}}\jdeq \refl{f(x)}$.
\end{lem}

Note that we are defining a new \emph{operation} on paths.

\begin{proof}[First proof]
  Let $D:\prd{x,y:A}{p:x=y}\type$ be $D(x,y,p)\defeq (f(x)= f(y))$.
  Then we have
  \begin{equation*}
    d\defeq\lambda x.\refl{f(x)}:\prd{x:A} D(x,x,\refl{x}).
  \end{equation*}
  Now $J$ gives us $\map f p\defeq f(x)= f(y)$ for each $p:x= y$.
  The conversion rule implies $\map f{\refl{x}}\jdeq\refl{f(x)}$ for each $x:A$.
\end{proof}

\begin{proof}[Second proof]
  By induction, it suffices to assume $p$ is $\refl{x}$.
  In this case, we may define $\map f p \defeq \refl{f(x)}:f(x)\jdeq f(x)$.
\end{proof}

Since \emph{dependently typed} functions are very important in type theory, we will also need a version of \autoref{lem:map} for these.
However, this is not quite so simple to state, because if $f:\prd{x:A} B(x)$ and $p:x=y$, then $f(x):B(x)$ and $f(y):B(y)$ are elements of distinct types, so that \emph{a priori} we cannot even ask whether they are equal.
The missing ingredient is that $p$ itself gives us a way to relate the types $B(x)$ and $B(y)$.

\begin{lem}[Transport]\label{lem:transport}
  Suppose that $P$ is a dependent type over $A$ and that $p:\id[A]xy$.
  Then there is a function $\transf{p}:P(x)\to P(y)$.
\end{lem}

\begin{proof}[First proof]
  Let $D:\prd{x,y:A}{p:\id{x}{y}} \type$ be $D(x,y,p)\defeq P(x)\to P(y)$.
  Then we have the function
  \begin{equation*}
    d\defeq\lambda x.\idfunc[P(x)]:\prd{x:A} D(x,x,\refl{x}),
  \end{equation*}
  so that the induction principle gives us $J_{D,d}(x,y,p):P(x)\to P(y)$ for $p:x= y$, which we define to be $\transf p$.
\end{proof}

\begin{proof}[Second proof]
  By induction, it suffices to assume $p$ is $\refl x$.
  But in this case, we can take $\transf{(\refl x)}:P(x)\to P(x)$ to be the identity function.
\end{proof}

Sometimes, it is necessary to notate the dependent type $P$ in which the transport operation happens.
In this case, we may write
\[\transfib P p - : P(x) \to P(y).\]

Recall that a dependent type $P$ over a type $A$ can be seen as a property of elements of $A$, which holds at $x$ in $A$ if $P(x)$ is inhabited.
Then the transportation lemma says that if $x$ is propositionally equal to $y$, then $P(x)$ holds if and only if $P(y)$ holds.
In fact, we will see later on that if $x=y$ then actually $P(x)$ and $P(y)$ are \emph{equivalent}.

Now we can prove the dependent version of \autoref{lem:map}.

\begin{lem}[Dependent map]\label{lem:mapdep}
  Suppose $f:\prd{x: A} P(x)$ and $p:\id[A]xy$.
  Then we have $\mapdep f p:\id[P(y)]{\trans p{f(x)}}{f(y)}$.
\end{lem}

\begin{proof}[First proof]
  Let $D:\prd{x,y:A}{p:x=y} \type$ be the dependent type given by
  \begin{equation*}
    D(x,y,p)\defeq \trans p {f(x)}= f(y).
  \end{equation*}
  Then $D(x,x,\refl{x})$ is $\trans{\refl{x}}{f(x)}= f(x)$.
  But since $\trans{(\refl{x})}{f(x)}\jdeq f(x)$, we get that $D(x,x,\refl{x})\jdeq (f(x)= f(x))$.
  Thus, we find the term
  \begin{equation*}
    d\defeq\lambda x.\refl{f(x)}:\prd{x:A} D(x,x,\refl{x})
  \end{equation*}
  and now $J$ gives us $\mapdep fp:\trans p{f(x)}= f(y)$ for each $p:x= y$.
\end{proof}

\begin{proof}[Second proof]
  By induction, it suffices to assume $p$ is $\refl x$.
  But in this case, we have $\trans{(\refl{x})}{f(x)}\jdeq f(x)$.
\end{proof}

From now on, we will stick to the second, more concise, style of proof by induction on identity types.

\subsection{Summary of the basic higher structure}
\label{sec:basics-summary}

Here we summarize the basic definitions made in this section.

\begin{itemize}
\item $\opp{p} : y=x$, for $p:x=y$, defined by
  \[\opp{\;\refl{x}}\jdeq \refl{x}.\]
\item $p\ct q :y=z$, for $p:x=y$ and $q:y=z$, defined by
  \[ \refl{x}\ct\refl{x}\jdeq\refl{x}.\]
\item If $P$ is a dependent type over $A$ then $\transf{p}:P(x)\ra P(y)$, for $p:x=y$, defined by
  \[\transf{(\refl{x})}\jdeq \idfunc[P(x)].\]
\item If $f:A\ra B$ then $\map{f}{p}:f(x)=f(y)$, for $p:x=y$, defined by
  \[\map{f}{\refl{x}}\jdeq \refl{f(x)}.\]
\item If $f:\prod_{x:A}P(x)$ then $\mapdep{f}{p}:\trans{p}{f(x)}=f(y)$, for $p:x=y$, defined by
  \[\mapdep{f}{\refl{x}}\jdeq \refl{f(x)}.\]
\end{itemize}



\section*{Exercises}
\label{basics:exercises}

\begin{ex}\label{ex:basics:concat}
  Show that the three obvious proofs of \autoref{lem:concat} are pairwise equal.
\end{ex}

\begin{ex}
  Show that the three equalities of proofs constructed in the previous exercise form a commutative triangle.
\end{ex}

\begin{ex}
  Give a fourth, different, proof of \autoref{lem:concat}, and prove that it is equal to the others.
\end{ex}


% Local Variables:
% TeX-master: "main"
% End:
