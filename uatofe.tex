\chapter{Univalence implies Function Extensionality}


TODO: fix numbering, make sure it does not use UA or FE.

We include in this appendix a proof that the Univalence Axiom implies Function
Extensionality. Thus, in this chapter we work \emph{without} either the Univalence Axiom
or Function Extensionality. We prove this in two steps. First we show that the Univalence
axiom implies the following weak form of function extensionality, defined below, which in turn implies the usual function extensionality.
We assume that $\type$ is a universe, which we will explicitly indicate where we assume that it is univalent. Also let $P : A \to \type$ be a family of types over a type $A$. % and let $C\defeq\prd{x:A}P(x)$.
\begin{defn}
The \emph{weak function extensionality principle} asserts that there is a function
\begin{equation*}
\left(\tprd{x:A}\iscontr(P(x))\right)
  \to\iscontr\left(\tprd{x:A}P(x)\right).
\end{equation*}
\end{defn}

\begin{lem} \label{UA-eqv-hom-eqv}
Assuming $\type$ is univalent, for any $A,B,X:\type$ and any $e:\eqv{A}{B}$, there is an equivalence
\begin{equation*}
\eqv{(X\to A)}{(X\to B)}
\end{equation*}
of which the underlying map is given by postcomposition with the underlying function of $e$.
\end{lem}

\begin{proof}
Immediate by induction on $\eqv{}{}$ (see \autoref{thm:equiv-induction}).
\end{proof}

\begin{cor}
Let the dependent type $P:A\to\type$ have fibers that are contractible, i.e., $\prd{x:A}\iscontr(P(x))$.
Then the projection $\proj{1}:(\sm{x:A}P(x))\to A$ is an equivalence. Assuming $\type$ is univalent, it follows immediately that precomposition with $\proj{1}$ gives an equivalence
\begin{equation*}
\alpha : \eqv{\left(A\to\tsm{x:A}P(x)\right)}{(A\to A)}.
\end{equation*}
\end{cor}

\begin{proof}
  Recall that for $\proj{1}:\sm{x:A}P(X)\to A$ and $x:A$ we have an equivalence
  \begin{equation*}
    \eqv{\hfiber{\proj{1}}{x}}{P(x)}.
  \end{equation*}
  therefore $\proj{1}$ is an equivalence. Now we apply \autoref{UA-eqv-hom-eqv}.
\end{proof}

In particular, the homotopy fiber of the above equivalence at $\idfunc[A]$ is contractible. Therefore, we can show that univalence implies weak function extensionality by showing that the product space $\prd{x:A}P(x)$ is a retract of $\hfiber{\alpha}{\idfunc[A]}$.

\begin{thm}
In a univalent universe $\type$, suppose that $P:A\to\type$ is a dependent type over $A$ with contractible fibers and let $\alpha$ be the function of the previous lemma. Then $\prd{x:A}P(x)$ is a retract of $\hfiber{\alpha}{\idfunc[A]}$. As a consequence, $\prd{x:A}P(x)$ is contractible. In other words, the univalence axiom implies the weak function extensionality principle.
\end{thm}

\begin{proof}
Define the functions
\begin{align*}
  \varphi &: \tprd{x:A}P(x)\to\hfiber{\alpha}{\idfunc[A]}\\
  \varphi(f) &\defeq (\lam{x} (x,f(x)),\refl{\idfunc[A]})\\
  \psi &: \hfiber{\alpha}{\idfunc[A]}\to \tprd{x:A}P(x) \\
  \psi(g,p) &\defeq \lam{x} \trans p{\proj{2} (g(x))}
\end{align*}
Then $\psi(\varphi(f))=\lam{x} f(x)$, which is $f$ by the $\eta$-rule for dependent products.
\end{proof}

We now show that weak function extensionality implies the usual function extensionality.
Recall from~\eqref{eq:happly} the function $\happly (f,g) : (f = g)\to(f\htpy g)$ which
converts equality of functions to homotopy.

\begin{thm}
Weak function extensionality implies the function extensionality \autoref{axiom:funext}.
\end{thm}

\begin{proof}
We want to show that
\begin{equation*}
\prd{A:\type}{P:A\to\type}{f,g:\prd{x:A}P(x)}\isequiv(\happly (f,g)).
\end{equation*}
Since a fiberwise map induces an equivalence on total spaces if and only if it is fiberwise an equivalence by \autoref{thm:total-fiber-equiv}, it suffices to show that the function of type
\begin{equation*}
\left(\sm{g:\prd{x:A}P(x)}(f= g)\right) \to \sm{g:\prd{x:A}P(x)}(f\htpy g)
\end{equation*}
induced by $\lam{g:\prd{x:A}P(x)} \happly (f,g)$ is an equivalence.
Since the type on the left is contractible by \autoref{thm:contr-paths}, it suffices to show that the type on the right:
\begin{equation}\label{eq:uatofesp}
\sm{g:\prd{x:A}P(x)}\prd{x:A}f(x)= g(x)
\end{equation}
is contractible.
Now \autoref{thm:ttac} says that this is equivalent to
\begin{equation}\label{eq:uatofeps}
\prd{x:A}\sm{u:P(x)}f(x)= u.
\end{equation}
The proof of \autoref{thm:ttac} uses function extensionality, but only for one of the composites.
Thus, without assuming function extensionality, we can conclude that~\eqref{eq:uatofesp} is a retract of~\eqref{eq:uatofeps}.
And~\eqref{eq:uatofeps} is a product of contractible types, which is contractible by the weak function extensionality principle; hence~\eqref{eq:uatofesp} is also contractible.
\end{proof}


%%% Local Variables: 
%%% mode: latex
%%% TeX-master: "main"
%%% End: 

