\newcommand{\inv}[1]{{#1}^{-1}}
\newcommand{\idtoiso}{\ensuremath{\mathsf{idtoiso}}\xspace}
\newcommand{\isotoid}{\ensuremath{\mathsf{isotoid}}\xspace}
\newcommand{\op}{^{\textrm{op}}}
\newcommand{\y}{\ensuremath{\mathbf{y}}\xspace}
\newcommand{\dgr}[1]{{#1}^{\dagger}}
\newcommand{\unitaryiso}{\mathrel{\cong^\dagger}}

\chapter{Category theory}
\label{cha:category-theory}

In this chapter we will develop some basic category theory.
Although homotopy type theory is well-adapted for \emph{higher} category theory, here we will restrict ourselves to 1-categories.
We will see that even in this case, the theory has a slightly different flavor from the classical set-based one (and this same difference carries over to the higher-categorical situation).

We assume the reader has some basic familiarity with the classical theory.

In this chapter (except, at the moment, for \autoref{sec:sip}) we will be universe polymorphic and typically ambiguous.
That is, when we write \type it denotes some universe of types, but perhaps a different one at different times, and everything we say will remain true for any consistent choice of universe levels.
This matches the ordinary practice of working category theorists.


\section{Categories and precategories}
\label{sec:cats}

In classical mathematics, there are many equivalent definitions of a category.
In our case, since we have dependent types, it is natural to choose the arrows to be a type family indexed by the objects.
This matches the way hom-types are always used in category theory: we never even consider comparing two arrows unless we know their sources and targets agree.
Furthermore, it seems clear that for a theory of 1-categories, the hom-types should all be sets.
This leads us to the following definition.

\begin{defn}\label{ct:precategory}
  A \textbf{precategory} $A$ consists of the following.
  \begin{enumerate}
  \item A type $A_0$ of \emph{objects}.  We write $a:A$ for $a:A_0$.
  \item For each $a,b:A$, a set $\hom_A(a,b)$ of \emph{arrows} or \emph{morphisms}.
  \item For each $a:A$, a morphism $1_a:\hom_A(a,a)$.
  \item For each $a,b,c:A$, a function
    \[  \hom_A(b,c) \to \hom_A(a,b) \to \hom_A(a,c) \]
    denoted infix by $g\mapsto f\mapsto g\circ f$, or sometimes simply by $gf$.
  \item For each $a,b:A$ and $f:\hom_A(a,b)$, we have $\id f {1_b\circ f}$ and $\id f {f\circ 1_a}$.
  \item For each $a,b,c,d:A$ and $f:\hom_A(a,b)$, $g:\hom_A(b,c)$, $h:\hom_A(c,d)$, we have $\id {h\circ (g\circ f)}{(h\circ g)\circ f}$.
  \end{enumerate}
\end{defn}

The problem with the notion of precategory is that for objects $a,b:A$, we have two possibly-different notions of ``sameness''.
On the one hand, we have $\id[A_0]{a}{b}$.
But on the other hand, there is the standard categorical notion of \emph{isomorphism}.

\begin{defn}\label{ct:isomorphism}
  A morphism $f:\hom_A(a,b)$ is an \textbf{isomorphism} if there is a morphism $g:\hom_A(b,a)$ such that $\id{g\circ f}{1_a}$ and $\id{f\circ g}{1_b}$.
  We write $a\cong b$ for the type of such isomorphisms.
\end{defn}

\begin{lem}\label{ct:isoprop}
  For any $f:\hom_A(a,b)$, the type ``$f$ is an isomorphism'' is a mere proposition.
  Therefore, for any $a,b:A$ the type $a\cong b$ is a set.
\end{lem}
\begin{proof}
  Suppose given $g:\hom_A(b,a)$ and $\eta:(\id{1_a}{g\circ f})$ and $\epsilon:(\id{f\circ g}{1_b})$, and similarly $g'$, $\eta'$, and $\epsilon'$.
We must show $\id{(g,\eta,\epsilon)}{(g',\eta',\epsilon')}$.
  But since all hom-sets are sets, their identity types are mere propositions, so it suffices to show $\id g {g'}$.
  For this we have
  \[g' = 1_a\circ g' = (g\circ f)\circ g' = g\circ (f\circ g') = g\circ 1_b = g\]
  using $\eta$ and $\epsilon'$.
\end{proof}

If $f:a\cong b$, then we write $\inv f$ for its inverse, which by \autoref{ct:isoprop} is uniquely determined.

The only relationship between these two notions of sameness that we have in a precategory is the following.

\begin{lem}[\textsf{idtoiso}]\label{ct:idtoiso}
  If $A$ is a precategory and $a,b:A$, then
  \[(\id a b)\to (a \cong b).\]
\end{lem}
\begin{proof}
  By induction on identity, we may assume $a$ and $b$ are the same.
  But then we have $1_a:\hom_A(a,a)$, which is clearly an isomorphism.
\end{proof}

Evidently, this situation is analogous to the issue that motivated us to introduce the univalence axiom.
In fact, we have the following:

\begin{eg}\label{ct:precatset}
  There is a precategory \uset, whose type of objects is \set, and with $\hom_{\uset}(A,B) \defeq (A\to B)$.
  The identity morphisms are identity functions and the composition is function composition.
  For this precategory, \autoref{ct:idtoiso} is equal to (the restriction to sets of) the identity-to-equivalence map from the chapter on univalence.
\end{eg}

Thus, it is natural to make the following definition.

\begin{defn}\label{ct:category}
  A \textbf{category} is a precategory such that for all $a,b:A$, the function $\idtoiso_{a,b}$ from \autoref{ct:idtoiso} is an equivalence.
\end{defn}

In particular, in a category, if $a\cong b$, then $a=b$.

\begin{eg}\label{ct:eg:set}
  The univalence axiom implies immediately that \uset is a category.
  One can also show, using univalence, that any precategory of set-level structures such as groups, rings, topological spaces, etc.\ is a category.
\end{eg}

We also note the following.

\begin{lem}\label{ct:obj-1type}
  In a category, the type of objects is a 1-type.
\end{lem}
\begin{proof}
  It suffices to show that for any $a,b:A$, the type $\id a b$ is a set.
  But $\id a b$ is equivalent to $a \cong b$, which is a set.
\end{proof}

We write $\isotoid$ for the inverse $(a\cong b) \to (\id a b)$ of the map $\idtoiso$ from \autoref{ct:idtoiso}.
The following relationship between the two is important.

\begin{lem}\label{ct:idtoiso-trans}
  For $p:\id a a'$ and $q:\id b b'$ and $f:\hom_A(a,b)$, we have
  \begin{equation}\label{ct:idtoisocompute}
    \id{\trans{(p,q)}{f}}
    {\idtoiso(q)\circ f \circ \inv{\idtoiso(p)}}
  \end{equation}
\end{lem}
\begin{proof}
  By induction, we may assume $p$ and $q$ are $\refl a$ and $\refl b$ respectively.
Then the left-hand side of~\eqref{ct:idtoisocompute} is simply $f$.
  But by definition, $\idtoiso(\refl a)$ is $1_a$, and $\idtoiso(\refl b)$ is $1_b$, so the right-hand side of~\eqref{ct:idtoisocompute} is $1_b\circ f\circ 1_a$, which is equal to $f$.
\end{proof}

Similarly, we can show
\begin{gather}
  \id{\idtoiso(\rev p)}{\inv {(\idtoiso(p))}}\\
  \id{\idtoiso(p\ct q)}{\idtoiso(q)\circ \idtoiso(p)}\\
  \id{\isotoid(f\circ e)}{\isotoid(e)\ct \isotoid(f)}
\end{gather}
and so on.

\begin{eg}\label{ct:orders}
  A precategory in which each set $\hom_A(a,b)$ is a mere proposition is equivalently a type $A_0$ equipped with a mere relation ``$\le$'' that is reflexive ($a\le a$) and transitive (if $a\le b$ and $b\le c$, then $a\le c$).
  We call this a \textbf{preorder}.

  In a preorder, a proof $f\colon a\le b$ is an isomorphism just when there exists some proof $g\colon b\le a$.
  Thus, $a\cong b$ is the mere proposition that $a\le b$ and $b\le a$.
  Therefore, a preorder $A$ is a category just when (1) each type $a=b$ is a mere proposition, and (2) for any $a,b:A_0$ there exists a function $(a\cong b) \to (a=b)$.
  In other words, $A_0$ must be a set, and $\le$ must be antisymmetric (if $a\le b$ and $b\le a$, then $a=b$).
  We call this a \textbf{(partial) order} or a \textbf{poset}.
\end{eg}

\begin{eg}\label{ct:gaunt}
  If $A$ is a category, then $A_0$ is a set if and only if for any $a,b:A_0$, the type $a\cong b$ is a mere proposition.
  This is equivalent to saying that every isomorphism in $A$ is an identity; thus it is rather stronger than the classical notion of ``skeletal'' category.
  Categories of this sort are sometimes called \textbf{gaunt}.
  There is not really any notion of ``skeletality'' for our categories, unless one considers \autoref{ct:category} itself to be such.
\end{eg}

\begin{eg}\label{ct:discrete}
  For any 1-type $X$, there is a category with $X$ as its type of objects and with $\hom(x,y) \defeq (x=y)$.
  If $X$ is a set, we call this the \textbf{discrete} category on $X$.
  In general, we call it a \textbf{groupoid} (see \autoref{ct:groupoids}).
\end{eg}

\begin{eg}\label{ct:fundgpd}
  For \emph{any} type $X$, there is a precategory with $X$ as its type of objects and with $\hom(x,y) \defeq \pizero{x=y}$.
  The composition operation
  \[ \pizero{y=z} \to \pizero{x=y} \to \pizero{x=z} \]
  is defined by induction on truncation from concatenation $(y=z)\to(x=y)\to(x=z)$.
  We call this the \emph{fundamental pregroupoid} of $X$.
\end{eg}

\begin{eg}\label{ct:hoprecat}
  There is a precategory whose type of objects is \type and with $\hom(X,Y) \defeq \pizero{X\to Y}$, and composition defined by induction on truncation from ordinary composition $(Y\to Z) \to (X\to Y) \to (X\to Z)$.
  We call this the \emph{homotopy precategory of types}.
\end{eg}

\begin{eg}\label{ct:rel}
  Let \urel be the following precategory:
  \begin{itemize}
  \item Its objects are sets.
  \item $\hom_{\urel}(X,Y) = X\to Y\to \prop$.
  \item For a set $X$, we have $1_X(x,x') \defeq (x=x')$.
  \item For $R:\hom_{\urel}(X,Y)$ and $S:\hom_{\urel}(Y,Z)$, their composite is defined by
    \[ (S\circ R)(x,z) \defeq \Brck{\sm{y:Y} R(x,y) \times S(y,z)}.\]
  \end{itemize}
  Suppose $R:\hom_{\urel}(X,Y)$ is an isomorphism, with inverse $S$.
  We observe the following.
  \begin{enumerate}
  \item If $R(x,y)$ and $S(y',x)$, then $(R\circ S)(y',y)$, and hence $y'=y$.
    Similarly, if $R(x,y)$ and $S(y,x')$, then $x=x'$.\label{item:rel1}
  \item For any $x$, we have $x=x$, hence $(S\circ R)(x,x)$.
    Thus, there merely exists a $y:Y$ such that $R(x,y)$ and $S(y,x)$.\label{item:rel2}
  \item Suppose $R(x,y)$.
    By~\ref{item:rel2}, there merely exists a $y'$ with $R(x,y')$ and $S(y',x)$.
    But then by~\ref{item:rel1}, merely $y'=y$, and hence $y'=y$ since $Y$ is a set.
    Therefore, by transporting $S(y',x)$ along this equality, we have $S(y,x)$.
    In conclusion, $R(x,y)\to S(y,x)$.
    Similarly, $S(y,x) \to R(x,y)$.\label{item:rel3}
  \item If $R(x,y)$ and $R(x,y')$, then by~\ref{item:rel3}, $S(y',x)$, so that by~\ref{item:rel1}, $y=y'$.
    Thus, for any $x$ there is at most one $y$ such that $R(x,y)$.
    And by~\ref{item:rel2}, there merely exists such a $y$, hence there exists such a $y$.
  \end{enumerate}
  In conclusion, if $R:\hom_{\urel}(X,Y)$ is an isomorphism, then for each $x:X$ there is exactly one $y:Y$ such that $R(x,y)$, and dually.
  Thus, there is a function $f:X\to Y$ sending each $x$ to this $y$, which is an equivalence; hence $X=Y$.
  With a little more work, we conclude that \urel is a category.
\end{eg}

In a textbook on category theory written for readers who had grown up with informal homotopy type theory, we would probably say very little about precategories from now on, restricting ourselves to the case of categories.
However, since this book has a somewhat different purpose and audience, we will develop many concepts for precategories as well as categories, in order to emphasize how much better-behaved categories are, as compared both to precategories and to ordinary categories in classical mathematics.

We will also see in \crefrange{sec:strict-categories}{sec:dagger-categories} that in slightly more exotic contexts, there are uses for certain kinds of precategories other than categories, each of which ``fixes'' the equality of objects in different ways.
This emphasizes the ``pre''-ness of precategories: they are the raw material out of which multiple important categorical structures can be defined.


\section{Functors and transformations}
\label{sec:transfors}

The following definitions are fairly obvious, and need no modification.

\begin{defn}\label{ct:functor}
  Let $A$ and $B$ be precategories.
  A \textbf{functor} $F:A\to B$ consists of
  \begin{enumerate}
  \item A function $F_0:A_0\to B_0$, generally also denoted $F$.
  \item For each $a,b:A$, a function $F_{a,b}:\hom_A(a,b) \to \hom_B(Fa,Fb)$, generally also denoted $F$.
  \item For each $a:A$, we have $\id{F(1_a)}{1_{Fa}}$.
  \item For each $a,b,c:A$ and $f:\hom_A(a,b)$ and $g:\hom_B(b,c)$, we have
    \[\id{F(g\circ f)}{Fg\circ Ff}.\]
  \end{enumerate}
\end{defn}

Note that by induction on identity, a functor also preserves \idtoiso.

\begin{defn}\label{ct:nattrans}
  For functors $F,G:A\to B$, a \textbf{natural transformation} $\gamma:F\to G$ consists of
  \begin{enumerate}
  \item For each $a:A$, a morphism $\gamma_a:\hom_B(Fa,Ga)$ (the ``components'').
  \item For each $a,b:A$ and $f:\hom_A(a,b)$, we have $\id{Gf\circ \gamma_a}{\gamma_b\circ Ff}$ (the ``naturality axiom'').
  \end{enumerate}
\end{defn}

Since each type $\hom_B(Fa,Gb)$ is a set, its identity type is a mere proposition.
Thus, the naturality axiom is a mere proposition, so identity of natural transformations is determined by identity of their components.
In particular, for any $F$ and $G$, the type of natural transformations from $F$ to $G$ is again a set.

Similarly, identity of functors is determined by identity of the functions $A_0\to B_0$ and (transported along this) of the corresponding functions on hom-sets.

\begin{defn}\label{ct:functor-precat}
  For precategories $A,B$, there is a precategory $B^A$ defined by
  \begin{itemize}
  \item $(B^A)_0$ is the type of functors from $A$ to $B$.
  \item $\hom_{B^A}(F,G)$ is the type of natural transformations from $F$ to $G$.
  \end{itemize}
\end{defn}
\begin{proof}
  We define $(1_F)_a\defeq 1_{Fa}$.
  Naturality follows by the unit axioms of a precategory.
  For $\gamma:F\to G$ and $\delta:G\to H$, we define $(\delta\circ\gamma)_a\defeq \delta_a\circ \gamma_a$.
  Naturality follows by associativity.
  Similarly, the unit and associativity laws for $B^A$ follow from those for $B$.
\end{proof}

\begin{lem}\label{ct:natiso}
  A natural transformation $\gamma:F\to G$ is an isomorphism in $B^A$ if and only if each $\gamma_a$ is an isomorphism in $B$.
\end{lem}
\begin{proof}
  If $\gamma$ is an isomorphism, then we have $\delta:G\to F$ that is its inverse.
  By definition of composition in $B^A$, $(\delta\gamma)_a\jdeq \delta_a\gamma_a$ and similarly.
  Thus, $\id{\delta\gamma}{1_F}$ and $\id{\gamma\delta}{1_G}$ imply $\id{\delta_a\gamma_a}{1_{Fa}}$ and $\id{\gamma_a\delta_a}{1_{Ga}}$, so $\gamma_a$ is an isomorphism.

  Conversely, suppose each $\gamma_a$ is an isomorphism, with inverse called $\delta_a$, say.
We define a natural transformation $\delta:G\to F$ with components $\delta_a$; for the naturality axiom we have
  \[ Ff\circ \delta_a = \delta_b\circ \gamma_b\circ Ff \circ \delta_a = \delta_b\circ Gf\circ \gamma_a\circ \delta_a = \delta_b\circ Gf. \]
  Now since composition and identity of natural transformations is determined on their components, we have $\id{\gamma\delta}{1_G}$ and $\id{\delta\gamma}{1_F}$.
\end{proof}

The following result is fundamental.

\begin{thm}\label{ct:functor-cat}
  If $A$ is a precategory and $B$ is a category, then $B^A$ is a category.
\end{thm}
\begin{proof}
  Let $F,G:A\to B$; we must show that $\idtoiso:(\id{F}{G}) \to (F\cong G)$ is an equivalence.

  To give an inverse to it, suppose $\gamma:F\cong G$ is a natural isomorphism.
  Then for any $a:A$, we have an isomorphism $\gamma_a:Fa \cong Ga$, hence an identity $\isotoid(\gamma_a):\id{Fa}{Ga}$.
  By function extensionality, we have an identity $\bar{\gamma}:\id[(A_0\to B_0)]{F_0}{G_0}$.

  Now since the last two axioms of a functor are mere propositions, to show that $\id{F}{G}$ it will suffice to show that for any $a,b:A$, the functions
  \begin{align*}
    F_{a,b}&:\hom_A(a,b) \to \hom_B(Fa,Fb)\mathrlap{\qquad\text{and}}\\
    G_{a,b}&:\hom_A(a,b) \to \hom_B(Ga,Gb)
  \end{align*}
  become equal when transported along $\bar\gamma$.
  By computation for function extensionality, when applied to $a$, $\bar\gamma$ becomes equal to $\isotoid(\gamma_a)$.
  But by \autoref{ct:idtoiso-trans}, transporting $Ff:\hom_B(Fa,Fb)$ along $\isotoid(\gamma_a)$ and $\isotoid(\gamma_b)$ is equal to the composite $\gamma_b\circ Ff\circ \inv{(\gamma_a)}$, which by naturality of $\gamma$ is equal to $Gf$.

  This completes the definition of a function $(F\cong G) \to (\id F G)$.
  Now consider the composite
  \[ (\id F G) \to (F\cong G) \to (\id F G). \]
  Since hom-sets are sets, their identity types are mere propositions, so to show that two identities $p,q:\id F G$ are equal, it suffices to show that $\id[\id{F_0}{G_0}]{p}{q}$.
  But in the definition of $\bar\gamma$, if $\gamma$ were of the form $\idtoiso(p)$, then $\gamma_a$ would be equal to $\idtoiso(p_a)$ (this can easily be proved by induction on $p$).
  Thus, $\isotoid(\gamma_a)$ would be equal to $p_a$, and so by function extensionality we would have $\id{\bar\gamma}{p}$, which is what we need.

  Finally, consider the composite
  \[(F\cong G)\to  (\id F G) \to (F\cong G). \]
  Since identity of natural transformations can be tested componentwise, it suffices to show that for each $a$ we have $\id{\idtoiso(\bar\gamma)_a}{\gamma_a}$.
  But as observed above, we have $\id{\idtoiso(\bar\gamma)_a}{\idtoiso((\bar\gamma)_a)}$, while $\id{(\bar\gamma)_a}{\isotoid(\gamma_a)}$ by computation for function extensionality.
  Since $\isotoid$ and $\idtoiso$ are inverses, we have $\id{\idtoiso(\bar\gamma)_a}{\gamma_a}$ as desired.
\end{proof}

In particular, naturally isomorphic functors between categories (as opposed to precategories) are equal.

\medskip

We now define all the usual ways to compose functors and natural transformations.

\begin{defn}
  For functors $F:A\to B$ and $G:B\to C$, their composite $G\circ F:A\to C$ is given by
  \begin{itemize}
  \item The composite $(G_0\circ F_0) : A_0 \to C_0$
  \item For each $a,b:A$, the composite
    \[(G_{Fa,Fb}\circ F_{a,b}):\hom_A(a,b) \to \hom_C(GFa,GFb).\]
  \end{itemize}
  It is easy to check the axioms.
\end{defn}

\begin{defn}
  For functors $F:A\to B$ and $G,H:B\to C$ and a natural transformation $\gamma:G\to H$, the composite $(\gamma F):GF\to HF$ is given by
  \begin{itemize}
  \item For each $a:A$, the component $\gamma_{Fa}$.
  \end{itemize}
  Naturality is easy to check.
  Similarly, for $\gamma$ as above and $K:C\to D$, the composite $(K\gamma):KG\to KH$ is given by
  \begin{itemize}
  \item For each $b:B$, the component $K(\gamma_b)$.
  \end{itemize}
\end{defn}

\begin{lem}\label{ct:interchange}
  For functors $F,G:A\to B$ and $H,K:B\to C$ and natural transformations $\gamma:F\to G$ and $\delta:H\to K$, we have
  \[\id{(\delta G)(H\gamma)}{(K\gamma)(\delta F)}.\]
\end{lem}
\begin{proof}
  It suffices to check componentwise: at $a:A$ we have
  \begin{align*}
    ((\delta G)(H\gamma))_a
    &\jdeq (\delta G)_{a}(H\gamma)_a\\
    &\jdeq \delta_{Ga}\circ H(\gamma_a)\\
    &= K(\gamma_a) \circ \delta_{Fa} \hspace{2cm}\text{(by naturality of $\delta$)}\\
    &\jdeq (K \gamma)_a\circ (\delta F)_a\\
    &\jdeq ((K \gamma)(\delta F))_a.\qedhere
  \end{align*}
\end{proof}

Classically, one defines the ``horizontal composite'' of $\gamma:F\to G$ and $\delta:H\to K$ to be the common value of ${(\delta G)(H\gamma)}$ and ${(K\gamma)(\delta F)}$.
We will refrain from doing this, because while equal, these two transformations are not \emph{definitionally} equal.
This also has the consequence that we can use the symbol $\circ$ (or juxtaposition) for all kinds of composition unambiguously: there is only one way to compose two natural transformations (as opposed to composing a natural transformation with a functor on either side).

\begin{lem}\label{ct:functor-assoc}
  Composition of functors is associative: $\id{H(GF)}{(HG)F}$.
\end{lem}
\begin{proof}
  Since composition of functions is associative, this follows immediately for the actions on objects and on homs.
  And since hom-sets are sets, the rest of the data is automatic.
\end{proof}

The equality in \autoref{ct:functor-assoc} is likewise not definitional.
(Composition of functions is definitionally associative, but the axioms that go into a functor must also be composed, and this breaks definitional associativity.)  For this reason, we need also to know about \emph{coherence} for associativity.

\begin{lem}\label{ct:pentagon}
  \autoref{ct:functor-assoc} is coherent, i.e.\ the following pentagon of equalities commutes:
  \[ \xymatrix{ & K(H(GF)) \ar[dl] \ar[dr]\\
    (KH)(GF) \ar[d] && K((HG)F) \ar[d]\\
    ((KH)G)F && (K(HG))F \ar[ll] }
  \]
\end{lem}
\begin{proof}
  As in \autoref{ct:functor-assoc}, this is evident for the actions on objects, and the rest is automatic.
\end{proof}

We will henceforth abuse notation by writing $H\circ G\circ F$ or $HGF$ for either $H(GF)$ or $(HG)F$, transporting along \autoref{ct:functor-assoc} whenever necessary.
We have a similar coherence result for units.

\begin{lem}\label{ct:units}
  For a functor $F:A\to B$, we have equalities $\id{(1_B\circ F)}{F}$ and $\id{(F\circ 1_A)}{F}$, such that given also $G:B\to C$, the following triangle of equalities commutes.
  \[ \xymatrix{
    G\circ (1_B \circ F) \ar[rr] \ar[dr] &&
    (G\circ 1_B)\circ F \ar[dl] \\
    & G \circ F.}
  \]
\end{lem}

See \autoref{ct:pre2cat,ct:2cat} for further development of these ideas.


\section{Adjunctions}
\label{sec:adjunctions}

\begin{defn}
  A functor $F:A\to B$ is a \textbf{left adjoint} if there exists
  \begin{itemize}
  \item A functor $G:B\to A$.
  \item A natural transformation $\eta:1_A \to GF$.
  \item A natural transformation $\epsilon:FG\to 1_B$.
  \item $\id{(\epsilon F)(F\eta)}{1_F}$.
  \item $\id{(G\epsilon)(\eta G)}{1_G}$.
  \end{itemize}
\end{defn}

\begin{lem}\label{ct:adjprop}
  If $A$ is a category (but $B$ may be only a precategory), then the type ``$F$ is a left adjoint'' is a mere proposition.
\end{lem}
\begin{proof}
  Suppose given $(G,\eta,\epsilon)$ with the triangle identities and also $(G',\eta',\epsilon')$.
  Define $\gamma:G\to G'$ to be $(G'\epsilon)(\eta' G)$, and $\delta:G'\to G$ to be $(G\epsilon')(\eta G')$.
  Then
  \begin{align*}
    \delta\gamma &=
    (G\epsilon')(\eta G')(G'\epsilon)(\eta'G)\\
    &= (G\epsilon')(G F G'\epsilon)(\eta G' F G)(\eta'G)\\
    &= (G\epsilon)(G\epsilon'FG)(G F \eta' G)(\eta G)\\
    &= (G\epsilon)(\eta G)\\
    &= 1_G
  \end{align*}
  using \autoref{ct:interchange} and the triangle identities.
  Similarly, we show $\id{\gamma\delta}{1_{G'}}$, so $\gamma$ is a natural isomorphism $G\cong G'$.
  By \autoref{ct:functor-cat}, we have an identity $\id G {G'}$.

  Now we need to know that when $\eta$ and $\epsilon$ are transported along this identity, they become equal to $\eta'$ and $\epsilon'$.
  By \autoref{ct:idtoiso-trans}, this transport is given by composing with $\gamma$ or $\delta$ as appropriate.
  For $\eta$, this yields
  \begin{equation*}
    (G'\epsilon F)(\eta'GF)\eta
    = (G'\epsilon F)(G'F\eta)\eta'
    = \eta'
  \end{equation*}
  using \autoref{ct:interchange} and the triangle identity.
  The case of $\epsilon$ is similar.
  Finally, the triangle identities transport correctly automatically, since hom-sets are sets.
\end{proof}

In \autoref{sec:yoneda} we will give another proof of \autoref{ct:adjprop}.


\section{Equivalences}
\label{sec:equivalences}

\begin{defn}
  A functor $F:A\to B$ is an \textbf{equivalence of (pre)categories} if it is a left adjoint for which $\eta$ and $\epsilon$ are isomorphisms.
  We write $A\simeq B$ for the type of equivalences of categories from $A$ to $B$.
\end{defn}

By \autoref{ct:adjprop,ct:isoprop}, if $A$ is a category, then the type ``$F$ is an equivalence of precategories'' is a mere proposition.

\begin{lem}\label{ct:adjointification}
  If for $F:A\to B$ there exists $G:B\to A$ and isomorphisms $GF\cong 1_A$ and $FG\cong 1_B$, then $F$ is an equivalence of precategories.
\end{lem}
\begin{proof}
  Just like the ``adjointification'' theorem for equivalences of types.
\end{proof}

\begin{defn}
  We say a functor $F:A\to B$ is \textbf{faithful} if for all $a,b:A$, the function
  \[F_{a,b}:\hom_A(a,b) \to \hom_B(Fa,Fb)\]
  is injective, and \textbf{full} if for all $a,b:A$ this function is surjective.
  If it is both (hence each $F_{a,b}$ is an equivalence) we say $F$ is \textbf{fully faithful}.
\end{defn}

\begin{defn}
  We say a functor $F:A\to B$ is \textbf{split essentially surjective} if for all $b:B$ there exists an $a:A$ such that $Fa\cong b$.
\end{defn}

\begin{lem}\label{ct:ffeso}
  For any precategories $A$ and $B$ and functor $F:A\to B$, the following types are equivalent.
  \begin{enumerate}
  \item $F$ is an equivalence of precategories.\label{item:ct:ffeso1}
  \item $F$ is fully faithful and split essentially surjective.\label{item:ct:ffeso2}
  \end{enumerate}
\end{lem}
\begin{proof}
  Suppose $F$ is an equivalence of precategories, with $G,\eta,\epsilon$ specified.
  Then we have the function
  \begin{equation*}
    \begin{array}{rcl}
      \hom_B(Fa,Fb) &\to& \hom_A(a,b)\\
      g &\mapsto& \inv{\eta_b}\circ G(g)\circ \eta_a.
    \end{array}
  \end{equation*}
  For $f:\hom_A(a,b)$, we have
  \[ \inv{\eta_{b}}\circ G(F(f))\circ \eta_{a}  =
  \inv{\eta_{b}} \circ \eta_{b} \circ f=
  f
  \]
  while for $g:\hom_B(Fa,Fb)$ we have
  \begin{align*}
    F(\inv{\eta_b} \circ G(g)\circ\eta_a)
    &= F(\inv{\eta_b})\circ F(G(g))\circ F(\eta_a)\\
    &= \epsilon_{Fb}\circ F(G(g))\circ F(\eta_a)\\
    &= g\circ\epsilon_{Fa}\circ F(\eta_a)\\
    &= g
  \end{align*}
  using naturality of $\epsilon$, and the triangle identities twice.
  Thus, $F_{a,b}$ is an equivalence, so $F$ is fully faithful.
  Finally, for any $b:B$, we have $Gb:A$ and $\epsilon_b:FGb\cong b$.

  On the other hand, suppose $F$ is fully faithful and split essentially surjective.
  Define $G_0:B_0\to A_0$ by sending $b:B$ to the $a:A$ given by the specified essential splitting, and write $\epsilon_b$ for the likewise specified isomorphism $FGb\cong b$.

  Now for any $g:\hom_B(b,b')$, define $G(g):\hom_A(Gb,Gb')$ to be the unique morphism such that $\id{F(G(g))}{\inv{(\epsilon_{b'})}\circ g \circ \epsilon_b }$ (which exists since $F$ is fully faithful).
  Finally, for $a:A$ define $\eta_a:\hom_A(a,GFa)$ to be the unique morphism such that $\id{F\eta_a}{\inv{\epsilon_{Fa}}}$.
  It is easy to verify that $G$ is a functor and that $(G,\eta,\epsilon)$ exhibit $F$ as an equivalence of precategories.

  Now consider the composite~\ref{item:ct:ffeso1}$\to$\ref{item:ct:ffeso2}$\to$\ref{item:ct:ffeso1}.
  We clearly recover the same function $G_0:B_0 \to A_0$.
  For the action of $G$ on hom-sets, we must show that for $g:\hom_B(b,b')$, $G(g)$ is the (necessarily unique) morphism such that $F(G(g)) = \inv{(\epsilon_{b'})}\circ g \circ \epsilon_b$.
  But this equation holds by the assumed naturality of $\epsilon$.
  We also clearly recover $\epsilon$, while $\eta$ is uniquely characterized by $\id{F\eta_a}{\inv{\epsilon_{Fa}}}$ (which is one of the triangle identities assumed to hold in the structure of an equivalence of precategories).
  Thus, this composite is equal to the identity.

  Finally, consider the other composite~\ref{item:ct:ffeso2}$\to$\ref{item:ct:ffeso1}$\to$\ref{item:ct:ffeso2}.
  Since being fully faithful is a mere proposition, it suffices to observe that we recover, for each $b:B$, the same $a:A$ and isomorphism $F a \cong b$.
  But this is clear, since we used this function and isomorphism to define $G_0$ and $\epsilon$ in~\ref{item:ct:ffeso1}, which in turn are precisely what we used to recover~\ref{item:ct:ffeso2} again.
  Thus, the composites in both directions are equal to identities, hence we have an equivalence \eqv{\ref{item:ct:ffeso1}}{\ref{item:ct:ffeso2}}.
\end{proof}

However, if $B$ is not a category, then neither type in \autoref{ct:ffeso} may necessarily be a mere proposition.
This suggests considering as well the following notions.

\begin{defn}
  A functor $F:A\to B$ is \textbf{essentially surjective} if for all $b:B$, there \emph{merely} exists an $a:A$ such that $Fa\cong b$.
  We say $F$ is a \textbf{weak equivalence} if it is fully faithful and essentially surjective.
\end{defn}

Being a weak equivalence is \emph{always} a mere proposition.
For categories, however, there is no difference between equivalences and weak ones.

\begin{lem}\label{ct:catweq}
  If $F:A\to B$ is fully faithful and $A$ is a category, then for any $b:B$ the type $\sm{a:A} (Fa\cong b)$ is a mere proposition.
  Hence a functor between categories is an equivalence if and only if it is a weak equivalence.
\end{lem}
\begin{proof}
  Suppose given $(a,f)$ and $(a',f')$ in $\sm{a:A} (Fa\cong b)$.
  Then $\inv{f'}\circ f$ is an isomorphism $Fa \cong Fa'$.
  Since $F$ is fully faithful, we have $g:a\cong a'$ with $Fg = \inv{f'}\circ f$.
  And since $A$ is a category, we have $p:a=a'$ with $\idtoiso(p)=g$.
  Now $Fg = \inv{f'}\circ f$ implies $\trans{(\map{(F_0)}{p})}{f} = f'$, hence (by the characterization of equalities in dependent sums) $(a,f)=(a',f')$.

  Thus, for fully faithful functors whose domain is a category, essential surjectivity is equivalent to split essential surjectivity, and so being a weak equivalence is equivalent to being an equivalence.
\end{proof}

This is an important advantage of our category theory over set-based approaches.
With a purely set-based definition of category, the statement ``every fully faithful and essentially surjective functor is an equivalence of categories'' is equivalent to the mere axiom of choice (that is, the strong one, not the trivial one).
Here we have it for free, as a category-theoretic version of the function comprehension principle.
(In fact, this property characterizes categories among precategories; see \autoref{sec:rezk}.)

On the other hand, the following characterization of equivalences of categories is perhaps even more useful.

\begin{defn}\label{ct:isocat}
  A functor $F:A\to B$ is an \textbf{isomorphism of (pre)categories} if $F$ is fully faithful and $F_0:A_0\to B_0$ is an equivalence of types.
\end{defn}

Note that being an isomorphism of precategories is always a mere property.
Let $A\cong B$ denote the type of isomorphisms of (pre)categories from $A$ to $B$.

\begin{lem}\label{ct:isoprecat}
  For precategories $A$ and $B$ and $F:A\to B$, the following are equivalent.
  \begin{enumerate}
  \item $F$ is an isomorphism of precategories.\label{item:ct:ipc1}
  \item There exist $G:B\to A$ and $\eta:1_A = GF$ and $\epsilon:FG=1_B$ such that\label{item:ct:ipc2}
    \begin{equation}
      \map{(\lambda H. F H)}{\eta} = \map{(\lambda K. K F)}{\opp\epsilon}.\label{eq:ct:isoprecattri}
    \end{equation}
  \item There merely exist $G:B\to A$ and $\eta:1_A = GF$ and $\epsilon:FG=1_B$.\label{item:ct:ipc3}
  \end{enumerate}
\end{lem}

Note that if $B_0$ is not a 1-type, then~\eqref{eq:ct:isoprecattri} may not be a mere proposition.

\begin{proof}
  First note that since hom-sets are sets, equalities between equalities of functors are uniquely determined by their object-parts.
  Thus, by function extensionality,~\eqref{eq:ct:isoprecattri} is equivalent to
  \begin{equation}
    \map{(F_0)}{\eta_0}_a = \opp{(\epsilon_0)}_{F_0 a}.\label{eq:ct:ipctri}
  \end{equation}
  for all $a:A_0$.
  Note that this is precisely the triangle identity for $G_0$, $\eta_0$, and $\epsilon_0$ to be a proof that $F_0$ is an equivalence of types.

  Now suppose~\ref{item:ct:ipc1}.
  Let $G_0:B_0 \to A_0$ be the inverse of $F_0$, with $\eta_0: \idfunc[A_0] = G_0 F_0$ and $\epsilon_0:F_0G_0 = \idfunc[B_0]$ satisfying the triangle identity, which is precisely~\eqref{eq:ct:ipctri}.
  Now define $G_{b,b'}:\hom_B(b,b') \to \hom_A(G_0b,G_0b')$ by
  \[ G_{b,b'}(g) \defeq
  \inv{(F_{G_0b,G_0b'})}\Big(\idtoiso(\opp{(\epsilon_0)}_{b'}) \circ g \circ \idtoiso((\epsilon_0)_b)\Big)
  \]
  (using the assumption that $F$ is fully faithful).
  Since \idtoiso takes opposites to inverses and concatenation to composition, and $F$ is a functor, it follows that $G$ is a functor.

  By definition, we have $(GF)_0 \jdeq G_0 F_0$, which is equal to $\idfunc[A_0]$ by $\eta_0$.
  To obtain $1_A = GF$, we need to show that when transported along $\eta_0$, the identity function of $\hom_A(a,a')$ becomes equal to the composite $G_{Fa,Fa'} \circ F_{a,a'}$.
  In other words, for any $f:\hom_A(a,a')$ we must have
  \begin{multline*}
    \idtoiso((\eta_0)_{a'}) \circ f \circ \idtoiso(\opp{(\eta_0)}_a)\\
    = \inv{(F_{GFa,GFa'})}\Big(\idtoiso(\opp{(\epsilon_0)}_{Fa'})
    \circ F_{a,a'}(f) \circ \idtoiso((\epsilon_0)_{Fa})\Big).
  \end{multline*}
  But this is equivalent to
  \begin{multline*}
    (F_{GFa,GFa'})\Big(\idtoiso((\eta_0)_{a'}) \circ f \circ \idtoiso(\opp{(\eta_0)}_a)\Big)\\
    = \idtoiso(\opp{(\epsilon_0)}_{Fa'})
    \circ F_{a,a'}(f) \circ \idtoiso((\epsilon_0)_{Fa}).
  \end{multline*}
  which follows from functoriality of $F$, the fact that $F$ preserves \idtoiso, and~\eqref{eq:ct:ipctri}.
  Thus we have $\eta:1_A = GF$.

  On the other side, we have $(FG)_0\jdeq F_0 G_0$, which is equal to $\idfunc[B_0]$ by $\epsilon_0$.
  To obtain $FG=1_B$, we need to show that when transported along $\epsilon_0$, the identity function of $\hom_B(b,b')$ becomes equal to the composite $F_{Gb,Gb'} \circ G_{b,b'}$.
  That is, for any $g:\hom_B(b,b')$ we must have
  \begin{multline*}
    F_{Gb,Gb'}\Big(\inv{(F_{Gb,Gb'})}\Big(\idtoiso(\opp{(\epsilon_0)}_{b'}) \circ g \circ \idtoiso((\epsilon_0)_b)\Big)\Big)\\
    = \idtoiso((\opp{\epsilon_0})_{b'}) \circ g \circ \idtoiso((\epsilon_0)_b).
  \end{multline*}
  But this is just the fact that $\inv{(F_{Gb,Gb'})}$ is the inverse of $F_{Gb,Gb'}$.
  And we have remarked that~\eqref{eq:ct:isoprecattri} is equivalent to~\eqref{eq:ct:ipctri}, so~\ref{item:ct:ipc2} holds.

  Conversely, suppose given~\ref{item:ct:ipc2}; then the object-parts of $G$, $\eta$, and $\epsilon$ together with~\eqref{eq:ct:ipctri} show that $F_0$ is an equivalence of types.
  And for $a,a':A_0$, we define $\overline{G}_{a,a'}: \hom_B(Fa,Fa') \to \hom_A(a,a')$ by
  \begin{equation}
    \overline{G}_{a,a'}(g) \defeq \idtoiso(\opp{\eta})_{a'} \circ G(g) \circ \idtoiso(\eta)_a.\label{eq:ct:gbar}
  \end{equation}
  By naturality of $\idtoiso(\eta)$, for any $f:\hom_A(a,a')$ we have
  \begin{align*}
    \overline{G}_{a,a'}(F_{a,a'}(f))
    &= \idtoiso(\opp{\eta})_{a'} \circ G(F(f)) \circ \idtoiso(\eta)_a\\
    &= \idtoiso(\opp{\eta})_{a'} \circ \idtoiso(\eta)_{a'} \circ f \\
    &= f.
  \end{align*}
  On the other hand, for $g:\hom_B(Fa,Fa')$ we have
  \begin{align*}
    F_{a,a'}(\overline{G}_{a,a'}(g))
    &= F(\idtoiso(\opp{\eta})_{a'}) \circ F(G(g)) \circ F(\idtoiso(\eta)_a)\\
    &= \idtoiso(\epsilon)_{Fa'}
    \circ F(G(g))
    \circ \idtoiso(\opp{\epsilon})_{Fa}\\
    &= \idtoiso(\epsilon)_{Fa'}
    \circ \idtoiso(\opp{\epsilon})_{Fa'}
    \circ g\\
    &= g.
  \end{align*}
  (There are lemmas needed here regarding the compatibility between \idtoiso and whiskering, which we leave it to the reader to state and prove.)
  Thus, $F_{a,a'}$ is an equivalence, so $F$ is fully faithful; i.e.~\ref{item:ct:ipc1} holds.

  Now the composite~\ref{item:ct:ipc1}$\to$\ref{item:ct:ipc2}$\to$\ref{item:ct:ipc1} is equal to the identity since~\ref{item:ct:ipc1} is a mere proposition.
  On the other side, tracing through the above constructions we see that the composite~\ref{item:ct:ipc2}$\to$\ref{item:ct:ipc1}$\to$\ref{item:ct:ipc2} essentially preserves the object-parts $G_0$, $\eta_0$, $\epsilon_0$, and the object-part of~\eqref{eq:ct:isoprecattri}.
  And in the latter three cases, the object-part is all there is, since hom-sets are sets.

  Thus, it suffices to show that we recover the action of $G$ on hom-sets.
  In other words, we must show that if $g:\hom_B(b,b')$, then
  \[ G_{b,b'}(g) =
  \overline{G}_{G_0b,G_0b'}\Big(\idtoiso(\opp{(\epsilon_0)}_{b'}) \circ g \circ \idtoiso((\epsilon_0)_b)\Big)
  \]
  where $\overline{G}$ is defined by~\eqref{eq:ct:gbar}.
  However, this follows from functoriality of $G$ and the \emph{other} triangle identity, which we have seen in \autoref{cha:equivalences} is equivalent to~\eqref{eq:ct:ipctri}.

  Now since~\ref{item:ct:ipc1} is a mere proposition, so is~\ref{item:ct:ipc2}, so it suffices to show they are co-inhabited with~\ref{item:ct:ipc3}.
  Of course,~\ref{item:ct:ipc2}$\to$\ref{item:ct:ipc3}, so let us assume~\ref{item:ct:ipc3}.
  Since~\ref{item:ct:ipc1} is a mere proposition, we may assume given $G$, $\eta$, and $\epsilon$.
  Then $G_0$ along with $\eta$ and $\epsilon$ imply that $F_0$ is an equivalence.
  Moreover, we also have natural isomorphisms $\idtoiso(\eta):1_A\cong GF$ and $\idtoiso(\epsilon):FG\cong 1_B$, so by \autoref{ct:adjointification}, $F$ is an equivalence of precategories, and in particular fully faithful.
\end{proof}

From \autoref{ct:isoprecat}\ref{item:ct:ipc2} and $\idtoiso$ in functor categories, we conclude immediately that any isomorphism of precategories is an equivalence.
For precategories, the converse can fail.

\begin{eg}\label{ct:chaotic}
  Let $X$ be a type and $x_0:X$ an element, and let $X_{\mathrm{ch}}$ denote the \emph{chaotic} or \emph{indiscrete} precategory on $X$.
  By definition, we have $(X_{\mathrm{ch}})_0\defeq X$, and $\hom_{X_{\mathrm{ch}}}(x,x') = 1$ for all $x,x'$.
  Then the unique functor $X_{\mathrm{ch}}\to 1$ is an equivalence of precategories, but not an isomorphism unless $X$ is contractible.

  This example also shows that a precategory can be equivalent to a category without itself being a category.
  Of course, if a precategory is \emph{isomorphic} to a category, then it must itself be a category.
\end{eg}

However, for categories, the two notions coincide.

\begin{lem}\label{ct:eqv-levelwise}
  For categories $A$ and $B$, a functor $F:A\to B$ is an equivalence of categories if and only if it is an isomorphism of categories.
\end{lem}
\begin{proof}
  Since both are mere properties, it suffices to show they are co-inhabited.
  So first suppose $F$ is an equivalence of categories, with $(G,\eta,\epsilon)$ given.
  We have already seen that $F$ is fully faithful.
  By \autoref{ct:functor-cat}, the natural isomorphisms $\eta$ and $\epsilon$ yield identities $\id{1_A}{GF}$ and $\id{FG}{1_B}$, hence in particular identities $\id{\idfunc[A]}{G_0\circ F_0}$ and $\id{F_0\circ G_0}{\idfunc[B]}$.
Thus, $F_0$ is an equivalence of types.

  Conversely, suppose $F$ is fully faithful and $F_0$ is an equivalence of types, with inverse $G_0$, say.
  Then for each $b:B$ we have $G_0 b:A$ and an identity $\id{FGb}{b}$, hence an isomorphism $FGb\cong b$.
  Thus, by \autoref{ct:ffeso}, $F$ is an equivalence of categories.
\end{proof}

Of course, there is yet a third notion of sameness for (pre)categories: equality.
However, the univalence axiom implies that it coincides with isomorphism.

\begin{lem}\label{ct:cat-eq-iso}
  If $A$ and $B$ are precategories, then the function
  \[(\id A B) \to (A\cong B)\]
  (defined by induction from the identity functor) is an equivalence of types.
\end{lem}
\begin{proof}
  As usual for dependent sum types, to give an element of $\id A B$ is equivalent to giving
  \begin{itemize}
  \item an identity $P_0:\id{A_0}{B_0}$,
  \item for each $a,b:A_0$, an identity
    \[P_{a,b}:\id{\hom_A(a,b)}{\hom_B(\trans {P_0} a,\trans {P_0} b)},\]
  \item identities $\id{\trans {(P_{a,a})} {1_a}}{1_{\trans {P_0} a}}$ and $\id{\trans {(P_{a,c})} {gf}}{\trans {(P_{b,c})} g \circ \trans {(P_{a,b})} f}$.
  \end{itemize}
  (Again, we use the fact that the identity types of hom-sets are mere propositions.)
  However, by univalence, this is equivalent to giving
  \begin{itemize}
  \item an equivalence of types $F_0:\eqv{A_0}{B_0}$,
  \item for each $a,b:A_0$, an equivalence of types
    \[F_{a,b}:\eqv{\hom_A(a,b)}{\hom_B(F_0 (a),F_0 (b))},\]
  \item and identities $\id{F_{a,a}(1_a)}{1_{F_0 (a)}}$ and $\id{F_{a,c}(gf)}{F_{b,c} (g)\circ F_{a,b} (f)}$.
  \end{itemize}
  But this consists exactly of a functor $F:A\to B$ that is an isomorphism of categories.
  And by induction on identity, this equivalence $\eqv{(\id A B)}{(A\cong B)}$ is equal to the one obtained by induction.
\end{proof}

Thus, for categories, equality also coincides with equivalence.
We can interpret this as saying that categories, functors, and natural transformations form, not just a pre-2-category, but a 2-category.

\begin{thm}\label{ct:cat-2cat}
  If $A$ and $B$ are categories, then the function
  \[(\id A B) \to (A\simeq B)\]
  (defined by induction from the identity functor) is an equivalence of types.
\end{thm}
\begin{proof}
  By \autoref{ct:cat-eq-iso,ct:eqv-levelwise}.
\end{proof}

As a consequence, the type of categories is a 2-type.
For since $A\simeq B$ is a subtype of the type of functors from $A$ to $B$, which are the objects of a category, it is a 1-type; hence the identity types $\id A B$ are also 1-types.


\section{The Yoneda lemma}
\label{sec:yoneda}

Recall that we have a category \uset whose objects are sets and whose morphisms are functions.
We now show that every precategory has a \uset-valued hom-functor.
First we need to define opposites and products of (pre)categories.

\begin{defn}
  For a precategory $A$, its \textbf{opposite} $A\op$ is a precategory with the same type of objects, with $\hom_{A\op}(a,b) \defeq \hom_A(b,a)$, and with identities and composition inherited from $A$.
\end{defn}

\begin{defn}
  For precategories $A$ and $B$, their \textbf{product} $A\times B$ is a precategory with $(A\times B)_0 \defeq A_0 \times B_0$ and
  \[\hom_{A\times B}((a,b),(a',b')) \defeq \hom_A(a,a') \times \hom_B(b,b').\]
  Identities are defined by $1_{(a,b)}\defeq (1_a,1_b)$ and composition by $(g,g')(f,f') \defeq ((gf),(g'f'))$.
\end{defn}

\begin{lem}\label{ct:functorexpadj}
  For precategories $A,B,C$, the following types are equivalent.
  \begin{enumerate}
  \item Functors $A\times B\to C$.
  \item Functors $A\to C^B$.
  \end{enumerate}
\end{lem}
\begin{proof}
  Given $F:A\times B\to C$, for any $a:A$ we obviously have a functor $F_a : B\to C$.
  This gives a function $A_0 \to (C^B)_0$.
  Next, for any $f:\hom_A(a,a')$, we have for any $b:B$ the morphism $F_{(a,b),(a',b)}(f,1_b):F_a(b) \to F_{a'}(b)$.
  These are the components of a natural transformation $F_a \to F_{a'}$.
  Functoriality in $a$ is easy to check, so we have a functor $\hat{F}:A\to C^B$.

  Conversely, suppose given $G:A\to C^B$.
  Then for any $a:A$ and $b:B$ we have the object $G(a)(b):C$, giving a function $A_0 \times B_0 \to C_0$.
  And for $f:\hom_A(a,a')$ and $g:\hom_B(b,b')$, we have the morphism
  \begin{equation*}
     G(a')_{b,b'}(g)\circ G_{a,a'}(f)_b = G_{a,a'}(f)_{b'} \circ  G(a)_{b,b'}(g)
  \end{equation*}
  in $\hom_C(G(a)(b), G(a')(b'))$.
  Functoriality is again easy to check, so we have a functor $\check{F}:A\times B \to C$.

  Finally, it is also clear that these operations are inverses.
\end{proof}

Now for any precategory $A$, we have a hom-functor
\[\hom_A : A\op \times A \to \uset.\]
It takes a pair $(a,b): (A\op)_0 \times A_0 \jdeq A_0 \times A_0$ to the set $\hom_A(a,b)$.
For a morphism $(f,f') : \hom_{A\op\times A}((a,b),(a',b'))$, by definition we have $f:\hom_A(a',a)$ and $f':\hom_A(b,b')$, so we can define 
\begin{align*}
  (\hom_A)_{(a,b),(a',b')}(f,f')
  &\defeq (g \mapsto (f'gf))\\
  &: \hom_A(a,b) \to \hom_A(a',b').
\end{align*}
Functoriality is easy to check.

By \autoref{ct:functorexpadj}, therefore, we have an induced functor $\y:A\to \uset^{A\op}$, which we call the \textbf{Yoneda embedding}.

\begin{thm}[The Yoneda lemma]\label{ct:yoneda}
  For any precategory $A$, any $a:A$, and any functor $F:\uset^{A\op}$, we have an isomorphism
  \begin{equation}\label{eq:yoneda}
    \hom_{\uset^{A\op}}(\y a, F) \cong Fa.
  \end{equation}
  Moreover, this is natural in both $a$ and $F$.
\end{thm}
\begin{proof}
  Given a natural transformation $\alpha:\y a \to F$, we can consider the component $\alpha_a : \y a(a) \to F a$.
  Since $\y a(a)\jdeq \hom_A(a,a)$, we have $1_a : \y a(a)$, so that $\alpha_a(1_a) : F a$.
  This gives a function $(\alpha \mapsto \alpha_a(1_a))$ from left to right in~\eqref{eq:yoneda}.

  In the other direction, given $x:F a$, we define $\alpha:\y a \to F$ by
  \[\alpha_{a'}(f) \defeq F_{a',a}(f)(x). \]
  Naturality is easy to check, so this gives a function from right to left in~\eqref{eq:yoneda}.

  To show that these are inverses, first suppose given $x:F a$.
  Then with $\alpha$ defined as above, we have $\alpha_a(1_a) = F_{a,a}(1_a)(x) = 1_{F a}(x) = x$.
  On the other hand, if we suppose given $\alpha:\y a \to F$ and define $x$ as above, then for any $f:\hom_A(a',a)$ we have
  \begin{align*}
    \alpha_{a'}(f)
    &= \alpha_{a'} (\y a_{a',a}(f))\\
    &= (\alpha_{a'}\circ \y a_{a',a}(f))(1_a)\\
    &= (F_{a',a}(f)\circ \alpha_a)(1_a)\\
    &= F_{a',a}(f)(\alpha_a(1_a))\\
    &= F_{a',a}(f)(x).
  \end{align*}
  Thus, both composites are equal to identities.
  We leave the proof of naturality to the reader.
\end{proof}

\begin{cor}\label{ct:yoneda-embedding}
  The Yoneda embedding $\y :A\to \uset^{A\op}$ is fully faithful.
\end{cor}
\begin{proof}
  By \autoref{ct:yoneda}, we have
  \[ \hom_{\uset^{A\op}}(\y a, \y b) \cong \y b(a) \jdeq \hom_A(a,b). \]
  It is easy to check that this isomorphism is in fact the action of \y on hom-sets.
\end{proof}

\begin{cor}\label{ct:yoneda-mono}
  If $A$ is a category, then $\y_0 : A_0 \to (\uset^{A\op})_0$ is a monomorphism.
  In particular, if $\y a = \y b$, then $a=b$.
\end{cor}
\begin{proof}
  By \autoref{ct:yoneda-embedding}, \y induces an isomorphism on sets of isomorphisms.
  But as $A$ and $\uset^{A\op}$ are categories and \y is a functor, this is equivalently an isomorphism on identity types, which is the definition of being mono.
\end{proof}

\begin{defn}\label{ct:representable}
  A functor $F:\uset^{A\op}$ is said to be \textbf{representable} if there exists $a:A$ and an isomorphism $\y a \cong F$.
\end{defn}

\begin{thm}\label{ct:representable-prop}
  If $A$ is a category, then the type ``$F$ is representable'' is a mere proposition.
\end{thm}
\begin{proof}
  By definition ``$F$ is representable'' is just the fiber of $\y_0$ over $F$.
  Since $\y_0$ is mono by \autoref{ct:yoneda-mono}, this fiber is a mere proposition.
\end{proof}

In particular, in a category, any two representations of the same functor are equal.
We can use this to give a different proof of \autoref{ct:adjprop}.
First we give a characterization of adjunctions in terms of representability.

\begin{lem}\label{ct:adj-repr}
  For any precategories $A$ and $B$ and a functor $F:A\to B$, the following types are equivalent.
  \begin{enumerate}
  \item $F$ is a left adjoint.\label{item:ct:ar1}
  \item For each $b:B$, the functor $(a \mapsto \hom_B(Fa,b))$ from $A\op$ to \uset is representable.\label{item:ct:ar2}
  \end{enumerate}
\end{lem}
\begin{proof}
  An element of the type~\ref{item:ct:ar2} consists of a function $G_0:B_0 \to A_0$ together with, for every $a:A$ and $b:B$ an isomorphism
  \[ \gamma_{a,b}:\hom_B(Fa,b) \cong \hom_A(a,G_0 b) \]
  such that $\gamma_{a,b}(g \circ Ff) = \gamma_{a',b}(g)\circ f$ for $f:\hom_{A}(a,a')$.
  
  Given this, for $a:A$ we define $\eta_a \defeq \gamma_{a,Fa}(1_{Fa})$, and for $b:B$ we define $\epsilon_b \defeq \inv{(\gamma_{Gb,b})}(1_{Gb})$.
  Now for $g:\hom_B(b,b')$ we define
  \[ G_{b,b'}(g) \defeq \gamma_{G b, b'}(g \circ \epsilon_b) \]
  The verifications that $G$ is a functor and $\eta$ and $\epsilon$ are natural transformations satisfying the triangle identities are exactly as in the classical case, and as they are all mere propositions we will not care about their values.
  Thus, we have a function~\ref{item:ct:ar2}$\to$\ref{item:ct:ar1}.
  % Then for $a:A$ we have
  % \begin{align*}
  %   \epsilon_{Fa} \circ F(\eta_a)
  %   &\jdeq \inv{(\gamma_{GFa,Fa})}(1_{GFa}) \circ F(\gamma_{a,Fa}(1_{Fa}))\\
  %   &= \inv{(\gamma_{GFa,Fa})} (1_{GFa} \circ \gamma_{a,Fa}(1_{Fa}))\\
  %   &= \inv{(\gamma_{a,Fa})} (\gamma_{a,Fa}(1_{Fa}))\\
  %   &= 1_{Fa}.
  % \end{align*}
  % We have $G(1_b) \jdeq \gamma_{Gb,b}(1_b \circ \epsilon_b) = 1_{Gb}$ by definition of $\epsilon_b$.
  % Before verifying that $G$ preserves composition, we check that $\epsilon$ is natural:
  % \begin{align*}
  %   \epsilon_{b'} \circ FGg
  %   &= \inv{(\gamma_{Gb',b'})}(1_{Gb'}) \circ FGg\\
  %   &= \inv{(\gamma_{Gb,b'})}(1_{Gb'} \circ Gg)\\
  %   &= \inv{(\gamma_{Gb,b'})}(Gg)\\
  %   &= g \circ \epsilon_b.
  % \end{align*}
  % Now we have
  % \begin{align*}
  %   G(g\circ f) &\jdeq \gamma_{G b, b''}(g \circ f \circ \epsilon_b)\\
  %   &= \gamma_{G b, b''}(g \circ \epsilon_{b'} \circ FGf)\\
  %   &= \gamma_{G b', b''}(g \circ \epsilon_{b'}) \circ Gf\\
  %   &= Gg \circ Gf.
  % \end{align*}
  % Thus, $G$ is a functor and $\epsilon$ is natural.

  In the other direction, if $F$ is a left adjoint, we of course have $G_0$ specified, and we can take $\gamma_{a,b}$ to be the composite
  \[ \hom_B(Fa,b)
  \xrightarrow{G_{Fa,b}} \hom_A(GFa,Gb)
  \xrightarrow{(-\circ \eta_a)} \hom_A(a,Gb).
  \]
  This is clearly natural since $\eta$ is, and it has an inverse given by
  \[ \hom_A(a,Gb)
  \xrightarrow{F_{a,Gb}} \hom_B(Fa,FGb)
  \xrightarrow{(\epsilon_b \circ -)} \hom_A(Fa,b)
  \]
  (by the triangle identities).
  Thus we also have~\ref{item:ct:ar1}$\to$~\ref{item:ct:ar2}.

  For the composite~\ref{item:ct:ar2}$\to$\ref{item:ct:ar1}$\to$~\ref{item:ct:ar2}, clearly the function $G_0$ is preserved, so it suffices to check that we get back $\gamma$.
  But the new $\gamma$ is defined to take $f:\hom_B(Fa,b)$ to
  \begin{align*}
    G(f) \circ \eta_a
    &\jdeq \gamma_{G Fa, b}(f \circ \epsilon_{Fa}) \circ \eta_a\\
    &= \gamma_{G Fa, b}(f \circ \epsilon_{Fa} \circ F\eta_a)\\
    &= \gamma_{G Fa, b}(f)
  \end{align*}
  so it agrees with the old one.

  Finally, for~\ref{item:ct:ar1}$\to$\ref{item:ct:ar2}$\to$~\ref{item:ct:ar1}, we certainly get back the functor $G$ on objects.
  The new $G_{b,b'}:\hom_B(b,b') \to \hom_A(Gb,Gb')$ is defined to take $g$ to
  \begin{align*}
    \gamma_{G b, b'}(g \circ \epsilon_b)
    &\jdeq G(g \circ \epsilon_b) \circ \eta_{Gb}\\
    &= G(g) \circ G\epsilon_b \circ \eta_{Gb}\\
    &= G(g)
  \end{align*}
  so it agrees with the old one.
  The new $\eta_a$ is defined to be $\gamma_{a,Fa}(1_{Fa}) \jdeq G(1_{Fa}) \circ \eta_a$, so it equals the old $\eta_a$.
  And finally, the new $\epsilon_b$ is defined to be $\inv{(\gamma_{Gb,b})}(1_{Gb}) \jdeq \epsilon_b \circ F(1_{Gb})$, which also equals the old $\epsilon_b$.
\end{proof}

\begin{cor}\label{ct:adjprop2}[\autoref{ct:adjprop}]
  If $A$ is a category and $F:A\to B$, then the type ``$F$ is a left adjoint'' is a mere proposition.
\end{cor}
\begin{proof}
  By \autoref{ct:representable-prop}, if $A$ is a category then the type in \autoref{ct:adj-repr}\ref{item:ct:ar2} is a mere proposition.
\end{proof}


\section{Strict categories}
\label{sec:strict-categories}

\begin{defn}
  A \textbf{strict category} is a precategory whose type of objects is a set.
\end{defn}

In accordance with the mathematical red herring principle, a strict category is not necessarily a category.
In fact, a category is a strict category precisely when it is gaunt (\autoref{ct:gaunt}).
Most of the time, category theory is about categories, not strict ones, but sometimes one wants to consider strict categories.
The main advantage of this is that strict categories have a stricter notion of ``sameness'' than equivalence, namely isomorphism (or equivalently, by \autoref{ct:cat-eq-iso}, equality).

Here is one origin of strict categories.

\begin{eg}
  Let $A$ be a precategory and $x:A$ an object.
  Then there is a precategory $\mathsf{mono}(A,x)$ as follows:
  \begin{itemize}
  \item Its objects consist of an object $y:A$ and a monomorphism $m:\hom_A(y,x)$.
  \item Its morphisms from $(y,m)$ to $(z,n)$ are arbitrary morphisms from $y$ to $z$ in $A$ (not necessarily respecting $m$ and $n$).
  \end{itemize}
  An equality $(y,m)=(z,n)$ of objects in $\mathsf{mono}(A,x)$ consists of an equality $p:y=z$ and an equality $\trans{p}{m}=n$, which by \autoref{ct:idtoiso-trans} is equivalently an equality $m=n\circ \idtoiso(p)$.
  Since hom-sets are sets, the type of such equalities is a mere proposition.
  But since $m$ and $n$ are monomorphisms, the type of morphisms $f$ such that $m = n\circ f$ is also a mere proposition.
  Thus, if $A$ is a category, then $(y,m)=(z,n)$ is a mere proposition, and hence $\mathsf{mono}(A,x)$ is a strict category.
\end{eg}

This example can be dualized, and generalized in various ways.
Here is an interesting application of strict categories.

\begin{eg}\label{ct:galois}
  Let $E/F$ be a finite Galois extension of fields, and $G$ its Galois group.
  Then there is a strict category whose objects are intermediate fields $F\subseteq K\subseteq E$, and whose morphisms are field homomorphisms which fix $F$ pointwise (but need not commute with the inclusions into $E$).
  There is another strict category whose objects are subgroups $H\subseteq G$, and whose morphisms are morphisms of $G$-sets $G/H \to G/K$.
  The fundamental theorem of Galois theory says that these two precategories are isomorphic (not merely equivalent).
\end{eg}


\section{\texorpdfstring{$\dagger$}{†}-categories}
\label{sec:dagger-categories}

It is also worth mentioning a useful kind of precategory whose type of objects is not a set, but which is not a category either.

\begin{defn}
  A \textbf{$\dagger$-precategory} is a precategory $A$ together with the following.
  \begin{enumerate}
  \item For each $x,y:A$, a function $\dgr{(-)}:\hom_A(x,y) \to \hom_A(y,x)$.
  \item For all $x:A$, we have $\dgr{(1_x)} = 1_x$.
  \item For all $f,g$ we have $\dgr{(g\circ f)} = \dgr f \circ \dgr g$.
  \item For all $f$ we have $\dgr{(\dgr f)} = f$.
  \end{enumerate}
\end{defn}

\begin{defn}
  A morphism $f:\hom_A(x,y)$ in a $\dagger$-precategory is \textbf{unitary} if $\dgr f \circ f = 1_x$ and $f\circ \dgr f = 1_y$.
\end{defn}

Of course, every unitary morphism is an isomorphism, and being unitary is a mere proposition.
Thus for each $x,y:A$ we have a set of unitary isomorphisms from $x$ to $y$, which we denote $(x\unitaryiso y)$.

\begin{lem}\label{ct:idtounitary}
  If $p:(x=y)$, then $\idtoiso(p)$ is unitary.
\end{lem}
\begin{proof}
  By induction, we may assume $p$ is $\refl x$.
  But then $\dgr{(1_x)} \circ 1_x = 1_x\circ 1_x = 1_x$ and similarly.
\end{proof}

\begin{defn}
  A \textbf{$\dagger$-category} is a $\dagger$-precategory such that for all $x,y:A$, the function
  \[ (x=y) \to (x \unitaryiso y) \]
  from \autoref{ct:idtounitary} is an equivalence.
\end{defn}

\begin{eg}
  The category \urel from \autoref{ct:rel} becomes a $\dagger$-precategory if we define $(\dgr R)(y,x) \defeq R(x,y)$.
  The proof that \urel is a category actually shows that every isomorphism is unitary; hence \urel is also a $\dagger$-category.
\end{eg}

\begin{eg}
  Any groupoid becomes a $\dagger$-category if we define $\dgr f \defeq \inv{f}$.
\end{eg}

\begin{eg}\label{ct:hilb}
  Let \uhilb be the following precategory.
  \begin{itemize}
  \item Its objects are finite-dimensional vector spaces equipped with an inner product $\langle -,-\rangle$.
  \item Its morphisms are arbitrary linear maps.
  \end{itemize}
  By standard linear algebra, any linear map $f:V\to W$ between finite dimensional inner product spaces has a uniquely defined adjoint $\dgr f:W\to V$, characterized by $\langle fv,w\rangle = \langle v,\dgr f w\rangle$.
  In this way, \uhilb becomes a $\dagger$-precategory.
  Moreover, a linear isomorphism is unitary precisely when it is an \emph{isometry}, i.e.\ $\langle fv,fw\rangle = \langle v,w\rangle$.
  It follows from this that \uhilb is a $\dagger$-category, though it is not a category (not every linear isomorphism is unitary).
\end{eg}

There has been a good deal of general theory developed for $\dagger$-categories under classical foundations.
It was observed early on that the unitary isomorphisms, not arbitrary isomorphisms, are the correct notion of ``sameness'' for objects of a $\dagger$-category, which has caused some consternation among category theorists.
Homotopy type theory resolves this issue by identifying $\dagger$-categories, like strict categories, as simply a different kind of precategory.


\section{The Structure Identity Principle}
\label{sec:sip}

The \emph{Structure Identity Principle (SIP)} is an informal principle
that expresses that isomorphic structures are identical.  We aim to
prove a general abstract result which can be applied to a wide family
of notions of structure, where structures may be many-sorted or even
dependently-sorted, infinitary, or even higher order.

The simplest kind of single-sorted structure consists of a type with
no additional structure.  The Univalence Axiom expresses SIP for that
notion of structure in a strong form: for types $A,B$, the
canonical function $(A=B)\to (A\simeq B)$ is an equivalence.

We start with a precategory $X$.  In application to SIP for
single-sorted first order structures, $X$ will be the category %\uset%
of $\bbU$-small sets, where $\bbU$ is a univalent type universe.

\begin{defn}\label{ct:sig}
  A \textbf{notion of structure} $(P,H)$ over $X$ consists of the following.
  \begin{enumerate}
  \item A function $P:X_0 \to \type$.
    For each $x:X_0$ the elements of $Px$ are called \emph{$(P,H)$-structures} on $x$.
  \item For $x,y:X_0$ and $\alpha:Px$, $\;\beta:Py$, to each $f:\hom_X(x,y)$ a mere proposition 
  \[ H_{\alpha\beta}(f).\]
    If $H_{\alpha\beta}(f)$ is true, we say that $f$ is a \emph{$(P,H)$-homomorphism} from $\alpha$ to $\beta$.
  \item For $x:X_0$ and $\alpha:Px$, we have $H_{\alpha\alpha}(1_x)$.\label{item:sigid}
  \item For $x,y,z:X_0$ and $\alpha:Px$, $\;\beta:Py$, $\;\gamma:Pz$, 
if $f:\hom_X(x,y)$, we have 
  \[ H_{\alpha\beta}(f)\to H_{\beta\gamma}(g)\to H_{\alpha\gamma}(g\circ   f).\]\label{item:sigcmp}  
   \end{enumerate}
  When $(P,H)$ is a notion of structure, for $\alpha,\beta:Px$ we define
  \[ \alpha\leq_x\beta \defeq H_{\alpha\beta}(1_x).\]
  By~\ref{item:sigid} and~\ref{item:sigcmp}, this is a preorder (\autoref{ct:orders}) with $Px$ its type of objects.
  We say that $(P,H)$ is a \textbf{standard notion of structure} if this preorder is in fact a partial order, for all $x:X$.
\end{defn}

Note that for a standard notion of structure, each type $Px$ must actually be a set.
We now define, for any notion of structure $(P,H)$, a precategory of \textbf{$(P,H)$-structures}, $A = \mathsf{Str}_{(P,H)}(X)$.
\begin{itemize}
\item The type of objects of $A$ is the type $A_0 \defeq \sm{x:X} Px$.
  If $a\jdeq (x,\alpha):A_0$, we may write $|a| \defeq x$.
\item For $(x,\alpha):A_0$ and $(y,\beta):A_0$, we define
  \[\hom_A((x,\alpha),(y,\beta)) \defeq \Big\{f:x \to y \;\Big|\; H_{\alpha\beta}(f)\Big\}.\]
\end{itemize}
The composition and identities are inherited from $X$; conditions~\ref{item:sigid} and~\ref{item:sigcmp} ensure that these lift to $A$.

Our abstract form of SIP is the following result.

\begin{thm}[Abstract SIP Theorem]\label{thm:sip}
  If $X$ is a category and $(P,H)$ is a standard notion of structure over $X$, then the precategory $\mathsf{Str}_{(P,H)}(X)$ is a category.
\end{thm}
\begin{proof}
  By the definition of equality in sum types, to give an equality $(x,\alpha)=(y,\beta)$ consists of
  \begin{itemize}
  \item An equality $p:x=y$, and
  \item An equality $\trans{p}{\alpha}=\beta$.
  \end{itemize}
  Since $P$ is set-valued, the latter is a mere proposition.
  On the other hand, it is easy to see that an isomorphism $(x,\alpha)\cong (y,\beta)$ in $\mathsf{Str}_{(P,H)}(X)$ consists of
  \begin{itemize}
  \item An isomorphism $f:x\cong y$ in $X$, such that
  \item $H_{\alpha\beta}(f)$ and $H_{\beta\alpha}(\inv f)$.
  \end{itemize}
  Of course, the second of these is also a mere proposition.
  And since $X$ is a category, the function $(x=y) \to (x\cong y)$ is an equivalence.
  Thus, it will suffice to show that for any $p:x=y$ and for any $(\alpha:Px)$, $(\beta:Py)$, we have $\trans{p}{\alpha}=\beta$ if and only if both  $H_{\alpha\beta}(\idtoiso (p))$ and $H_{\beta\alpha}(\inv{\idtoiso(p)})$.

  The ``only if'' direction is just the existence of the function $\idtoiso$ for the category $\mathsf{Str}_{(P,H)}(X)$.
  For the ``if'' direction, by induction on $p$ we may assume that $y\jdeq x$ and $p\jdeq\refl x$.
  However, in this case $\idtoiso (p)\jdeq 1_x$ and therefore $\inv{\idtoiso(p)}=1_x$.
  Thus, $\alpha\leq_x \beta$ and $\beta\leq_x \alpha$, which implies $\alpha=\beta$ since $(P,H)$ is a standard notion of structure.
\end{proof}

As an example, this methodology gives an alternative way to express the proof of \autoref{ct:functor-cat}.

\begin{eg}\label{ct:sip-functor-cat}
  Let $A$ be a precategory and $B$ a category.
  There is a precategory $B^{A_0}$ whose objects are functions $A_0 \to B_0$, and whose set of morphisms from $F_0:A_0 \to B_0$ to $G_0:A_0 \to B_0$ is $\prd{a:A_0} \hom_B(F_0 a, G_0 a)$.
  Composition and identities are inherited directly from those in $B$.
  It is easy to show that $\gamma:\hom_{B^{A_0}}(F_0, G_0)$ is an isomorphism exactly when each component $\gamma_a$ is an isomorphism, so that we have $\eqv{(F_0 \cong G_0)}{\prd{a:A_0} (F_0 a \cong G_0 a)}$.
  Moreover, the map $\idtoiso : (F_0 = G_0) \to (F_0 \cong G_0)$ of $B^{A_0}$ is equal to the composite
  \[ (F_0 = G_0) \to \prd{a:A_0} (F_0 a  = G_0 a) \to \prd{a:A_0} (F_0 a \cong G_0 a) \to (F_0 \cong G_0) \]
  in which the first map is an equivalence by function extensionality, the second because it is a dependent product of equivalences (since $B$ is a category), and the third as remarked above.
  Thus, $B^{A_0}$ is a category.

  Now we define a notion of structure on $B^{A_0}$ for which $P(F_0)$ is the type of operations $F:\prd{a,a':A_0} \hom_A(a,a') \to \hom_B(F_0 a,F_0 a')$ which extend $F_0$ to a functor (i.e.\ preserve composition and identities).
  This is a set since each $\hom_B(-,-)$ is so.
  Given such $F$ and $G$, we define $\gamma:\hom_{B^{A_0}}(F_0, G_0)$ to be a homomorphism if it forms a natural transformation.
  In \autoref{ct:functor-precat} we essentially verified that this is a notion of structure.
  Moreover, if $F$ and $F'$ are both structures on $F_0$ and the identity is a natural transformation from $F$ to $F'$, then for any $f:\hom_A(a,a')$ we have $F'f = F'f \circ 1_{F_0 a} = 1_{F_0 a}\circ F f = F f$.
  Applying function extensionality, we conclude $F = F'$.
  Thus, we have a \emph{standard} notion of structure, and so by \autoref{thm:sip}, the precategory $B^A$ is a category.
\end{eg}

As another example, we consider categories of structures first-order for a first-order (FO) signature.
We define an \textbf{FO-signature}, $\Omega$, to consist of sets $\Omega_0$ and $\Omega_1$ of function symbols, $\omega:\Omega_0$, and relation symbols, $\omega:\Omega_1$, each having an arity $|\omega|$ that is a set.
An \textbf{$\Omega$-structure} $a$ consists of a set $|a|$ together with an assignment of an $|\omega|$-ary function $\omega^a:|a|^{|\omega|}\to |a|$ on $|a|$ to each function symbol, $\omega$, and an assignment of an $|\omega|$-ary relation $\omega^a$ on $|a|$, assigning a mere proposition $\omega^ax$ to each $x:|a|^{|\omega|}$, to each relation symbol.
And given $\Omega$-structures $a,b$, a function $f:|a|\to |b|$ is a \textbf{homomorphism $a\to b$} if it preserves the structure; i.e.\ if for each symbol $\omega$ of the signature and each $x:|a|^{|\omega|}$,
\begin{enumerate}
\item $f(\omega^ax) = \omega^b(f\circ x)$ if $\omega:\Omega_0$, and
\item $\omega^ax\to\omega^b(f\circ x)$ if $\omega:\Omega_1$.
\end{enumerate}
Note that each $x:|a|^{|\omega|}$ is a function $x:|\omega|\to |a|$ so that $f\circ x : b^\omega$.

Now we assume given a univalent universe $\bbU$ and a $\bbU$-small signature $\Omega$; i.e. $|\Omega|$ is a $\bbU$-small set and, for each $\omega:|\Omega|$, the set $|\omega|$ is $\bbU$-small.
We form the precategory $\uset_\bbU$ of $\bbU$-small sets, which is a category because $\bbU$ is univalent.  We want to define the precategory of $\bbU$-small $\Omega$-structures over $\uset_\bbU$ and use the abstract SIP theorem to show that it is a category.

We use the FO-signature $\Omega$ to give us a standard notion of structure $(P,H)$ over $\uset_\bbU$.  

%%%%%%%%%%%%%%%%%%%%%%%%%%%%%%%%%%%%%%%%%%%%%
%We assume given a function $fsymbol :|\Omega|\to\bbB$, where $\bbB$ is the %type of boolean values, having canonical elements $\bbt,\bbf$, so that if %$\omega :\Omega$ then we take
% \begin{center}
% ($\omega$ is a function symbol) 
% \end{center}
% to be the proposition $(fsymbol(x)=\bbt)$.  
%%%%%%%%%%%%%%%%%%%%%%%%%%%%%%%%%%%%%%%%%%%%%%%%%%%
\begin{defn}\label{defn:fo-notion-of-structure}
$\;$
\begin{enumerate}
\item For each $\bbU$-small set $x$ define 
  \[ Px := P_0x\times P_1x.\]  
Here
  \[ \begin{array}{ll}
 P_0x &:= \Pi_{\omega:\Omega_0}\; x^{|\omega|}\to x,
\mbox{ and } \\
     P_1x &:= \Pi_{\omega:\Omega_1}\; x^{|\omega|}\to \bbP_\bbU,
  \end{array}\]
where $\bbP_\bbU$ is the set of $\bbU$-small propositions.
\item For $\bbU$-small sets $x,y$ and 
$\alpha:P^\omega x,\;\beta:P^\omega y,\; f:x\to y$, define
  \[ H_{\alpha\beta}(f) := H_{0,\alpha\beta}(f)\wedge H_{1,\alpha\beta}(f).\]
Here
  \[ \begin{array}{ll}
H_{0,\alpha\beta}(f) &:= 
\forall_{\omega:\Omega_0}\forall_{u:x^{|\omega|}}\;\; f(\alpha u)=\;\beta(f\circ u),
\mbox{ and }\\
H_{1,\alpha\beta}(f) &:= 
  \forall_{\omega:\Omega_1}\forall_{u:x^{|\omega|}}\;\;\;\;\;\;\;\alpha u\to\beta(f\circ u).
  \end{array}\]


% {%%%%%%%%%%%%%%%%%%%%%%%%%%%%%%%%%
% \item Let $\omega:|\Omega|$
% \begin{enumerate}
% \item
% For each $\bbU$-small set $x$ define
%   \[ P^\omega x := \left\{
%          \begin{array}{ll}
% x^{|\omega|}\to x &\mbox{ if } fsymbol(x)=\bbt\\
% x^{|\omega|}\to \bbP_\bbU &\mbox{ if } fsymbol(x)=\bbf
%          \end{array}\right.
%   \]
% where $\bbP_\bbU$ is the set of $\bbU$-small propositions, and
% \item For $\bbU$-small sets $x,y$ and 
% $\alpha:P^\omega x,\;\beta:P^\omega y,\; f:x\to y$, define
%   \[ H^\omega_{\alpha\beta}(f) := \left\{
%          \begin{array}{ll}
% \forall_{u:x^{|\omega|}}\; [f(\alpha u)=\beta(f\circ u)]  &\mbox{ if } fsymbol(x)=\bbt\\
% \forall_{u:x^{|\omega|}}\; [\alpha u\to\beta(f\circ u)] &\mbox{ if } fsymbol(x)=\bbf
%          \end{array}\right.
%   \]
% \end{enumerate}
% \item 
% \begin{enumerate}
% \item For each $\bbU$-small set $x$ define $Px := \Pi_{\omega :|\Omega|}P^\omega x$, and
% \item for $\bbU$-small sets $x,y$ and $\alpha :Px,\; \beta :Py,\; f:x\to y$
% define 
%   \[ H_{\alpha\beta}(f) := \forall_{\omega :|\Omega|}\; H^\omega_{(\alpha\omega)(\beta\omega)}(f).\]
% \end{enumerate}
% }%%%%%%%%%%%%%%%%%%%%%%%%%%%%%%%%%%%%%%%%%%%%%%%%
\end{enumerate}
\end{defn}
It is now routine to check that $(P,H)$ is a standard notion of structure over $\uset_\bbU$ and hence we may use the Abstract SIP Theorem to get that the precategory $Str_{(P,H)}(\uset_\bbU)$ is a category.  It only remains to observe that this is essentially the same as the precategory of $\bbU$-small $\Omega$-structures over $\uset_\bbU$.


\section{The Rezk completion}
\label{sec:rezk}

In this section we will give a universal way to replace a precategory by a category.
In fact, we will give two.
Both rely on the fact that ``categories see weak equivalences as equivalences''.

To prove this, we begin with a couple of lemmas which are completely standard category theory, phrased carefully so as to make sure we are using the eliminator for $\truncf{-1}$ correctly.
One would have to be similarly careful in classical category theory if one wanted to avoid the axiom of choice: any time we want to define a function, we need to characterize its values uniquely somehow.

\begin{lem}
  If $A,B,C$ are precategories and $H:A\to B$ is an essentially surjective functor, then $(-\circ H):C^B \to C^A$ is faithful.
\end{lem}
\begin{proof}
  Let $F,G:B\to C$, and $\gamma,\delta:F\to G$ be such that $\gamma H = \delta H$; we must show $\gamma=\delta$.
  Thus let $b:B$; we want to show $\gamma_b=\delta_b$.
  This is a mere proposition, so since $H$ is essentially surjective, we may assume given an $a:A$ and an isomorphism $f:Ha\cong b$.
  But now we have
  \[ \gamma_b = G(f) \circ \gamma_{Ha} \circ F(\inv{f}) 
  = G(f) \circ \delta_{Ha} \circ F(\inv{f})
  = \delta_b.\qedhere
  \]
\end{proof}

\begin{lem}
  If $A,B,C$ are precategories and $H:A\to B$ is essentially surjective and full, then $(-\circ H):C^B \to C^A$ is fully faithful.
\end{lem}
\begin{proof}
  It remains to show fullness.
  Thus, let $F,G:B\to C$ and $\gamma:FH \to GH$.
  We claim that for any $b:B$, the type
  \begin{equation}\label{eq:fullprop}
    \sm{g:\hom_C(Fb,Gb)} \prd{a:A}{f:Ha\cong b} (\gamma_a =  \inv{Gf}\circ g\circ Ff)
  \end{equation}
  is contractible.
  Since contractibility is a mere property, and $H$ is essentially surjective, we may assume given $a_0:A$ and $h:Ha_0\cong b$.

  Now take $g\defeq Gh \circ \gamma_{a_0} \circ \inv{Fh}$.
  Then given any other $a:A$ and $f:Ha\cong b$, we must show $\gamma_a =  \inv{Gf}\circ g\circ Ff$.
  Since $H$ is full, there merely exists a morphism $k:\hom_A(a,a_0)$ such that $Hk = \inv{h}\circ f$.
  And since our goal is a mere proposition, we may assume given some such $k$.
  Then we have
  \begin{align*}
    \gamma_a &= \inv{GHk}\circ \gamma_{a_0} \circ FHk\\
    &= \inv{Gf} \circ Gh \circ \gamma_{a_0} \circ \inv{Fh} \circ Ff\\
    &= \inv{Gf}\circ g\circ Ff.
  \end{align*}
  Thus,~\eqref{eq:fullprop} is inhabited.
  It remains to show it is a mere proposition.
  Let $g,g':\hom_C(Fb, Gb)$ be such that for all $a:A$ and $f:Ha\cong b$, we have both $(\gamma_a =  \inv{Gf}\circ g\circ Ff)$ and $(\gamma_a =  \inv{Gf}\circ g'\circ Ff)$.
  The dependent product types are mere propositions, so all we have to prove is $g=g'$.
  But this is a mere proposition, so we may assume $a_0:A$ and $h:Ha_0\cong b$, in which case we have
  \[ g = Gh \circ \gamma_{a_0} \circ \inv{Fh} = g'.\]

  This proves that~\eqref{eq:fullprop} is contractible for all $b:B$.
  Now we define $\delta:F\to G$ by taking $\delta_b$ to be the unique $g$ in~\eqref{eq:fullprop} for that $b$.
  To see that this is natural, suppose given $f:\hom_B(b,b')$; we must show $Gf \circ \delta_b = \delta_{b'}\circ Ff$.
  As before, we may assume $a:A$ and $h:Ha\cong b$, and likewise $a':A$ and $h':Ha'\cong b'$.
  Since $H$ is full as well as essentially surjective, we may also assume $k:\hom_A(a,a')$ with $Hk = \inv{h'}\circ f\circ h$.

  Since $\gamma$ is natural, $GHk\circ \gamma_a = \gamma_{a'} \circ FHk$.
  Using the definition of $\delta$, we have
  \begin{align*}
    Gf \circ \delta_b
    &= Gf \circ Gh \circ \gamma_a \circ \inv{Fh}\\
    &= Gh' \circ GHk\circ \gamma_a \circ \inv{Fh}\\
    &= Gh' \circ \gamma_{a'} \circ FHk \circ \inv{Fh}\\
    &= Gh' \circ \gamma_{a'} \circ \inv{Fh'} \circ Ff\\
    &= \delta_{b'} \circ Ff.
  \end{align*}
  Thus, $\delta$ is natural.
  Finally, for any $a:A$, applying the definition of $\delta_{Ha}$ to $a$ and $1_a$, we obtain $\gamma_a = \delta_{Ha}$.
  Hence, $\delta \circ H = \gamma$.
\end{proof}

The rest of the theorem follows almost exactly the same lines, with the category-ness of $C$ inserted in one crucial step, which we have bolded below for emphasis.
This is the point at which we are trying to define a function into \emph{objects} without using choice, and so we must be careful about what it means for an object to be ``uniquely specified''.
In classical category theory, all one can say is that this object is specified up to unique isomorphism, but in set-theoretic foundations this is not a sufficient amount of uniqueness to give us a function without invoking AC.
In univalent foundations, however, if $C$ is a category, then isomorphism is equality, and we have the appropriate sort of uniqueness (namely, living in a contractible space).

\begin{thm}\label{ct:cat-weq-eq}
  If $A,B$ are precategories, $C$ is a category, and $H:A\to B$ is a weak equivalence, then $(-\circ H):C^B \to C^A$ is an isomorphism.
\end{thm}
\begin{proof}
  By \autoref{ct:functor-cat}, $C^B$ and $C^A$ are categories.
  Thus, by \autoref{ct:eqv-levelwise} it will suffice to show that $(-\circ H)$ is an equivalence.
  But since we know from the preceeding two lemmas that it is fully faithful, by \autoref{ct:catweq} it will suffice to show that it is essentially surjective.
  Thus, suppose $F:A\to C$; we want there to merely exist a $G:B\to C$ such that $GH\cong F$.

  For each $b:B$, let $X_b$ be the type whose elements consist of:
  \begin{enumerate}
  \item An element $c:C$; and
  \item For each $a:A$ and $h:Ha\cong b$, an isomorphism $k_{a,h}:Fa\cong c$; such that\label{item:eqvprop2}
  \item For each $(a,h)$ and $(a',h')$ as in~\ref{item:eqvprop2} and each $f:\hom_A(a,a')$ such that $h'\circ Hf = h$, we have $k_{a',h'}\circ Ff = k_{a,h}$.\label{item:eqvprop3}
  \end{enumerate}
  We claim that for any $b:B$, the type $X_b$ is contractible.
  As this is a mere proposition, we may assume given $a_0:A$ and $h_0:Ha_0 \cong b$.
  Let $c^0\defeq Fa_0$.
  Next, given $a:A$ and $h:Ha\cong b$, since $H$ is fully faithful there is a unique isomorphism $g_{a,h}:a\to a_0$ with $Hg_{a,h} = \inv{h_0}\circ h$; define $k^0_{a,h} \defeq Fg_{a,h}$.
  Finally, if $h'\circ Hf = h$, then $\inv{h_0}\circ h'\circ Hf = \inv{h_0}\circ h$, hence $g_{a',h'} \circ f = g_{a,h}$ and thus $k^0_{a',h'}\circ Ff = k^0_{a,h}$.
  Therefore, $X_b$ is inhabited.

  Now suppose given another $(c^1,k^1): X_b$.
  Then $k^1_{a_0,h_0}:c^0 \jdeq Fa_0 \cong c^1$.
  \textbf{Since $C$ is a category, we have $p:c^0=c^1$ with $\idtoiso(p) = k^1_{a_0,h_0}$.}
  And for any $a:A$ and $h:Ha\cong b$, by~\ref{item:eqvprop3} for $(c^1,k^1)$ with $f\defeq g_{a,h}$, we have
  \[k^1_{a,h} = k^1_{a_0,h_0} \circ k^0_{a,h} = \trans{p}{k^0_{a,h}}\]
  This gives the requisite data for an equality $(c^0,k^0)=(c^1,k^1)$, completing the proof that $X_b$ is contractible.

  Now since $X_b$ is contractible for each $b$, the type $\prd{b:B} X_b$ is also contractible.
  In particular, it is inhabited, so we have a function assigning to each $b:B$ a $c$ and a $k$.
  Define $G_0(b)$ to be this $c$; this gives a function $G_0 :B_0 \to C_0$.

  Next we need to define the action of $G$ on morphisms.
  For each $b,b':B$ and $f:\hom_B(b,b')$, let $Y_f$ be the type whose elements consist of:
  \begin{enumerate}[resume]
  \item A morphism $g:\hom_C(Gb,Gb')$, such that
  \item For each $a:A$ and $h:Ha\cong b$, and each $a':A$ and $h':Ha'\cong b'$, and any $\ell:\hom_A(a,a')$, we have\label{item:eqvprop5}
    \[ (h' \circ H\ell = f \circ h)
    \to
    (k_{a',h'} \circ F\ell = g\circ k_{a,h}). \]
  \end{enumerate}
  We claim that for any $b,b'$ and $f$, the type $Y_f$ is contractible.
  As this is a mere proposition, we may assume given $a_0:A$ and $h_0:Ha_0\cong b$, and each $a'_0:A$ and $h'_0:Ha'_0\cong b'$.
  Then since $H$ is fully faithful, there is a unique $\ell_0:\hom_A(a_0,a_0')$ such that $h'_0 \circ H\ell_0 = f \circ h_0$.
  Define $g_0 \defeq k_{a_0',h_0'} \circ F \ell_0 \circ \inv{(k_{a_0,h_0})}$.

  Now for any $a,h,a',h'$, and $\ell$ such that $(h' \circ H\ell = f \circ h)$, we have $\inv{h}\circ h_0:Ha_0\cong Ha$, hence there is a unique $m:a_0\cong a$ with $Hm = \inv{h}\circ h_0$ and hence $h\circ Hm = h_0$.
  Similarly, we have a unique $m':a_0'\cong a'$ with $h'\circ Hm' = h_0'$.
  Now by~\ref{item:eqvprop3}, we have $k_{a,h}\circ Fm = k_{a_0,h_0}$ and $k_{a',h'}\circ Fm' = k_{a_0',h_0'}$.
  We also have
  \begin{align*}
    Hm' \circ H\ell_0 
    &= \inv{(h')} \circ h_0' \circ H\ell_0\\
    &= \inv{(h')} \circ f \circ h_0\\
    &= \inv{(h')} \circ f \circ h \circ \inv{h} \circ h_0\\
    &= H\ell \circ Hm
  \end{align*}
  and hence $m'\circ \ell_0 = \ell\circ m$ since $H$ is fully faithful.
  Finally, we can compute
  \begin{align*}
    g_0 \circ k_{a,h}
    &= k_{a_0',h_0'} \circ F \ell_0 \circ \inv{(k_{a_0,h_0})} \circ k_{a,h}\\
    &= k_{a_0',h_0'} \circ F \ell_0 \circ \inv{Fm}\\
    &= k_{a_0',h_0'} \circ \inv{(Fm')} \circ F\ell\\
    &= k_{a',h'}\circ F\ell.
  \end{align*}
  Whew!  We've shown that $Y_f$ is inhabited.
  To show it is contractible, since hom-sets are sets, it thankfully suffices to take another $g_1:\hom_C(Gb,Gb')$ satisfying~\ref{item:eqvprop5} and show $g_0=g_1$.
  However, we still have our specified $a_0,h_0,a_0',h_0',\ell_0$ around, and~\ref{item:eqvprop5} implies both $g_0$ and $g_1$ must be equal to $k_{a_0',h_0'} \circ F \ell_0 \circ \inv{(k_{a_0,h_0})}$.

  This completes the proof that $Y_f$ is contractible for each $b,b':B$ and $f:\hom_B(b,b')$.
  Therefore, there is a function assigning to each such $f$ its unique inhabitant; denote this function $G_{b,b'}:\hom_B(b,b') \to \hom_C(Gb,Gb')$.
  The proof that $G$ is a functor is straightforward; in each case we can choose $a,h$ and apply~\ref{item:eqvprop5}.

  Finally, for any $a_0:A$, defining $c\defeq Fa_0$ and $k_{a,h}\defeq F g$, where $g:\hom_A(a,a_0)$ is the unique isomorphism with $Hg = h$, gives an element of $X_{Ha_0}$.
  Thus, it is equal to the specified one; hence $GHa=Fa$.
  Similarly, for $f:\hom_A(a_0,a_0')$ we can define an element of $Y_{Hf}$ by transporting along these equalities, which must therefore be equal to the specified one.
  Hence, we have $GH=F$, and thus $GH\cong F$ as desired.
\end{proof}

Therefore, if a precategory $A$ admits a weak equivalence functor $A\to \hat{A}$, then that is its ``reflection'' into categories: any functor from $A$ into a category will factor essentially uniquely through $\widehat{A}$.
We now give two constructions of such a weak equivalence.

\begin{thm}
  For any precategory $A$, there is a category $\widehat A$ and a weak equivalence $A\to\widehat{A}$.
\end{thm}

\begin{proof}[First proof]
  Let $\widehat{A}_0 \defeq \setof{ F:\uset^{A\op} | \Brck{\sm{a:A} (\y a \cong F)}}$, with hom-sets inherited from $\uset^{A\op}$.
  Then the inclusion $\widehat{A} \to \uset^{A\op}$ is fully faithful and a monomorphism on objects.
  Since $\uset^{A\op}$ is a category (by \autoref{ct:functor-cat}, since \uset is so by univalence), $\widehat A$ is also a category.

  Let $A\to\widehat A$ be the Yoneda embedding.
  This is fully faithful by \autoref{ct:yoneda-embedding}, and essentially surjective by definition of $\widehat{A}_0$.
  Thus it is a weak equivalence.
\end{proof}

This proof is very slick, but it has the drawback that it increases universe level.
If $A$ is a category in a universe \bbU, then in this proof \uset must be at least as large as $\uset_\bbU$.
Then $\uset_\bbU$ and $(\uset_\bbU)^{A\op}$ are not themselves categories in \bbU, but only in a higher universe, and \emph{a priori} the same is true of $\widehat A$.
One could imagine a resizing axiom that could deal with this, but it is also possible to give a direct construction using higher inductive types.

\begin{proof}[Second proof]
  We define a higher inductive 1-type $\widehat A_0$ with the following constructors:
  \begin{itemize}
  \item A function $i:A_0 \to \widehat A_0$.
  \item For each $a,b:A$ and $e:a\cong b$, an equality $je:\id{ia}{ib}$.
  \item For each $a,b:A$ and $p:\id a b$, an equality $\id{j(\idtoiso(p))}{\map i p}$.
  \item For each $a:A$, an equality $\id{j(1_a)}{\refl{ia}}$.
  \item For each $(a,b,c:A)$, $(f:a\cong b)$, and $(g:b\cong c)$, an equality $\id{j(g \circ f)}{j(g)\ct j(f)}$.
  % \item The usual 1-truncation constructor: for all $x,y:A_0'$ and $p,q:\id x y$ and $r,s:\id p q$, an identity $\id r s$.
  \end{itemize}
  This will be the type of objects of $\widehat A$; we now build all the rest of the structure.
  (The following proof is of the sort that could benefit a lot from the help of a proof assistant: it is wide and shallow with many short cases to consider, and with a large part of the work consisting of writing down what needs to be checked.)

  \medskip

  \textbf{Step 1:} We define a family $\hom_{\widehat A}:\widehat A_0\to \widehat A_0 \to \set$ by double induction on $\widehat A_0$, which is possible since \set is a 1-type.
  When $x$ and $y$ are of the form $ia$ and $ib$, we take $\hom_{\widehat A}(ia,ib) \defeq \hom_A(a,b)$.
  It remains to consider all the other possible pairs of constructors.

  Let us keep $x=ia$ fixed at first.
  If $y$ varies along the identity $je:\id{ib}{ib'}$, for some $e:b\cong b'$, we require an identity $\id{\hom_A(a,b)}{\hom_A(a,b')}$.
  By univalence, it suffices to give an equivalence $\eqv{\hom_A(a,b)}{\hom_A(a,b')}$.
  We take this to be the function $(e\circ -):\hom_A(a,b)\to \hom_A(a,b')$.
  To see that this is an equivalence, we give its inverse as $(\inv e\circ -)$, with witnesses to inversion coming from the fact that $\inv e$ is the inverse of $e$ in $A$.
  
  Next, as $y$ varies along the identity $\id{j(\idtoiso(p))}{\map ip}$, for $p:\id{b}{b'}$, we require an identity $\id{(\idtoiso(p)\circ -)}{\map{\hom_A(a,-)}{p}}$.
  This is immediate by induction on $p$.

  As $y$ varies along the identity $\id{j(1_b)}{\refl{ib}}$, we require an identity $\id{(1_b\circ -)}{\refl{\hom_A(a,b)}}$; this follows from the identity axiom $\id{1_b\circ g}{g}$ of a precategory.
  Similarly, as $y$ varies along the identity $\id{j(g\circ f)}{j(g)\ct j(f)}$, we require an identity $\id{((g\circ f)\circ -)}{(g\circ (f\circ -))}$, which follows from associativity.
  % Finally, as $y$ varies along the 1-truncation constructor, we need only to observe that \set is 1-truncated.

  Now we consider the other constructors for $x$.
  Say that $x$ varies along the identity $j(e):\id{ia}{ia'}$, for some $e:a \cong a'$; we again must deal with all the constructors for $y$.
  If $y$ is $ib$, then we require an identity $\id{\hom_A(a,b)}{\hom_A(a',b)}$.
  By univalence, this may come from an equivalence, and for this we can use $(-\circ \inv e)$, with inverse $(-\circ e)$.

  Still with $x$ varying along $j(e)$, suppose now that $y$ also varies along $j(f)$ for some $f:b\cong b'$.
  Then we need to know that the two concatenated identities
  \begin{gather*}
    \hom_A(a,b) = \hom_A(a',b) = \hom_A(a',b') \mathrlap{\qquad\text{and}}\\
    \hom_A(a,b) = \hom_A(a,b') = \hom_A(a',b')
  \end{gather*}
  are identical.
  This follows from associativity: $(f\circ -)\circ \inv e = f\circ (-\circ \inv e)$.
  The rest of the constructors for $y$ are trivial, since they are 2-fold equalities in sets.

  For the last three constructors of $x$, all but the first constructor for $y$ is likewise trivial.
  When $x$ varies along the equality $\id{j(\idtoiso(p))}{\map i p}$ for $p:a=a'$ and $y$ is $ib$, we require $(-\circ \idtoiso(p) = \map{\hom_A(-,b)}{p}$, which follows by induction on $p$.
  Finally, when $x$ varies along $j(1_a)=\refl{ia}$, we use the identity axiom again, and when $x$ varies along $\id{j(g\circ f)}{j(g)\ct j(f)}$, we use associativity again.
  This completes the construction of $\hom_{\widehat A}:\widehat A_0 \to \widehat A_0 \to \set$.

  \medskip

  \textbf{Step 2:} We give the precategory structure on $\widehat A$, always by induction on $\widehat A_0$.
  % The reader is probably getting bored at this point, so we skip the details.
  We are now eliminating into sets (the hom-sets of $\widehat A$), so all but the first two constructors are trivial to deal with.

  For identities, if $x$ is $ia$ then we have $\hom_{\widehat A}(x,x) \jdeq \hom_A(a,a)$ and we define $1_x \defeq 1_{ia}$.
  If $x$ varies along $je$ for $e:a\cong a'$, we must show that $\trans{je}{1_{ia}} = 1_{ia'}$.
  Here the transport is with respect to the type family $x\mapsto \hom_{\widehat A}(x,x)$.
  But by definition of $\hom_{\widehat A}$, transporting along $je$ is given by composing with $e$ and $\inv e$, and we have $e\circ 1_{ia} \circ \inv{e} = 1_{ia'}$.

  For composition, if $x,y,z$ are $ia,ib,ic$ respectively, then $\hom_{\widehat A}$ reduces to $\hom_A$ and we can define composition in $\widehat A$ to be composition in $A$.
  And when $x$, $y$, or $z$ varies along $je$, then we verify the following equalities:
  \begin{align*}
    e \circ (g\circ f) &= (e\circ g) \circ f\\
    g\circ f &= (g\circ \inv e) \circ (e\circ f)\\
    (g\circ f) \circ \inv e &= g \circ (f\circ \inv e)
  \end{align*}
  Finally, the associativity and unitality axioms are mere propositions, so all constructors except the first are trivial.
  But in that case, we have the corresponding axioms in $A$.

  \medskip

  \textbf{Step 3}: We show that $\widehat A$ is a category.
  That is, we must show that for all $x,y:\widehat A$, the function $\idtoiso:(x=y) \to (x\cong y)$ is an equivalence.
  First we define, for all $x,y:\widehat A$, a function $k_{x,y}:(x\cong y) \to (x=y)$ by induction.
  As before, since our goal is a set, it suffices to deal with the first two constructors.

  When $x$ and $y$ are $ia$ and $ib$ respectively, we have $\hom_{\widehat A}(ia,ib)\jdeq \hom_A(a,b)$, with composition and identities inherited as well, so that $(ia\cong ib)$ is equivalent to $(a\cong b)$.
  But now we have the constructor $j:(a\cong b) \to (ia=ib)$.

  Next, if $y$ varies along $j(e)$ for some $e:b\cong b'$, we must show that for $f:a\cong b$ we have $j(\trans{j(e)}{f}) = j(e) \ct j(f)$.
  But by definition of $\hom_{\widehat A}$ on equalities, transporting along $j(e)$ is equivalent to postcomposing with $e$, so this equality follows from the last constructor of $\widehat A_0$.
  The remaining case when $x$ varies along $j(e)$ for $e:a\cong a'$ is similar.
  This completes the definition of $k:\prd{x,y:\widehat A_0} (x\cong y) \to (x=y)$.

  Now one thing we must show is that if $p:x=y$, then $k(\idtoiso(p))=p$.
  By induction on $p$, we may assume it is $\refl x$, and hence $\idtoiso(p)\jdeq 1_x$.
  Now we argue by induction on $x:\widehat A_0$, and since our goal is a mere proposition (since $\widehat A_0$ is a 1-type), all constructors except the first are trivial.
  But if $x$ is $ia$, then $k(1_{ia}) \jdeq j(1_a)$, which is equal to $\refl{ia}$ by the penultimate constructor of $\widehat A_0$.

  To complete the proof that $\widehat A$ is a category, we must show that if $f:x\cong y$, then $\idtoiso(k(f))=f$.
  By induction we may assume that $x$ and $y$ are $ia$ and $ib$ respectively, in which case $f$ must arise from an isomorphism $g:a\cong b$ and we have $k(f)\jdeq j(g)$.
  However, for any $p$ we have $\idtoiso(p) = \trans{p}{1}$, so in particular $\idtoiso (j(g)) = \trans{j(g)}{1_{ia}}$.
  And by definition of $\hom_{\widehat A}$ on equalities, this is given by composing $1_{ia}$ with the equivalence $g$, hence is equal to $g$.

  \medskip

  \textbf{Step 4}: We define a weak equivalence $I:A \to \widehat A$.
  We take $I_0 \defeq i : A_0 \to \widehat A_0$, and by construction of $\hom_{\widehat A}$ we have functions $I_{a,b}:\hom_A(a,b) \to \hom_{\widehat A}(Ia,Ib)$ forming a functor $I:A \to \widehat A$.
  This functor is fully faithful by construction, so it remains to show it is essentially surjective.
  That is, for all $x:\widehat A$ we want there to merely exist an $a:A$ such that $Ia\cong x$.
  As always, we argue by induction on $x$, and since the goal is a mere proposition, all but the first constructor are trivial.
  But if $x$ is $ia$, then of course we have $a:A$ and $Ia\jdeq ia$, hence $Ia \cong ia$.
  (Note that if we were trying to prove $I$ to be \emph{split} essentially surjective, we would be stuck, because we know nothing about equalities in $A_0$ and thus have no way to deal with any further constructors.)
\end{proof}

We call the construction $A\mapsto \widehat A$ the \textbf{Rezk completion}, although there is also an argument (coming from higher topos semantics) for calling it the \textbf{stack completion}.

We have seen that most precategories arising in practice are categories, since they are constructed from \uset, which is a category by the univalence axiom.
However, there are a few cases in which the Rezk completion is necessary to obtain a category.

\begin{eg}
  Recall from \autoref{ct:fundgpd} that for any type $X$ there is a pregroupoid with $X$ as its type of objects and $\hom(x,y) \defeq \pizero{x=y}$.
  Its Rezk completion is the \emph{fundamental groupoid} of $X$.
  Recalling that groupoids are equivalent to 1-types, it is not hard to identify this groupoid with $\trunc1X$.
\end{eg}

\begin{eg}\label{ct:hocat}
  Recall from \autoref{ct:hoprecat} that there is a precategory whose type of objects is \type and with $\hom(X,Y) \defeq \pizero{X\to Y}$.
  Its Rezk completion may be called the \emph{homotopy category of types}.
  Its type of objects can be identified with $\trunc1\type$ (see \autoref{ct:ex:hocat}).
\end{eg}

The Rezk completion also allows us to show that the notion of ``category'' is determined by the notion of ``weak equivalence of precategories''.
Thus, insofar as the latter is inevitable, so is the former.

\begin{thm}
  A precategory $C$ is a category if and only if for every weak equivalence of precategories $H:A\to B$, the induced functor $(-\circ H):C^B \to C^A$ is an isomorphism of precategories.
\end{thm}
\begin{proof}
  ``Only if'' is \autoref{ct:cat-weq-eq}.
  In the other direction, let $H$ be $I:A\to\widehat A$.
  Then since $(-\circ I)_0$ is an equivalence, there exists $R:\widehat A\to A$ such that $RI=1_A$.
  Hence $IRI=I$, but again since $(-\circ I)_0$ is an equivalence, this implies $IR =1_{\widehat A}$.
  By \autoref{ct:isoprecat}\ref{item:ct:ipc3}, $I$ is an isomorphism of precategories.
  But then since $\widehat A$ is a category, so is $A$.
\end{proof}


\newpage

\section*{Notes}
\label{sec:ct:notes}

The usefulness of a definition of categories of this sort, where the type of objects contains all the categorical isomorphisms, was first strongly emphasized (in the context of set-based definitions of higher categories) by Charles Rezk~\cite{rezk01css}.

The term ``gaunt'' (\autoref{ct:gaunt}) was introduced by Barwick and Schommer-Pries~\cite{bsp12infncats}.

\autoref{ct:galois} was brought up by Peter May on the categories mailing list in May 2010.


\section*{Exercises}
\label{sec:ct:exercises}

\begin{ex}
  For a precategory $A$ and $a:A$, define the \emph{slice precategory} $A/a$.
  Show that if $A$ is a category, so is $A/a$.
\end{ex}

\begin{ex}
  For any set $X$, prove that the slice category $\uset/X$ is equivalent to the functor category $\uset^X$, where in the latter case we regard $X$ as a discrete category.
\end{ex}

\begin{ex}
  Prove that a functor is an equivalence of categories if and only if it is a \emph{right} adjoint whose unit and counit are isomorphisms.
\end{ex}

\begin{ex}\label{ct:pre2cat}
  Define a \emph{pre-2-category} to consist of the structure formed by precategories, functors, and natural transformations in \autoref{sec:transfors}.
  Similarly, define a \emph{pre-bicategory} by replacing the equalities in \autoref{ct:functor-assoc,ct:units} with natural isomorphisms satisfying analogous coherence conditions.
  Define a function from pre-2-categories to pre-bicategories, and show that it becomes an equivalence when restricted and corestricted to those whose hom-precategories are categories.
\end{ex}

\begin{ex}\label{ct:2cat}
  Define a \emph{2-category} to be a pre-2-category satisfying a condition analogous to that of \autoref{ct:category}.
  Verify that the pre-2-category of categories \ucat is a 2-category.
  How much of this chapter can be done internally to an arbitrary 2-category?
\end{ex}

\begin{ex}\label{ct:groupoids}
  Define a 2-category whose objects are 1-types, whose morphisms are functions, and whose 2-morphisms are homotopies.
  Prove that it is equivalent, in an appropriate sense, to the full sub-2-category of \ucat spanned by the \emph{groupoids} (categories in which every arrow is an isomorphism).
\end{ex}

\begin{ex}
  A \emph{strict category} is a precategory whose type of objects is a set.
  Prove that the pre-2-category of strict categories is equivalent to the following pre-2-category.
  \begin{itemize}
  \item Its objects are categories $A$ equipped with a surjection\footnote{Recall that a function $f:X\to Y$ is a \emph{surjection} if for every $y:Y$, there \emph{merely exists} an $x:X$ such that $f(x)=y$.  This is to be distinguished from a \emph{split surjection}, which has the property that for every $y:Y$ there \emph{exists} an $x:X$ such that $f(x)=y$.} $p_A:A_0'\to A_0$, where $A_0'$ is a set.
  \item Its morphisms are functors $F:A\to B$ equipped with a function $F_0':A_0' \to B_0'$ such that $p_B \circ F_0' = F_0 \circ p_A$.
  \item Its 2-morphisms are simply natural transformations.
  \end{itemize}
\end{ex}

\begin{ex}
  Define the pre-2-category of $\dagger$-categories, which has $\dagger$-structures on its hom-precategories.
  Show that two $\dagger$-categories are equal precisely when they are ``unitarily equivalent'' in a suitable sense.
\end{ex}

\begin{ex}\label{ct:ex:hocat}
  Prove that a function $X\to Y$ is an equivalence if and only if its image in the homotopy category of \autoref{ct:hocat} is an isomorphism.
  Show that the type of objects of this category is $\trunc1\type$.
\end{ex}

\begin{ex}
  Construct the $\dagger$-Rezk completion of a $\dagger$-precategory into a $\dagger$-category, and give it an appropriate universal property.
\end{ex}

% Local Variables:
% TeX-master: "main"
% End:

