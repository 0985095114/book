\chapter*{Preface}
\label{cha:preface}

% Uncomment this if you think Preface should appear in TOC
%\addcontentsline{toc}{chapter}{Preface}

This book was written as a collaborative effort during the 2012-13 Special Year on Univalent Foundations at the Institute for Advanced Study, School of Mathematics, organized by Steve Awodey, Thierry Coquand, and Vladimir Voevodsky.  This \emph{Univalent Foundations Program} officially consisted of the following people.

%\medskip
%\noindent The participants in the special year were:
%\centerline{\emph{Participants}}
%
\begin{multicols}{3}{
\begin{itemize}
\item[] Peter Aczel
\item[] Benedikt Ahrens
\item[] Thorsten Altenkirch
\item[] Steve Awodey
\item[] Bruno Barras
\item[] Andrej Bauer
\item[] Yves Bertot
\item[] Marc Bezem
\item[] Thierry Coquand
\item[] Eric Finster
\item[] Daniel Grayson
\item[] Hugo Herbelin
\item[] Andr\'e Joyal
\item[] Chris Kapulkin
\item[] Dan Licata
\item[] Peter LeFanu Lumsdaine
\item[] Assia Mahboubi
\item[] Per Martin-L\"of
\item[] Sergey Melikhov
\item[] Alvaro Pelayo
\item[] Andrew Polonsky
\item[] Michael Shulman
\item[] Kristina Sojakova
\item[] Matthieu Sozeau
\item[] Bas Spitters
\item[] Benno van den Berg
\item[] Vladimir Voevodsky
\item[] Michael Warren
\item[] Noam Zeilberger
\end{itemize}
}
\end{multicols}

\noindent There were also the following students, whose participation was no less valuable.

%\centerline{\emph{Students}}
%
\begin{multicols}{3}{
\begin{itemize}
\item[] Carlo Angiuli
\item[] Anthony Bordg
\item[] Guillaume Brunerie
\item[] Chris Kapulkin
\item[] Egbert Rijke
\item[] Kristina Sojakova
\end{itemize}
}
\end{multicols}

\noindent In addition, there were the following short- and long-term visitors, including student visitors, whose contributions to the Program were also essential.

%\centerline{\emph{Visitors}}
%
\begin{multicols}{3}{
\begin{itemize}
\item[] Jeremy Avigad
\item[] Cyril Cohen
\item[] Robert Constable
\item[] Pierre-Louis Curien
\item[] Peter Dybjer
\item[] Mart{\'\i}n Escard{\'o}
\item[] Kuen-Bang Hou
\item[] Nicola Gambino
\item[] Richard Garner
\item[] Georges Gonthier
\item[] Thomas Hales
\item[] Robert Harper
\item[] Martin Hofmann
\item[] Pieter Hofstra
\item[] Joachim Kock
\item[] Nicolai Kraus
\item[] Nuo Li
\item[] Zhaohui Luo
\item[] Michael Nahas
\item[] Erik Palmgren
\item[] Emily Riehl
\item[] Dana Scott
\item[] Philip Scott
\item[] Sergei Soloviev
\end{itemize}
}
\end{multicols}

Each of the above-mentioned individuals contributed something to the Program --- and so to this book --- in the form of ideas, words, or deeds.  A few people were responsible for the primary writing of the book, however, and they deserve to be recognized.  They are Peter Aczel (who originally conceived of this project), Benedikt Ahrens, Thorsten Altenkirch,  Carlo Angiuli, Steve Awodey, Andrej Bauer, Guillaume Brunerie, Thierry Coquand, Chris Kapulkin, Dan Licata, Peter Lumsdaine, Egbert Rijke, Kristina Sojakova, Bas Spitters --- and above all, Michael Shulman.  Of the visitors, Robert Harper, Nicolai Kraus, and Michael Nahas made especially valuable contributions to this book.  Let it also be recorded that the spirit of collaboration that prevailed throughout the year was truly extraordinary. 

% To get the list of contributers to github run the command:
% git log --format='%aN' | sort -u

We did not set out to write a book. The present work has its origins in our collective attempts to develop a new style of ``informal type theory" that can be read and understood by a human being, as a complement to a formal proof that can be checked by a machine. Univalent Foundations is closely tied to the idea of a foundation of mathematics that can be implemented in a computer proof assistant.  Although such a formalization is not part of this book, much of the material presented here was actually done first in the fully formalized setting inside a proof assistant, and only later ``unformalized" to arrive at the presentation you find before you --- a remarkable inversion of the usual state of affairs in formalized mathematics.  

% Large and growing libraries of code currently exist in the proof assistants Coq and Agda, in which much of the theory presented here has been formalized.  The interested reader can find these libraries at {\tt homotopytypetheory.org}.

Special thanks are due to the Institute for Advanced Study, without which this book would obviously never have come to be --- but which, moreover, proved to be a stimulating and thoroughly enjoyable forge for this new branch of mathematics.  We herewith commit it to the flames of public opinion; may it be sufficiently hardened and tempered to stand the test. 

\bigskip

\begin{flushright}
The Univalent Foundations Program\\
Institute for Advanced Study\\
Princeton, April 2013
\end{flushright}

%%%%%%%%%%%%% end of scope of local macros
% Local Variables:
% TeX-master: "main"
% End:
