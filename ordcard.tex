\newcommand{\cd}[1]{\left|#1\right|}
\newcommand{\inj}{\ensuremath{\mathsf{inj}}}

\section{Cardinal numbers}
\label{sec:cardinals}

\begin{defn}
  The \textbf{type of cardinal numbers} is the 0-truncation of \set:
  \[ \card \defeq \pizero{\set} \]
  Thus, a \textbf{cardinal number}, or \textbf{cardinal}, is an inhabitant of $\card\jdeq \pizero\set$.
\end{defn}

\begin{rmk}
  As in some previous sections, here we are being universe polymorphic and typically ambiguous.
  Technically, there is a separate type ``\card'' associated to each universe ``\type'', but with these conventions we can state theorems beginning with ``for all cardinal numbers\dots''\ and give them exactly the same sort of meaning as those beginning ``for all types\dots''.
\end{rmk}

If $A$ is a set, we write $\cd{A}$ for its image under the canonical projection $\set \to \card$.
Of course, by definition, \card is a set.
It also inherits the structure of a semiring from \set.

\begin{defn}
  The operation of \textbf{cardinal addition}
  \[ \blank+\blank : \card \to \card \to \card \]
  is defined by induction on truncation:
  \[ \cd{A} + \cd{B} \defeq \cd{A+B} .\]
\end{defn}
\begin{proof}
  Since $\card\to\card$ is a set, to define $\alpha+\blank:\card\to\card$ for all $\alpha:\card$, by induction it suffices to assume that $\alpha$ is $\cd{A}$ for some $A:\set$.
  Now we want to define $\cd{A}+\blank :\card\to\card$, i.e.\ we want to define $\cd{A}+\beta :\card$ for all $\beta:\card$.
  However, since $\card$ is a set, by induction it suffices to assume that $\beta$ is $\cd{B}$ for some $B:\set$.
  But now we can define $\cd{A}+\cd{B}$ to be $\cd{A+B}$.
\end{proof}

\begin{defn}
  Similarly, the operation of \textbf{cardinal multiplication}
  \[ \blank\cdot\blank : \card \to \card \to \card \]
  is defined by induction on truncation:
  \[ \cd{A} \cdot \cd{B} \defeq \cd{A\times B} \]
\end{defn}

\begin{lem}\label{card:semiring}
  \card is a commutative semiring, i.e.\ for $\alpha,\beta,\gamma:\card$ we have the following.
  \begin{align*}
    (\alpha+\beta)+\gamma &= \alpha+(\beta+\gamma)\\
    \alpha+0 &= \alpha\\
    \alpha + \beta &= \beta + \alpha\\
    (\alpha \cdot \beta) \cdot \gamma &= \alpha \cdot (\beta\cdot\gamma)\\
    \alpha \cdot 1 &= \alpha\\
    \alpha\cdot\beta &= \beta\cdot\alpha\\
    \alpha\cdot(\beta+\gamma) &= \alpha\cdot\beta + \alpha\cdot\gamma
  \end{align*}
  where $0 \defeq \cd{\emptyset}$ and $1\defeq\cd{\unit}$.
\end{lem}
\begin{proof}
  We prove the commutativity of multiplication, $\alpha\cdot\beta = \beta\cdot\alpha$; the others are exactly analogous.
  Since \card is a set, the type $\alpha\cdot\beta = \beta\cdot\alpha$ is a mere proposition, and in particular a set.
  Thus, by induction it suffices to assume $\alpha$ and $\beta$ are of the form $\cd{A}$ and $\cd{B}$ respectively, for some $A,B:\set$.
  Now $\cd{A}\cdot \cd{B} \jdeq \cd{A\times B}$ and $\cd{B}\times\cd{A} \jdeq \cd{B\times A}$, so it suffices to show $A\times B = B\times A$.
  Finally, by univalence, it suffices to give an equivalence $A\times B \simeq B\times A$.
  But this is easy: take $(a,b) \mapsto (b,a)$ and its obvious inverse.
\end{proof}

\begin{defn}
  The operation of \textbf{cardinal exponentiation} is also defined by induction on truncation:
  \[ \cd{A}^{\cd{B}} \defeq \cd{B\to A}. \]
\end{defn}

\begin{lem}\label{card:exp}
  For $\alpha,\beta,\gamma:\card$ we have
  \begin{align*}
    \alpha^0 &= 1\\
    1^\alpha &= 1\\
    \alpha^1 &= \alpha\\
    \alpha^{\beta+\gamma} &= \alpha^\beta \cdot \alpha^\gamma\\
    \alpha^{\beta\cdot \gamma} &= (\alpha^{\beta})^\gamma\\
    (\alpha\cdot\beta)^\gamma &= \alpha^\gamma \cdot \beta^\gamma
  \end{align*}
\end{lem}
\begin{proof}
  Exactly like \autoref{card:semiring}.
\end{proof}

\begin{defn}
  The relation of \textbf{cardinal inequality}
  \[ \blank\le\blank : \card\to\card\to\prop \]
  is defined by induction on truncation:
  \[ \cd{A} \le \cd{B} \defeq \brck{\inj(A,B)} \]
  where $\inj(A,B)$ is the type of injections from $A$ to $B$.
  In other words, $\cd{A} \le \cd{B}$ means that there merely exists an injection from $A$ to $B$.
\end{defn}

\begin{lem}
  Cardinal inequality is a preorder, i.e.\ for $\alpha,\beta:\card$ we have
  \begin{gather*}
    \alpha \le \alpha\\
    (\alpha \le \beta) \to (\beta\le\gamma) \to (\alpha\le\gamma)
  \end{gather*}
\end{lem}
\begin{proof}
  As before, by induction on truncation.
  For instance, since $(\alpha \le \beta) \to (\beta\le\gamma) \to (\alpha\le\gamma)$ is a mere proposition, by induction on 0-truncation we may assume $\alpha$, $\beta$, and $\gamma$ are $\cd{A}$, $\cd{B}$, and $\cd{C}$ respectively.
  Now since $\cd{A} \le \cd{C}$ is a mere proposition, by induction on $(-1)$-truncation we may assume given injections $f:A\to B$ and $g:B\to C$.
  But then $g\circ f$ is an injection from $A$ to $C$, so $\cd{A} \le \cd{C}$ holds.
  Reflexivity is even easier.
\end{proof}

We may likewise show that cardinal inequality is compatible with the semiring operations.

\begin{lem}\label{thm:injsurj}
  Consider the following statements:
  \begin{enumerate}
  \item There is an injection $A\to B$.\label{item:cle-inj}
  \item There is a surjection $B\to A$.\label{item:cle-surj}
  \end{enumerate}
  Then, assuming excluded middle:
  \begin{itemize}
  \item Given $a_0:A$, we have~\ref{item:cle-inj}$\to$\ref{item:cle-surj}.
  \item Therefore, if $A$ is merely inhabited, we have~\ref{item:cle-inj} $\to$ merely \ref{item:cle-surj}.
  \item Assuming the axiom of choice, we have~\ref{item:cle-surj} $\to$ merely \ref{item:cle-inj}.
  \end{itemize}
\end{lem}
\begin{proof}
  If $f:A\to B$ is an injection, define $g:B\to A$ at $b:B$ as follows.
  Since $f$ is injective, the fiber of $f$ at $b$ is a mere proposition.
  Therefore, by excluded middle, either there is an $a:A$ with $f(a)=b$, or not.
  In the first case, define $g(b)\defeq a$; otherwise set $g(b)\defeq a_0$.
  Then for any $a:A$, we have $a = g(f(a))$, so $g$ is surjective.

  The second statement follows from this by induction on truncation.
  For the third, if $g:B\to A$ is surjective, then by the axiom of choice, there merely exists a function $f:A\to B$ with $g(f(a)) = a$ for all $a$.
  But then $f$ must be injective.
\end{proof}

\begin{thm}[Schroeder-Bernstein]
  Assuming excluded middle, for sets $A$ and $B$ we have
  \[ \inj(A,B) \to \inj(B,A) \to (A\cong B) \]
\end{thm}
\begin{proof}
  The usual ``back-and-forth'' argument applies without significant changes.
  Note that it actually constructs an isomorphism $A\cong B$ (assuming excluded middle so that we can decide whether a given element belongs to a cycle, an infinite chain, a chain beginning in $A$, or a chain beginning in $B$).
\end{proof}

\begin{cor}
  Assuming excluded middle, cardinal inequality is a partial order, i.e.\ for $\alpha,\beta:\card$ we have
  \[ (\alpha\le\beta) \to (\beta\le\alpha) \to (\alpha=\beta). \]
\end{cor}
\begin{proof}
  Since $\alpha=\beta$ is a mere proposition, by induction on truncation we may assume $\alpha$ and $\beta$ are $\cd{A}$ and $\cd{B}$, respectively, and that we have injections $f:A\to B$ and $g:B\to A$.
  But then the Schroeder-Bernstein theorem gives an isomorphism $A\simeq B$, hence an equality $\cd{A}=\cd{B}$.
\end{proof}

Finally, we can reproduce Cantor's theorem, showing that for every cardinal there is a greater one.

\begin{thm}[Cantor]
  For $A:\set$, there is no surjection $A \to (A\to \mathbf{2})$.
\end{thm}
\begin{proof}
  Suppose $f:A \to (A\to \mathbf{2})$ is any function, and define $g:A\to \mathbf{2}$ by $g(a) \defeq \neg f(a)(a)$.
  If $g = f(a_0)$, then $g(a_0) = f(a_0)(a_0)$ but $g(a_0) = \neg f(a_0)(a_0)$, a contradiction.
  Thus, $f$ is not surjective.
\end{proof}

\begin{cor}
  Assuming excluded middle, for any $\alpha:\card$, there is a cardinal $\beta$ such that $\alpha\le\beta$ and $\alpha\neq\beta$.
\end{cor}
\begin{proof}
  Let $\beta = 2^\alpha$.
  Now we want to show a mere propositon, so by induction we may assume $\alpha$ is $\cd{A}$, so that $\beta\jdeq \cd{A\to \mathbf{2}}$.
  Using excluded middle, we have a function $f:A\to (A\to \mathbf{2})$ defined by
  \[f(a)(a') \defeq
  \begin{cases}
    \top &\quad a=a'\\
    \bot &\quad a\neq a'.
  \end{cases}
  \]
  And if $f(a)=f(a')$, then $f(a')(a) = f(a)(a) = \top$, so $a=a'$; hence $f$ is injective.
  Thus, $\alpha \jdeq \cd{A} \le \cd{A\to \mathbf{2}} \jdeq 2^\alpha$.

  On the other hand, if $2^\alpha \le \alpha$, then we would have an injection $(A\to\mathbf{2})\to A$.
  By \autoref{thm:injsurj}, since we have $(\lam{x} \bot):A\to \mathbf{2}$ and excluded middle, there would then be a surjection $A \to (A\to \mathbf{2})$, contradicting Cantor's theorem.
\end{proof}

\section{Ordinal numbers}
\label{sec:ordinals}

\newcommand{\acc}{\ensuremath{\mathsf{acc}}}

\begin{defn}
  Let $A$ be a set and
  \[\blank<\blank:A\to A\to \prop\]
  a binary relation on $A$.
  We define by induction what it means for an element $a:A$ to be \textbf{accessible} by $<$:
  \begin{itemize}
  \item If $b$ is accessible for every $b<a$, then $a$ is accessible.
  \end{itemize}
  We write $\acc(a)$ to mean that $a$ is accessible.
\end{defn}

It may seem that such an inductive definition can never get off the ground, but of course if $a$ has the property that there are \emph{no} $b$ such that $b<a$, then $a$ is vacuously accessible.

\begin{lem}
  Accessibility is a mere property.
\end{lem}
\begin{proof}
  We claim that for any $a_1,a_2:A$, any $p:a_1=a_2$, and any $s_1:\acc(a_1)$ and $s_2:\acc(a_2)$, we have $\trans{p}{s_1}=s_2$.
  By induction, we may assume that $s_1$ is given by a function assigning to each $b_1<a_1$ a proof $s_1(b_1):\acc(b_1)$, and moreover that for any $b_2:A$, any $q:b_1=b_2$, and $t_2:\acc(b_2)$, we have $\trans{q}{s_1(b_1)}=t_2$.
  Similarly, we may assume that $s_2$ is given by a function assigning to each $b_2<a_2$ a proof $s_2(b_2):\acc(b_2)$, and that for any $b_1:A$, any $q:b_1=b_2$, and $t_1:\acc(b_1)$, we have $\trans{q}{t_1}=s_2(b_2)$.
  
  Now by function extensionality, to show $\trans{p}{s_1}=s_2$ it suffices to show that for any $b_2<a_2$ we have $\trans{p}{s_1}(b_2) = s_2(b_2)$.
  However, we can obtain this from the induction hypothesis for $s_2$ with $q\defeq \refl{b_2}$ and $t_1 \defeq \trans{p}{s_1}(b_2)$.
  This proves the claim.

  Finally, we instantiate the claim with $a_2\defeq a_1\defeq a$ and $p\defeq \refl{a}$.
  Thus, for any $a:A$ and $s_1,s_2:\acc(a)$, we have $s_1=s_2$, as desired.
\end{proof}

\begin{defn}
  A binary relation $<$ on a set $A$ is \textbf{well-founded} if every element of $A$ is accessible.
\end{defn}

\begin{lem}
  Well-foundedness is a mere property.
\end{lem}
\begin{proof}
  Well-foundedness of $<$ is the type $\prd{a:A} \acc(a)$, which is a mere proposition since each $\acc(a)$ is.
\end{proof}

\begin{eg}\label{thm:nat-wf}
  Perhaps the most familiar well-founded relation is the usual strict ordering on \nat.
  To show that this is well-founded, we must show that $n$ is accessible for each $n:\nat$.
  This is just the usual proof of ``strong induction'' from ordinary induction on \nat.

  Specifically, we prove by induction on $n:\nat$ that $k$ is accessible for all $k\le n$.
  The base case is just that $0$ is accessible, which is vacuously true since nothing is strictly less than $0$.
  For the inductive step, we assume that $k$ is accessible for all $k\le n$, which is to say for all $k<n+1$; hence by definition $n+1$ is also accessible.

  A different relation on \nat which is also well-founded is obtained by setting only $n < \suc(n)$ for all $n:\nat$.
  Well-foundedness of this relation is almost exactly the ordinary induction principle of \nat.
\end{eg}

\begin{eg}\label{thm:wtype-wf}
  Let $A:\set$ and $B : A \to \set$ be a family of sets.
  Recall from \autoref{sec:w-types} that the $W$-type $\wtype{a:A} B(a)$ is inductively generated by the single constructor
  \begin{itemize}
  \item $\supp : \prd{a:A} (B(a) \to \wtype{x:A} B(x)) \to \wtype{x:A} B(x)$
  \end{itemize}
  We define the relation $<$ on $\wtype{x:A} B(x)$ by recursion on its second argument:
  \begin{itemize}
  \item For any $a:A$ and $f:B(a) \to \wtype{x:A} B(x)$, we define $w<\supp(a,f)$ to mean that there merely exists a $b:B(a)$ such that $w = f(b)$.
  \end{itemize}
  Now we prove that every $w:\wtype{x:A} B(x)$ is accessible for this relation, using the usual induction principle for $\wtype{x:A}B(x)$.
  This means we assume given $a:A$ and $f:B(a) \to \wtype{x:A} B(x)$, and also a lifting $f' : \prd{b:B(a)} \acc(f(b))$.
  But then by definition of $<$, we have $\acc(w)$ for all $w<\supp(a,f)$; hence $\supp(a,f)$ is accessible.
\end{eg}

Well-foundedness allows us to define functions by recursion and prove statements by induction, such as for instance the following.

\begin{lem}\label{thm:wfrec}
  Suppose $B$ is a set and we have a function
  \[ g : \mathcal{P}B \to B \]
  Then if $<$ is a well-founded relation on $A$, there is a function $f:A\to B$ such that for all $a:A$ we have
  \begin{equation*}
    f(a) = g\Big(\setof{ f(a') | a'<a }\Big).
  \end{equation*}
\end{lem}
\begin{proof}
  We first define, for every $a:A$ and $s:\acc(a)$, an element $\bar f(a,s):B$.
  By induction, it suffices to assume that $s$ is a function assigning to each $a'<a$ a proof $s(a'):\acc(a')$, and that moreover for each such $a'$ we have an element $\bar f(a',s(a')):B$.
  In this case, we define
  \begin{equation*}
    \bar f(a,s) \defeq g\Big(\setof{ \bar f(a',s(a')) | a'<a }\Big).
  \end{equation*}

  Now since $<$ is well-founded, we have a function $w:\prd{a:A} \acc(a)$.
  Thus, we can define $f(a)\defeq \bar f (a,w(a))$.
\end{proof}

In classical logic, well-foundedness has a more well-known reformulation.

\begin{lem}\label{thm:wfmin}
  Assuming excluded middle, $<$ is well-founded if and only if every nonempty subset $B\subseteq A$ merely has a minimal element.
\end{lem}
\begin{proof}
  Suppose first $<$ is well-founded, and suppose $B\subseteq A$ is a subset with no minimal element.
  That is, for any $a:A$ with $a\in B$, there merely exists a $b:A$ with $b<a$ and $b\in B$.

  We claim that for any $a:A$ and $s:\acc(a)$, we have $a\notin B$.
  By induction, we may assume $s$ is a function assigning to each $a'<a$ a proof $s(a'):\acc(a)$, and that moreover for each such $a'$ we have $a'\notin B$.
  If $a\in B$, then by assumption, there would merely exist a $b<a$ with $b\in B$, which contradicts this assumption.
  Thus, $a\notin B$; this completes the induction.
  Since $<$ is well-founded, we have $a\notin B$ for all $a:A$, i.e. $B$ is empty.

  Now suppose each nonempty subset merely has a minimal element.
  Let $B = \setof{ a:A | \neg \acc(a) }$.
  Then if $B$ is nonempty, it merely has a minimal element.
  Thus there merely exists an $a:A$ with $a\in B$ such that for all $b<a$, we have $\acc(b)$.
  But then by definition (and induction on truncation), $a$ is merely accessible, and hence accessible, contradicting $a\in B$.
  Thus, $B$ is empty, so $<$ is well-founded.
\end{proof}

\begin{defn}
  A well-founded relation $<$ on a set $A$ is \textbf{extensional} if for any $a,b:A$, we have
  \[ \Big(\prd{c:A} (c<a) \leftrightarrow (c<b)\Big) \to (a=b). \]
\end{defn}

Note that since $A$ is a set, extensionality is a mere proposition.
This notion of ``extensionality'' is unrelated to function extensionality, and also unrelated to the extensionality of identity types.
Rather, it is a ``local'' counterpart of the axiom of extensionality in classical set theory.

\begin{thm}
  The type of extensional well-founded relations is a set.
\end{thm}
\begin{proof}
  By the univalence axiom, it suffices to show that if $(A,<)$ is extensional and well-founded and $f:(A,<) \cong (A,<)$, then $f=\idfunc[A]$.
  We prove by induction on $<$ that $f(a)=a$ for all $a:A$.
  The inductive hypothesis is that for all $a'<a$, we have $f(a')=a'$.

  Now since $A$ is extensional, to conclude $f(a)=a$ it is sufficient to show
  \[\prd{c:A}(c<f(a)) \leftrightarrow (c<a).\]
  However, since $f$ is an automorphism, we have $(c<a) \leftrightarrow (f(c)<f(a))$.
  But $c<a$ implies $f(c)=c$ by the induction hypothesis, so $(c<a) \to (c<f(a))$.
  On the other hand, if $c<f(a)$, then $f^{-1}(c)<a$, and so $c = f(f^{-1}(c)) = f^{-1}(c)$ by the induction hypothesis again; thus $c<a$.
  Therefore, we have $(c<a) \leftrightarrow (c<f(a))$ for any $c:A$, so $f(a)=a$.
\end{proof}

\begin{defn}\label{def:simulation}
  If $(A,<)$ and $(B,<)$ are extensional and well-founded, a \textbf{simulation} is a function $f:A\to B$ such that
  \begin{enumerate}
  \item if $a<a'$, then $f(a)<f(a')$, and\label{item:sim1}
  \item for all $a:A$ and $b:B$, if $b<f(a)$, then there merely exists an $a'<a$ with $f(a')=b$.\label{item:sim2}
  \end{enumerate}
\end{defn}

\begin{lem}
  Any simulation is injective.
\end{lem}
\begin{proof}
  We prove by double well-founded induction that for any $a,b:A$, if $f(a)=f(b)$ then $a=b$.
  The induction hypothesis for $a:A$ says that for any $a'<a$, and any $b:B$, if $f(a')=f(b)$ then $a=b$.
  The inner induction hypothesis for $b:A$ says that for any $b'<b$, if $f(a')=f(b')$ then $a'=b'$.

  Suppose $f(a)=f(b)$; we must show $a=b$.
  By extensionality, it suffices to show that for any $c:A$ we have $(c<a)\leftrightarrow (c<b)$.
  If $c<a$, then $f(c)<f(a)$ by \autoref{def:simulation}\ref{item:sim1}.
  Hence $f(c)<f(b)$, so by \autoref{def:simulation}\ref{item:sim2} there merely exists $c':A$ with $c'<b$ and $f(c)=f(c')$.
  By the induction hypothesis for $a$, we have $c=c'$, hence $c<b$.
  The dual argument is symmetrical.
\end{proof}

In particular, this implies that the word ``merely'' in \autoref{def:simulation}\ref{item:sim2} could be omitted without change of sense.

\begin{cor}
  If $f:A\to B$ is a simulation, then for all $a:A$ and $b:B$, if $b<f(a)$, there \emph{purely} exists an $a'<a$ with $f(a')=b$.
\end{cor}
\begin{proof}
  Since $f$ is injective, $\sm{a:A} (f(a)=b)$ is a mere proposition.
\end{proof}

We say that a subset $C :\mathcal{P}B$ is an \textbf{initial segment} if $c\in C$ and $b<c$ imply $b\in C$.
The image of a simulation must be an initial segment, while the inclusion of any initial segment is a simulation.
Thus, by univalence, every simulation $A\to B$ is \emph{equal} to the inclusion of some initial segment of $B$.

\begin{thm}
  For a set $A$, let $P(A)$ be the type of extensional well-founded relations on $A$.
  If $\mathord{<_A} : P(A)$ and $\mathord{<_B} : P(B)$ and $f:A\to B$, let $H_{\mathord{<_A}\mathord{<_B}}(f)$ be the mere proposition that $f$ is a simulation.
  Then $(P,H)$ is a standard notion of structure over \uset in the sense of \autoref{sec:sip}.
\end{thm}
\begin{proof}
  We leave it to the reader to verify that identities are simulations, and that composites of simulations are simulations.
  Thus, we have a notion of structure.
  For standardness, we must show that if $<$ and $\prec$ are two extensional well-founded relations on $A$, and $\idfunc[A]$ is a simulation in both directions, then $<$ and $\prec$ are equal.
  Since extensionality and well-foundedness are mere propositions, for this it suffices to have $\prd{a,b:A} (a<b) \leftrightarrow (a\prec b)$.
  But this follows from \autoref{def:simulation}\ref{item:sim1} for $\idfunc[A]$.
\end{proof}

\begin{cor}\label{thm:wfcat}
  There is a category whose objects are sets equipped with extensional well-founded relations, and whose morphisms are simulations.
\end{cor}

In fact, this category is a poset.

\begin{lem}
  For extensional and well-founded $(A,<)$ and $(B,<)$, there is at most one simulation $f:A\to B$.
\end{lem}
\begin{proof}
  Suppose $f,g:A\to B$ are simulations.
  Since being a simulation is a mere property, it suffices to show $\prd{a:A}(f(a)=g(a))$.
  By induction on $<$, we may suppose $f(a')=g(a')$ for all $a'<a$.
  And by extensionality of $B$, to have $f(a)=g(a)$ it suffices to have $\prd{b:B}(b<f(a)) \leftrightarrow (b<g(a))$.

  But since $f$ is a simulation, if $b<f(a)$, then we have $a'<a$ with $f(a')=b$.
  By the inductive hypothesis, we have also $g(a')=b$, hence $b<g(a)$.
  The dual argument is symmetrical.
\end{proof}

Thus, if $A$ and $B$ are equipped with extensional and well-founded relations, we may write $A\le B$ to mean there exists a simulation $f:A\to B$.
\autoref{thm:wfcat} implies that if $A\le B$ and $B\le A$, then $A=B$.

\begin{defn}
  An \textbf{ordinal} is a set $A$ with an extensional well-founded relation which is \emph{transitive}, i.e.\ $\prd{a,b,c:A}(a<b)\to (b<c) \to (a<c)$.
\end{defn}

\begin{eg}
  Of course, the usual strict order on \nat is transitive.
  It is easily seen to be extensional as well; thus it is an ordinal.
  As usual, we denote this ordinal by $\omega$.
\end{eg}

Let \ord denote the type of ordinals.
By the previous results, \ord is a set and has a natural partial order.
We now show that \ord also admits a well-founded relation.

If $A$ is an ordinal and $a:A$, let $\ordsl A a$ denote the initial segment $\setof{ b:A | b<a}$.
Note that if $\ordsl A a = \ordsl A b$ as ordinals, then that isomorphisms must respect their inclusions into $A$ (since simulations form a poset), and hence they are equal as subsets of $A$.
Therefore, since $A$ is extensional, $a=b$.
Thus the function $a\mapsto \ordsl A a$ is an injection $A\to \ord$.

\begin{defn}
  For ordinals $A$ and $B$, a simulation $f:A\to B$ is said to be \textbf{bounded} if there exists $b:B$ such that $A = \ordsl B b$.
\end{defn}

The remarks above imply that such a $b$ is unique when it exists, so that boundedness is a mere property.

We write $A<B$ if there exists a bounded simulation from $A$ to $B$.
Since simulations are unique, $A<B$ is also a mere proposition.

\begin{thm}\label{thm:ordord}
  $(\ord,<)$ is an ordinal.
\end{thm}

\begin{rmk}
  Note the use of universe polymorphism and typical ambiguity.
  If universe levels were made explicit, this theorem would state that the set of ordinals in one universe is an ordinal in the next higher universe.
\end{rmk}

\begin{proof}
  Let $A$ be an ordinal; we first show that $\ordsl A a$ is accessible (in \ord) for all $a:A$.
  By induction, suppose $\ordsl A b$ is accessible for all $b:A$.
  By definition of accessibility, we must show that $B$ is accessible in \ord for all $B<\ordsl A a$.
  However, if $B<\ordsl A a$ then there is some $b<a$ such that $B = \ordsl{(\ordsl A a)}{b} = \ordsl A b$, which is accessible by the inductive hypothesis.
  Thus, $\ordsl A a$ is accessible for all $a:A$.

  Now to show that $A$ is accessible in \ord, by definition we must show $B$ is accessible for all $B<A$.
  But as before, $B<A$ means $B=\ordsl A a$ for some $a:A$, which is accessible as we just proved.
  Thus, \ord is well-founded.

  For extensionality, suppose $A$ and $B$ are ordinals such that $\prd{C:\ord} (C<A) \leftrightarrow (C<B)$.
  Then for every $a:A$, since $\ordsl A a<A$, we have $\ordsl A a<B$, hence there is $b:B$ with $\ordsl A a = \ordsl B b$.
  Define $f:A\to B$ to take each $a$ to the corresponding $b$; it is straightforward to verify that $f$ is an isomorphism.
  Thus $A\cong B$, hence $A=B$ by univalence.

  Finally, it is easy to see that $<$ is transitive.
\end{proof}

Treating \ord as an ordinal is often very convenient, but it has its pitfalls as well.
For instance, consider the following lemma, for whose statement we drop briefly into \emph{explicit} universe polymorphism.

\begin{lem}\label{thm:ordsucc}
  Let \bbU be a universe.
  For any $A:\ord_\bbU$, there is a $B:\ord_\bbU$ such that $A<B$.
\end{lem}
\begin{proof}
  Let $B=A+\unit$, with the element $\star:\unit$ being greater than all elements of $A$.
  Then $B$ is an ordinal and it is easy to see that $A\cong \ordsl B \star$.
\end{proof}

This lemma illustrates a potential pitfall of typical ambiguity.
Consider the following alternative proof of it.

\begin{proof}[Another putative proof of \autoref{thm:ordsucc}]
  Note that $C<A$ if and only if $C=\ordsl A a$ for some $a:A$.
  This gives an isomorphism $A \cong \ordsl \ord A$, so that $A<\ord$.
  Thus we may take $B\defeq\ord$.
\end{proof}

The second proof would be valid if we had stated \autoref{thm:ordsucc} in a typically ambiguous style.
But the resulting lemma would be less useful, because the second proof would constrain the second ``\ord'' in the lemma statement to refer to a higher universe level than the first one.
The first proof allows both universes to be the same.

Similar remarks apply to the next lemma, which could be proved in a less useful way by observing that $A\le \ord$ for any $A:\ord$.

\begin{lem}\label{thm:ordunion}
  Let \bbU be a universe.
  For any $X:\type_\bbU$ and $F:X\to \ord_\bbU$, there exists $B:\ord_\bbU$ such that $Fx\le B$ for all $x:X$.
\end{lem}
\begin{proof}
  Let $B$ be the quotient of the equivalence relation on $\sm{x:X} Fx$ defined as follows:
  \[ (x,y) \sim (x',y')
  \;\defeq\;
  \Big(\ordsl{(Fx)}{y} \cong \ordsl{(Fx')}{y'}\Big).
  \]
  Define $(x,y)<(x',y')$ if $\ordsl{(Fx)}{y} < \ordsl{(Fx')}{y'}$.
  This clearly descends to the quotient, and can be seen to make $B$ into an ordinal.
  Moreover, for each $x:X$ the induced map $Fx\to B$ is a simulation.
\end{proof}



\section{Classical well-orderings}
\label{sec:wellorderings}

We now show the equivalence of our ordinals with the more familiar classical well-orderings.

\begin{lem}
  Assuming excluded middle, every ordinal is trichotomous:
  \[ \prd{a,b:A} (a<b) \vee (a=b) \vee (b<a). \]
\end{lem}
\begin{proof}
  By induction on $a$, we may assume that for every $a'<a$ and every $b':A$, we have $(a'<b') \vee (a'=b') \vee (b'<a')$.
  Now by induction on $b$, we may assume that for every $b'<b$, we have $(a<b') \vee (a=b') \vee (b'<a)$.

  By excluded middle, either there merely exists a $b'<b$ such that $a<b'$, or there merely exists a $b'<b$ such that $a=b'$, or for every $b'<b$ we have $b'<a$.
  In the first case, merely $a<b$ by transitivity, hence $a<b$ as it is a mere proposition.
  Similarly, in the second case, $a<b$ by transport.
  Thus, suppose $\prd{b':A}(b'<b)\to (b'<a)$.

  Now analogously, either there merely exists $a'<a$ such that $b<a'$, or there merely exists $a'<a$ such that $a'=b$, or for every $a'<a$ we have $a'<b$.
  In the first and second cases, $b<a$, so we may suppose $\prd{a':A}(a'<a)\to (a'<b)$.
  However, by extensionality, our two suppositions now imply $a=b$.
\end{proof}

\begin{lem}
  A well-founded relation contains no cycles, i.e.\
  \[ \prd{n:\mathbb{N}}{a:\mathbb{N}_n\to A} \neg\Big((a_0<a_1) \wedge \dots \wedge (a_{n-1}<a_n)\wedge (a_n<a_0)\Big). \]
\end{lem}
\begin{proof}
  We prove by induction on $a:A$ that there is no cycle containing $a$.
  Thus, suppose by induction that for all $a'<a$, there is no cycle containing $a'$.
  But in any cycle containing $a$, there is some element less than $a$ and contained in the same cycle.
\end{proof}

\begin{thm}\label{thm:wellorder}
  Assuming excluded middle, $(A,<)$ is an ordinal if and only if every nonempty subset $B\subseteq A$ has a least element.
\end{thm}
\begin{proof}
  If $A$ is an ordinal, then by \autoref{thm:wfmin} every nonempty subset merely has a minimal element.
  But trichotomy implies that any minimal element is a least element.
  Moreover, least elements are unique when they exist, so merely having one is as good as having one.

  Conversely, if every nonempty subset has a least element, then $A$ is well-founded by \autoref{thm:wfmin}.
  We also have trichotomy, since for any $a,b$ the set $\setof{a,b}$ merely has a least element, which must be either $a$ or $b$.
  This implies transitivity, since if $a<b$ and $a<c$, then either $a=c$ or $c<a$ would produce a cycle.
  Similarly, it implies extensionality, for if $\prd{c:A}(c<a)\leftrightarrow (c<b)$, then $a<b$ implies (letting $c$ be $a$) that $a<a$, which is a cycle, and similarly if $b<a$; hence $a=b$.
\end{proof}

In classical mathematics, the characterization of \autoref{thm:wellorder} is taken as the definition of a \textbf{well-ordering}, with the \emph{ordinals} being a canonical set of representatives of isomorphism classes for well-orderings.
In our context, the Structure Identity Principle means that there is no need to look for such representatives: any well-ordering is as good as any other.

We now move on to consider consequences of the axiom of choice.

\begin{thm}\label{thm:wop}
  Assuming excluded middle, the following are equivalent.
  \begin{enumerate}
  \item For every set $X$, there merely exists a function
    $ f: \mathcal{P}_+X \to X $
    such that $f(Y)\in Y$ for all $Y:\mathcal{P}X$.\label{item:wop1}
  \item Every set merely admits the structure of an ordinal.\label{item:wop2}
  \end{enumerate}
\end{thm}

Of course,~\ref{item:wop1} is a standard classical version of the axiom of choice.

\begin{proof}
  One direction is easy: suppose~\ref{item:wop2}.
  Since we aim to prove the mere proposition~\ref{item:wop1}, we may assume $A$ is an ordinal.
  But then we can define $f(B)$ to be the least element of $B$.

  Now suppose~\ref{item:wop1}.
  As before, since~\ref{item:wop2} is a mere proposition, we may assume given such an $f$.
  We extend $f$ to a function
  \[ \bar f:\mathcal{P}X \cong (\mathcal{P}_+ X) + \unit \longrightarrow X+\unit
  \]
  in the obvious way.
  Now for any ordinal $A$, we can define $g_A:A\to X+\unit$ by well-founded recursion:
  \[ g_A(a) \defeq 
    \bar f\Big(X \setminus \setof{ g_A(b) | \rule{0pt}{1em} (b<a) \wedge (g_A(b) \in X) }\Big)
  \]
  (regarding $X$ as a subset of $X+\unit$ in the obvious way).

  Let $A'$ be the preimage of $X$; then we claim the restriction $g_A':A' \to X$ is injective.
  For if $a,a':A$ with $a\neq a'$, then by trichotomy and without loss of generality, we may assume $a'<a$.
  Thus $g_A(a') \in \setof{ g_A(b) | b<a }$, so since $f(Y)\in Y$ for all $Y$ we have $g_A(a) \neq g_A(a')$.

  Moreover, $A'$ is an initial segment of $A$.
  For $g_A(a)$ lies in \unit if and only if $\setof{g_A(b)|b<a} = X$, and if this holds then it also holds for any $a'>a$.
  Thus, $A'$ is itself an ordinal.

  Finally, since \ord is an ordinal, we can take $A\defeq\ord$.
  Let $X'$ be the image of $g_\ord':\ord' \to X$; then the inverse of $g_\ord'$ yields an injection $H:X'\to \ord$.
  By \autoref{thm:ordunion}, there is an ordinal $C$ such that $Hx\le C$ for all $x:X'$.
  Then by \autoref{thm:ordsucc}, there is a further ordinal $D$ such that $C<D$, hence $Hx<D$ for all $x:X'$.
  Now we have
  \begin{align*}
    g_{\ord}(D) &= \bar f\Big( X \setminus \setof{ g_\ord(B) | \rule{0pt}{1em} B<D \wedge (g_\ord(B) \in X)} \Big)\\
    &=\bar f\Big( X \setminus \setof{ g_\ord(B) | \rule{0pt}{1em} B:\ord \wedge (g_\ord(B) \in X)} \Big)
  \end{align*}
  since if $B:\ord$ and $(g_\ord(B) \in X)$, then $B = Hx$ for some $x:X'$, hence $B<D$.
  Now if
  \[\setof{ g_\ord(B) | \rule{0pt}{1em} B:\ord \wedge (g_\ord(B) \in X)}\]
  is not all of $X$, then $g_\ord(D)$ would lie in $X$ but not in this subset, which would be a contradiction since $D$ is itself a potential value for $B$.
  So this set must be all of $X$, and hence $g_\ord'$ is surjective as well as injective.
  Thus, we can transport the ordinal structure on $\ord'$ to $X$.
\end{proof}

\begin{rmk}
  If we had given the wrong proof of \autoref{thm:ordsucc} or \autoref{thm:ordunion}, then the resulting proof of \autoref{thm:wop} would be invalid: there would be no way to consistently assign universe levels.
\end{rmk}

\begin{cor}
  Assuming the axiom of choice, the function $\ord\to\set$ (which forgets the order structure) is a surjection.
\end{cor}

Note that \ord is a set, while \set is a 1-type.
In general, there is no reason for a 1-type to admit any surjective function from a set.
Even the axiom of choice does not appear to imply that \emph{every} 1-type does so, but it readily implies that this is so for 1-types constructed out of \set, such as the types of objects of categories of structures as in \autoref{sec:sip}.
The following corollary also applies to such categories.

\begin{cor}
  Assuming AC, \uset admits a weak equivalence functor from a strict category.
\end{cor}
\begin{proof}
  Let $X_0\defeq \ord$, and for $A,B:X_0$ let $\hom_X(A,B) \defeq (A\to B)$.
  Then $X$ is a strict category, since \ord is a set, and the above surjection $X_0 \to \set$ extends to a weak equivalence functor $X\to \uset$.
\end{proof}

Now recall from \autoref{sec:cardinals} that we have a further surjection $\cd{-}:\set\to\card$, and hence a composite surjection $\ord\to\card$ which sends each ordinal to its cardinality.

\begin{thm}
  Assuming AC, the surjection $\ord\to\card$ has a section.
\end{thm}
\begin{proof}
  There is an easy and wrong proof of this: since \ord and \card are both sets, AC implies that any surjection between them \emph{merely} has a section.
  However, we actually have a canonical \emph{specified} section: because \ord is an ordinal, every nonempty subset of it has a uniquely specified least element.
  Thus, we can map each cardinal to the least element in the corresponding fiber.
\end{proof}

It is traditional in set theory to identify cardinals with their image in \ord: the least ordinal having that cardinality.

It follows that \card also canonically admits the structure of an ordinal: in fact, one isomorphic to \ord.
Specifically, we define by well-founded recursion a function $\aleph:\ord\to\ord$, such that $\aleph(A)$ is the least ordinal having cardinality greater than $\aleph({\ordsl A a})$ for all $a:A$.
Then (assuming AC) the image of $\aleph$ is exactly the image of \card.

% Local Variables:
% TeX-master: "main"
% End:
