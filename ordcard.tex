\newcommand{\card}{\ensuremath{\mathsf{Card}}\xspace}
\newcommand{\cd}[1]{\left|#1\right|}
\newcommand{\inj}{\ensuremath{\mathsf{inj}}}

\section{Cardinal numbers}
\label{sec:cardinals}

\begin{defn}
  The \textbf{type of cardinal numbers} is the 0-truncation of \set:
  \[ \card \defeq \pizero{\set} \]
  Thus, a \textbf{cardinal number}, or \textbf{cardinal}, is an inhabitant of $\card\jdeq \pizero\set$.
\end{defn}

\begin{rmk}
  As in some previous sections, here we are being universe polymorphic and typically ambiguous.
  Technically, there is a separate type ``\card'' associated to each universe ``\type'', but with these conventions we can state theorems beginning with ``for all cardinal numbers\dots''\ and give them exactly the same sort of meaning as those beginning ``for all types\dots''.
\end{rmk}

If $A$ is a set, we write $\cd{A}$ for its image under the canonical projection $\set \to \card$.
Of course, by definition, \card is a set.
It also inherits the structure of a semiring from \set.

\begin{defn}
  The operation of \textbf{cardinal addition}
  \[ \blank+\blank : \card \to \card \to \card \]
  is defined by induction on truncation:
  \[ \cd{A} + \cd{B} \defeq \cd{A+B} .\]
\end{defn}
\begin{proof}
  Since $\card\to\card$ is a set, to define $\alpha+\blank:\card\to\card$ for all $\alpha:\card$, by induction it suffices to assume that $\alpha$ is $\cd{A}$ for some $A:\set$.
  Now we want to define $\cd{A}+\blank :\card\to\card$, i.e.\ we want to define $\cd{A}+\beta :\card$ for all $\beta:\card$.
  However, since $\card$ is a set, by induction it suffices to assume that $\beta$ is $\cd{B}$ for some $B:\set$.
  But now we can define $\cd{A}+\cd{B}$ to be $\cd{A+B}$.
\end{proof}

\begin{defn}
  Similarly, the operation of \textbf{cardinal multiplication}
  \[ \blank\cdot\blank : \card \to \card \to \card \]
  is defined by induction on truncation:
  \[ \cd{A} \cdot \cd{B} \defeq \cd{A\times B} \]
\end{defn}

\begin{lem}\label{card:semiring}
  \card is a commutative semiring, i.e.\ for $\alpha,\beta,\gamma:\card$ we have the following.
  \begin{align*}
    (\alpha+\beta)+\gamma &= \alpha+(\beta+\gamma)\\
    \alpha+0 &= \alpha\\
    \alpha + \beta &= \beta + \alpha\\
    (\alpha \cdot \beta) \cdot \gamma &= \alpha \cdot (\beta\cdot\gamma)\\
    \alpha \cdot 1 &= \alpha\\
    \alpha\cdot\beta &= \beta\cdot\alpha\\
    \alpha\cdot(\beta+\gamma) &= \alpha\cdot\beta + \alpha\cdot\gamma
  \end{align*}
  where $0 \defeq \cd{\emptyset}$ and $1\defeq\cd{\unit}$.
\end{lem}
\begin{proof}
  We prove the commutativity of multiplication, $\alpha\cdot\beta = \beta\cdot\alpha$; the others are exactly analogous.
  Since \card is a set, the type $\alpha\cdot\beta = \beta\cdot\alpha$ is a mere proposition, and in particular a set.
  Thus, by induction it suffices to assume $\alpha$ and $\beta$ are of the form $\cd{A}$ and $\cd{B}$ respectively, for some $A,B:\set$.
  Now $\cd{A}\cdot \cd{B} \jdeq \cd{A\times B}$ and $\cd{B}\times\cd{A} \jdeq \cd{B\times A}$, so it suffices to show $A\times B = B\times A$.
  Finally, by univalence, it suffices to give an equivalence $A\times B \simeq B\times A$.
  But this is easy: take $(a,b) \mapsto (b,a)$ and its obvious inverse.
\end{proof}

\begin{defn}
  The operation of \textbf{cardinal exponentiation} is also defined by induction on truncation:
  \[ \cd{A}^{\cd{B}} \defeq \cd{B\to A}. \]
\end{defn}

\begin{lem}\label{card:exp}
  For $\alpha,\beta,\gamma:\card$ we have
  \begin{align*}
    \alpha^0 &= 1\\
    1^\alpha &= 1\\
    \alpha^1 &= \alpha\\
    \alpha^{\beta+\gamma} &= \alpha^\beta \cdot \alpha^\gamma\\
    \alpha^{\beta\cdot \gamma} &= (\alpha^{\beta})^\gamma\\
    (\alpha\cdot\beta)^\gamma &= \alpha^\gamma \cdot \beta^\gamma
  \end{align*}
\end{lem}
\begin{proof}
  Exactly like \autoref{card:semiring}.
\end{proof}

\begin{defn}
  The relation of \textbf{cardinal inequality}
  \[ \blank\le\blank : \card\to\card\to\prop \]
  is defined by induction on truncation:
  \[ \cd{A} \le \cd{B} \defeq \brck{\inj(A,B)} \]
  where $\inj(A,B)$ is the type of injections from $A$ to $B$.
  In other words, $\cd{A} \le \cd{B}$ means that there merely exists an injection from $A$ to $B$.
\end{defn}

\begin{lem}
  Cardinal inequality is a preorder, i.e.\ for $\alpha,\beta:\card$ we have
  \begin{gather*}
    \alpha \le \alpha\\
    (\alpha \le \beta) \to (\beta\le\gamma) \to (\alpha\le\gamma)
  \end{gather*}
\end{lem}
\begin{proof}
  As before, by induction on truncation.
  For instance, since $(\alpha \le \beta) \to (\beta\le\gamma) \to (\alpha\le\gamma)$ is a mere proposition, by induction on 0-truncation we may assume $\alpha$, $\beta$, and $\gamma$ are $\cd{A}$, $\cd{B}$, and $\cd{C}$ respectively.
  Now since $\cd{A} \le \cd{C}$ is a mere proposition, by induction on $(-1)$-truncation we may assume given injections $f:A\to B$ and $g:B\to C$.
  But then $g\circ f$ is an injection from $A$ to $C$, so $\cd{A} \le \cd{C}$ holds.
  Reflexivity is even easier.
\end{proof}

We may likewise show that cardinal inequality is compatible with the semiring operations.

\begin{lem}\label{thm:injsurj}
  Consider the following statements:
  \begin{enumerate}
  \item There is an injection $A\to B$.\label{item:cle-inj}
  \item There is a surjection $B\to A$.\label{item:cle-surj}
  \end{enumerate}
  Then, assuming excluded middle:
  \begin{itemize}
  \item Given $a_0:A$, we have~\ref{item:cle-inj}$\to$\ref{item:cle-surj}.
  \item Therefore, if $A$ is merely inhabited, we have~\ref{item:cle-inj} $\to$ merely \ref{item:cle-surj}.
  \item Assuming the axiom of choice, we have~\ref{item:cle-surj} $\to$ merely \ref{item:cle-inj}.
  \end{itemize}
\end{lem}
\begin{proof}
  If $f:A\to B$ is an injection, define $g:B\to A$ at $b:B$ as follows.
  Since $f$ is injective, the fiber of $f$ at $b$ is a mere proposition.
  Therefore, by excluded middle, either there is an $a:A$ with $f(a)=b$, or not.
  In the first case, define $g(b)\defeq a$; otherwise set $g(b)\defeq a_0$.
  Then for any $a:A$, we have $a = g(f(a))$, so $g$ is surjective.

  The second statement follows from this by induction on truncation.
  For the third, if $g:B\to A$ is surjective, then by the axiom of choice, there merely exists a function $f:A\to B$ with $g(f(a)) = a$ for all $a$.
  But then $f$ must be injective.
\end{proof}

\begin{thm}[Schroeder-Bernstein]
  Assuming excluded middle, for sets $A$ and $B$ we have
  \[ \inj(A,B) \to \inj(B,A) \to (A\cong B) \]
\end{thm}
\begin{proof}
  The usual ``back-and-forth'' argument applies without significant changes.
  Note that it actually constructs an isomorphism $A\cong B$ (assuming excluded middle so that we can decide whether a given element belongs to a cycle, an infinite chain, a chain beginning in $A$, or a chain beginning in $B$).
\end{proof}

\begin{cor}
  Assuming excluded middle, cardinal inequality is a partial order, i.e.\ for $\alpha,\beta:\card$ we have
  \[ (\alpha\le\beta) \to (\beta\le\alpha) \to (\alpha=\beta). \]
\end{cor}
\begin{proof}
  Since $\alpha=\beta$ is a mere proposition, by induction on truncation we may assume $\alpha$ and $\beta$ are $\cd{A}$ and $\cd{B}$, respectively, and that we have injections $f:A\to B$ and $g:B\to A$.
  But then the Schroeder-Bernstein theorem gives an isomorphism $A\simeq B$, hence an equality $\cd{A}=\cd{B}$.
\end{proof}

Finally, we can reproduce Cantor's theorem, showing that for every cardinal there is a greater one.

\begin{thm}[Cantor]
  For $A:\set$, there is no surjection $A \to (A\to \mathbf{2})$.
\end{thm}
\begin{proof}
  Suppose $f:A \to (A\to \mathbf{2})$ is any function, and define $g:A\to \mathbf{2}$ by $g(a) \defeq \neg f(a)(a)$.
  If $g = f(a_0)$, then $g(a_0) = f(a_0)(a_0)$ but $g(a_0) = \neg f(a_0)(a_0)$, a contradiction.
  Thus, $f$ is not surjective.
\end{proof}

\begin{cor}
  Assuming excluded middle, for any $\alpha:\card$, there is a cardinal $\beta$ such that $\alpha\le\beta$ and $\alpha\neq\beta$.
\end{cor}
\begin{proof}
  Let $\beta = 2^\alpha$.
  Now we want to show a mere propositon, so by induction we may assume $\alpha$ is $\cd{A}$, so that $\beta\jdeq \cd{A\to \mathbf{2}}$.
  Using excluded middle, we have a function $f:A\to (A\to \mathbf{2})$ defined by
  \[f(a)(a') \defeq
  \begin{cases}
    \top &\quad a=a'\\
    \bot &\quad a\neq a'.
  \end{cases}
  \]
  And if $f(a)=f(a')$, then $f(a')(a) = f(a)(a) = \top$, so $a=a'$; hence $f$ is injective.
  Thus, $\alpha \jdeq \cd{A} \le \cd{A\to \mathbf{2}} \jdeq 2^\alpha$.

  On the other hand, if $2^\alpha \le \alpha$, then we would have an injection $(A\to\mathbf{2})\to A$.
  By \autoref{thm:injsurj}, since we have $(\lambda x.\bot):A\to \mathbf{2}$ and excluded middle, there would then be a surjection $A \to (A\to \mathbf{2})$, contradicting Cantor's theorem.
\end{proof}

\section{Ordinal numbers}
\label{sec:ordinals}

\newcommand{\acc}{\ensuremath{\mathsf{acc}}}

\begin{defn}
  Let $A$ be a set and
  \[\blank<\blank:A\to A\to \prop\]
  a binary relation on $A$.
  We define by induction what it means for an element $a:A$ to be \textbf{accessible} by $<$:
  \begin{itemize}
  \item If $b$ is accessible for every $b<a$, then $a$ is accessible.
  \end{itemize}
  We write $\acc(a)$ to mean that $a$ is accessible.
\end{defn}

It may seem like such an inductive definition can never get off the ground, but of course if $a$ has the property that there are \emph{no} $b$ such that $b<a$, then $a$ is vacuously accessible.

\begin{lem}
  Accessibility is a mere property.
\end{lem}
\begin{proof}
  We claim that for any $a_1,a_2:A$, any $p:a_1=a_2$, and any $s_1:\acc(a_1)$ and $s_2:\acc(a_2)$, we have $\trans{p}{s_1}=s_2$.
  By induction, we may assume that $s_1$ is given by a function assigning to each $b_1<a_1$ a proof $s_1(b_1):\acc(b_1)$, and moreover that for any $b_2:A$, any $q:b_1=b_2$, and $t_2:\acc(b_2)$, we have $\trans{q}{s_1(b_1)}=t_2$.
  Similarly, we may assume that $s_2$ is given by a function assigning to each $b_2<a_2$ a proof $s_2(b_2):\acc(b_2)$, and that for any $b_1:A$, any $q:b_1=b_2$, and $t_1:\acc(b_1)$, we have $\trans{q}{t_1}=s_2(b_2)$.
  
  Now by function extensionality, to show $\trans{p}{s_1}=s_2$ it suffices to show that for any $b_2<a_2$ we have $\trans{p}{s_1}(b_2) = s_2(b_2)$.
  However, we can obtain this from the induction hypothesis for $s_2$ with $q\defeq \refl{b_2}$ and $t_1 \defeq \trans{p}{s_1}(b_2)$.
  This proves the claim.

  Finally, we instantiate the claim with $a_2\defeq a_1\defeq a$ and $p\defeq \refl{a}$.
  Thus, for any $a:A$ and $s_1,s_2:\acc(a)$, we have $s_1=s_2$, as desired.
\end{proof}

\begin{defn}
  A binary relation $<$ on a set $A$ is \textbf{well-founded} if every element of $A$ is accessible.
\end{defn}

\begin{lem}
  Well-foundedness is a mere property.
\end{lem}
\begin{proof}
  Well-foundedness of $<$ is the type $\prd{a:A} \acc(a)$, which is a mere proposition since each $\acc(a)$ is.
\end{proof}

Well-foundedness allows us to define functions by recursion and prove statements by induction, such as for instance the following.

\begin{lem}
  Suppose $B$ is a set and we have a function
  \[ g : \mathcal{P}B \to B \]
  Then if $<$ is a well-founded relation on $A$, there is a function $f:A\to B$ such that for all $a:A$ we have
  \begin{equation*}
    f(a) = g\Big(\big\{ f(a') \;\big|\; a'<a \big\}\Big).
  \end{equation*}
\end{lem}
\begin{proof}
  We first define, for every $a:A$ and $s:\acc(a)$, an element $\bar f(a,s):B$.
  By induction, it suffices to assume that $s$ is a function assigning to each $a'<a$ a proof $s(a'):\acc(a')$, and that moreover for each such $a'$ we have an element $\bar f(a',s(a')):B$.
  In this case, we define
  \begin{equation*}
    \bar f(a,s) \defeq g\Big(\big\{ \bar f(a',s(a')) \;\big|\; a'<a \big\}\Big).
  \end{equation*}

  Now since $<$ is well-founded, we have a function $w:\prd{a:A} \acc(a)$.
  Thus, we can define $f(a)\defeq \bar f (a,w(a))$.
\end{proof}

In classical logic, well-foundedness has a more well-known reformulation.

\begin{lem}
  Assuming excluded middle, $<$ is well-founded if and only if every nonempty subset $B\subseteq A$ merely has a minimal element.
\end{lem}
\begin{proof}
  Suppose first $<$ is well-founded, and suppose $B\subseteq A$ is a subset with no minimal element.
  That is, for any $a:A$ with $a\in B$, there merely exists a $b:A$ with $b<a$ and $b\in B$.

  We claim that for any $a:A$ and $s:\acc(a)$, we have $a\notin B$.
  By induction, we may assume $s$ is a function assigning to each $a'<a$ a proof $s(a'):\acc(a)$, and that moreover for each such $a'$ we have $a'\notin B$.
  If $a\in B$, then by assumption, there would merely exist a $b<a$ with $b\in B$, which contradicts this assumption.
  Thus, $a\notin B$; this completes the induction.
  Since $<$ is well-founded, we have $a\notin B$ for all $a:A$, i.e. $B$ is empty.

  Now suppose each nonempty subset merely has a minimal element.
  Let $B = \{ a:A \;|\; \neg \acc(a) \}$.
  Then if $B$ is nonempty, it has a minimal element.
  Thus there merely exists an $a:A$ with $a\in B$ such that for all $b<a$, we have $\acc(b)$.
  But then by definition (and induction on truncation), $a$ is merely accessible, and hence accessible, contradicting $a\in B$.
  Thus, $B$ is empty, so $<$ is well-founded.
\end{proof}

\begin{defn}
  An \textbf{ordinal} is a set $A$ equipped with a well-founded and transitive relation.
\end{defn}


% Local Variables:
% TeX-master: "main"
% End:
