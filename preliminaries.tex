\chapter{Type theory}
\label{cha:typetheory}

\section{Type theory versus set theory}
\label{sec:types-vs-sets}

Homotopy type theory is (among other things) a foundational language for mathematics, i.e., an alternative to Zermelo-Fraenkel set theory.
However, it behaves differently from set theory in several important ways, and that can take some getting used to.
Explaining these differences carefully requires us to be more formal here than we will be in the rest of the book.
As stated in the introduction, our goal is to write type theory \emph{informally}; but for a mathematician accustomed to set theory, more precision at the beginning can help avoid some common misconceptions and mistakes.

We note that a set-theoretic foundation has two ``layers'': the deductive system of first-order logic, and, formulated inside this system, the axioms of a particular theory, such as ZFC.
Thus, set theory is not only about sets, but rather about the interplay between sets (the objects of the second layer) and propositions (the objects of the first layer).

By contrast, type theory is its own deductive system: it need not be formulated inside any superstructure, such as first-order logic.
Instead of the two basic notions of set theory, sets and propositions, type theory has one basic notion: \emph{types}.
Propositions (statements which we can prove, disprove, assume, negate, and so on\footnote{Confusingly, it is also a common practice (dating back to Euclid) to use the word ``proposition'' synonomously with ``theorem''.
  We will confine ourselves to the logician's usage, according to which a \emph{proposition} is a statement \emph{susceptible to} proof, whereas a \emph{theorem} (or ``lemma'' or ``corollary'') is such a statement that \emph{has been} proven.
Thus ``$0=1$'' and its negation ``$\neg(0=1)$'' are both propositions, but only the latter is a theorem.}) are identified with particular types, via the correspondence shown in Table~\ref{tab:pov} on page~\pageref{tab:pov}.
Thus, the mathematical activity of \emph{proving a theorem} is identified with a special case of the mathematical activity of \emph{constructing an object}---in this case, an inhabitant of a type that represents a proposition.

This leads us to another difference between type theory and set theory, but to explain it we must say a little about deductive systems in general.
Informally, a deductive system is a collection of rules for deriving things called \emph{judgments}.
If we think of a deductive system as a formal game, then the judgments are the ``positions'' in the game which we reach by following the game rules.
We can also think of a deductive system as a sort of algebraic theory, in which case the judgments are the elements (like the elements of a group) and the deductive rules are the operations (like the group multiplication).
From a logical point of view, the judgments can be considered to be the ``external'' statements, living in the metatheory, as opposed to the ``internal'' statements of the theory itself.

In the deductive system of first-order logic (on which set theory is based), there is only one judgment: that a given proposition has a proof.
A rule of first-order logic such as ``from $A$ and $B$ infer $A\wedge B$'' is actually a rule of ``proof construction'' which says that given the judgments ``$A$ has a proof'' and ``$B$ has a proof'', we may deduce that ``$A\wedge B$ has a proof''.
Note that the judgment ``$A$ has a proof'' exists at a different level from the \emph{proposition} $A$ itself, which is an internal statement of the theory.
% In particular, we cannot manipulate it to construct propositions such as ``if $A$ has a proof, then $B$ does not have a proof''---unless we are using our set-theoretic foundation as a meta-theory with which to talk about some other axiomatic system.

The basic judgment of type theory, analogous to ``$A$ has a proof'', is written ``$a:A$'' and pronounced as ``the term $a$ has type $A$'', or more loosely ``$a$ is an element of $A$'' (or, in homotopy type theory, a ``point of $A$'').
When $A$ is a type representing a proposition, then $a$ may be called a \emph{witness} to the provability of $A$, or \emph{evidence} of the truth of $A$.
In this case, the judgment $a:A$ is derivable in type theory (for some $a$) precisely when the analogous judgment ``$A$ has a proof'' is derivable in first-order logic (modulo differences in the axioms assumed and in the encoding of mathematics, as we will discuss throughout the book).
% (Already here type theory has an advantage in that the term $a$ records the essence of the proof of $A$, whereas in first-order logic that information is only available by backtracking through the derivation of ``$A$ has a proof''.)

On the other hand, if the type $A$ is being treated more like a set than like a proposition (although as we will see, the distinction can become blurry), then ``$a:A$'' may be regarded as analogous to the set-theoretic statement ``$a\in A$''.
However, there is an essential difference in that ``$a:A$'' is a \emph{judgment} whereas ``$a\in A$'' is a \emph{proposition}.
In particular, when working internally in type theory, we cannot make statements such as ``if $a:A$ then it is not the case that $b:B$'', nor can we ``disprove'' the judgment ``$a:A$''.

A good way to think about this is that in set theory, ``membership'' is a relation which may or may not hold between two pre-existing objects ``$a$'' and ``$A$'', while in type theory we cannot talk about an element ``$a$'' in isolation: every element \emph{by its very nature} is an element of some type, and that type is (generally speaking) uniquely determined.
Thus, when we say informally ``let $x$ be a natural number'', in set theory this is shorthand for ``let $x$ be a thing and assume that $x\in\nat$'', whereas in type theory ``let $x:\nat$'' is an atomic statement: we cannot introduce a variable without specifying its type.

At first glance, this may seem an uncomfortable restriction, but it is arguably closer to the intuitive mathematical meaning of ``let $x$ be a natural number''.
In practice, it seems that whenever we actually \emph{need} ``$a\in A$'' to be a proposition rather than a judgment, there is always an ambient set $B$ of which $a$ is known to be an element and $A$ is known to be a subset.
This situation is also easy to represent in type theory, by taking $a$ to be an element of the type $B$, and $A$ to be a predicate on $B$; see \autoref{subsec:prop-subsets}.

A last difference between type theory and set theory is the treatment of equality.
The familiar notion of equality in mathematics is a proposition: e.g.\ we can disprove an equality or assume an equality as a hypothesis.
Since in type theory, propositions are types, this means that equality is a type: for elements $a,b:A$ (that is, both $a:A$ and $b:A$) we have a type ``$\id[A]ab$''.
(In \emph{homotopy} type theory, of course, this equality proposition can behave in unfamiliar ways: see \autoref{sec:identity-types}, \autoref{cha:basics}, and the rest of the book).

However, in type theory there is also a need for an equality \emph{judgment}, existing at the same level as the judgment ``$x:A$''.
This is called \emph{judgmental equality} or \emph{definitional equality}, and we write it as $a\jdeq b$ or $a \jdeq_A b$.
It is helpful to think of this as meaning ``equal by definition''.
For instance, if we define a function $f:\nat\to\nat$ by the equation $f(x)=x^2$, then the expression $f(3)$ is equal to $3^2$ \emph{by definition}.
It does not make sense to negate or assume an equality-by-definition; we cannot say ``if $x$ is equal to $y$ by definition, then $z$ is not equal to $w$ by definition''.
Whether or not two expressions are equal by definition is just a matter of expanding out the definitions; in particular, it is algorithmically decidable.

As type theory becomes more complicated, judgmental equality can get more subtle than this, but it is a good intuition to start from.
Alternatively, if we regard a deductive system as an algebraic theory, then judgmental equality is simply the equality in that theory, analogous to the equality between elements of a group---the only potential for confusion is that there is \emph{also} an object \emph{inside} the deductive system of type theory (namely the type ``$a=b$'') which behaves internally as a notion of ``equality''.

The reason we \emph{want} a judgmental notion of equality is so that it can control the other form of judgment, ``$a:A$''.
For instance, suppose we have given a proof that $3^2=9$, i.e.\ we have derived the judgment $p:(3^2=9)$ for some term $p$.
Then the same proof $p$ ought to count as a proof that $f(3)=9$, since $f(3)$ is $3^2$ \emph{by definition}.
The best way to represent this is with a rule saying that given the judgments $a:A$ and $A\jdeq B$, we can derive the judgment $a:B$.

Thus, for us, type theory will be a deductive system based on two forms of judgment:
\begin{center}
\begin{tabular}{c|l}
  \textbf{Judgment} & \textbf{Meaning}\\\hline
  $a : A$ & $a$ is an object of type $A$.\\
  $a \jdeq_A b$ & $a$ and $b$ are definitionally equal objects of type $A$.
\end{tabular}
\end{center}
When introducing a definitional equality, i.e., defining one thing is equal to another, we will use the symbol ``$\defeq$''.
Thus, the above definition of the function $f$ would be written as $f(x)\defeq x^2$.

Because judgments cannot be put together into more complicated statements, the symbols ``$:$'' and ``$\jdeq$'' bind more loosely than anything else.%
\footnote{In formalized type theory, commas and turnstiles can bind even more loosely.
  For instance, $x:A,y:B\vdash c:C$ is parsed as $((x:A),(y:B))\vdash (c:C)$.
  However, in this book we refrain from such notation until \autoref{cha:rules}.}\nopagebreak~
Thus, for instance, ``$p:x=y$'' should be parsed as ``$p:(x=y)$'', which makes sense since ``$x=y$'' is a type, and not as ``$(p:x)=y$'', which is senseless since ``$p:x$'' is a judgment and cannot be equal to anything.
Similarly, ``$A\jdeq x=y$'' can only be parsed as ``$A\jdeq(x=y)$'', although in extreme cases such as this, one ought to add parentheses anyway to aid reading comprehension.
This is perhaps also an appropriate place to mention that the common mathematical notation ``$f:A\to B$'', expressing the fact that $f$ is a function from $A$ to $B$, can be regarded as a typing judgment, if we use ``$A\to B$'' as notation for the type of functions from $A$ to $B$ (which is standard practice in type theory; see \autoref{sec:pi-types}).

Judgments may depend on assumptions of the form $x:A$, where $x$ is a variable and $A$ is a type.
For example, we may construct an object $m + n : \Nat$ under the assumptions that $m,n : \Nat$.
Another example is that assuming $A$ is a type, $x,y : A$, and $p : x =_A y$, we may construct an element $p^{-1} : y =_A x$.
The sequence of all such assumptions is called the \emph{context}; from a topological point of view it may be thought of as a ``parameter space''.
Later assumptions may depend on previous ones, as in the example: the assumption $x:A$ can only be made \textbf{after} the assumption $A$ is a type has been introduced.

If the type $A$ in an assumption $x:A$ represents a proposition, then the assumption is a type-theoretic version of a \emph{hypothesis}: we assume that the proposition $A$ holds.
When types are regarded as propositions, we may omit the names of their proofs.
Thus, in the second example above we may instead say that assuming $x =_A y$, we can prove $y =_A x$.
However, since we are doing proof-relevant mathematics, we will frequently refer back to proofs as objects.
In the example, we may want to establish that $p^{-1}$ together with the proofs of transitivity and reflexivity behave like a groupoid; see \autoref{cha:basics}.

In the rest of this chapter, we attempt to give an informal presentation of Type Theory, sufficient for the purposes of this book; we will give a more formal account in \autoref{cha:rules}.
Aside from some fairly obvious rules (such as the fact that judgmentally equal things can always be substituted for each other), the rules of type theory can be grouped into \emph{type formers}.
Each type former consists of a way to construct types (possibly making use of previously constructed types), together with rules for the construction and behavior of elements of that type.
In most cases, these rules follow a fairly predictable pattern, but we will not attempt to make this precise.


\section{Function types}
\label{sec:function-types}

Given types $A$ and $B$, we can construct the type $A \to B$ of functions with domain $A$
and codomain $B$. Unlike in set theory, functions are not defined as
functional relations; rather they a primitive concept in type theory.
We explain the function type by prescribing what we can do with functions, 
how to construct them and what equalities they induce.

Given a function $f : A \to B$ and an element of the domain $a : A$, we
can \textbf{apply} the function to obtain an element of the codomain,
denoted $f(a) : B$. 

But how can we construct elements of $A \to B$? There are two equivalent ways:
either by direct definition or by using
$\lambda$-abstraction. Introducing a function by definition means that
we introduce a function by giving it a name --- let's say, $f$ --- and saying
we define $f : A \to B$ by giving an equation
\[ f(x) \defeq b \]
where $x$ is a variable and $b$ is an expression which may use $x$.
In order for this to be valid, we have to check that $b : B$ assuming $x:A$.

Now we can compute $f(a)$ by replacing the variable $x$ in $b$ with
$a$. As an example, consider the function $f : \Nat \to \Nat$ which is
defined by $f(x) \defeq x+x$. Now $f(2)$ is judgmentally equal to $2+2$.

If we don't want to introduce a name for the function we can use
$\lambda$-abstraction. Given an expression $b : B$ which may use $x:A$, as above, we can
construct 
\[ (\lamt{x:A}b) : A \to B \]
We generally omit the type of the variable $x$ and write $\lam{x}b$, since the typing $x:A$ is inferrable from the judgment that the function $\lam x b$ has type $A\to B$.
Another equivalent notation is
\[ (x \mapsto b) : A \to B. \]
Applying a $\lambda$-abstraction to an argument $a:A$ introduces the
definitional equality\footnote{Use of this equality is often referred to as $\beta$-conversion or $\beta$-reduction.}
\[(\lamu{x:A}b)(a) \jdeq b'\]
 where $b'$ is the
expression $b$ in which all occurences of $x$ have been replaced by $a$.
Reusing the same example, we have the typing judgment
\[ (\lamu{x:\nat}x+x) : \nat \to \nat \]
and the judgmental equality
\[ (\lamu{x:\nat}x+x)(2) \jdeq 2+2. \]
By convention, the ``scope'' of the variable binding ``$\lam{x}$'' is the entire rest of the expression, unless delimited with parentheses.
Thus, for instance, $\lam{x} x+x$ should be parsed as $\lam{x} (x+x)$, not as $(\lam{x}x)+x$ (which would, in this case, be ill-typed anyway).

Note that from any function $f:A\to B$, we can construct a lambda abstraction function $\lam{x} f(x)$.
Since this is by definition ``the function which applies $f$ to its argument'' we consider it to be definitionally equal to $f$:\footnote{Use of this equality is often referred to as $\eta$-conversion or $\eta$-expansion.}
\[ f \jdeq (\lam{x} f(x)). \]

The introduction of functions by definitions with explicit parameters can be reduced
to simple definitions by using $\lambda$-abstraction: i.e., we can read 
a definition of $f: A\to B$ by
\[ f(x) \defeq b \]
as 
\[ f \defeq \lamu{x:A}b .\]

When doing calculations involving variables, we have to be 
careful when replacing a variable with an expression that also involves
variables, because we want to preserve the binding structure of
expressions. By the \emph{binding structure} we mean the
invisible link generated by binders such as $\lambda$, $\Pi$ and
$\Sigma$ (the latter we are going to meet soon) between the place where the variable is introduced and where it is used. As an example, consider $f : \nat \to (\nat \to \nat)$
defined as 
\[ f(x) \defeq \lamu{y:\nat} x + y \] 
Now if we have assumed somewhere that $y : \nat$, then what is $f(y)$? It would be wrong to just naively replace $x$ by $y$ everywhere, obtaining $\lamu{y:\nat} y + y$, because this means that $y$ gets \textbf{captured}. Previously, the substituted $y$ was referring to our assumption, but now it is referring to the parameter of the anonymous function. Hence, this naive substitution would destroy the binding structure, allowing us to perform calculations which are semantically unsound.

But what \emph{is} $f(y)$ in this example? Note that bound
variables such as $y$ in the expression $\lamu{y:\nat} x + y$
have only a local meaning, and can be consistently replaced by any
other variable, preserving the binding structure. Indeed, $\lamu{y:\nat} x + y$ is declared to be judgmentally equal\footnote{Use of this equality is called $\alpha$-conversion.} to
$\lamu{z:\nat} x + z$.  It follows that 
$f(y)$ is judgmentally equal to  $\lamu{z:\nat} y + z$, and that answers our question.  (Instead of $z$,
any variable distinct from $y$ could have been used, yielding an equal result.)

We have seen how to define functions in one variable. One
way to define functions in several variables would be to use the
cartesian product, which will be introduced later; a function with
parameters $A$ and $B$ and results in $C$ would be given the type 
$f : A \times B \to C$. However, there is another choice that avoids
using product types --- it is called \textbf{currying} in functional
programming; we write $f : A \to (B \to C)$.
We may also write it without the parentheses, as $f : A \to B \to C$, with
associativity to the right as the default convention.  Given $a : A$ and $b : B$
we can apply $f$ to $a$ and then apply the result to $b$ to obtain 
$f(a)(b) : C$. To avoid the proliferation of parentheses we allow ourselves to
write $f(a)(b)$ as $f(a,b)$ even though there are no products
involved. Our notation for definitions with explicit parameters extends to
this situation: we can define a named function $f : A \to B \to C$ by
giving an equation
\[ f(x,y) \defeq c\]
where $c:C$ assuming $x:A$ and $y:B$. Using $\lambda$-abstraction this
corresponds to
\[ f \defeq \lamu{x:A}{y:B} c, \]
which may also be written as 
\[ f \defeq x \mapsto y \mapsto c. \] 
Currying a function of three or more arguments is a straightforward extension of what we have just described.
 

\section{Universes and families}
\label{sec:universes}

So far, we have been using the expression ``$A$ is a type'' informally. We
are going to make this more precise by introducing \textbf{universes}.
A universe is a type whose elements are types. As in naive set theory,
we might wish for a universe of all types $\UU_\infty$ including itself
(that is, with $\UU_\infty : \UU_\infty$).
However, as in set
theory, this is unsound, i.e.\ we can deduce from it that every type,
including the empty type representing the proposition False, is inhabited. Indeed, using a
representation of sets as trees, we can directly encode Russell's
paradox \cite{coquand:paradox}.
%  or alternatively, in order to avoid the use of
% inductive types to define trees, we can follow Girard \cite{girard:paradox} and encode the Burali-Forti paradox,
% which shows that the collection of all ordinals cannot be an ordinal.

To avoid the paradox we introduce a hierarchy of universes
\[ \UU_0 : \UU_1 : \UU_2 : \cdots \]
where every universe $\UU_i$ is an element of the next universe
$\UU_{i+1}$. Moreover we assume that our universes are
\emph{cumulative}, that is that all the elements of the $i$th
universe are also elements of the $(i+1)^{\mathrm{st}}$ universe, i.e.\ if
$A:\UU_i$ then also $A:\UU_{i+1}$.
% \footnote{This rule breaks the principle that every element
%   has a uniquely determined type. This could be avoided by introducing
%   an explicit lifting operator.}

% We say that our type theory is predicative if types in a universe $\UU_i$ are introduced only with reference to types in the same or lower universes. Most vconstructions introduced in this book are predicative. An impredicative type is one which is obtained by quantifying over all types at a certain level including itself. In classical mathematics we can quantify over all truth values, which are identified with the type of booleans. Doing the same in type theory would involve quantifying over all types at the current level and is hence impredicative. As a consequence, assuming the principle of excluded middle, which identifies the small type of booleans with the universe, leads to an impredicative theory.

When we say that $A$ is a type, we mean that it inhabits some universe
$\UU_i$. We usually want to avoid mentioning the level $i$ explicitly,
and just assume that levels can be assigned in a consistent way; thus we
may write $A:\UU$ omitting the level. This way we can even write
$\UU:\UU$, which can be read as $\UU_i:\UU_{i+1}$, having left the
indices implicit.  Writing universes in this style is referred to as
{\em typical ambiguity}.

To model a collection of types varying over a given type $A$, we use functions $B : A \to \UU$  whose
codomain is a universe. These functions are called
\textbf{families of types}; they correspond to families of sets as used in
set theory. An example of a family is the family of finite sets $\Fin
: \nat \to \UU$, where $\Fin(n)$ is a type with exactly $n$ elements, denoted $0_n,1_n,\dots,(n-1)_n$.  (We will be able to define this family soon.) 


\section{Dependent function types (\texorpdfstring{$\Pi$}{Π}-types)}
\label{sec:pi-types}

In type theory we often use a more general version of function
types, called a $\Pi$-type or dependent function type. The elements of
a $\Pi$-type are functions whose codomain type can vary depending on the
element of the domain to which the function is applied. The name ``$\Pi$-type''
is used because this type can also be regarded as the infinite cartesian
product over a given type.

Given a type $A:\UU$ and a family $B:A \to \UU$, we may construct
the type of dependent functions $\prd{x:A}B(x) : \UU$.
There are many alternative notations for this type, such as
\[ \tprd{x:A} B(x) \hspace{2cm} \dprd{x:A}B(x) \hspace{2cm} \lprd{x:A} B(x). \]
If $B$ is constant, then the dependent product type is the ordinary function type:
\[\tprd{x:A} B \jdeq (A \to B).\]
Indeed, all the constructions of $\Pi$-types are generalisations of the corresponding constructions on ordinary function types.

We can introduce dependent functions by explicit definitions: to
define $f : \prd{x:A}B(x)$, where $f$ is the name of a dependent function to be
defined, we need an expression $b : B(x)$ involving $x:A$, and we write
\[ f(x) \defeq b \qquad \mbox{for $x:A$}\]
Alternatively, we can use $\lambda$-abstraction 
\[ \lamu{x:A} b \ :\ \prd{x:A} B(x). \]
The equalities are the same as for the ordinary function type, i.e.\
given $a:A$ we have $f(a) \jdeq b'$ and  
$(\lamu{x:A} b)(a) \jdeq b'$, where $b' $ is obtained by replacing all
occurences of $x$ in $b$ by $a$.
Similarly, we have $f\jdeq (\lam{x} f(x))$ for any $f:\prd{x:A} B(x)$.

An example of a dependent function is $\fmax : \prd{x:\nat} \Fin(n+1)$
which returns the largest element in a nonempty finite type, $\fmax(n) \defeq
n_{n+1}$. 
Another source of dependent function types
are functions which are \emph{polymorphic} over a given universe.
This means a function which takes a type as one of its arguments, and then acts on elements of that type (or other types constructed from it).
An example is the polymorphic identity function $\idfunc : \prd{A:U} A \to A$, which we define by $\idfunc[A] \defeq \lamu{x:A}x$.

As for ordinary functions, we use currying to define dependent functions with
several arguments. However, in the dependent case the second domain may
depend on the first one, and the codomain may depend on both. That is,
given $A:U$ and type families $B : A \to U$ and $C : \prd{x:A}B(x) \to \UU$, we may construct
the type $\prd{x:A}{y : B(x)} C(x,y)$ of functions with two
arguments.  Likewise, given $f:\prd{x:A}{y : B(x)} C(x,y)$ and arguments $a:A$ and $b:B(a)$, we have $f(a)(b) : C(a,b)$, which,
as before, we write as $f(a,b) : C(a,b)$.



\section{Product types}
\label{sec:finite-product-types}

Given types $A,B:\UU$ we introduce the type $A\times B:\UU$, which we call their \emph{cartesian product}.
We also introduce a nullary product type, called the \emph{unit type} $\unit : \UU$.
We intend the elements of $A\times B$ to be pairs $\tup{a}{b} : A \times B$, where $a:A$ and $b:B$, and the only element of $\unit$ to be some particular object $\ttt : \unit$.
However, unlike in set theory, where we define ordered pairs to be particular sets and then collect them all together into the cartesian product, in type theory, ordered pairs are a primitive concept, as are functions.
Just as we did for the function type, we explain the product type by prescribing what we can do with pairs, how to construct them, and what equalities they satisfy.

The way to construct pairs is obvious: given $a:A$ and $b:B$, we may form $(a,b):A\times B$.
Similarly, there is a unique way to construct elements of $\unit$, namely we have $\ttt:\unit$.
However, we do not assert as a rule of type theory that ``every element of $A\times B$ is a pair'' --- it turns out that we can \emph{prove} that using the principles to be introduced below.

Now, what can we \emph{do} with pairs, i.e.\ how can we define functions out of a product type?
Let us first consider the definition of a non-dependent function $f : A\times B \to C$.
Since we intend the only elements of $A\times B$ to be pairs, in order to define such a function, it should be enough to prescibe the result
when $f$ is applied to a pair $\tup{a}{b}$.  We prescribe these results by
providing a function $g : A \to B \to C$ and defining
\[ f(\tup{a}{b}) \defeq g(a)(b). \]
We avoid writing $g(a,b)$ here, in order to emphasize that $g$ is not a
function on a product. 

In particular, we can derive the \emph{projection} functions
\begin{eqnarray*}
  \fst & : & A \times B \to A \\
  \snd & : & A \times B \to B
\end{eqnarray*}
with the defining equations 
\begin{eqnarray*}
  \fst(\tup{a}{b}) & \defeq & a \\
  \snd(\tup{a}{b}) & \defeq & b
\end{eqnarray*}

We can package this principle into a constant which we call the
\emph{recursor} for product
types.  Given types $A,B : \UU$, we have
\[\rec{A\times B} : \prd{C:\UU'}(A \to B \to C) \to A \times B \to C\]
with the defining equation
\[\rec{A\times B}(C,g,\tup{a}{b}) = g(a)(b). \]
We may also speak of the \emph{recursion principle} for cartesian products, meaning the fact that we can define a function $f:A\times B\to C$ as above by giving its value on pairs.

The name ``recursor'' is a bit unfortunate here, since no recursion is taking place.
It comes from a general framework of type definitions, and will make more sense once we consider inductive types, such as
the natural numbers.

Note that the universes $\UU,\UU'$ may be different, i.e.\ we can
construct functions in higher or lower universes. 
We leave it as a simple exercise to show that the recursor can be
derived from the projections and vice versa.
% Ex: Derive from projections

We also have a recursor for the unit type:
\[\rec{\unit} : \prd{C:\UU'}C \to \unit \to C\]
with the defining equation
\[ \rec{\unit}(C,c,\star) \defeq c \]
However, this is completely useless,
because we could have defined such a function directly
by simply ignoring the argument of type $\unit$.

To be able to define \emph{dependent} functions over the product type, we have
to generalize the recursor. Given $C: A \times B \to U$, we may
define a function $f : \prd{x : A \times B} C(x)$ by providing a
function $g : \prd{x:A}\prd{y:B} C(\tup{x}{y})$ and the defining equation
\[ f(x,y) \defeq g(x)(y). \] 
For example, in this way we can prove that every element of $A\times B$ is a pair; this is called the principle of \emph{surjective pairing}.
Specifically, we can construct a function
\[ \spr : \prd{x:A \times B} (\tup{\fst {(x)}}{\snd {(x)}} =_{A\times B} x). \]
Here we are using the identity type, which we are going to introduce below in \autoref{sec:identity-types}.
However, all we need here is to know that there is a reflexivity element $\refl{x} : x =_A x$ for any $x:A$.
Given this, we can define
\[ \spr(\tup{a}{b}) \defeq \refl{\tup{a}{b}} \]
This construction works, because in the case that $x \defeq \tup{a}{b}$ we can 
calculate 
\[ \tup{\fst(\tup{a}{b})}{\snd{(\tup{a}{b})}} \jdeq \tup{a}{b} \]
using the defining equations for the projections. Therefore,
\[ \refl{\tup{a}{b}} : \tup{\fst(\tup{a}{b})}{\snd{(\tup{a}{b})}} = \tup{a}{b} \]
is well-typed, since both sides of the equality are judgmentally equal.

More generally, the ability to define dependent functions in this way means that to prove a property for all elements of a product, it is enough 
to prove it for its canonical elements, the tuples.
When we come to inductive types such as the natural numbers, the analogous property will be the ability to write proofs by induction.
Thus, when we package this principle into another constant, we call it
\emph{induction} for product types: given $A,B : \UU$ we have
\[ \ind{A\times B} : \prd{C:A \times B \to \UU'} (\prd{x:A}{y:B}
C(\tup{x}{y})) \to \prd{x:A \times B} C(x) \]
with the defining equation 
\[ \ind{A\times B}(C,g,\tup{a}{b}) \defeq g(a)(b). \]
Similarly, we may speak of a dependent function defined on pairs being obtained from the \emph{induction principle} of the cartesian product.
It is easy to see that the recursor is just the special case of induction
in the case that the family $C$ is constant.
Induction is also called the \emph{(dependent) eliminator} of the product type, and recursion the \emph{non-dependent eliminator}.

% We can read induction propositionally as saying that a property which
% is true for all pairs holds for all elements of the product type.

Induction for the unit type turns out to be more useful then the
recursor: 
\[ \ind{\unit} : \prd{C:\unit \to \UU'} C(\star) \to \prd{x:\unit}C(x)\]
with the defining equation
\[ \ind{\unit}(C,c,\star) \defeq c \]
Induction enables us to \emph{prove} that the only inhabitant of the
unit type is $\star$, that is we can construct
\[\un : \prd{x:\unit} x = * \]
by using the defining equations
\[\un(\star) \defeq \refl{\star} \]
or equivalently by using induction:
\[\un \defeq \ind{\unit}(\lamu{x:\unit} x = *,\refl{\star}) \]

\section{Dependent pair types (\texorpdfstring{$\Sigma$}{Σ}-types)}
\label{sec:sigma-types}

Just as we generalized function types (\autoref{sec:function-types}) to dependent function types (\autoref{sec:pi-types}), it is often useful to generalize the product types from \autoref{sec:finite-product-types} to allow the type of
the second component of a pair to vary depending on the choice of the first
component. This is called a $\Sigma$-type, because in set theory it
corresponds to an infinite sum (in the sense of coproducts or
disjoint union) over a given type.

Given a type $A:\UU$ and a family $B : A \to \UU$, the dependent
pair type is written as $\sm{x:A} B(x) : \UU$.
Alternative notations are 
\[ \tsm{x:A} B(x) \hspace{2cm} \dsm{x:A}B(x) \hspace{2cm} \lsm{x:A} B(x). \]
The way to construct elements of this type is by pairing: we have
$\tup{a}{b} : \sm{x:A} B(x)$ given $a:A$ and $b:B(a)$.
If $B$ is constant, then the dependent pair type is the
ordinary cartesian product type:
\[ (\sm{x:A} B) \jdeq (A \times B).\]
All the constructions on $\Sigma$-types arise as straightforward generalisations of the ones for product types, with dependent functions often replacing non-dependent ones.

For instance, the recursion principle says that to define a non-dependent function out of a $\Sigma$-type
$f : (\sm{x:A} B(x)) \to C$, we provide a function 
$g : \prd{x:A} B(x) \to C$, and then we can define $f$ via the defining
equation
\[ f(\tup{a}{b}) \defeq g(a)(b) \]
For instance, we can derive the first projection from a $\Sigma$-type:
\begin{eqnarray*}
  \fst & : & (\sm{x : A}B(x)) \to A.
  % \snd & : & \prd{p:\sm{x : A}B(x)}B(\fst(p)).
\end{eqnarray*}
by the defining equation
\begin{eqnarray*}
  \fst(\tup{a}{b}) & \defeq & a.
  % \snd(\tup{a}{b}) & \defeq & b
\end{eqnarray*}
However, since the type of the second component of a pair $(a,b):\sm{x:A} B(x)$ is $B(a)$, the second projection must be a \emph{dependent} function, whose type involves the first projection function:
\[ \snd : \prd{p:\sm{x : A}B(x)}B(\fst(p)). \]
Thus we need the \emph{induction} principle for $\Sigma$-types (the ``dependent eliminator'').
This says that to construct a dependent function out of a $\Sigma$-type into a family $C : (\sm{x:A} B(x)) \to \UU'$, we need a function
\[ g : \prd{x:A} B(x) \to C(\tup{a}{b}). \]
We can then derive a function 
\[ f : \prd{p : \sm{x:A}B(x)} C(p) \]
with  defining equation
\[ f(\tup{a}{b}) \defeq g(a)(b).\]
Applying this with $C(p)\defeq B(\fst(p))$, we can define $\snd : \prd{p:\sm{x : A}B(x)}B(\fst(p))$ with the obvious equation
\[ \snd(\tup{a}{b})  \defeq  b. \]
To convince ourselves that this is correct, we note that $B (\fst(\tup{a}{b})) \jdeq B(a)$, using the defining equation for $\fst$, and
indeed $b : B(a)$.

We can package the recursion and induction principles into the recursor for $\Sigma$:
\[ \rec{\sm{x:A}B(x)} : \dprd{C:\UU'}\Big(\tprd{x:A} B(x) \to C\Big) \to
\big(\tsm{x:A}B(x)\big) \to C \]
with the defining equation
\[ \rec{\sm{x:A}B(x)}(C,g,\tup{a}{b}) \defeq g(a)(b) \]
and the corresponding induction operator:
\begin{align*}
  \ind{\Sigma{x:A}B(x)} : & \dprd{C:(\sm{x:A} B(x)) \to \UU'}
    \Big(\tprd{x:A} B(x) \to C(\tup{a}{b})\Big)
    \to \dprd{p : \sm{x:A}B(x)} C(p)
\end{align*}
with the defining equation 
\[ \ind{\sm{x:A}B(x)}(C,g,\tup{a}{b}) \defeq g(a)(b). \]
As before, the recursor is the special case of induction
when the family $C$ is constant.

As a further example, consider the following principle, where $A$ and $B$ are types and $R:A\to B\to \UU$.
\[ \ac : \Big(\tprd{x:A} \tsm{y :B} R(x,y)\Big) \to
\Big(\tsm{f:A\to B} \tprd{x:A} R(x,f(x))\Big)
\]
We may regard $R$ as a ``proof-relevant relation'' between $A$ and $B$, with $R(a,b)$ the type of witnesses for relatedness of $a:A$ and $b:B$.
Then $\ac$ says intuitively that if we have a dependent function $g$ assigning to every $a:A$ a dependent pair $(b,r)$ where $b:B$ and $r:R(a,b)$, then we have a function $f:A\to B$ and a dependent function assigning to every $a:A$ a witness that $R(a,f(a))$.
Our intuition tells us that we can just split up the values of $g$ into their components.
Indeed, using the projections we have just defined, we can define:
\[ \ac(g) \defeq \Big(\lamu{x:A} \fst(g(x)),\, \lamu{x:A} \snd(g(x))\Big). \]
To verify that this is well-typed, note that $g:\prd{x:A} \sm{y :B} R(x,y)$, we have
\begin{eqnarray*}
\lamu{x:A} \fst(g(x)) & : &  A \to  B \\
\lamu{x:A} \snd(g(x)) & : &  \tprd{x:A} R(a,\fst(g(x))).
\end{eqnarray*}
Moreover, the type $\prd{x:A} R(a,\fst(g(x)))$ is the result of substituting the functior $\lamu{x:A} \fst(g(x))$ for $f$ in the family being summed over in the codomain of \ac:
\[ \tprd{x:A} R(x,\fst(f(x))) \jdeq
\Big(\lamu{f:A\to B} \tprd{x:A} R(x,f(x))\Big)\big(\lamu{x:A} \fst(g(x))\big). \]
Thus, we have
\[ \Big(\lamu{x:A} \fst(g(x)),\, \lamu{x:A} \snd(g(x))\Big) : \tsm{f:A\to B} \tprd{x:A} R(x,f(x))\]
as required.

If we read $\Pi$ as ``for all'' and $\Sigma$ as ``there exists'', then the type of the function $\ac$ expresses:
\emph{if for all $x:A$ there is a $y:B$ such that $R(x,y)$, then there is a function $f : A \to B$ such that for all $x:A$ we have $R(x,f(x))$}.
Since this sounds like a version of the axiom of choice, the function \ac has traditionally been called the \emph{type-theoretic axiom of choice}.
However, there is clearly no choice actually involved, since the choices have already been given to us in the premise: all we have to do is take it apart into two functions: one representing the choice and the other its correctness.
In \autoref{sec:axiom-choice} we will give a correct formulation of ``the axiom of choice''.


\section{Coproduct types}
\label{sec:coproduct-types}

Given $A,B:\UU$, we introduce their \emph{coproduct} type $A+B:\UU$.
This corresponds to the \emph{disjoint union} in set theory, and we may also use that name for it.
In type theory, as was the case with functions and products, the coproduct must be a fundamental construction, since there is no previously given notion of ``union of types''.
We also introduce a
nullary version: the empty type $\emptyt:\UU$.

There are two ways to construct elements of $A+B$, either as $\inl(a) : A+B$ for $a:A$, or as
$\inr(b):A+B$ for $b:B$. There are no ways to construct elements of the empty type. 

To construct a non-dependent function $f : A+B \to C$, we need 
functions $g_0 : A \to C$ and $g_1 : B \to C$. Then $f$ is defined
via the defining equations
\begin{eqnarray*}
  f(\inl(a)) & \defeq & g_0(a) \\
  f(\inl(b)) & \defeq & g_1(b).
\end{eqnarray*}
That is, the function $f$ is defined by cases. As before, we can
derive the recursor:
\[ \rec{A+B} : \dprd{C:\UU}(A \to C) \to (B\to C) \to A+B \to C\]
with the defining equations
\begin{eqnarray*}
\rec{A+B}(C,g_0,g_1,\inl(a)) & \defeq & g_0(a) \\
\rec{A+B}(C,g_0,g_1,\inr(b)) & \defeq & g_1(b)
\end{eqnarray*}

We can always construct a function $f : \emptyt \to C$ without
having to give any defining equations, because there are no canonical elements of \emptyt on which to define $f$.
Thus, the recursor for $\emptyt$ is
\[\rec{\emptyt} : \tprd{C:\UU'} \emptyt \to C,\]
which constructs the canonical function from the empty type to any other type.
Logically, it corresponds to the principle \emph{ex falso quod libet}. 

To construct a dependent function $f:\prd{x:A+B}C(x)$ out of a coproduct, we assume as given the family 
$C: (A + B) \to \UU'$, and 
require 
\begin{eqnarray*}
  g_0 & : & \prd{a:A} C(\inl(a)) \\
  g_1 & : & \prd{b:B} C(\inr(b)).
\end{eqnarray*}
This yields $f$ with the defining equations:
\begin{eqnarray*}
  f(\inl(a)) & \defeq & g_0(a) \\
  f(\inl(b)) & \defeq & g_1(b) .
\end{eqnarray*}
We package this scheme into the induction principle for coproducts:
\begin{align*}
  \ind{A+B} :  \dprd{C: (A + B) \to \UU'}
  \Big(\tprd{a:A} C(\inl(a))\Big) \to \Big(\tprd{b:B} C(\inr(b))\Big)
  \to \tprd{x:A+B}C(x) 
\end{align*}
As before, the recursor arises in the case that the family $C$ is
constant. 

The induction principle for the empty type
\[ \ind{\emptyt} : \tprd{C:\emptyt \to \UU}{z:\emptyt} C(z) \]
gives us a way to define a trivial dependent function out of the
empty type. % In the presence of $\eta$-equality it is derivable
% from the recursor.
% ex

\section{The type of booleans}
\label{sec:type-booleans}

The type of booleans $\bool:\UU$ is intended to have exactly two elements 
$\btrue,\bfalse : \bool$. It is clear that we could construct this
type out of coproduct and unit types as $\unit + \unit$. However,
since it is used frequently, we give the explicit rules here.
Indeed, we are going to observe that we can also go the other way
and derive binary coproducts from $\Sigma$-types and $\bool$.

To derive a function $f : \bool \to C$ we need $c_0,c_1 : C$ and
add the defining equations
\begin{eqnarray*}
  f(\bfalse) & \defeq & c_0 \\
  f(\btrue) & \defeq & c_1
\end{eqnarray*}
The recursor corresponds to the if-then-else construct in
functional programming:
\[ \rec{\bool} : \prd{C:\UU'}  C \to C \to \bool \to C \]
with the defining equations
\begin{eqnarray*}
  \rec{\bool}(C,c_0,c_1,\bfalse) & \defeq & c_0 \\
  \rec{\bool}(C,c_0,c_1,\btrue) & \defeq & c_1
\end{eqnarray*}

Given a family $C : \bool \to \UU$, to derive a dependent function 
$f : \prd{x:\bool}C(x)$ we need $c_0:C(\bfalse)$ and $c_1 : C(\btrue)$, in which case we can give the defining equations
\begin{eqnarray*}
  f(\bfalse) & \defeq & c_0 \\
  f(\btrue) & \defeq & c_1.
\end{eqnarray*}
We package this up into the induction principle
\[ \ind{\bool} : \dprd{C:\bool \to \UU'}  C(\bfalse) \to C(\btrue)
\to \tprd{x:\bool} C(x) \]
with the defining equations
\begin{eqnarray*}
  \ind{\bool}(C,c_0,c_1,\bfalse) & \defeq & c_0 \\
  \ind{\bool}(C,c_0,c_1,\btrue) & \defeq & c_1
\end{eqnarray*}

We have remarked that $\Sigma$-types can be regarded as analogous to indexed disjoint unions, while coproducts are binary disjoint unions.
It is natural to expect that a binary disjoint union could be constructed as an indexed one over the two-element type \bool, and indeed we could have defined
\[ A + B \defeq \sm{x:\bool} \rec{\bool}(\UU,A,B,x) \]
with
\begin{eqnarray*}
  \inl(a) & \defeq & \tup{\bfalse}{a} \\
  \inr(a) & \defeq & \tup{\btrue}{a}.
\end{eqnarray*}
We leave it as an exercise to derive the induction principle of a coproduct type from this definition.
(See also \autoref{ex:sum-via-bool,sec:appetizer-univalence}.)

We can apply the same idea to products and $\Pi$-types: we could have defined
\[ A \times B \defeq \prd{x:\bool}\rec{\bool}(\UU,A,B,x) \]
Pairs could then be constructed using induction for \bool:
\[ \tup{a}{b} \defeq \ind{\bool}(\rec{\bool}(\UU,A,B),a,b) \]
while the projections are straightforward applications
\begin{eqnarray*}
  \fst(p) & \defeq & p(\btrue) \\
  \snd(p) & \defeq & p(\bfalse).
\end{eqnarray*}
The derivation of the induction principle for binary products defined in this way is a bit more involved, and requires function extensionality, which we will introduce in \autoref{sec:compute-pi}.
Moreover, we do not get the same judgmental equalities; see \autoref{ex:prod-via-bool}.
This is a recurrent issue when encoding one type as another; we will return to it in \autoref{sec:htpy-inductive}. 

We may occasionally refer to the elements $\btrue$ and $\bfalse$ of $\bool$ as ``true'' and ``false'' respectively.
However, note that unlike in classical mathematics, we do not use elements of $\bool$ as truth values or as propositions.
(Instead we identify propositions with types; see \autoref{sec:pat}.)
In particular, the type $A \to \bool$ is not generally the power set of $A$; it represents only the ``decidable'' subsets of $A$.


\section{Identity types}
\label{sec:identity-types}

While the previous constructions can be seen as generalisations of
standard set theoretic constructions, the identity type seems to be
a particular feature of type theory. Moreover, homotopy type theory is
based on a reinterpretation of identity types embracing a
proof-relevant understanding of identity types: there can be more than
one reason that two objects are equal, corresponding to the fact that
there can be more than one path between two points of a space.

Given a type $A:\UU$ and two elements $a,b:A$ we can form the type $a=_A b:\UU$ in the same universe. The idea is that $a=_A b:\UU$ is the type of proofs that $a$ and $b$ are equal, or from the homotopy point of view that there is a path between $a$ and $b$. In particular, there is a canonical element
\[\refl{} : \prd{a:A} a =_A a\] --- the proof of reflexivity.

There are two different but equivalent elimination principles for
identity types:
\begin{description}
\item[Martin-L\"{o}f Rule:] 
Given a family 
\[ C : \prd{x,y:A}\prd{p:x =_A y} \UU \]
and a function
\[ c :  \prd{x:A} C(x,x,\refl{x})\]
we can construct a function
\[ f : \prd{x,y:A}{p:x =_A y} C(x,y,p) \]
with the defining equality 
\[ f(x,x,\refl{x}) \defeq c(x) \]

The Martin-L\"of rule corresponds to the following induction principle:
\begin{align*}
 \ind{=_A}^{ML} : & \prd{C : \prd{x,y:A}\prd{p:x =_A y} \UU}  \\
     & \prd{x:A} C(x,x,\refl{x})  \\
     & \to  \prd{x,y:A}{p:x =_A y} C(x,y,p)
\end{align*}
with the equality
\[ \ind{=_A}^{ML}(C,c,x,x,\refl{x}) \defeq c(x) \]
This induction constant is also called $J$.

\item[Paulin-Mohring Rule:] 

Fix an element $a:A$, given a family
\[ C : \prd{x:A}{p : a =_A x} \UU \]
and an element
\[ c : C(a,\refl{a}) \]
we obtain a function
\[ f : \prd{x:A}{p:a = x} C(x,p) \]
with the defining equality
\[ f(a,\refl{a}) \defeq c \]

Correspondingly, there is an induction constant
\begin{align*}
\ind{=_A}^{PM} : & \prd{a:A}{C : \prd{x:A}{p : a =_A x} \UU} \\
& C(a,\refl{a}) \\
& \to \prd{x:A}{p : a =_A x} C(x,p) 
\end{align*}
with the equality
\[ \ind{=_A}^{PM}(a,C,c,a,\refl{a}) \defeq c \]
%\[ g(x)(x,\refl{x}) \defeq d(x) \]
\end{description}
These principles arise from different views on the identity type,
while the Martin-L\"of rule is based on the view that the identity
type is the least reflexive relation, the Paulin-Mohring rule defines 
equality with respect to a given element $a:A$ as the least predicate
over $A$ which holds for $a$.

It is easy to see that the Martin-L\"of rule follows from the
Paulin-Mohring rule: We assume 
the premises of the Martin-L\"of rule
\begin{eqnarray*}
C & : & \prd{x,y:A}\prd{p:x =_A y} \UU  \\
c & :  & \prd{x:A} C(x,x,\refl{x})
\end{eqnarray*}
now given an element $x:A$ we can instantiate both obtaining
\begin{eqnarray*}
C' & : & \prd{y:A}\prd{p:x =_A y} \UU  \\
C' & \defeq & C(x) \\
c' & : & C'(x,\refl{x}) \\
c' & \defeq & c(x)
\end{eqnarray*}
Clearly, $C'$ and $c'$ match the premises of the Paulin-Mohring rule and hence we can construct 
\begin{eqnarray*}
g : \prd{y:A}{p : x = y} C'(y,p)
\end{eqnarray*}
with the defining equality
\[ g(x,\refl{x}) \defeq c' \]
Now we observe that $g$'s codomain is equal to $C(x,y,p)$ and discharging our assumption
$x:A$ we can derive a function 
\[ f : \prd{x,y:A}{p : x =_A y} C(x,y,p) \]
with the required definitional equality.

In terms of induction constants we can derive $\ind{=_A}^{ML}$ from $\ind{=_A}^{PM}$ using
\[ \ind{=_A}^{ML}(C,c,x,y,p) \defeq \ind{=_A}^{PM}(x,C(x),c(x),y,p) \]

The other direction is a bit more tricky: To derive the Paulin-Mohring rule from the Martin-L\"of rule 
we assume $a : A$ and the premises of the Paulin-Mohring rule:
\begin{eqnarray*}
C & : & \prd{x:A}{p : a =_A x} \UU \\  
c & : & C(a,\refl{a})
\end{eqnarray*}
It is not clear how we can use a particular instance of the Martin-L\"of rule to derive the conclusion of 
the Paulin-Mohring rule. However, we can generalize this and construct an instance of the Martin-L\"of rule which shows 
all possible instantiations of the Paulin-Mohring rule:
\begin{eqnarray*}
D & : & \prd{x,y:A}\prd{p:x =_A y} \UU' \\
D(x,y,p) & \defeq & \prd{C : \prd{z:A}{p : x =_A z} \UU} C(x,\refl{x}) \to C(y,p)
\end{eqnarray*}
we can construct the function
\begin{eqnarray*}
d & : & \prd{x : A} D(x,x,\refl{x}) \\
d & \defeq & \lamu{C:\prd{z:A}{p : x =_A z} \UU}\lamu{x:C(x,\refl{x})} x
\end{eqnarray*}
and hence using the Martin-L\"of rule obtain
\[ f : \prd{x,y:A}{p:x =_A y} D(x,y,p) \]
with $f(x,x,\refl{x}) \defeq d(x)$. Unfolding the definition of $D$ we can expand the type of $f$:
\[ f : \prd{x,y:A}{p:x =_A y}{C : \prd{z:A}{p : x =_A z} \UU} C(x,\refl{x}) \to C(y,p) \]
now given $x:A$ and $p:a =_A x$ we can derive the conclusion of the Paulin-Mohring rule:
\[ f(a,x,p,C,c) : C(x,p) \]
We notice that we can also derive the correct definitional equality.

In terms of induction constants the construction becomes:
\begin{align*}
\ind{=_A}^{PM}(a,C,c,x,p) \defeq \\
\ind{=_A}^{ML}( & \lamu{x,y:A}{p:x =_A y} \prd{C : \prd{z:A}{p : x =_A z} \UU} \\
&\qquad C(x,\refl{x}) \to C(y,p),\\
& \lamu{C:\prd{z:A}{p : x =_A z} \UU}\lamu{x:C(x,\refl{x})} x,\\
& a,x,p,C,c) 
\end{align*}

The construction given above uses universes: that is if we want to
model $\ind{=_A}^{PM}$ with $C : \prd{x:A}{p : a =_A x} \UU_i$ we need
to use $\ind{=_A}^{ML}$ with $D:\prd{x,y:A}\prd{p:x =_A y} \UU_{i+1}$
since $D$ quantifies over all $C$ of the given type. While this is
compatible with our definition of universes, one may wonder wether we
can derive $\ind{=_A}^{PM}$ without using universes. This is indeed
possible: we can show that $\ind{=_A}^{ML}$ entails \autoref{lem:transport} and  \autoref{thm:contr-paths} and that these two principles allow us to derive $ind{=_A}^{PM}$ without using universes. We leave the details of this construction as an exercise.

We can use the eliminator for equality to establish that equality is an equivalence relation, that every function preserves equality and that every family is stable under equality. We leave the details to the next chapter where these principles are explained and derived in the context of homotopy type theory. 



\section{The natural numbers}
\label{sec:inductive-types}

The rules we have introduced so far do not allow us to construct any infinite types.
The simplest infinite type we can think of (and which is also, of course, extremely useful) is the type $\nat : \UU$ of natural numbers.
The elements of $\nat$ are constructed using $0 : \nat$ and the successor operation $\suc : \nat \to \nat$.
When denoting natural numbers, we adopt the usual decimal notation $1 \defeq \suc(0)$, $2 \defeq \suc(1)$, $3 \defeq \suc(2)$, \dots.

The essential property of the natural numbers is that we can define functions by recursion and perform proofs by induction --- where now the words ``recursion'' and ``induction'' have a more familiar meaning.
To construct a non-dependent function $f : \nat \to C$ out of the natural numbers by recursion, it is enough to provide a starting point $c_0 : C$ and a ``next step'' function $c_s : \nat \to C \to C$.
This gives rise to $f$ with the defining equations
\begin{eqnarray*}
  f(0) & \defeq & c_0 \\
  f(\suc(n)) & \defeq & c_s(n,f(n)).
\end{eqnarray*}
We say that $f$ is defined by \textbf{primitive recursion}.

As an example, we look at how to define a function on natural numbers which doubles its argument.
In this case we have $C\defeq \nat$.
We first need to supply the value of $\dbl(0)$, which is easy: we put $c_0 \defeq 0$.
Next, to compute the value of $\dbl(\suc(n))$ for a natural number $n$, we first compute the value of $\dbl(n)$ and then perform the successor operation twice.
This is captured by the recurrence $c_s(n,y) \defeq \suc(\suc(y))$.
Note that the second argument $y$ of $c_s$ stands for the result of the \emph{recursive call} $\dbl(n)$.

Defining $\dbl:\nat\to\nat$ by primitive recursion in this way, therefore, we obtain the defining equations:
\begin{align*}
  \dbl(0) &\defeq 0\\
  \dbl(\suc(n)) &\defeq \suc(\suc(\dbl(n))).
\end{align*}
This indeed has the correct computational behavior: for example, we have 
\begin{align*}
  \dbl(2) &\jdeq \dbl(\suc(\suc(0)))\\
  & \jdeq c_s(\suc(0), \dbl(\suc(0))) \\
                 & \jdeq \suc(\suc(\dbl(\suc(0)))) \\
                 & \jdeq \suc(\suc(c_s(0,\dbl(0)))) \\
                 & \jdeq \suc(\suc(\suc(\suc(\dbl(0))))) \\
                 & \jdeq \suc(\suc(\suc(\suc(c_0)))) \\
                 & \jdeq \suc(\suc(\suc(\suc(0))))\\
                 &\jdeq 4.
\end{align*}
We can define multi-variable functions by primitive recursion as well, by currying and allowing $C$ to be a function type.
For example, we define addition $\add : \nat \to \nat \to \nat$ with $C = \nat \to \nat$ and the following data:
\begin{align*}
  \add_0 & : \nat \to \nat \\
  \add_0 (n) & \defeq n \\
  \add_s & : \nat \to (\nat \to \nat) \to (\nat \to \nat) \\
  \add_s(n,g,m) & = \suc(g(m))
\end{align*}
We thus obtain $\add : \nat \to \nat \to \nat$ satisfying the definitional equalities
\begin{eqnarray*}
  \add(0,n) & \jdeq & n \\
  \add(\suc(m),n) & \jdeq & \suc(\add(m,n)) 
\end{eqnarray*}
As usual, we write $\add(m,n)$ as $m+n$.
The reader is invited to verify that $2+2\jdeq 4$.
% ex: define multiplication and exponentiation.

As in previous cases, we can package the principle of primitive recursion into a recursion constant:
\[\rec{\nat}  : \dprd{C:\UU} C \to (\nat \to C \to C) \to \nat \to C \]
with the defining equations
\begin{eqnarray*}
\rec{\nat}(C,c_0,c_s,0)  & \defeq & c_0 \\
\rec{\nat}(C,c_0,c_s,\suc(n)) & \defeq & c_s(n,\rec{\nat}(C,C_0,c_s,n)  
\end{eqnarray*}
%ex derive rec from it
Using $\rec{\nat}$ we can present $\dbl$ and $\add$ as follows:
\begin{align}
\dbl &\defeq \rec\nat\big(\nat,\, 0,\, \lamu{n:\nat}{y:\nat} \suc(\suc(y))\big) \label{eq:dbl-as-rec}\\
\add &\defeq \rec{\nat}\big(\nat \to \nat,\, \lamu{n:\nat} n,\, \lamu{n:\nat}{g:\nat \to \nat}{m :\nat} \suc(g(m))\big)
\end{align}
Of course, all functions definable only using the primitive recursion principle will be \emph{computable}.
(The presence of higher function types does, however, mean we can define more than the usual primitive recursive functions; see e.g.~\autoref{ex:ackermann}.)
This is appropriate in constructive mathematics; in \autoref{sec:intuitionism,sec:axiom-choice} we will see how to augment type theory so that we can define more general mathematical functions.

We now follow the same approach as for other types, generalizing primitive recursion to dependent functions to obtain an \emph{induction principle}.
Thus, assume as given a family $C : \nat \to \UU$, an element $c_0 : C(n)$, and a function $c_s : \prd{n:\nat} C(n) \to C(\suc(n))$; then we can construct $f : \prd{n:\nat} C(n)$ with the defining equations:
\begin{eqnarray*}
  f(0) & \defeq & c_0 \\
  f(\suc(n)) & \defeq & c_s(n,f(n))
\end{eqnarray*}
We can also package this into an induction constant
\[\ind{\nat}  : \dprd{C:\nat\to \UU} C(0) \to \big(\tprd{n : \nat} C(n) \to C(\suc(n))\big) \to \tprd{n : \nat} C(n) \]
with the defining equations
\begin{eqnarray*}
\ind{\nat}(C,c_0,c_s,0)  & \defeq & c_0 \\
\ind{\nat}(C,c_0,c_s,\suc(n)) & \defeq & c_s(n,\ind{\nat}(C,C_0,c_s,n)  
\end{eqnarray*}
Here we finally see the connection to the classical notion of proof by induction.
Recall that in type theory we represent propositions by types, and proving a proposition by inhabiting the corresponding type.
In particular, a \emph{property} of natural numbers is represented by a family of types $P:\nat\to\type$.
From this point of view, the above induction principle says that if we can prove $P(0)$, and if for any $n$ we can prove $P(\suc(n))$ assuming $P(n)$, then we have $P(n)$ for all $n$.
This is, of course, exactly the usual principle of proof by induction on natural numbers.

As an example consider how we might represent an explicit proof that $+$ is associative.
(We will not actually write out proofs in this style, but it serves as a useful example for understanding how induction is represented formally in type theory.)
To derive
\[\ass : \prd{i,j,k:\nat} i + (j + k) = (i + j) + k, \]
it is sufficient to supply
\[ \ass_0 :  \prd{j,k:\nat} 0 + (j + k) = (0+ j) + n \]
and
\begin{multline*}
  \ass_s  : \prd{n:\nat} \big(\prd{j,k:\nat} i + (j + k) = (i + j) + n\big) \\
   \to \prd{j,k:\nat} \suc(n) + (j + k) = (\suc(n) + j) + n.
\end{multline*}
To derive $\ass_0$, recall that $0+n \jdeq n$, and hence  $0 + (j + k) \jdeq j+k \jdeq (0+ j) + k$.
Hence we can just set
\[ ass_0(j,k) \defeq \refl{j+k} \]
For $\ass_s$, recall that the definition of $+$ gives $\suc(m)+n \jdeq \suc(m+n)$, and hence 
\begin{eqnarray*}
   \suc(n) + (j + k)  & \jdeq & \suc(n+(j+k)) \\
   (\suc(n)+j)+k & \jdeq & \suc((n+j)+k.
\end{eqnarray*}
Thus, the output type of $\ass_s$ is equivalently $\suc(n+(j+k)) = \suc((n+j)+k)$.
But its input (the ``inductive hypothesis'') yields $n+(j+k)=(n+j)+k$, so it suffices to invoke the fact that if two natural numbers are equal, then so are their successors.
(We will prove this obvious fact formally from the induction principle of identity types in \autoref{lem:map}.)
We call this latter fact
$\apfunc{\suc} : %\prd{m,n:\nat}
(m =_\nat n) \to (\suc(m) =_\nat \suc(n))$, so we can define
\[\ass_s(n,h,j,k) \defeq \apfunc{\suc}( %n+(j+k),(n+j)+k,
h(j,k)). \]
Putting these together with $\ind{\nat}$, we obtain a proof of associativity.


\section{Pattern matching and recursion}
\label{sec:pattern-matching}

The natural numbers introduce an additional subtlety over the types considered up until now.
In the case of coproducts, for instance, we could define a function $f:A+B\to C$ either with the recursor:
\[ f \defeq \rec{A+B}(C, g_0, g_1) \]
or by giving the defining equations:
\begin{align*}
  f(\inl(a)) &\defeq g_0(a)\\
  f(\inr(b)) &\defeq g_1(b).
\end{align*}
To go from the former expression of $f$ to the latter, we simply use the computation rules for the recursor.
Conversely, given any defining equations
\begin{align*}
  f(\inl(a)) &\defeq c_0\\
  f(\inr(b)) &\defeq c_1
\end{align*}
where $c_0$ and $c_1$ are expressions that may involve the variables $a$ and $b$, we can express these equations equivalently in terms of the recursor by using $\lambda$-abstraction:
\[ f\defeq \rec{A+B}(C, \lam{a} c_0, \lam{b} c_1).\]
In the case of the natural numbers, however, the ``defining equations'' of a function such as $\dbl$:
\begin{align}
  \dbl(0) &\defeq 0 \label{eq:dbl0}\\
  \dbl(\suc(n)) &\defeq \suc(\suc(\dbl(n)))\label{eq:dblsuc}
\end{align}
involve \emph{the function $\dbl$ itself} on the right-hand side.
However, we would still like to be able to give these equations, rather than~\eqref{eq:dbl-as-rec}, as the definition of \dbl, since they are much more convenient and readable.
The solution is to read the expression ``$\dbl(n)$'' on the right-hand side of~\eqref{eq:dblsuc} as standing in for the result of the recursive call, which in a definition of the form $\dbl\defeq \rec{\nat}(\nat,c_0,c_s)$ would be the second argument of $c_s$.

More generally, if we have a ``definition'' of a function $f:\nat\to C$ such as
\begin{align*}
  f(0) &\defeq c_0\\
  f(\suc(n)) &\defeq c_s
\end{align*}
where $c_0$ is an expression of type $C$, and $c_s$ is an expression of type $C$ which may involve the variable $n$ and also the symbol ``$f(n)$'', we may translate it to a definition
\[ f \defeq \rec{\nat}(C,\,c_0,\,\lam{n}{r} c_s') \]
where $c_s'$ is obtained from $c_s$ by replacing all occurrences of ``$f(n)$'' by the new variable $r$.

This style of defining functions by recursion (or, more generally, dependent functions by induction) is so convenient that we frequently adopt it.
It is called definition by \textbf{pattern matching}.
Of course, it is very similar to how a computer programmer may define a recursive function with a body that literally contains recursive calls to itself.
However, unlike the programmer, we are restricted in what sort of recursive calls we can make: in order for such a definition to be re-expressible using the recursion principle, the function $f$ being defined can only appear in the body of $f(\suc(n))$ as part of the composite symbol ``$f(n)$''.
Otherwise, we could write nonsense functions such as
\begin{align*}
  f(0)&\defeq 0\\
  f(\suc(n)) &\defeq f(\suc(\suc(n))).
\end{align*}
If a programmer wrote such a function, it would simply call itself forever on any positive input, never returning a value until it ran out of memory.
In mathematics, however, to be worthy of the name, a \emph{function} must always associate a unique output value to every input value, so this would be unacceptable.

This point will be even more important when we introduce more complicated inductive types in \autoref{cha:induction,cha:hits,cha:real-numbers}.
Whenever we introduce a new kind of inductive definition, we always begin by deriving its induction principle.
Only then do we introduce an appropriate sort of ``pattern matching'' which can be justified as a shorthand for the induction principle.%
\footnote{Formally speaking, one may describe type theory by taking either pattern matching or induction principles as basic and deriving the other.
  However, pattern matching in general is not yet well understood in \emph{homotopy} type theory, and can be dangerous unless sufficient care is taken.
  Thus, we will always regard the induction principle as the basic property of an inductive definition, with pattern matching justified in terms of induction.}


\section{Propositions as types}
\label{sec:pat}

To state that a proposition is true in type theory corresponds to exhibiting an element of a type corresponding to a proposition. We will however, in general not construct an explicit term representing this proof but instead present the proofs in prose so that \emph{in principle} they could be translated into an element of a type. This is no different to reasoning in classical set theory where we don't expect to see an explicit derivation using the rules of predicate logic and the axioms of set theory. One important difference is, however, that we cannot use certain classical principle such as the law of excluded middle or proof by contradiction, i.e., proving $A$ by deriving a contradiction from $\lnot A$.

We start with propositional logic: a proposition can be translated into a type following the following rules:
\begin{center}
\begin{tabular}{c|l}
  \textbf{English} & \textbf{Translation into Type Theory}\\\hline
  True & $\unit$ \\
  False & $\emptyt$ \\
  $A$ and $B$ & $A \times B$ \\
  $A$ or $B$ & $A + B$ \\
  If $A$ then $B$ & $A \to B$ \\
  $A$ if, and only if, $B$ & $(A \to B) \times (B \to A)$ \\
  Not $A$ &  $A \to \emptyt$
\end{tabular}
\end{center}
As an exercise you should verify 
the tautology \emph{``If not ($A$ or $B$), then  (not $A$) and (not $B$)}'' by exhibiting an element of 
\[ ((A + B) \to \emptyt) \to (A \to \emptyt) \times (B \to \emptyt), \]
for any types $A$, $B$ using the rules we have just introduced. However, not all classical tautologies hold: for example the rule 
\emph{If not ($A$ and $B$) then (not $A$) or (not $B$)} is not constructively valid and hence we cannot, in general, construct an element of the corresponding type
\[ ((A \times B) \to \emptyt) \to (A \to \emptyt) + (B \to \emptyt).\]
The intuitive reason is that the premise provides no information that would help us decide whether we should prove the disjunction using $\inl$ or $\inr$.

In predicate logic types play a dual role: they serve as propositions and as types in the conventional sense, i.e., domains we quantify over. Indeed, we are going to exploit this duality meaning that we can also quantify over the type of proofs of a proposition. A predicate over a type $A$ is representing as a family $P : A \to U$ assigning to every element $a : A$ a type $P(a)$ corresponding to the proposition that $P$ holds for $a$. We extend our translation with an explanation of the quantifiers:
\begin{center}
\begin{tabular}{c|l}
  \textbf{English} & \textbf{Translation into Type Theory}\\\hline
  For all $x:A$, $P(x)$ holds & $\prd{x:A} P(x)$ \\
  There exists $x:A$ such that $P(x)$ & $\sm{x:A}$ $P(x)$ \\
\end{tabular}
\end{center}
As before we can translate tautologies of predicate logic into types of predicate logic, for example \emph{If for all $x:A$, $P(x)$ and $Q(x)$ then (for all $x:A$, $P(x)$) and  (for all $x:A$, $Q(x)$)  } which translates to
\[ (\prd{x:A} P(x) \times Q(x)) \to (\prd{x:A} P(x)) \times (\prd{x:A} Q(x)) \]
and we can verify the tautology by constructing an element for any $A:\UU$ and $P,Q : A \to \UU$. 

We can use the universes to represent higher order logic even though in a limited way if we do not add additional assumptions. For example we can represent the proposition \emph{for all properties $P : A \to \UU$ if $P(a)$ then $P(b)$} can be represented as
\[ \prd{P : A \to \UU} P(a) \to P(b) \]
where $A : \UU$ and $a,b : A$. However, unless we assume the existence
of an impredicative universe of propositions, this
proposition  lives in a different, higher universe, then the
propositions we are quantifying over, that is
\[ \prd{P : A \to \UU_i} P(a) \to P(b) : \UU_{i+1} \]

We have descibed here a proof-relevant translation of propositions, where the proofs of disjunctions and existential statements carry some information. Using the language of homotopy type theory which we are going to introduce in the next section there is an alternative way to interpret the word \emph{proposition} as a (-1)-type, i.e. a type which doesn't carry any information. In this case we would say \emph{there merely exists} or \emph{merely $A$ or $B$}.

\section*{Exercises}

\begin{ex}
Derive the non-dependent eliminator for products $\rec{A\times B} $ using only the projections and verify that the definitional equalities are valid. Do the same for $\Sigma$-types.
\end{ex}

\begin{ex}
Derive the dependent eliminator for products $\ind{A\times B}$ using only the projections and $\spr$ and verify that the definitional equalities are valid. Generalize $\spr$ to $\Sigma$-types and do the same for $\Sigma$-types.
\end{ex}

\begin{ex}
Assuming as given only the iterator for natural numbers
\[\ite : \prd{C:\UU} C \to (C \to C) \to \nat \to C \]
with the defining equations
\begin{eqnarray*}
\ite(C,c_0,c_s,0)  & \defeq & c_0 \\
\ite(C,c_0,c_s,\suc(n)) & \defeq & c_s(\ite(C,C_0,c_s,n)  
\end{eqnarray*}
derive the recursor $\rec{\nat}$.
\end{ex}

\begin{ex}\label{ex:sum-via-bool}
Derive $\ind{A+B}$ for  $A + B \defeq \sm{x:\bool} \rec{\bool}(\UU,A,B,x)$ and show that the definitional equalities stating for native $\ind{A+B}$ hold.
\end{ex}

\begin{ex}\label{ex:prod-via-bool}
Derive $\ind{A\times B}$ for  $A \times B \defeq \prd{x:\bool}\rec{\bool}(\UU,A,B,x)$ and show that the definitional equalities stating for native $\ind{A+B}$ hold propositionally (i.e. using equality types).
\end{ex}

\begin{ex}
Give an alternative derivation of $\ind{=_A}^{PM}$ from $\ind{=_A}^{ML}$ which avoids the use of universes.
\end{ex}

\begin{ex}
Define multiplication and exponentiation using $\rec{\Nat}$. Formally verify that $(\nat,+,0,\times,1)$ is a semiring using only $\ind{\Nat}$.  
\end{ex}

\begin{ex}\label{ex:ackermann}
  Show that the Ackermann function $\ack : \nat \to \nat \to \nat$ is definable using only $\rec{\nat}$ satisfying the following equations:
  \begin{eqnarray*}
    \ack(0,n) & \jdeq & \suc(n) \\
    \ack(\suc(m),0) & \jdeq & 1 \\
    \ack(\suc(m),\suc(n)) & \jdeq & \ack(m,\ack(\suc(m),n))
  \end{eqnarray*}
\end{ex}

\begin{ex}
Using the propositions as types interpretation formally derive the double negation of the principle of excluded middle, i.e. show 
that \emph{not (not ($P$ or not $P$))}  is provable.
\end{ex}

% Local Variables:
% TeX-master: "main"
% End:
