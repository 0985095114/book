\chapter{Homotopy type theory}
\label{cha:basics}

\section{Types are higher groupoids}
\label{sec:equality}

Recall that for any type $A$, and any $x,y:A$, we have a identity type $\id[A]{x}{y}$, also written $\idtype[A]{x}{y}$ or just $x=y$.
We can think of inhabitants of $x=y$ as either
\begin{enumerate}
\item proofs or evidence that $x$ and $y$ are equal,
\item identifications of $x$ with $y$,
\item paths from $x$ to $y$, or
\item isomorphisms/equivalences from $x$ to $y$.
\end{enumerate}
The first is more traditional in type theory; but in homotopy type theory we often take the latter perspectives as well.
It turns out that the defining rules of identity types, as described in the previous chapter, give them structure which corresponds precisely to that of a space or a higher groupoid.

Recall that the induction principle for the identity types $\id[A]{x}{y}$ (with $A$ a fixed type) says that if
\begin{itemize}
\item for every $x,y:A$ and every $p:\id[A]xy$ we have a type $D(x,y,p)$, and
\item for every $a:A$ we have an element $d(a):D(a,a,\refl a)$, 
\end{itemize}
then
\begin{itemize}
\item there exists an element $J_{D,d}(x,y,p):D(x,y,p)$ for \emph{every} two elements $x,y:A$ and $p:\id[A]xy$, such that $J_{D,d}(a,a,\refl a) \jdeq d(a)$.
\end{itemize}
In other words, given dependent functions
\begin{align*}
D & :\prd{x,y:A}{p:\id{x}{y}} \type\\
d & :\prd{a:A} D(a,a,\refl{a})
\end{align*}
there is a dependent function
\[J_{D,d}:\prd{x,y:A}{p:\id{x}{y}} D(x,y,p)\]
such that 
\begin{equation}\label{eq:Jconv}
J_{D,d}(a,a,\refl{a})\jdeq d(a)
\end{equation}
for every $a:A$.
The notation $J$ is traditional for this function, but we will not use it very much.
Usually, every time we apply this induction rule we will either not care about the specific function being defined, or we will immediately give it a different name.

Informally, the induction principle for identity types says that if we want to construct an object (or prove a statement) which depends on an inhabitant $p:\id[A]xy$ of an identity type, then it suffices to perform the construction (or the proof) in the special case when $x$ and $y$ are the same (judgmentally) and $p$ is a reflexivity term $\refl{x}$ (judgmentally).
When writing informally, we may express this with a phrase such as ``by induction, it suffices to assume\dots''.
This reduction to the ``reflexivity case'' is analogous to the reduction to the ``base case'' and ``inductive step'' in an ordinary proof by induction on the natural numbers, and also to the ``left case'' and ``right case'' in a proof by case analysis on a disjoint union or disjunction.

The ``conversion rule''~\eqref{eq:Jconv} is less familiar in the context of proof by induction on natural numbers, but there is an analogous notion in the related concept of definition by recursion.
If a sequence $(a_n)_{n\in \mathbb{N}}$ is defined by giving $a_0$ and specifying $a_{n+1}$ in terms of $a_n$, then in fact the $0^{\mathrm{th}}$ term of the resulting sequence \emph{is} the given one, and the given recurrence relation relating $a_{n+1}$ to $a_n$ holds for the resulting sequence.
(This may seem so obvious as to not be worth saying, but if we view a definition by recursion as an algorithm for calculating values of a sequence, then it is precisely the process of executing that algorithm.)
The rule~\eqref{eq:Jconv} is analogous: it says that if we define an object $f(p)$ for all $p:x=y$ by specifying what the value should be when $p$ is $\refl{x}:x=x$, then the value we specified is in fact the value of $f(\refl{x})$.

We now derive from this induction principle all the structure of a higher groupoid.
We begin with symmetry of equality, which, in topological language, means that ``paths can be reversed''.

\begin{lem}\label{lem:opp}
  For every type $A$ and every $x,y:A$ there is a function
  \begin{equation*}
    (x= y)\to(y= x)
  \end{equation*}
  denoted $p\mapsto \opp{p}$, such that $\opp{\refl{x}}\jdeq\refl{x}$ for each $x:A$.
\end{lem}
\begin{proof}[First proof]
  Let $D:\prd{x,y:A}{p:x= y} \type$ be the type family defined by $D(x,y,p)\defeq (y= x)$.
  In other words, $D$ is a function assigning to any $x,y:A$ and $p:x=y$ a type, namely the type $y=x$.
  Then we have
  \begin{equation*}
    d\defeq \lambda x.\refl{x}:\prd{x:A} D(x,x,\refl{x}).
  \end{equation*}
  Thus, the eliminator $J$ for identity types gives us a term $J_{D,d}(x,y,p): (y= x)$ for each $p:(x= y)$.
  We can now define the desired function $\opp{(-)}$ to be $\lambda p. J_{D,d}(x,y,p)$, i.e.\ we set $\opp{p} \defeq J_{D,d}(x,y,p)$.
  The conversion rule~\eqref{eq:Jconv} gives $\opp{\refl{x}}\jdeq \refl{x}$.
\end{proof}

We have written out this proof in a very formal style, which may be helpful while the induction rule on identity types is unfamiliar.
However, eventually we prefer to use more natural language, such as in the following equivalent proof.

\begin{proof}[Second proof]
  We want to construct, for each $x,y:A$ and $p:x=y$, an element $\opp{p}:y=x$.
  By induction, it suffices to do this in the case when $y$ is $x$ and $p$ is $\refl{x}$.
  But in this case, the type $x=y$ of $p$ and the type $y=x$ in which we are trying to construct $\opp{p}$ are both simply $x=x$.
  Thus, in the ``reflexivity case'', we can define $\opp{\refl{x}}$ to be simply $\refl{x}$.
  The general case then follows by the induction principle, and the conversion rule $\opp{\refl{x}}\jdeq\refl{x}$ is precisely the proof in the reflexivity case that we gave.
\end{proof}

We will write out the next few proofs in both styles, to help the reader become accustomed to the latter one.
Next we prove the transitivity of equality, or equivalently we ``concatenate paths''.

\begin{lem}\label{lem:concat}
  For every type $A$ and every $x,y,z:A$ there is a function
  \begin{equation*}
    (x= y) \to (y= z)\to (x=  z)
  \end{equation*}
  written $(p,q)\mapsto p\ct q$, such that $\refl{x}\ct \refl{x}\jdeq \refl{x}$ for any $x:A$.
\end{lem}

\begin{proof}[First proof]
  Let $D:\prd{x,y:A}{p:x=y} \type$ be the type family
  \begin{equation*}
    D(x,y,p)\defeq \prd{z:A}{q:y=z} (x=z).
  \end{equation*}
  Note that $D(x,x,\refl x) \jdeq \prd{z:A}{q:x=z} (x=z)$.
  Thus, in order to apply the induction principle for identity types to this $D$, we need a function of type
  \begin{equation}\label{eq:concatD}
    \prd{x:A} D(x,x,\refl{x})
    \jdeq \prd{x,z:A}{q:x=z} (x=z).
  \end{equation}
  Now let $E:\prd{x,z:A}{q:x=z}\type$ be the type family $E(x,z,q)\defeq (x=z)$.
  Note that $E(x,x,\refl x) \jdeq (x=x)$.
  Thus, we have the function
  \begin{equation*}
    e(x) \defeq \refl{x} : E(x,x,\refl{x}).
  \end{equation*}
  By the induction principle for identity types applied to $E$, we obtain a function
  \begin{equation*}
    d(x,z,q) : \prd{x,z:A}{q:x=z} E(x,z,q) \jdeq \prd{x,z:A}{q:x=z} (x=z)
  \end{equation*}
  which is~\eqref{eq:concatD}.
  Thus, we can use this function $d$ and apply the induction principle for identity types to $D$, to obtain our desired function of type
  \begin{equation*}
    \prd{x,y,z:A}{q:y=z}{p:x=y} (x=z).
  \end{equation*}
  The conversion rules for the two induction principles give us $\refl{x}\ct \refl{x}\jdeq \refl{x}$ for any $x:A$.
\end{proof}

\begin{proof}[Second proof]
  We want to construct, for every $x,y,z:A$ and every $p:x=y$ and $q:y=z$, an element of $x=z$.
  By induction on $p$, it suffices to assume that $y$ is $x$ and $p$ is $\refl{x}$.
  In this case, the type $y=z$ of $q$ is $x=z$.
  Now by induction on $q$, it suffices to assume also that $z$ is $x$ and $q$ is $\refl{x}$.
  But in this case, $x=z$ is $x=x$, and we have $\refl{x}:(x=x)$.
\end{proof}

The reader may well feel that we have given an overly convoluted proof of this lemma.
In fact, we could stop after the induction on $p$, since at that point what we want to produce is an equality $x=z$, and we already have such an equality, namely $q$.
Why do we go on to do another induction on $q$?

The answer is that, as described in the introduction, we are doing \emph{proof-relevant} mathematics.
When we prove a lemma, we are defining an inhabitant of some type, and it can matter what \emph{specific} element we defined in the course of the proof, not merely the type that that element inhabits (that is, the \emph{statement} of the lemma).
\autoref{lem:concat} has three obvious proofs: we could do induction over $p$, induction over $q$, or induction over both of them.
If we proved it three different ways, we would have three different elements of the same type.
It's not hard to show that these three elements would be (provably) \emph{equal} (see \autoref{ex:basics:concat}), but there can still be reasons to prefer a particular definition over a provably equal one.

In the case of \autoref{lem:concat}, the difference hinges on the computation rule.
If we proved the lemma using a single induction over $p$, then we would end up with a computation rule of the form $\refl{y} \ct q \jdeq q$.
If we proved it with a single induction over $q$, we would have instead $p\ct\refl{x}\jdeq p$, while proving it with a double induction (as we did) gives only $\refl{x}\ct\refl{x} \jdeq \refl{x}$.

The asymmetrical computation rules can sometimes be convenient when doing formalized mathematics, as they allow the computer to simplify more things automatically.
However, in informal mathematics, and arguably even in the formalized case, it can be confusing to have a concatenation operation which behaves asymmetrically and to have to remember which side is the ``special'' one.
Treating both sides symmetrically makes for more robust proofs; this is why we have given the proof that we did.
(However, this is admittedly a stylistic choice.)

The table below summarizes the ``equality'' and ``homotopical'' points of view on what we have done so far.
\begin{center}
  \begin{tabular}{c|c}
    Equality & Homotopy \\\hline
    reflexivity & constant path\\
    symmetry & inversion of paths\\
    transitivity & concatenation of paths
  \end{tabular}
\end{center}

However, proof-relevance also means that we can't stop after proving ``symmetry'' and  ``transitivity'' of equality: we need to know that these \emph{operations} on equalities are well-behaved.
(This issue is invisible to set-level mathematics, where symmetry and transitivity are mere \emph{properties} of equality, rather than structure on paths.)
For instance, we need to know that concatenation is \emph{associative}, and that inversion provides \emph{inverses} with respect to concatenation.
This is to be expected from the topological point of view, where these are regarded as \emph{operations} on paths.

\begin{lem}\label{thm:omg}%[The $\omega$-groupoid structure of types]
  Suppose $A:\type$, that $x,y,z,w:A$ and that $p:x= y$ and $q:y = z$ and $r:z=w$.
  We have the following:
  \begin{enumerate}
  \item $p= p\ct \refl{y}$ and $p = \refl{x} \ct p$.\label{item:omg1}
  \item $\opp{p}\ct p=  \refl{y}$ and $p\ct \opp{p}= \refl{x}$.
  \item $\opp{(\opp{p})}= p$.
  \item $p\ct (q\ct r)=  (p\ct q)\ct r$.\label{item:omg4}
  \end{enumerate}
\end{lem}

Note, in particular, that~\ref{item:omg1}--\ref{item:omg4} are themselves propositional equalities, living in the identity types of the identity types $x=y$.
Topologically, they are \emph{paths of paths}, and we are familiar topologically with the idea that concatenating a path with the reversed path only gives a constant path \emph{up to homotopy}, i.e.\ up to a higher path.
The paths~\ref{item:omg1}--\ref{item:omg4} also satisfy their own higher coherence laws, which are themselves higher paths, and so on all the way up.

However, for most purposes it is unnecessary to make the whole infinite-dimensional structure explicit.
One of the nice things about homotopy type theory is that all of this structure can be \emph{proven} starting from only the inductive property of identity types, so we can make explicit as much or as little of it as we need.
In particular, often we don't need the complicated combinatorics involved in making precise notions such as ``coherent structure at all higher levels''.

\begin{proof}[Proof of~\autoref{thm:omg}]
  All the proofs use the induction principle for equalities.
  \begin{enumerate}
  \item \emph{(First proof)} Let $D:\prd{x,y:A}{p:x=y} \type$ be the type family given by 
    \begin{equation*}
      D(x,y,p)\defeq (p= p\ct \refl{y}).
    \end{equation*}
    Then $D(x,x,\refl{x})$ is $\refl{x}=\refl{x}\ct\refl{x}$.
    Since $\refl{x}\ct\refl{x}\jdeq\refl{x}$, it follows that $D(x,x,\refl{x})\jdeq (\refl{x}=\refl{x})$.
    Thus, there is a term
    \begin{equation*}
      d\defeq\lambda x.\refl{\refl{x}}:\prd{x:A} D(x,x,\refl{x}).
    \end{equation*}
    Now $J$ gives a term $J(D,d,p):(p= p\ct\refl{y})$ for each $p:x= y$.
    The other equality is proven similarly.

    \noindent
    \emph{(Second proof)} By induction on $p$, it suffices to assume that $y$ is $x$ and that $p$ is $\refl x$.
    But in this case, we have $\refl{x}\ct\refl{x}\jdeq\refl{x}$.
  \item \emph{(First proof)} Let $D:\prd{x,y:A}{p:x=y} \type$ be the type family given by 
    \begin{equation*}
      D(x,y,p)\defeq (\opp{p}\ct p=  \refl{y}).
    \end{equation*}
    Then $D(x,x,\refl{x})$ is $\opp{\refl{x}}\ct\refl{x}=\refl{x}$.
    Since $\opp{\refl{x}}\jdeq\refl{x}$ and $\refl{x}\ct\refl{x}\jdeq\refl{x}$, we get that $D(x,x,\refl{x})\jdeq (\refl{x}=\refl{x})$.
    Hence we find the function
    \begin{equation*}
      d\defeq\lambda x.\refl{\refl{x}}:\prd{x:A} D(x,x,\refl{x}).
    \end{equation*}
    Now $J$ gives a term $J(D,d,p):\opp{p}\ct p=\refl{y}$ for each $p:x= y$ in $A$.
    The other equality is similar.

    \noindent \emph{(Second proof)} By induction, it suffices to assume $p$ is $\refl x$.
    But in this case, we have $\opp{p} \ct p \jdeq \opp{\refl x} \ct \refl x \jdeq \refl x$.
  \item \emph{(First proof)} Let $D:\prd{x,y:A}{p:x=y} \type$ be the type family given by
    \begin{equation*}
      D(x,y,p)\defeq (\opp{\opp{p}}= p).
    \end{equation*}
    Then $D(x,x,\refl{x})$ is the type $(\opp{\opp{\refl x}}=\refl{x})$.
    But since $\opp{\refl{x}}\jdeq \refl{x}$ for each $x:A$, we have $\opp{\opp{\refl{x}}}\jdeq \opp{\refl{x}} \jdeq\refl{x}$, and thus $D(x,x,\refl{x})\jdeq(\refl{x}=\refl{x})$.
    Hence we find the function
    \begin{equation*}
      d\defeq\lambda x.\refl{\refl{x}}:\prd{x:A} D(x,x,\refl{x}).
    \end{equation*}
    Now $J$ gives a term $J(D,d,p):\opp{\opp{p}}= p$ for each $p:x= y$.

    \noindent \emph{(Second proof)} By induction, it suffices to assume $p$ is $\refl x$.
    But in this case, we have $\opp{\opp{p}}\jdeq \opp{\opp{\refl x}} \jdeq \refl x$.
  \item \emph{(First proof)} Let $D_1:\prd{x,y:A}{p:x=y} \type$ be the type family given by
    \begin{equation*}
      D_1(x,y,p)\defeq\prd{z,w:A}{q:y= z}{r:z= w} \big(p\ct (q\ct r)=  (p\ct q)\ct r\big).
    \end{equation*}
    Then $D_1(x,x,\refl{x})$ is
    \begin{equation*}
      \prd{z,w:A}{q:x= z}{r:z= w} \big(\refl{x}\ct(q\ct r)= (\refl{x}\ct q)\ct r\big).
    \end{equation*}
    To construct a term in this type, let $D_2:\prd{x,z:A}{q:x=z} \type$ be the type family
    \begin{equation*}
      D_2 (x,z,q) \defeq \prd{w:A}{r:z=w} \big(\refl{x}\ct(q\ct r)= (\refl{x}\ct q)\ct r\big).
    \end{equation*}
    Then $D_2(x,x,\refl{x})$ is
    \begin{equation*}
      \prd{w:A}{r:x=w} \big(\refl{x}\ct(\refl{x}\ct r)= (\refl{x}\ct \refl{x})\ct r\big).
    \end{equation*}
    To construct a term in \emph{this} type, let $D_3:\prd{x,w:A}{r:x=w} \type$ be the type family
    \begin{equation*}
      D_3(x,w,r) \defeq \big(\refl{x}\ct(\refl{x}\ct r)= (\refl{x}\ct \refl{x})\ct r\big).
    \end{equation*}
    Then $D_3(x,x,\refl{x})$ is
    \begin{equation*}
      \big(\refl{x}\ct(\refl{x}\ct \refl{x})= (\refl{x}\ct \refl{x})\ct \refl{x}\big)
    \end{equation*}
    which is definitionally equal to the type $(\refl{x} = \refl{x})$, and is therefore inhabited by $\refl{\refl{x}}$.
    Applying the identity elimination rule three times, therefore, we obtain a term of the overall desired type.

    \noindent \emph{(Second proof)} By induction, it suffices to assume $p$, $q$, and $r$ are all $\refl x$.
    But in this case, we have
    \begin{align*}
      p\ct (q\ct r)
      &\jdeq \refl{x}\ct(\refl{x}\ct \refl{x})\\
      &\jdeq \refl{x}\\
      &\jdeq (\refl{x}\ct \refl x)\ct \refl x\\
      &\jdeq (p\ct q)\ct r.
    \end{align*}
    Thus, we have $\refl{\refl{x}}$ inhabiting this type.\qedhere
  \end{enumerate}
\end{proof}

\begin{rmk}
  There are other ways to define all of these higher paths.
  For instance, in \autoref{thm:omg}\ref{item:omg4} we might do induction only over one or two paths rather than all three.
  All possibilities will produce \emph{definitionally} different proofs, but they will always be propositionally the same.
  Such an equality between any two particular proofs can, again, be proven by induction, reducing all the paths in question to reflexivities and then observing that both proofs reduce themselves to reflexivities.
\end{rmk}

\section{Functions are functors}
\label{sec:functors}

Now we wish to establish that functions $f:A\to B$ behave functorially on paths.
In traditional type theory, this is equivalently the statement that functions respect equality.
Topologically, this corresponds to saying that every function is ``continuous'', i.e.\ preserves paths.

\begin{lem}\label{lem:map}
  Suppose that $f:A\to B$ is a function and that $p:(\id[A]xy)$.
  Then for any $x,y:A$ there is an operation
  \begin{equation*}
    \apfunc f : (x=y) \to (f(x)= f(y)).
  \end{equation*}
  Moreover, for each $x:A$ we have $\apfunc{f}(\refl{x})\jdeq \refl{f(x)}$.
\end{lem}

The notation $\apfunc f$ can be read either as the \underline{ap}plication of $f$ to a path, or as the \underline{a}ction on \underline{p}aths of $f$.

\begin{proof}[First proof]
  Let $D:\prd{x,y:A}{p:x=y}\type$ be the type family defined by
  \[D(x,y,p)\defeq (f(x)= f(y)).\]
  Then we have
  \begin{equation*}
    d\defeq\lambda x.\refl{f(x)}:\prd{x:A} D(x,x,\refl{x}).
  \end{equation*}
  Applying $J$, we obtain $\apfunc f : \prd{x,y:A}{p:x=y}(f(x)=g(x))$.
  The conversion rule implies $\apfunc f({\refl{x}})\jdeq\refl{f(x)}$ for each $x:A$.
\end{proof}

\begin{proof}[Second proof]
  By induction, it suffices to assume $p$ is $\refl{x}$.
  In this case, we may define $\apfunc f(p) \defeq \refl{f(x)}:f(x)\jdeq f(x)$.
\end{proof}

We will often write $\apfunc f (p)$ as simply $\ap f p$.
This is strictly speaking ambiguous, but generally no confusion arises.
It matches the common convention in category theory of using the same symbol for the application of a functor to objects and to morphisms.

Now, since \emph{dependently typed} functions are very important in type theory, we will also need a version of \autoref{lem:map} for these.
However, this is not quite so simple to state, because if $f:\prd{x:A} B(x)$ and $p:x=y$, then $f(x):B(x)$ and $f(y):B(y)$ are elements of distinct types, so that \emph{a priori} we cannot even ask whether they are equal.
The missing ingredient is that $p$ itself gives us a way to relate the types $B(x)$ and $B(y)$.

\begin{lem}[Transport]\label{lem:transport}
  Suppose that $P$ is a type family over $A$ and that $p:\id[A]xy$.
  Then there is a function $\transf{p}:P(x)\to P(y)$.
\end{lem}

\begin{proof}[First proof]
  Let $D:\prd{x,y:A}{p:\id{x}{y}} \type$ be the type family defined by
  \[D(x,y,p)\defeq P(x)\to P(y).\]
  Then we have the function
  \begin{equation*}
    d\defeq\lambda x.\idfunc[P(x)]:\prd{x:A} D(x,x,\refl{x}),
  \end{equation*}
  so that the induction principle gives us $J_{D,d}(x,y,p):P(x)\to P(y)$ for $p:x= y$, which we define to be $\transf p$.
\end{proof}

\begin{proof}[Second proof]
  By induction, it suffices to assume $p$ is $\refl x$.
  But in this case, we can take $\transf{(\refl x)}:P(x)\to P(x)$ to be the identity function.
\end{proof}

Sometimes, it is necessary to notate the type family $P$ in which the transport operation happens.
In this case, we may write
\[\transfib P p - : P(x) \to P(y).\]

Recall that a type family $P$ over a type $A$ can be seen as a property of elements of $A$, which holds at $x$ in $A$ if $P(x)$ is inhabited.
Then the transportation lemma says that if $x$ is propositionally equal to $y$, then $P(x)$ holds if and only if $P(y)$ holds.
In fact, we will see later on that if $x=y$ then actually $P(x)$ and $P(y)$ are \emph{equivalent}.

Now we can prove the dependent version of \autoref{lem:map}.

\begin{lem}[Dependent map]\label{lem:mapdep}
  Suppose $f:\prd{x: A} P(x)$; then we have
  \[\apdfunc f :(x=y) \to \big(\id[P(y)]{\trans p{f(x)}}{f(y)}\big).\]
\end{lem}

\begin{proof}[First proof]
  Let $D:\prd{x,y:A}{p:\id{x}{y}} \type$ be the type family defined by
  \begin{equation*}
    D(x,y,p)\defeq \trans p {f(x)}= f(y).
  \end{equation*}
  Then $D(x,x,\refl{x})$ is $\trans{(\refl{x})}{f(x)}= f(x)$.
  But since $\trans{(\refl{x})}{f(x)}\jdeq f(x)$, we get that $D(x,x,\refl{x})\jdeq (f(x)= f(x))$.
  Thus, we find the term
  \begin{equation*}
    d\defeq\lambda x.\refl{f(x)}:\prd{x:A} D(x,x,\refl{x})
  \end{equation*}
  and now $J$ gives us $\apdfunc f(p):\trans p{f(x)}= f(y)$ for each $p:x= y$.
\end{proof}

\begin{proof}[Second proof]
  By induction, it suffices to assume $p$ is $\refl x$.
  But in this case, we have $\trans{(\refl{x})}{f(x)}\jdeq f(x)$ judgmentally.
\end{proof}

Recall that a non-dependently typed function $f:A\to B$ is just the special case of a dependently typed function $f:\prd{x:A} P(x)$ when $P$ is a constant type family, $P(x) \defeq B$.
In this case, $\apdfunc{f}$ and $\apfunc{f}$ are closely related, because of the following lemma:

\begin{lem}\label{thm:trans-trivial}
  If $P:A\to\type$ is defined by $P(x) \defeq B$ for a fixed $B:\type$, then for any $x,y:A$ and $p:x=y$ and $b:B$ we have $\transfib P p b = b$.
\end{lem}
\begin{proof}[First proof]
  Fix a $b:B$, and let $D:\prd{x,y:A}{p:\id{x}{y}} \type$ be the type family defined by
  \[ D(x,y,p) \defeq (\transfib P p b = b). \]
  Then $D(x,x,\refl x)$ is $(\transfib P{\refl{x}}{b} = b)$, which is judgmentally equal to $(b=b)$ by the computation rule for transporting.
  Thus, we have the term
  \[ d \defeq \lambda x.\refl{b} : \prd{x:A} D(x,x,\refl x). \]
  Now $J$ gives us a term in $\prd{x,y:A}{p:x=y}(\transfib P p b = b)$, as desired.
\end{proof}
\begin{proof}[Second proof]
  By induction, it suffices to assume $y$ is $x$ and $p$ is $\refl x$.
  But $\transfib P {\refl x} b \jdeq b$, so in this case what we have to prove is $b=b$, and we have $\refl{b}$ for this.
\end{proof}

Thus, by concatenating with the path defined in \autoref{thm:trans-trivial}, for any $x,y:A$ and $p:x=y$ and $f:A\to B$ we have functions
\begin{align}
  \big(f(x) = f(y)\big) &\to \big(\trans{p}{f(x)} = f(y)\big)\label{eq:ap-to-apd}
  \qquad\text{and} \\
  \big(\trans{p}{f(x)} = f(y)\big) &\to \big(f(x) = f(y)\big).\label{eq:apd-to-ap}
\end{align}
In fact, these functions are inverse equivalences (in the sense to be introduced in \S\ref{sec:basics-equivalences}), and they relate $\apfunc f (p)$  to $\apdfunc f (p)$.
But because the types of $\apdfunc{f}$ and $\apfunc{f}$ are different, it is often clearer to use different notations for them.
We may sometimes use a notation $\apd f p$ for $\apdfunc{f}(p)$, which is similar to the notation $\ap f p$ for $\apfunc{f}$.

\begin{thm}[Path lifting property]\label{thm:path_lifting}
Let $P:A\to\type$ be a type family over $A$ and assume we have $u:P(x)$ for $x:A$. Then we have the identity
\begin{equation*}
\mathsf{lift}(u,p):(x,u)=(y,\trans{p}{u})
\end{equation*}
in $\sm{x:A}P(x)$ for any $p:x=y$.
\end{thm}

\begin{proof}[First proof]
Let $D:\prd{x,y:A}{p:x=y}\type$ be defined by
\begin{equation*}
D(x,y,p)\defeq (x,u)=(y,\trans{p}{u}).
\end{equation*}
Then $D(x,x,\refl{x})\defeq (x,u)=(x,\trans{\refl{x}}{u})$. By the conversion rule we have $\trans{\refl{x}}{u}\defeq u$, so we see that $D(x,x,\refl{x})\defeq (x,u)=(x,u)$. Therefore we find $d(x)\defeq\refl{(x,u)}:D(x,x,\refl{x})$. Now $J$ gives a term of type $\prd{x,y:A}{p:x=y}(x,u)=(y,\trans{p}{u})$.
\end{proof}
\begin{proof}[Second proof] 
  By induction, it suffices to find a term of type $(x,u)=(x,\trans{\refl{x}}{u})$.
  Note that $\trans{\refl{x}}{u}\jdeq u$, so we really need to find a term of type $(x,u)=(x,u)$.
  But here we can take the reflexivity term.
\end{proof}

\medskip

At this point, we hope the reader is starting to get a feel for proofs by induction on identity types.
From now on we desist from giving both styles of proofs, allowing ourselves to use whatever is most clear and convenient (and often the second, more concise one).


\section{Summary of the basic higher structure}
\label{sec:basics-summary}

Here we summarize the basic definitions made in the previous two sections.

\begin{itemize}
\item $\opp{p} : y=x$, for $p:x=y$, defined by
  \[\opp{\;\refl{x}}\jdeq \refl{x}.\]
\item $p\ct q :y=z$, for $p:x=y$ and $q:y=z$, defined by
  \[ \refl{x}\ct\refl{x}\jdeq\refl{x}.\]
\item If $P$ is a type family over $A$ then $\transf{p}:P(x)\ra P(y)$, for $p:x=y$, defined by
  \[\transf{(\refl{x})}\jdeq \idfunc[P(x)].\]
\item If $f:A\ra B$ then $\map{f}{p}:f(x)=f(y)$, for $p:x=y$, defined by
  \[\map{f}{\refl{x}}\jdeq \refl{f(x)}.\]
\item If $f:\prod_{x:A}P(x)$ then $\mapdep{f}{p}:\trans{p}{f(x)}=f(y)$, for $p:x=y$, defined by
  \[\mapdep{f}{\refl{x}}\jdeq \refl{f(x)}.\]
\end{itemize}


\section{Homotopies and equivalences}
\label{sec:basics-equivalences}

So far, we have seen how the identity type $\id[A]xy$ can be regarded as a type of \emph{identifications}, \emph{paths}, or \emph{equivalences} between two elements $x$ and $y$ of a type $A$.
Now we investigate the appropriate notions of ``identification'' or ``sameness'' between \emph{functions} and between \emph{types}.
In \S\ref{sec:computational}, we will see that homotopy type theory allows us to identify these with instances of the identity type, but before we can do that we need to understand them in their own right.

Traditionally, we regard two functions as the same if they take equal values on all inputs.
Under the propositions-as-types interpretation, this suggests that two functions $f$ and $g$ should be the same if the type $\prd{x:A} (f(x)=g(x))$ is inhabited.
Under the homotopical interpretation, this dependent function type consists of \emph{continuous} paths or \emph{functorial} equivalences, and thus may be regarded as the type of \emph{homotopies} or of \emph{natural isomorphisms}.
We will adopt the topological terminology for this.
Note that it makes perfect sense even when $f$ and $g$ are dependent functions.

\begin{defn}
  Let $f,g:\prd{x:A} P(x)$ be two sections of a type family $P:A\to\type$.
  A \textbf{homotopy} from $f$ to $g$ is a term of type
  \begin{equation*}
    (f\htpy g)\;\defeq\; \prd{x:A} (f(x)=g(x))
  \end{equation*}
\end{defn}

Note that a homotopy is not the same as an identification $(f=g)$.
In \S\ref{sec:compute-pi} we will show that homotopies and identifications are nevertheless ``equivalent''.

The following proofs are left to the reader.

\begin{lem}\label{lem:homotopy-props}
  Homotopy is an equivalence relation on each function type $A\ra B$.
  That is, we have elements of the types
  \begin{gather*}
    \prd{f:A\to B} (f\htpy f)\\
    \prd{f,g:A\to B} (f\htpy g) \to (g\htpy f)\\
    \prd{f,g,h:A\to B} (f\htpy g) \to (g\htpy h) \to (f\htpy h).
  \end{gather*}
\end{lem}

\begin{lem}
  Composition is associative and unital up to homotopy.
  That is:
  \begin{enumerate}
  \item If $f:A\ra B$ then $f\circ \idfunc[A]\htpy f\htpy \idfunc[B]\circ f$.
  \item If $f:A\ra B, g:B\ra C$ and $h:C\ra D$ then $h\circ (g\circ f)\;\htpy\; (h\circ g)\circ f$.
  \end{enumerate}
\end{lem}

The first level of the continuity/naturality of homotopies can be expressed as follows:

\begin{lem}\label{lem:htpy_natural}
  Suppose $H:f\htpy g$ is a homotopy between functions $f,g:A\to B$ and let $p:\id[A]xy$.  Then there is a term of type
  \begin{equation*}
    H(x)\ct\ap{g}{p}=\ap{f}{p}\ct H(y).
  \end{equation*}
  We may also draw this as a commutative diagram:
  \begin{align*}
    \xymatrix{
      f(x) \ar[r]^{\ap fp} \ar[d]_{H(x)} & f(y) \ar[d]^{H(y)} \\
      g(x) \ar[r]_{\ap gp} & g(y)
    }
  \end{align*}
\end{lem}
\begin{proof}
  By induction, we may assume $p$ is $\refl x$.
  Since $\apfunc{f}$ and $\apfunc g$ compute on reflexivity, in this case what we must show is
  \[ H(x) \ct \refl{g(x)} = \refl{f(x)} \ct H(x). \]
  But this follows since both sides are equal to $H(x)$.
\end{proof}

\newcommand{\com}[3]{\mathtt{swap}_{#1,#2}(#3)}
\begin{cor}\label{cor:hom-fg}
Let $H : f \htpy \idfunc[A]$ be a homotopy, with $f : A \to A$. Then for any $x : A$ we have \[ H(f(x)) = \ap f{H(x)}. \] The above path will be denoted by $\com{h}{f}{x}$.
\end{cor}
\begin{proof}
By naturality of $H$, the following diagram commutes:
\begin{align*}
\xymatrix{
ffx \ar[r]^{\ap f{Hx}} \ar[d]_{H(fx)} & fx \ar[d]^{Hx} \\
fx \ar[r]_{Hx} & x
}
\end{align*}
Canceling $H(x)$, we see that $h(fx) = f(hx)$ as desired.
\end{proof}

Moving on to types, from a traditional perspective one may say that a function $f:A\to B$ is an \emph{isomorphism} if there is a function $g:B\to A$ such that both composites $f\circ g$ and $g\circ f$ are pointwise equal to the identity, i.e.\ such that $f \circ g \htpy \idfunc[B]$ and $g\circ f \htpy \idfunc[A]$.
A homotopical perspective suggests that this should be called a \emph{homotopy equivalence}, and from a categorical one, it should be called an \emph{equivalence of (higher) groupoids}.
However, when doing proof-relevant mathematics, the corresponding type
\begin{equation}
  \sm{g:B\to A} \big((f \circ g \htpy \idfunc[B]) \times (g\circ f \htpy \idfunc[A])\big)\label{eq:qinvtype}
\end{equation}
is poorly behaved.
For instance, for a single function $f:A\to B$ there may be multiple inhabitants of~\eqref{eq:qinvtype} which are not provably equal.
(This is closely related to the observation in higher category theory that often one needs to consider \emph{adjoint} equivalences rather than plain equivalences.)
For this reason, we give~\eqref{eq:qinvtype} the following historically accurate, but slightly derogatory-sounding name instead.

\begin{defn}
  For a function $f:A\to B$, a \textbf{quasi-inverse} of $f$ is a triple $(g,\alpha,\beta)$ consisting of a function $g:B\to A$ and homotopies
$\alpha:f\circ g\htpy \idfunc[B]$ and $\beta:g\circ f\htpy \idfunc[A]$.
\end{defn}

Thus,~\eqref{eq:qinvtype} is \emph{the type of quasi-inverses of $f$}; we may denote it by $\qinv(f)$.

\begin{eg}\label{eg:idequiv}
  The identity function $\idfunc[A]:A\to A$ has a quasi-inverse given by $\idfunc[A]$ itself, together with homotopies defined by $\alpha(y) \defeq \refl{y}$ and $\beta(x) \defeq \refl{x}$.
\end{eg}

In general, we will only use the word \emph{isomorphism} (and similar words such as \emph{bijection}) in the special case when the types $A$ and $B$ ``behave like sets'' (see \S\ref{sec:basics-sets}).
In this case, the type~\eqref{eq:qinvtype} is unproblematic.
We will reserve the word \emph{equivalence} for an improved notion with the following properties:
\begin{enumerate}
\item For each $f:A\to B$ there is a function $\qinv(f) \to \isequiv (f)$.\label{item:be1}
\item Similarly, for each $f$ we have $\isequiv (f) \to \qinv(f)$; thus the two are ``logically equivalent''.\label{item:be2}
\item For any two inhabitants $e_1,e_2:\isequiv(f)$ we have $e_1=e_2$.\label{item:be3}
\end{enumerate}
In Chapter~\ref{cha:equivalences} we will see that there are many different definitions of $\isequiv(f)$ which satisfy these three properties, but that all of them are equivalent.
For now, to convince the reader that such things exist, we mention only the easiest such definition (though it is not the one we will eventually settle on in Chapter~\ref{cha:equivalences}):
\begin{equation}
  \isequiv(f) \;\defeq\;
  \big(\sm{g:B\to A} (f\circ g \htpy \idfunc[B])\big)
  \times
  \big(\sm{h:B\to A} (h\circ f \htpy \idfunc[A])\big)\label{eq:isequiv-invertible}
\end{equation}
We can show~\ref{item:be1} and~\ref{item:be2} for this definition now.
A function $\qinv(f) \to \isequiv (f)$ is easy to define by taking $(g,\alpha,\beta)$ to $(g,\alpha,g,\beta)$.
In the other direction, given $(g,\alpha,h,\beta)$, let $\gamma$ be the composite homotopy
\[ g \overset{\beta}{\htpy} h\circ f\circ g \overset{\alpha}{\htpy} h \]
and let $\beta':g\circ f\htpy \idfunc[A]$ be obtained from $\gamma$ and $\beta$.
Then $(g,\alpha,\beta'):\qinv(f)$.

Property~\ref{item:be3} for this definition is not too hard to prove either, but it requires identifying the identity types of cartesian products and dependent pair types, which we will discuss in \S\ref{sec:computational}.
Thus, we postpone it as well.
At this point, the main thing to take away is that there is a well-behaved type which we can pronounce as ``$f$ is an equivalence'', and that we can prove $f$ to be an equivalence by exhibiting a quasi-inverse to it.
In practice, this is the most common approach.

In accord with the proof-relevant philosophy, \emph{an equivalence} from $A$ to $B$ is defined to be a function $f:A\to B$ together with an inhabitant of $\isequiv (f)$, i.e.\ a proof that it is an equivalence.
We write $(\eqv A B)$ for the type of equivalences from $A$ to $B$, i.e.\ the type
\[ (\eqv A B) \;\defeq \; \sm{f:A\to B} \isequiv(f). \]
Property~\ref{item:be3} above will ensure that if two equivalences are equal as functions (that is, the underlying elements of $A\to B$ are equal), then they are also equal as equivalences (see \S\ref{sec:compute-sigma}).

% Local Variables:
% TeX-master: "main"
% End:


\newcommand{\fcomp}{\circ}

\section{The computational behavior of type formers}
\label{sec:computational}

In \autoref{cha:typetheory}, we introduced many ways to form new types: cartesian products, disjoint unions, dependent products, dependent sums, etc.
In the previous sections of this chapter, we have seen that \emph{all} types in homotopy type theory behave like spaces or higher groupoids.
Our goal in this section is to make explicit how this higher structure behaves, for particular types defined as in \autoref{cha:typetheory}.

It turns out that for many types $A$, the equality types $\id[A]xy$ can be characterized, up to equivalence, in terms of whatever data was used to construct $A$.
For instance, if $A$ is a cartesian product $B\times C$, and $x\jdeq (b,c)$ and $y\jdeq(b',c')$, then we have an equivalence
\begin{equation}\label{eq:prodeqv}
  \eqv{\Big((b,c)=(b',c')\Big)}{\Big((b=b')\times (c=c')\Big)}.
\end{equation}
In more traditional language, two ordered pairs are equal just when their components are equal (but the equivalence~\eqref{eq:prodeqv} says rather more than this).
The higher structure of the identity types can also be expressed in terms of these equivalences; for instance, concatenating two equalities between pairs corresponds to pairwise concatenation.

Similarly, when a type family $P:A\to\type$ is built up fiberwise using the type forming rules from \autoref{cha:typetheory}, the operation $\transfib{P}{p}{-}$ can be characterized, up to homotopy, in terms of the corresponding operations on the data that went into $P$.
For instance, if $P(x) \jdeq B(x)\times C(x)$, then we have
\[\transfib{P}{p}{(b,c)} = \left(\transfib{B}{p}{b},\transfib{C}{p}{c}\right).\]

Finally, the type forming rules are also functorial, and if a function $f$ is built from this functoriality, then the operations $\apfunc f$ and $\apdfunc f$ can be computed based on the corresponding ones on the data going into $f$.
For instance, if $g:B\to B'$ and $h:C\to C'$ and we define $f:B\times C \to B'\times C'$ by $f(b,c)\defeq (g(b),h(c))$, then modulo the equivalence~\eqref{eq:prodeqv}, we can identify $\apfunc f$ with ``$(\apfunc g,\apfunc h)$''.

In this section, we will state and prove theorems of this sort for all the basic type forming rules.

\begin{rmk}\label{rmk:computational-hope}
  In the type theory we are working with, identity types are defined simultaneously for all types by their induction principle.
  The characterizations for particular types to be discussed in this chapter are then theorems which we have to discover and prove.
  An alternative presentation of type theory might take these characterizations as \emph{definitions} of the identity types (by induction over the construction of types), with the general induction principle then being provable.
  While such a type theory has not yet been made precise except in very simple cases, it is still helpful to think of the rules to be presented in this section as ``computation'' rules for ``evaluating'' identity types, transport, and function application.
  It is important to note, though, that not \emph{all} identity types can be ``defined'' by induction over the construction of types; counterexamples include most nontrivial higher inductive types (see \autoref{cha:hits,cha:homotopy}).
\end{rmk}

\subsection{Cartesian product types}
\label{sec:compute-cartprod}

Given types $A$ and $B$, consider the cartesian product type $A \times B$.  
For any elements $x,y:A\times B$ and a path $p:\id[A\times B]{x}{y}$, by functoriality we can extract paths $\ap{\proj1}p:\id[A]{\proj1(x)}{\proj1(y)}$ and $\ap{\proj2}p:\id[B]{\proj2(x)}{\proj2(y)}$.
Thus, we have a function
\begin{equation}
  (\id[A\times B]{x}{y}) \;\to\; (\id[A]{\proj1(x)}{\proj1(y)}) \times (\id[B]{\proj2(x)}{\proj2(y)}).\label{eq:path-prod}
\end{equation}

\begin{thm}\label{thm:path-prod}
  For any $x$ and $y$, the function~\eqref{eq:path-prod} is an equivalence.
\end{thm}

Read logically, this says that two pairs are equal if they are equal
componentwise.  Read category-theoretically, this says that the
morphisms in a product groupoid are pairs of morphisms.  Read
homotopy-theoretically, this says that the paths in a product
space are pairs of paths.

\begin{proof}
  We need a function in the other direction:
  \begin{equation}
    (\id[A]{\proj1(x)}{\proj1(y)}) \times (\id[B]{\proj2(x)}{\proj2(y)}) \;\to\; (\id[A\times B]{x}{y}) .\label{eq:path-prod-inverse}
  \end{equation}
  By the induction rule for cartesian products, we may assume that $x$ and $y$ are both pairs, i.e.\ $x\jdeq (a,b)$ and $y\jdeq (a',b')$ for some $a,a':A$ and $b,b':B$.
  In this case, what we want is a function
  \begin{equation*}
    (\id[A]{a}{a'}) \times (\id[B]{b}{b'}) \;\to\; \big(\id[A\times B]{(a,b)}{(a',b')}\big).
  \end{equation*}
  Now by induction for the cartesian product in its domain, we may assume given $p:a=a'$ and $q:b=b'$.
  And by two path inductions, we may assume that $a\jdeq a'$ and $b\jdeq b'$ and both $p$ and $q$ are reflexivity.
  But in this case, we have $(a,b)\jdeq(a',b')$ and so we can take the output to also be reflexivity.

  It remains to prove that~\eqref{eq:path-prod-inverse} is quasi-inverse to~\eqref{eq:path-prod}.
  This is a simple sequence of inductions, but they have to be done in the right order.

  In one direction, let us start with $r:\id[A\times B]{x}{y}$.
  We first do a path induction on $r$ in order to assume that $x\jdeq y$ and $r$ is reflexivity.
  In this case, since $\apfunc{\proj1}$ and $\apfunc{\proj2}$ are defined by path induction,~\eqref{eq:path-prod} takes $r\jdeq \refl{x}$ to the pair $(\refl{\proj1x},\refl{\proj2x})$.
  Now by induction on $x$, we may assume $x\jdeq (a,b)$, so that this is $(\refl a, \refl b)$.
  Thus,~\eqref{eq:path-prod-inverse} takes it by definition to $\refl{(a,b)}$, which (under our current assumptions) is $r$.
  
  In the other direction, if we start with $s:(\id[A]{\proj1(x)}{\proj1(y)}) \times (\id[B]{\proj2(x)}{\proj2(y)})$, then we first do induction on $x$ and $y$ to assume that they are pairs $(a,b)$ and $(a',b')$, and then induction on $s:(\id[A]{a}{a'}) \times (\id[B]{b}{b'})$ to reduce it to a pair $(p,q)$ where $p:a=a'$ and $q:b=b'$.
  Now by induction on $p$ and $q$, we may assume they are reflexivities $\refl a$ and $\refl b$, in which case~\eqref{eq:path-prod-inverse} yields $\refl{(a,b)}$ and then~\eqref{eq:path-prod} returns us to $(\refl a,\refl b)\jdeq (p,q)\jdeq s$.
\end{proof}

In particular, we have shown that~\eqref{eq:path-prod} has an inverse~\eqref{eq:path-prod-inverse}, which we may denote by
\[
\pairpath : (\id{\proj{1}(x)}{\proj{1}(y)}) \times (\id{\proj{1}(x)}{\proj{1}(y)}) \;\to\; {(\id x y)}
\]
Note that a special case of this yields the $\eta$-equivalence rule for products: $z = (\proj1(z),\proj2(z))$.

It can be helpful to view \pairpath as a \emph{constructor} or \emph{introduction rule} for $\id x y$, analogous to the ``pairing'' constructor of $A\times B$ itself, which introduces the pair $(a,b)$ given $a:A$ and $b:B$.
From this perspective, the two components of~\eqref{eq:path-prod}:
\begin{align*}
  \projpath{1} &: (\id{x}{y}) \to (\id{\proj{1}(x)}{\proj{1} (y)})\\
  \projpath{2} &: (\id{x}{y}) \to (\id{\proj{2}(x)}{\proj{2} (y)})
\end{align*}
are \emph{elimination} rules.
Similarly, the two homotopies which witness~\eqref{eq:path-prod-inverse} as quasi-inverse to~\eqref{eq:path-prod} consist respectively, of \emph{computation} or \emph{$\beta$-reduction} rules:
\begin{align*}
  {\projpath{1}{(\pairpath(p, q)})}
  &= %_{(\id{\proj{1} x}{\proj{1} y})}
  {p} \qquad\text{for } p:\id{\proj{1} x}{\proj{1} y} \\
  {\projpath{2}{(\pairpath(p,q)})}
  &= %_{(\id{\proj{2} x}{\proj{2} y})}
  {q} \qquad\text{for } q:\id{\proj{2} x}{\proj{2} y}
\end{align*}
and an \emph{$\eta$-equivalence} rule:
\[
\id{r}{\pairpath(\projpath{1} (r), \projpath{2} (r)) }
\qquad\text{for } r : \id[A \times B] x y
\]

We can also characterize the reflexivity, inverses, and composition of paths in $A\times B$ componentwise:
\begin{align*}
  {\refl{(z : A \times B)}}
  &= {\pairpath (\refl{\proj{1} z},\refl{\proj{2} z})} \\
  {\opp{p}}
  &= {\pairpath \big(\opp{\projpath{1} (p)},\, \opp{\projpath{2} (p)}\big)} \\
  {{p \ct q}}
  &= {\pairpath \big({\projpath{1} (p)} \ct {\projpath{1} (q)},\,{\projpath{2} (p)} \ct {\projpath{2} (q)}\big)}
\end{align*}
The same is true for the rest of the higher groupoid structure considered in \autoref{sec:equality}.
All of these equations can be derived by using path induction on the given paths and then returning reflexivity.  

We now consider transport in a pointwise product of type families.
Given type families $ A, B : Z \to \type$, we abusively write $A\times B:Z\to \type$ for the type family defined by $(A\times B)(z) \defeq A(z) \times B(z)$.
Now given $p : \id[Z]{z}{w}$ and $x : A(z) \times B(z)$, we can transport $x$ along $p$ to obtain an element of $A(w)\times B(w)$.

\begin{thm}\label{thm:trans-prod}
  In the above situation, we have
  \[
  \id[A(y) \times B(y)]
  {\transfib{A\times B}px}
  {(\transfib{A}{p}{\proj{1}x}, \transfib{B}{p}{\proj{2}x})}
  \]
\end{thm}
\begin{proof}
  By path induction, we may assume $p$ is reflexivity, in which case we have
  \begin{align*}
    \transfib{A\times B}px&\jdeq x\\
    \transfib{A}{p}{\proj{1}x}&\jdeq \proj1x\\
    \transfib{A}{p}{\proj{2}x}&\jdeq \proj2x.
  \end{align*}
  Thus, it remains to show $x = (\proj1 x, \proj2x)$.
  But this is $\eta$-equivalence, which as we remarked above follows from \autoref{thm:path-prod}.
\end{proof}

Finally, we consider the functoriality of $\apfunc{}$ under cartesian products.
Suppose given types $A,B,A',B'$ and functions $g:A\to A'$ and $h:B\to B'$; then we can define a function $f:A\times B\to A'\times B'$ by $f(x) \defeq (g(\proj1x),h(\proj2x))$.

\begin{thm}\label{thm:ap-prod}
  In the above situation, given $x,y:A\times B$ and $p:\proj1x=\proj1y$ and $q:\proj2x=\proj2y$, we have
  \[ \id[(f(x)=f(y))]{\ap{f}{\pairpath(p,q)}} {\pairpath(\ap{g}{p},\ap{h}{q})}. \]
\end{thm}
\begin{proof}
  Note first that the above equation is well-typed.
  On the one hand, since $\pairpath(p,q):x=y$ we have $\ap{f}{\pairpath(p,q)}:f(x)=f(y)$.
  On the other hand, since $\proj1(f(x))\jdeq g(\proj1x)$ and $\proj2(f(x))\jdeq h(\proj2x)$, we also have $\pairpath(\ap{g}{p},\ap{h}{q}):f(x)=f(y)$.

  Now, by induction, we may assume $x\jdeq(a,b)$ and $y\jdeq(a',b')$, in which case we have $p:a=a'$ and $q:b=b'$.
  Thus, by path induction, we may assume $p$ and $q$ are reflexivity, in which case the desired equation holds judgmentally.
\end{proof}


\subsection{\texorpdfstring{$\Sigma$}{Σ}-types}
\label{sec:compute-sigma}

Let $A$ be a type and $B:A\to\type$ a type family.
Recall that the $\Sigma$-type, or dependent pair type, $\sm{x:A} B(x)$ is a generalization of the cartesian product type.
Thus, we expect its higher groupoid structure to also be a generalization of the previous section.
In particular, its paths should be pairs of paths, but it takes a little thought to give the correct types of these paths.

Suppose that we have a path $p:w=w'$ in $\sm{x:A}P(x)$.
Then we get $\ap{\proj{1}}{p}:\proj{1}(w)=\proj{1}(w')$.
However, we cannot directly ask whether $\proj{2}(w)$ is identical to $\proj{2}(w')$ since they don't have to be in the same type.
But we can transport $\proj{2}(w)$ along the path $\ap{\proj{1}}{p}$, and this does give us an element of the same type as $\proj{2}(w')$.
By path induction, we do in fact obtain a path $\trans{\ap{\proj{1}}{p}}{\proj{2}(w)}=\proj{2}(w')$.

Recall from the discussion preceeding \autoref{lem:mapdep} that $\trans{\ap{\proj{1}}{p}}{\proj{2}(w)}=\proj{2}(w')$ can be regarded as the type of paths from $\proj2(w)$ to $\proj2(w')$ which lie over the path $\ap{\proj1}{p}$ in $A$.
Thus, we are saying that a path $w=w'$ in the total space determines (and is determined by) a path $p:\proj1(w)=\proj1(w')$ in $A$ together with a path from $\proj2(w)$ to $\proj2(w')$ lying over $p$, which seems sensible.

\begin{rmk}
  Note that if we have $x:A$ and $u,v:P(x)$ such that $(x,u)=(x,v)$, it does not follow that $u=v$.
  All we can conclude is that there exists $p:x=x$ such that $\trans p u = v$.
  This is a well-known source of confusion for newcomers to type theory, but it makes sense from a topological viewpoint: the existence of a path $(x,u)=(x,v)$ in the total space of a fibration betweeen two points that happen to lie in the same fiber does not imply the existence of a path $u=v$ lying entirely \emph{within} that fiber.
\end{rmk}

The next theorem states that we can also reverse this process.
Since it is a direct generalization of \autoref{thm:path-prod}, we will be more concise.

\begin{thm}\label{thm:path-sigma}
Suppose that $P:A\to\type$ is a type family over a type $A$ and let $w,w^\prime:\sm{x:A}P(x)$. Then there is an equivalence
\begin{equation*}
\eqvspaced{(w=w')}{\dsm{p:\proj{1}(w)=\proj{1}(w')} \trans{p}{\proj{2}(w)}=\proj{2}(w^\prime)}.
\end{equation*}
\end{thm}

\begin{proof}
We define for any $w,w':\sm{x:A}P(x)$, a function
\begin{equation*}
f: (w=w') \;\to\; \dsm{p:\proj{1}(w)=\proj{1}(w')} \trans{p}{\proj{2}(w)}=\proj{2}(w^\prime)
\end{equation*}
by path induction, with
\begin{equation*}
f(w,w,\refl{w})\defeq(\refl{\proj{1}(w)},\refl{\proj{2}(w)}).
\end{equation*}
We want to show that $f$ is an equivalence.

In the reverse direction, we define
\begin{equation*}
  g : \dprd{w,w':\sm{x:A}P(x)} \left(\dsm{p:\proj{1}(w)=\proj{1}(w')}\trans{p}{\proj{2}(w)}=\proj{2}(w^\prime)\right)
  \;\to\; (w=w')
\end{equation*}
by first inducting on $w$ and $w'$, which splits them into $(w_1,w_2)$ and
$(w_1',w_2')$ respectively, so it suffices to show 
\begin{equation*}
\left(\dsm{p:w_1 = w_1'}\trans{p}{w_2}=w_2^\prime\right) \;\to\; ((w_1,w_2)=(w_1',w_2'))
\end{equation*}
Next, given a pair $\sm{p:w_1 = w_1'}\trans{p}{w_2}=w_2^\prime$, we can
use $\Sigma$-induction to get $p : w_1 = w_1'$ and $q :
\trans{p}{w_2}=w_2^\prime$.  Inducting on $p$, we have $q :
\trans{\refl{}}{w_2}=w_2^\prime$, and it suffices to show 
$(w_1,w_2)=(w_1,w_2')$.  But $\trans{\refl{}}{w_2} \jdeq w_2$, so
inducting on $q$ reduces to the goal to 
$(w_1,w_2)=(w_1,w_2)$, which we can prove with $\refl{(w_1,w_2)}$.  

Next we show that $f \circ g$ is the identity for all $w$, $w'$ and
$r$, where $r$ has type
\[\dsm{p:\proj{1}(w)=\proj{1}(w')} (\trans{p}{\proj{2}(w)}=\proj{2}(w^\prime)).\]
First, we break apart the pairs $w$, $w'$, and $r$ by pair induction, as in the
definition of $g$, and then use two path inductions to reduce both components
of $r$ to \refl{}.  Then it suffices to show that 
$f (g(\refl{},\refl{})) = \refl{}$, which is true by definition.

Similarly, to show that $g \circ f$ is the identity for all $w$, $w'$,
and $p : w = w'$, we can do path-induction $p$, and then induction to
split $w$, at which point it suffices to show that
$g(f (\refl{(w_1,w_2)})) = \refl{(w_1,w_2)}$, which is true by
definition.

Thus, $f$ has a quasi-inverse, and is therefore an equivalence.  
\end{proof}

As we did in the case of cartesian products, we have $\eta$-equivalence as a special case.

\begin{cor}\label{thm:eta-sigma}
  For $z:\sm{x:A} P(x)$, we have $z = (\proj1(z),\proj2(z))$.
\end{cor}
\begin{proof}
  We have $\refl{\proj1(z)} : \proj1(z) = \proj1(\proj1(z),\proj2(z))$, so by \autoref{thm:path-sigma} it will suffice to exhibit a path $\trans{(\refl{\proj1(z)})}{\proj2(z)} = \proj2(\proj1(z),\proj2(z))$.
  But both sides are judgmentally equal to $\proj2(z)$.
\end{proof}

Like with binary cartesian products, we can think of 
the backward direction of \autoref{thm:path-sigma} as
an introduction form (\pairpath{}{}), the forward direction as
elimination forms (\projpath{1} and \projpath{2}), and the equivalence
as giving $\beta$ and $\eta$ rules for these.  

Note that the lifted path $\mathsf{lift}(u,p)$  of $p:x=y$ at $u:P(x)$ defined in \autoref{thm:path-lifting} may be identified with the special case of the introduction form
\[\pairpath(p,\refl{\trans p u}):(x,u) = (y,\trans p u).\]
This appears in the statement of action of transport on $\Sigma$-types, which is also a generalization of the action for binary cartesian products:

\begin{thm}
  Suppose we have type families $P:A\to\type$ and $Q:\big(\sm{x:A} P(x)\big)\to\type$.
  Then we can construct the type family over $A$ defined by
  \begin{equation*}
    x \;\mapsto\;  \sm{u:P(x)} Q(x,u).
  \end{equation*}
  For any path $p:x=y$ and any $(u,z):\sum(u:P(x)),\ Q(x,u)$ we have
  \begin{equation*}
    \trans{p}{u,z}=\big(\trans{p}{u},\,\trans{\pairpath(p,\refl{\trans pu})}{z}\big).
  \end{equation*}
\end{thm}

\begin{proof}
Immediate by path induction.
\end{proof}

We leave it to the reader to state and prove a generalization of
\autoref{thm:ap-prod} (see \autoref{ex:ap-sigma}), and to characterize
the reflexivity, inverses, and composition of $\Sigma$-types
componentwise.


\subsection{The unit type}
\label{sec:compute-unit}

Trivial cases are sometimes important, so we mention briefly the case of the unit type \unit.

\begin{thm}\label{thm:path-unit}
  For any $x,y:\unit$, we have $\eqv{(x=y)}{\unit}$.
\end{thm}
\begin{proof}
  A function $(x=y)\to\unit$ is easy to define by sending everything to \ttt.
  Conversely, for any $x,y:\unit$ we may assume by induction that $x\jdeq \ttt\jdeq y$.
  In this case we have $\refl{\ttt}:x=y$, yielding a constant function $\unit\to(x=y)$.

  To show that these are inverses, consider first an element $u:\unit$.
  We may assume that $u\jdeq\ttt$, but this is also the result of the composite $\unit \to (x=y)\to\unit$.

  On the other hand, suppose given $p:x=y$.
  By path induction, we may assume $x\jdeq y$ and $p$ is $\refl x$.
  We may then assume that $x$ is \ttt, in which case the composite $(x=y) \to \unit\to(x=y)$ takes $p$ to $\refl x$, i.e.\ to $p$.
\end{proof}

In particular, any two elements of $\unit$ are equal.
We leave it to the reader to formulate this equivalence in terms of introduction, elimination, $\beta$, and $\eta$ rules.
The transport lemma for \unit is simply the transport lemma for constant type families.


\subsection{\texorpdfstring{$\Pi$}{Π}-types}
\label{sec:compute-pi}

Given a type $A$ and a type family $B : A \to \type$, consider the dependent function type $\prd{x:A}B(x)$.
We expect paths $f=g$ in $\prd{x:A} B(x)$ to be equivalent to pointwise paths:
\begin{equation}
  \eqvspaced{(\id{f}{g})}{\big(\prd{x:A} (\id[B(x)]{f(x)}{g(x)})\big)}\label{eq:path-forall}
\end{equation}
From a traditional perspective, this would say that two functions which are equal at each point are equal as functions.
From a topological perspective, it would say that a path in a function space is the same as a continuous homotopy.
And from a categorical perspective, it would say that an isomorphism in a functor category is a natural family of isomorphisms.

Unlike the case in the previous sections, however, the basic type theory presented in \autoref{cha:typetheory} is insufficient to prove~\eqref{eq:path-forall}.
All we can say is that there is a function
\begin{equation}
  \happly : {(\id{f}{g})}\;\to\; \prd{x:A} (\id[B(x)]{f(x)}{g(x)})\label{eq:happly}
\end{equation}
which is easily defined by path induction.
For the moment, therefore, we will assume:

\begin{axiom}[Function extensionality]\label{axiom:funext}
  For any $A$, $B$, $f$, and $g$, the function~\eqref{eq:happly} is an equivalence.
\end{axiom}

We will see in later chapters that this axiom follows both from univalence (see \autoref{sec:compute-universe} and \autoref{cha:univalence}) and from an interval type (see \autoref{sec:interval}).
Moreover, it could be more naturally built into a computational version of homotopy type theory, as envisioned in \autoref{rmk:computational-hope}.

In particular,~\eqref{eq:happly} has a quasi-inverse
\[
\funext : \left(\prd{x:A} (\id{f(x)}{g(x)})\right) \to {(\id{f}{g})}
\]
which is also called ``function extensionality''.
As we did with $\pairpath$ in \autoref{sec:compute-cartprod}, we can regard $\funext$ as an \emph{introduction rule} for the type $\id f g$.
From this point of view, $\happly$ is the \emph{elimination rule}, while the homotopies witnessing $\funext$ as quasi-inverse to $\happly$ become a $\beta$-reduction rule
\[
\id{\happly({\funext{(h)}},x)}{h(x)} \qquad\text{for }h:\prd{x:A} (\id{f(x)}{g(x)})
\]
and an $\eta$-equivalence rule:
\[
\id{p}{\funext (x \mapsto \happly(p,{x}))} \qquad\text{for } p: (\id f g)
\]
%% FIXME: where do the rules for \alpha[\delta] go in this style?

We can also compute the identity, inverses, and composition in $\Pi$-types; they are simply given by pointwise operations.
\begin{align*}
\refl{f} &= \funext(x \mapsto \refl{f(x)}) \\
\opp{\alpha} &= \funext (x \mapsto \opp{\happly (\alpha)(x)})  \\
{\alpha} \ct \beta &= \funext (x \mapsto {\happly({\alpha})(x) \ct \happly({\beta})(x)})
\end{align*}

Since the non-dependent function type $A\to B$ is a special case of the dependent function type $\prd{x:A} B(x)$ when $B$ is independent of $x$, everything we have said above applies in non-dependent cases as well.
The rules for transport, however, are somewhat simpler in the non-dependent case.
Given a type $X$, a path $p:\id[X]{x_1}{x_2}$, type families $A,B:X\to \type$, and a function $f : A(x_1) \to B(x_1)$,  we have
\begin{align}\label{eq:transport-arrow}
  \transfib{A\to B}{p}{f} &=
  \; \Big(x \mapsto \transfib{B}{p}{f(\transfib{A}{\opp p}{x})}\Big)
\end{align}
where $A\to B$ denotes abusively the type family $X\to \type$ defined by
\[(A\to B)(x) \defeq (A(x)\to B(x)).\]
In other words, when we transport a function $f:A(x_1)\to B(x_1)$ along a path $p:x_1=x_2$, we obtain the function $A(x_2)\to B(x_2)$ which transports its argument backwards along $p$ (in the type family $A$), applies $f$, and then transports the result forwards along $p$ (in the type family $B$).
This can be proven easily by path induction.

Transporting dependent functions is similar, but more complicated.
Suppose given $X$ and $p$ as before, type families $A:X\to \type$ and $B:\prd{x:X} (A(x)\to\type)$, and also a dependent function $f : \prd{a:A(x_1)} B(x_1,a)$.
Then for $p:\id[A]{x_1}{x_2}$ and $a:A(x_2)$, we have
\begin{align*}
  \transfib{\Pi_A(B)}{p}{f}(a) =
  \Transfib{\hat{B}}{\opp{(\pairpath(\opp{p},\refl{ \trans{\opp p}{a} }))}}{f(\transfib{A}{\opp p}{a})}
\end{align*}
where $\Pi_A(B)$ and $\hat{B}$ denote respectively the type families
\begin{equation}\label{eq:transport-arrow-families}
\begin{array}{rclcl}
\Pi_A(B) &\defeq& \big(x\mapsto \prd{a:A(x)} B(x,a) \big) &:& X\to \type\\
\hat{B} &\defeq& \big(w \mapsto B(\proj1w,\proj2w) \big) &:& \big(\sm{x:X} A(x)\big) \to \type.
\end{array}
\end{equation}
If these formulas look a bit intimidating, don't worry about the details.
The basic idea is just the same as for the non-dependent function type: we transport the argument backwards, apply the function, and then transport the result forwards again.

Now recall that for a general type family $P:X\to\type$, in \autoref{sec:functors} we defined the type of \emph{dependent paths} over $p:\id[X]xy$ from $u:P(x)$ to $v:P(y)$ to be $\id[P(y)]{\trans{p}{u}}{v}$.
When $P$ is a family of function types, there is an equivalent way to represent this which is often more convenient.

\begin{lem}\label{thm:dpath-arrow}
  Given type families $A,B:X\to\type$ and $p:\id[X]xy$, and also $f:A(x)\to B(x)$ and $g:A(y)\to B(y)$, we have an equivalence
  \[ \eqvspaced{ \big(\trans{p}{f} = {g}\big) } { \prd{a:A(x)}  (\trans{p}{f(a)} = g(\trans{p}{a})) }. \]
  Moreover, if $q:\trans{p}{f} = {g}$ corresponds under this equivalence to $\hat q$, then for $a:A(x)$, the path
  \[ \happly(q,\trans p a) : (\trans p f)(\trans p a) = g(\trans p a)\]
  is equal to the composite
  \begin{alignat*}{2}
    (\trans p f)(\trans p a)
    &= \trans p {f (\trans {\opp p}{\trans p a})}
    &&\quad\text{by~\eqref{eq:transport-arrow}}\\
    &= \trans p {f(a)}\\
    &= g(\trans p a)
    &&\quad\text{by $\hat{q}$.}
  \end{alignat*}
\end{lem}
\begin{proof}
  By path induction, we may assume $p$ is reflexivity, in which case the desired equivalence reduces to function extensionality.
  The second statement then follows by the computation rule for function extensionality.
\end{proof}

As usual, the case of dependent functions is similar, but more complicated.

\begin{lem}\label{thm:dpath-forall}
  Given type families $A:X\to\type$ and $B:\prd{x:X} A(x)\to\type$ and $p:\id[X]xy$, and also $f:\prd{a:A(x)} B(x,a)$ and $g:\prd{a:A(y)} B(y,a)$, we have an equivalence
  \[ \eqvspaced{ \big(\trans{p}{f} = {g}\big) } { \Big(\prd{a:A(x)}  \transfib{\hat{B}}{\pairpath(p,\refl{\trans pa})}{f(a)} = g(\trans{p}{a}) \Big) } \]
  with $\hat{B}$ as in~\eqref{eq:transport-arrow-families}.
\end{lem}

We leave it to the reader to prove this and to formulate a suitable computation rule.


\subsection{The Universe}
\label{sec:compute-universe}

Given two types $A$ and $B$, we may consider them as elements of some universe type \type, and thereby form the identity type $\id[\type]AB$.
As mentioned in the introduction, \emph{univalence} is the identification of $\id[\type]AB$ with the type $(\eqv AB)$ of equivalences from $A$ to $B$, which we described in \autoref{sec:basics-equivalences}.
We perform this identification by way of the following canonical function.

\begin{lem}
  For types $A,B:\type$, there is a function
  \begin{equation}\label{eq:uidtoeqv}
    \idtoeqv : (\id[\type]AB) \to (\eqv A B).
  \end{equation}
\end{lem}
\begin{proof}
  We could construct this directly by induction on equality, but the following description is more convenient.
  Note that the identity function $\idfunc[\type]:\type\to\type$ may be regarded as a type family indexed by the universe \type; it assigns to each type $X:\type$ the type $X$ itself.
  Thus, given a path $p:A =_\type B$, we have a transport function $\transf{p}:A \to B$.
  We claim that $\transf{p}$ is an equivalence.
  But by induction, it suffices to assume that $p$ is $\refl A$, in which case $\transf{p} \jdeq \idfunc[A]$, which is an equivalence by \autoref{eg:idequiv}.
  Thus, we can define $\idtoeqv(p)$ to be $\transf{p}$ (together with the above proof that it is an equivalence).
\end{proof}

We would like to say that \idtoeqv is an equivalence.
However, as with $\happly$ for function types, the type theory described in \autoref{cha:typetheory} is insufficient to guarantee this.
Thus, as we did for function extensionality, we formulate this property as an axiom: Voevodsky's \emph{univalence axiom}.

\begin{axiom}[Univalence]\label{axiom:univalence}
  For any $A,B:\type$, the function~\eqref{eq:uidtoeqv} is an equivalence,
  \[
\eqv{(\id[\type]{A}{B})}{(\eqv A B)}.
\]
\end{axiom}
%
As with function extensionality, the univalence axiom could eventually be built into a more computational version of homotopy type theory.

\begin{rmk}
  It is important for the univalence axiom that we defined $\eqv AB$ using a ``good'' version of $\isequiv$ as described in \autoref{sec:basics-equivalences}, rather than (say) as $\sm{f:A\to B} \qinv(f)$.
\end{rmk}

In particular, univalence means that \emph{equivalent types may be identified}.
As we did in previous sections, it is useful to break this equivalence into:

\begin{itemize}
\item An introduction rule for {(\id[\type]{A}{B})}:
  \[
  \ua : ({\eqv A B}) \to (\id[\type]{A}{B})
  \]
\item The elimination rule, which is $\idtoeqv$:
  \[
  \idtoeqv \jdeq \transfibf{X \mapsto X} : (\id[\type]{A}{B}) \to (\eqv A B)
  \]
\item $\beta$-reduction: 
  \[
  \begin{array}{l}
  \transfib{X \mapsto X}{\ua(f)}{x} = f(x)
  \end{array}
  \]
\item $\eta$-equivalence: For any $p : \id A B$
  \[
  \id{p}{\ua(\transfibf{X \mapsto X}(p))}
  \]
\end{itemize}

We can also identify the reflexivity, concatenation, and inverses of equalities in the universe with the corresponding operations on equivalences:
\begin{align*}
  \refl{A} &= \ua(\idfunc[A]) \\
  \ua(f) \ct \ua(g) &= \ua(g\circ f) \\
  \opp{\ua(f)} &= \ua(f^{-1})
\end{align*}
The first of these follows because $\idfunc[A] = \idtoeqv(\refl{A})$ by definition of \idtoeqv, and \ua is the inverse of \idtoeqv.
For the second, if we define $p \defeq \ua(f)$ and $q\defeq \ua(g)$, then we have
\[ \ua(g\circ f) = \ua(\idtoeqv(q) \circ \idtoeqv(p)) = \ua(\idtoeqv(p\cdot q)) = p\cdot q\]
using \autoref{thm:transport-concat} and the definition of $\idtoeqv$.
The third is similar.

The following observation, which is a special case of \autoref{thm:transport-compose}, is often useful when applying the univalence axiom.

\begin{lem}\label{thm:transport-is-ap}
  For any type family $B:A\to\type$ and $x,y:A$ with a path $p:x=y$ and $u:B(x)$, we have
  \begin{align*}
    \transfib{B}{p}{u} &= \transfib{X\mapsto X}{\apfunc{B}(p)}{u}\\
    &= \idtoeqv(\apfunc{B}(p))(u).
  \end{align*}
\end{lem}


\subsection{Identity Type}
\label{sec:compute-paths}

Just as the type \id[A]{a}{a'} is characterized up to isomorphism, with
a separate ``definition'' for each $A$, there is no simple
characterization of the type \id[{\id[A]{a}{a'}}]{p}{q} of paths between
paths $p,q : \id[A]{a}{a'}$.
However, our other general classes of theorems do extend to identity types, such as the fact that they respect equivalence.

\begin{thm}\label{thm:paths-respects-equiv}
  If $f : A \to B$ is an equivalence, then for all $a,a':A$, so is
  \[\apfunc{f} : (\id[A]{a}{a'}) \to (\id[B]{f(a)}{f(a')}).\]
\end{thm}
\begin{proof}
  Let $\opp f$ be a quasi-inverse of $f$, with homotopies $\alpha:\prd{b:B} (f(\opp f(b))=b)$ and $\beta:\prd{a:A} (\opp f(f(a)) = a)$.
  The quasi-inverse of $\apfunc{f}$ is, essentially, \apfunc{\opp f}.
  However, the type of \apfunc{\opp f} is
  \[\apfunc{\opp f} : (\id{f(a)}{f(a')}) \to (\id{\opp f(f(a))}{\opp f(f(a'))}).\]
  Thus, in order to obtain an element of $\id[A]{a}{a'}$ we must concatenate with the paths $\opp{\beta(a)}$ and $\beta (a')$ on either side.
  To show that this gives a quasi-inverse of $\apfunc{f}$, on one hand we must show that for any $p:a=a'$ we have
  \[ \opp{\beta(a)} \ct \apfunc{\opp f}(\apfunc{f}(p)) \ct \beta(a') = p. \]
  This follows from the functoriality of $\apfunc{}$ on function composition (\autoref{lem:ap-functor}\ref{item:apfunctor-compose}) and the naturality of homotopies (\autoref{lem:htpy-natural}).
  On the other hand, we must show that for any $q:f(a)=f(a')$ we have
  \[ \apfunc{f}\big( \opp{\beta(a)} \ct \apfunc{\opp f}(q) \ct \beta(a') \big) = q. \]
  This follows in the same way, using also the functoriality of $\apfunc{}$ on path-concatenation and inverses (\autoref{lem:ap-functor}\ref{item:apfunctor-ct} and~\ref{item:apfunctor-opp}).
\end{proof}

Thus, if for some type $A$ we have a full characterization of $\id[A]{a}{a'}$, the type $\id[{\id[A]{a}{a'}}]{p}{q}$ is determined as well.  
For example:
\begin{itemize}
\item Paths $p = q$, where $p,q : \id[A \times B]{w}{w'}$, are equivalent to pairs of paths
  \[\id[{\id[A]{\proj{1} w}{\proj{1} w'}}]{\projpath{1}{p}}{\projpath{1}{q}}
  \quad\text{and}\quad
  \id[{\id[B]{\proj{2} w}{\proj{2} w'}}]{\projpath{2}{p}}{\projpath{2}{q}}.
  \]
\item Paths $p = q$, where $p,q : \id[\prd{x:A} B(x)]{f}{g}$, are equivalent to homotopies
  \[\prd{x:A} (\id[B(x)] {\happly(p)(x)}{\happly(q)(x)}).\]
\end{itemize}

Next we consider transport in families of paths, i.e.\ transport in $C:A\to\type$ where each $C(x)$ is an identity type.
The simplest case is when $C(x)$ is a type of paths in $A$ itself, perhaps with one endpoint fixed.

\begin{lem}\label{cor:transport-path-prepost}
  For any $A$ and $a:A$, with $p:x_1=x_2$, we have
  \begin{alignat*}{2}
    \transfib{x \mapsto (\id{a}{x})} {p} {q} &= q \ct p
    && \qquad\text{for } q:a=x_1\\
    \transfib{x \mapsto (\id{x}{a})} {p} {q} &= \opp {p} \ct q 
    && \qquad\text{for } q:x_1=a\\
    \transfib{x \mapsto (\id{x}{x})} {p} {q} &= \opp{p} \ct q \ct p
    && \qquad\text{for } q:x_1=x_1.
  \end{alignat*}
\end{lem}
\begin{proof}
  Path induction on $p$, followed by the unit laws for composition.
\end{proof}

In other words, transporting with ${x \mapsto \id{c}{x}}$ is post-composition, and transporting with ${x \mapsto \id{x}{c}}$ is contravariant pre-composition.
These may be familiar as the functorial actions of the covariant and contravariant hom-functors $\hom(c,-)$ and $\hom(-,c)$ in category theory.

Combining \autoref{cor:transport-path-prepost,thm:transport-compose}, we obtain a more general form:

\begin{thm}\label{thm:transport-path}
  For $f,g:A\to B$, with $p : \id[A]{a}{a'}$ and $q : \id[B]{f(a)}{g(a)}$, we have
  \begin{equation*}
    \id[f(a') = g(a')]{\transfib{x \mapsto \id[B]{f(x)}{g(x)}}{p}{q}}
    {\opp{(\apfunc{f}{p})} \ct q \ct \apfunc{g}{p}}.
  \end{equation*}
\end{thm}

Because $\apfunc{(x \mapsto x)}$ is the identity function and $\apfunc{(x \mapsto c)}$ (where $c$ is a constant) is \refl{c}, \autoref{cor:transport-path-prepost} is a special case.
A yet more general version is when $B$ can be a family of types indexed on $A$:

\begin{thm}\label{thm:transport-path2}
  Let $B : A \to \type$ and $f,g : \prd{x:A} B(x)$, with $p : \id[A]{a}{a'}$ and $q : \id[B(a)]{f(a)}{g(a)}$.
  Then we have
  \begin{equation*}
    \transfib{x \mapsto \id[B(x)]{f(x)}{g(x)}}{p}{q} = 
    \opp{(\apfunc{f}{p})} \ct \apdfunc{(\transfibf{A}{p})}(q) \ct \apfunc{g}{p}
  \end{equation*}
\end{thm}

Finally, as in \autoref{sec:compute-pi}, for families of identity types there is another equivalent characterization of dependent paths.

\begin{thm}\label{thm:dpath-path}
  For $p:\id[A]a{a'}$ with $q:a=a$ and $r:a'=a'$, we have
  \[ \eqvspaced{ \big(\transfib{x\mapsto (x=x)}{p}{q} = r \big) }{ \big( q \ct p = p \ct r \big) } \]
\end{thm}
\begin{proof}
  Path induction on $p$, followed by the fact that composing with the unit equalities $q\ct 1 = q$ and $r = 1\ct r$ is an equivalence.
\end{proof}

There are more general equivalences involving the application of functions, akin to \autoref{thm:transport-path,thm:transport-path2}.


\subsection{Coproducts}
\label{sec:compute-coprod}

So far, most of the type formers we have considered have been what are called \emph{negative}.
Intuitively, this means that their elements are determined by their behavior under the elimination rules: a (dependent) pair is determined by its projections, and a (dependent) function is determined by its values.
The identity types of negative types can almost always be characterized straightforwardly, along with all of their higher structure, as we have done in \crefrange{sec:compute-cartprod}{sec:compute-pi}.
The universe is not exactly a negative type, but its identity types behave similarly: we have a straightforward characterization (univalence) and a description of the higher structure.
Identity types themselves, of course, are a special case.

We now consider our first example of a \emph{positive} type former.
A positive type is one which is ``presented'' by certain constructors, with the universal property of a presentation being expressed by its elimination rule.
(Categorically speaking, a positive type has a ``mapping out'' universal property, while a negative type has a ``mapping in'' universal property.)
Because computing with presentations is, in general, an uncomputable problem, for positive types we cannot always expect a straightforward characterization of the identity type.
However, in many particular cases, a characterization or partial characterization does exist, and can be obtained by the general method we describe here and in later sections.

(Technically, our chosen presentation of cartesian products and $\Sigma$-types is also positive.
However, because these types also admit a negative presentation which differs only slightly, their identity types have a direct characterization that does not require the method to be described here.)

Consider the coproduct type $A+B$, which is ``presented'' by the injections $\inl:A\to A+B$ and $\inr:B\to A+B$.
Intuitively, we expect that $A+B$ contains exact copies of $A$ and $B$ disjointly, so that we should have
\begin{align}
  {(\inl(a_1)=\inl(a_2))}&\simeq {(a_1=a_2)} \label{eq:inlinj}\\
  {(\inr(b_1)=\inr(b_2))}&\simeq {(b_1=b_2)}\\
  {(\inl(a)= \inr(b))} &\simeq {\emptyt} \label{eq:inlrdj}
\end{align}
We prove this as follows.
Fix an element $a_0:A$; we will characterize the type family
\begin{equation}
  (x\mapsto (\inl(a_0)=x)) : A+B \to \type.\label{eq:sumcodefam}
\end{equation}
A similar argument would characterize the analogous family $x\mapsto (x = \inr(b_0))$ for any $b_0:B$.
Together, these characterizations imply~\eqref{eq:inlinj}--\eqref{eq:inlrdj}.

In order to characterize~\eqref{eq:sumcodefam}, we will define a type family $\code:A+B\to\type$ and show that $\prd{x:A+B} (\eqv{(\inl(a)=x)}{\code(x)})$.
Since we want to conclude~\eqref{eq:inlinj} from this, we should have $\code(\inl(a)) = (a_0=a)$, and since we also want to conclude~\eqref{eq:inlrdj}, we should have $\code (\inr(b)) = \emptyt$.
The essential insight is that we can use the elimination property of $A+B$ to \emph{define} $\code:A+B\to\type$ by these two equations:
\begin{align*}
  \code(\inl(a)) &\defeq (a_0=a)\\
  \code(\inr(b)) &\defeq \emptyt
\end{align*}
This is a very simple example of a proof technique that is used quite a
bit in formalizing homotopy theory in HoTT; see
e.g.\ \autoref{sec:pi1-s1-intro}.  

We can now show:

\begin{thm}\label{thm:path-coprod}
  For all $x:A+B$ we have $\eqv{(\inl(a)=x)}{\code(x)}$.
\end{thm}
\begin{proof}
  The key to the following proof is that we do it for all points $x$ together, enabling us to use the elimination principle for the coproduct.
  We first define a function
  \[ \encode : \prd{x:A+B}{p:\inl(a_0)=x} \code(x) \]
  by transporting reflexivity along $p$:
  \[ \encode(x,p) \defeq \transfib{\code}{p}{\refl{a_0}}. \]
  Note that $\refl{a_0} : \code(\inl(a_0))$, since $\code(\inl(a_0))\jdeq (a_0=a_0)$ by definition of \code.
  Next, we define a function
  \[ \decode : \prd{x:A+B}{c:\code(x)} (\inl(a_0)=x) \]
  To define $\decode(x,c)$, we may first use the elimination principle of $A+B$ to divide into cases based on whether $x$ is of the form $\inl(a)$ or the form $\inr(b)$.

  In the first case, where $x\jdeq \inl(a)$, then $\code(x)\jdeq (a_0=a)$, so that $c$ is an identification between $a_0$ and $a$.
  Thus, $\apfunc{\inl}(c):(\inl(a_0)=\inl(a))$ so we can define this to be $\decode(\inl(a),c)$.

  In the second case, where $x\jdeq \inr(b)$, then $\code(x)\jdeq \emptyt$, so that $c$ inhabits the empty type.
  Thus, the elimination rule of $\emptyt$ yields a value for $\decode(\inr(b),c)$.

  This completes the definition of \decode; we now show that $\encode(x,-)$ and $\decode(x,-)$ are quasi-inverses for all $x$.
  On the one hand, suppose given $x:A+B$ and $p:\inl(a_0)=x$; we want to show $\decode(x,\encode(x,p))=p$.
  But now by path induction (in the Paulin-Mohring style), we may assume that $x\jdeq\inl(a_0)$ and $p\jdeq \refl{\inl(a_0)}$.
  In this case we have
  \begin{align*}
    \decode(x,\encode(x,p))
    &\jdeq \decode(\inl(a_0),\encode(\inl(a_0),\refl{\inl(a_0)}))\\
    &\jdeq \decode(\inl(a_0),\transfib{\code}{\refl{\inl(a_0)}}{\refl{a_0}})\\
    &\jdeq \decode(\inl(a_0),\refl{a_0})\\
    &\jdeq \ap{\inl}{\refl{a_0}}\\
    &\jdeq \refl{\inl(a_0)}\\
    &\jdeq p.
  \end{align*}
  On the other hand, let $x:A+B$ and $c:\code(x)$; we want to show $\encode(x,\decode(x,c))=c$.
  We may again divide into cases based on $x$.
  If $x\jdeq\inl(a)$, then $c:a_0=a$ and $\decode(x,c)\jdeq \apfunc{\inl}(c)$, so that
  \begin{alignat*}{2}
    \encode(x,\decode(x,c))
    &\jdeq \transfib{\code}{\apfunc{\inl}(c)}{\refl{a_0}}\\
    &= \transfib{a\mapsto (a_0=a)}{c}{\refl{a_0}}
    &&\quad\text{by \autoref{thm:transport-compose}}\\
    &= \refl{a_0} \ct c
    &&\quad\text{by \autoref{cor:transport-path-prepost}}\\
    &= c.\qedhere
  \end{alignat*}
\end{proof}

Of course, there is a corresponding theorem if we fix $b_0:B$ instead of $a_0:A$.

In particular, for any $a:A$ we have the function
\[ \encode(a,-) : (\inl(a_0)=\inl(a)) \to (a_0=a).\]
Similarly, for any $b:B$ we have
\[ \encode(b,-) : (\inl(a_0)=\inr(b)) \to \emptyt \] which can be stated as
``$\inl(a_0)$ is not equal to $\inr(b)$'', i.e.\ the images of \inl and \inr are
disjoint.  Traditionally, with identity types viewed as propositions, these
results may be stated as ``$\inl$ is injective'' and similarly for $\inr$.  The
homotopical version gives more information: the types $\inl(a_0)=\inl(a)$ and
$a_0=a$ are actually equivalent, as are $\inr(a_0)=\inr(a)$ and $a_0=a$.

\begin{rmk}\label{rmk:true-neq-false}
  In particular, since the two-element type $\bool$ is equivalent to $\unit+\unit$, we have $\btrue\neq\bfalse$.
\end{rmk}

As usual, we can also characterize the action of transport in coproduct types.
Given a type $X$, a path $p:\id[X]{x_1}{x_2}$, and type families $A,B:X\to\type$, we have
\begin{align*}
  \transfib{A+B}{p}{\inl(a)} &= \inl (\transfib{A}{p}{a})\\
  \transfib{A+B}{p}{\inr(b)} &= \inr (\transfib{B}{p}{b})
\end{align*}
where as usual, $A+B$ denotes abusively the type family $x\mapsto A(x)+B(x)$.
The proof is easy using path induction.


\subsection{Natural numbers}
\label{sec:compute-nat}

The same technique we used for coproducts applies to the natural numbers, which are also a positive type.
In this case the codes for identities are a type family
\[\code:\N\to\N\to\type,\]
defined by double recursion over \N as follows:
\begin{align*}
  \code(0,0) &= \unit\\
  \code(\suc(m),0) &= \emptyt\\
  \code(0,\suc(n)) &= \emptyt\\
  \code(\suc(m),\suc(n)) &= \code(m,n).
\end{align*}
We also define by recursion a dependent function $r:\prd{n:\N} \code(n,n)$, with
\begin{align*}
  r(0) &= \ttt\\
  r(\suc(n)) &= r(n).
\end{align*}

\begin{thm}\label{thm:path-nat}
  For all $m,n:\N$ we have $\eqv{(m=n)}{\code(m,n)}$.
\end{thm}
\begin{proof}
  We define
  \[ \encode : \prd{m,n:\N} (m=n) \to \code(m,n) \]
  by transporting, $\encode(m,n,p) \defeq \transfib{\code(m,-)}{p}{r(m)}$.
  And we define
  \[ \decode : \prd{m,n:\N} \code(m,n) \to (m=n) \]
  by double induction on $m,n$.
  When $m$ and $n$ are both $0$, we need a function $\unit \to (0=0)$, which we define to send everything to $\refl{0}$.
  When $m$ is a successor and $n$ is $0$ or vice versa, the domain $\code(m,n)$ is \emptyt, so the eliminator for \emptyt suffices.
  And when both are successors, we can define $\decode(\suc(m),\suc(n))$ to be the composite
  \[ \code(\suc(m),\suc(n))\jdeq\code(m,n) \xrightarrow{\decode(m,n)} (m=n) \xrightarrow{\apfunc{\suc}} (\suc(m)=\suc(n)). \]
  Next we show that $\encode(m,n)$ and $\decode(m,n)$ are quasi-inverses for all $m,n$.

  On one hand, if we start with $p:m=n$, then by induction on $p$ it suffices to show $\decode(n,n,\encode(n,n,\refl{n}))=\refl{n}$.
  But $\encode(n,n,\refl{n}) \jdeq r(n)$, so it suffices to show that $\decode(n,n,r(n)) =\refl{n}$.
  We can prove this by induction on $n$.
  If $n\jdeq 0$, then $\decode(0,0,r(0)) =\refl{0}$ by definition of \decode.
  And in the case of a successor, by the inductive hypothesis we have $\decode(n,n,r(n)) = \refl{n}$, so it suffices to observe that $\apfunc{\suc}(\refl{n}) \jdeq \refl{\suc(n)}$.

  On the other hand, if we start with $c:\code(m,n)$, then we proceed by double induction on $m$ and $n$.
  If both are $0$, then $\decode(0,0,c) \jdeq \refl{0}$, while $\encode(0,0,\refl{0})\jdeq r(0) \jdeq \ttt$.
  Thus, it suffices to recall from \autoref{sec:compute-unit} that every inhabitant of $\unit$ is equal to \ttt.
  If $m$ is $0$ but $n$ is a successor, or vice versa, then $c:\emptyt$, so we are done.
  And in the case of two successors, we have
  \begin{multline*}
    \encode(\suc(m),\suc(n),\decode(\suc(m),\suc(n),c))\\
    \begin{split}
    &= \encode(\suc(m),\suc(n),\apfunc{\suc}(\decode(m,n,c)))\\
    &= \transfib{\code(\suc(m),-)}{\apfunc{\suc}(\decode(m,n,c))}{r(\suc(m))}\\
    &= \transfib{\code(\suc(m),\suc(-))}{\decode(m,n,c)}{r(\suc(m))}\\
    &= \transfib{\code(m,-)}{\decode(m,n,c)}{r(m)}\\
    &= \encode(m,n,\decode(m,n,c))\\
    &= c
  \end{split}
  \end{multline*}
  using the inductive hypothesis.
\end{proof}

In particular, we have
\begin{equation}
  \encode(\suc(m),0) : (\suc(m)=0) \to \emptyt\label{eq:zero-not-succ}
\end{equation}
which shows that ``$0$ is not the successor of any natural number''.
We also have the composite
\begin{equation}
  (\suc(m)=\suc(n)) \xrightarrow{\encode} \code(\suc(m),\suc(n)) \jdeq \code(m,n) \xrightarrow{\decode} (m=n)\label{eq:suc-injective}
\end{equation}
which shows that the function $\suc$ is injective.

We will study more general positive types in Chapters~\ref{cha:induction} and~\ref{cha:hits}.
In \autoref{cha:homotopy} we will see that the same technique used here to characterize the identity types of coproducts and \nat can also be used to calculate homotopy groups of spheres.

% \subsection{Higher Inductives}
% \label{sec:compute-hits}

% \newcommand{\sone}{\mathsf{S^1}}

% Consider a higher inductive type such as $\sone$.  The definition of the
% higher inductive type does not immediately characterize
% \id[\sone]{x}{y}---which is good, because the calculation of higher
% homotopy groups can be a significant theorem, so we don't want it to be
% baked into the definitions.  However, we will often be able to prove a
% theorem characterizing the loop space, which follows the above form.
% For example, the proof in \autoref{cha:homotopy} that the fundamental
% group of the circle is the integers plays this role:

% \begin{itemize}
% \item An introduction rule for \id[\sone]{\mathsf{base}}{\mathsf{base}}:
%   \[
%   \mathsf{loopToThe} : \mathbb{Z} \to \id{\mathsf{base}}{\mathsf{base}}
%   \]
% \item An elimination rule:
%   \[
%   \mathsf{encode} : \id{\mathsf{base}}{\mathsf{base}} \to \mathbb{Z}
%   \]
% \item With $\beta$ and $\eta$ rules stating that these are mutually inverse.
% \end{itemize}

% It's less clear that you want to think about identity, inverses, and
% composition as being defined through this encoding (rather than thinking
% of them as constructors), but you can:

% \[
% \begin{array}{l}
% \refl{\mathsf{base}} = \mathsf{loopToThe} \: 0 \\
% \opp{\alpha} = \mathsf{loopToThe} \: (- (\mathsf{encode} \: \alpha)) \\
% \alpha \ct \beta = \mathsf{loopToThe} \: ((\mathsf{encode} \: \alpha) + (\mathsf{encode} \: \beta)) \\
% \end{array}
% \]

% This changes the representation of the group structure
% from the identity type to an explicit representation, as the free group
% on one generator (the additive group on the integers).  

% FIXME: say something about map



% Local Variables:
% TeX-master: "main"
% End:



\section{Universal properties}
\label{sec:universal-properties}

By combining the path computation rules described in \S\ref{sec:computational}, we can show that various type forming operations satisfy the expected universal properties.
For instance, given types $X,A,B$, we have a function
\begin{equation}
  (X\to A\times B) \;\to \; (X\to A)\times (X\to B)\label{eq:prod-ump-map}
\end{equation}
defined by $f \mapsto (\proj1 \circ f, \proj2\circ f)$.

\begin{thm}\label{thm:prod-ump}
  \eqref{eq:prod-ump-map} is an equivalence.
\end{thm}
\begin{proof}
  We define a quasi-inverse to send $(g,h)$ to the function $x\mapsto (g(x),h(x))$.
  (Technically, we have used the induction principle for the cartesian product $(X\to A)\times (X\to B)$, to reduce to the case of a pair.)

  Now given $f:X\to A\times B$, the round-trip composite yields the function
  \begin{equation}
    x\mapsto (\proj1(f(x)),\proj2(f(x))).\label{eq:prod-ump-rt1}
  \end{equation}
  By \autoref{thm:path-prod}, for any $x:X$ we have $(\proj1(f(x)),\proj2(f(x))) = f(x)$.
  Thus, by function extensionality, the function~\eqref{eq:prod-ump-rt1} is equal to $f$.

  On the other hand, given $(g,h)$, the round-trip composite yields the pair $(x\mapsto g(x),x\mapsto h(x))$.
  By function extensionaility, the two components of this are equal to $g$ and $h$ respectively, so by \autoref{thm:path-prod}, the pair is equal to $(g,h)$.
\end{proof}

In fact, we also have a dependently typed version of this universal property.
Suppose given a type $X$ and type families $A,B:X\to \type$.
Then we have a function
\begin{equation}\label{eq:prod-umpd-map}
  \Big(\prd{x:X} (A(x)\times B(x))\Big) \;\to\; \Big(\prd{x:X} A(x)\Big) \times \Big(\prd{x:X} B(x)\Big)
\end{equation}
defined as before by $f \mapsto (\proj1 \circ f, \proj2\circ f)$.

\begin{thm}\label{thm:prod-umpd}
  \eqref{eq:prod-umpd-map} is an equivalence.
\end{thm}
\begin{proof}
  Left to the reader.
\end{proof}

Just as $\Sigma$-types are a generalization of cartesian products, they satisfy a generalized version of this universal property.
Jumping right to the dependently typed version, suppose we have a type $X$ and type families $A:X\to \type$ and $P:\prd{x:X} A(x)\to\type$.
Then we have a function
\begin{multline}
  \label{eq:sigma-ump-map}
  \Big(\prd{x:X} \textstyle\sum_{a:A(x)} P(x,a)\Big) \\ \;\to\;
  \Big(\sm{g:\textstyle\prod_{x:X} A(x)} \textstyle\prod_{x:X} P(x,g(x))\Big)
\end{multline}
Note that if we have $P(x,a) \defeq B(x)$ for some $B:X\to\type$, then~\eqref{eq:sigma-ump-map} reduces to~\eqref{eq:prod-umpd-map}.

\begin{thm}\label{thm:ttac}
  \eqref{eq:sigma-ump-map} is an equivalence.
\end{thm}
\begin{proof}
  As before, we define a quasi-inverse to send $(g,h)$ to the function $x\mapsto (g(x),h(x))$.
  Now given $f:\prd{x:X} \sm{a:A(x)} P(x,a)$, the round-trip composite yields the function
  \begin{equation}
    x\mapsto (\proj1(f(x)),\proj2(f(x))).\label{eq:prod-ump-rt1}
  \end{equation}
  Now for any $x:X$, by \autoref{thm:eta-sigma} ($\eta$-equivalence for $\Sigma$-types) we have $(\proj1(f(x)),\proj2(f(x))) = f(x)$.
  Thus, by function extensionality,~\eqref{eq:prod-ump-rt1} is equal to $f$.

  On the other hand, given $(g,h)$, the round-trip composite yields the pair $(x\mapsto g(x),x\mapsto h(x))$.
  But $x\mapsto g(x)$ and $x\mapsto h(x)$ are judgmentally equal to $g$ and $h$, respectively, and hence this pair of functions is also equal to $(g,h)$.
\end{proof}

This is noteworthy because the propositions-as-types interpretation of~\eqref{eq:sigma-ump-map} is ``the axiom of choice''.
If we read $\Sigma$ as ``there exists'' and $\Pi$ (sometimes) as ``for all'', we can pronounce:
\begin{itemize}
\item $\prd{x:X} \sm{a:A(x)} P(x,a)$ as ``for all $x:X$ there exists an $a:A(x)$ such that $P(x,a)$'', and
\item $\sm{g:\prd{x:X} A(x)} \prd{x:X} P(x,g(x))$ as ``there exists a choice function $g:\prd{x:X} A(x)$ such that for all $x:X$ we have $P(x,g(x))$''.
\end{itemize}
Thus, \autoref{thm:ttac} says that not only is the axiom of choice ``true'', it hypotheses are equivalent to its conclusion.
(On the other hand, it should also be clear to the classical mathematician that~\eqref{eq:sigma-ump-map} does not carry the intended meaning of the axiom of choice, since we have specified the values of $g$ already and there are no choices left to be made.
We will return to this point in Chapter~\ref{cha:logic}.

The above universal property for pair types is for ``mapping in'', which is familiar from the category-theoretic notion of products.
However, pair types also have a universal property for ``mapping out'', which may look less familiar.
In the case of cartesian products, the non-dependent version simply expresses the cartesian closedness adjunction:
\[ (A\times B) \to C \;\simeq\; A\to (B\to C).\]
The dependent version of this is formulated for a type family $C:A\times B\to \type$:
\[ \prd{w:A\times B} C(w) \;\simeq\; \prd{x:A}{y:B} C(x,y). \]
Here the left-to-right function is simply the induction principle for $A\times B$, while the right-to-left is evaluation at a pair.
We leave it to the reader to prove that these are quasi-inverses.
There is also a version for $\Sigma$-types:
\begin{equation}
  \prd{w:\textstyle\sum_{x:A} B(x)} C(w) \;\simeq\; \prd{x:A}{y:B(x)} C(x,y)\label{eq:sigma-lump}
\end{equation}
Again, the left-to-right function is the induction principle.

In fact, basically every induction principles in type theory is part of a universal property of this sort.
For instance, the case of identity types gives
\begin{equation}
  \label{eq:path-lump}
  \prd{x:A}{p:a=x} B(x,p) \;\simeq\; B(a,\refl a)
\end{equation}
for any $a:A$ and type family $B:\prd{x:A} (a=x) \to\type$.


\section{Sets}
\label{sec:basics-sets}

While types in general behave like higher groupoids, there is a subclass of them that behave more like the sets in a traditional set-theoretic system.
Categorically, we may consider \emph{discrete} groupoids, which are determined by a set of objects and only identity morphisms and higher morphisms, while topologically we may consider sets with the discrete topology.
More generally, we may consider groupoids or spaces that are \emph{equivalent} to ones of this sort; since everything we do in type theory is up to homotopy, we can't expect to tell the difference.

Intuitively, we would expect a type to ``be a set'' in this sense if it has no higher homotopy information: any two parallel paths are equal (up to homotopy), and similarly for parallel higher paths at all dimensions.
Fortunately, because everything in homotopy type theory is automatically functorial/continuous, it turns out to be sufficient to ask this at the bottom level.

\begin{defn}\label{defn:set}
  A type $A$ is a \textbf{set} if for all $x,y:A$ and all $p,q:x=y$, we have $p=q$.
\end{defn}

More precisely, the proposition $\isset(A)$ is defined to be the type
\[ \isset(A) \defeq \prd{x,y:A}{p,q:x=y} (p=q). \]
In Chapter~\ref{cha:hlevels} we will make precise the sense in which this ``suffices for all higher levels'', but as an example, we observe that it suffices for the next level up.

\begin{lem}\label{thm:isset-is1type}
  If $A$ is a set (that is, $\isset(A)$ is inhabited), then for any $x,y:A$ and $p,q:x=y$ and $r,s:p=q$, we have $r=s$.
\end{lem}
\begin{proof}
  Suppose $f:\isset(A)$; then for any $x,y:A$ and $p,q:x=y$ we have $f(x,y,p,q):p=q$.
  Fix $x$, $y$, and $p$, and define $g: \prd{q:x=y} (p=q)$ by $g(q) \defeq f (x,y,p,q)$.
  Then for any $r:q=q'$, we have $\apdfunc{g}(r) : \trans{r}{g(q)} = g(q')$.
  By \autoref{thm:transport-paths}, therefore, we have $g(q) \ct r = g(q')$.

  In particular, suppose given $x,y,p,q,r,s$ as in the lemma statement, and define $g$ as above.
  Then $g(p) \ct r = g(q)$ and also $g(p) \ct s = g(q)$, hence by cancellation we have $r=s$.
\end{proof}

As mentioned in \S\ref{sec:types-vs-sets},
the sets in homotopy type theory are not like the sets in ZF set theory, in that there is no global ``membership predicate'' $\in$.
They are more like the sets used in structural mathematics and in category theory, whose elements are ``abstract points'' to which we give structure with functions and relations.
This is all we need in order to use them as a foundational system for most set-based mathematics; we will see some examples in Chapter~\ref{cha:set-math}.

Which types are sets?
In Chapter~\ref{cha:hlevels} we will study a more general form of this question in depth, but for now we can observe some easy examples.

\begin{eg}
  The type \unit is a set.
  For by \autoref{thm:path-unit}, for any $x,y:\unit$ we have $\eqv{(x=y)}{\unit}$.
  Since any two elements of \unit are equal, this implies that any two elements of $x=y$ are equal.
\end{eg}

\begin{eg}
  The type $\emptyset$ is a set, for given any $x,y:\emptyset$ we may deduce anything we like by contradiction.
\end{eg}

\begin{eg}
  The type \nat of natural numbers is also a set.
  This follows from \autoref{thm:path-nat}, since all equality types $\id[\nat]xy$ are equivalent to either \unit or \emptyt, and any two inhabitants of \unit or \emptyt are equal.
  We will see another proof of this fact in Chapter~\ref{cha:hlevels}.
\end{eg}

Most of the type forming operations we have considered so far also preserve sets.

\begin{eg}\label{thm:isset-prod}
  If $A$ and $B$ are sets, then so is $A\times B$.
  For given $x,y:A\times B$ and $p,q:x=y$, by \autoref{thm:path-prod} we have $p= \pairpath(\projpath1(p),\projpath2(p))$ and $q= \pairpath(\projpath1(q),\projpath2(q))$.
  But $\projpath1(p)=\projpath1(q)$ since $A$ is a set, and $\projpath2(p)=\projpath2(q)$ since $B$ is a set; hence $p=q$.
\end{eg}

\begin{eg}\label{thm:isset-forall}
  If $A$ is \emph{any} type and $B:A\to \type$ is such that for each $x:A$, the type $B(x)$ is a set, then the type $\prd{x:A} B(x)$ is a set.
  For suppose $f,g:\prd{x:A} B(x)$ and $p,q:f=g$.
  By function extensionality, we have $p = {\funext (x \mapsto \happly(p,x))}$ and likewise $q = {\funext (x \mapsto \happly(q,x))}$.
  But for any $x:A$, we have $\happly(p,x):f(x)=g(x)$ and also $\happly(q,x):f(x)=g(x)$, so since $B(x)$ is a set we have $\happly(p,x) = \happly(q,x)$.
  Now using function extensionality again, we conclude that the dependent functions $(x \mapsto \happly(p,x))$ and $(x \mapsto \happly(q,x))$ are equal, and hence (applying $\apfunc{\funext}$) so are $p$ and $q$.
\end{eg}

For more examples, see Exercises~\ref{ex:isset-coprod} and~\ref{ex:isset-sigma}.
However, not all types are sets.

\begin{eg}\label{thm:type-is-not-a-set}
  The universe \type is not a set.
  To prove this, it suffices to exhibit a type $A$ and a path $p:A=A$ which is not equal to $\refl A$.
  Take $A=\unit+\unit$, and let $f:A\to A$ be defined by $f(\inl(\ttt))\defeq \inr(\ttt)$ and $f(\inr(\ttt))\defeq \inl(\ttt)$.
  Then $f(f(x))=x$ for all $x$ (by an easy case analysis), so $f$ is an equivalence.
  Hence, by univalence, $f$ gives rise to a path $p:A=A$.

  If $p$ were equal to $\refl A$, then (again by univalence) $f$ would equal the identity function of $A$.
  But this would imply that $\inl(\ttt)=\inr(\ttt)$, contradicting the disjointness of injections from \S\ref{sec:compute-coprod}.
\end{eg}

We will study other types that are not sets in more detail starting in Chapter~\ref{cha:hits}.  An even more special kind of type than a set is a set with at most one element --- such types may be regarded as \emph{propositions in a narrow sense}, and their study is just what is usually called "logic".

%%%%%%%%%%%%%%%%%%%%%%%%%%%
\section{Logic}
\label{sec:logic}

Type theory, formal or informal, is a collection of rules for manipulating types and their elements.
But when writing mathematics informally in human language, we generally use familiar words, particularly logical connectives such as ``and'' and ``or'', and logical quantifiers such as ``for all'' and ``there exists''.
In contrast to set theory, type theory offers us more than one choice for how to regard these English phrases as operations on types.
This potential ambiguity needs to be resolved, by setting out local or global conventions, by introducing new annotations to informal mathematics, or both.
This requires some getting used to, but is offset by the fact that because type theory permits this finer analysis of logic, we can represent mathematics more faithfully, with fewer ``abuses of language'' than in set-theoretic foundations.
In this section we will explain the issues involved, and justify the choices we have made.


\subsection{Propositions as types?}
\label{subsec:pat?}

Until now, we have been following the straightforward ``propositions as types'' philosophy described in \S\ref{sec:pat}, according to which English phrases such as ``there exists an $x:A$ such that $P(x)$'' are interpreted by corresponding types such as $\sm{x:A} P(x)$, with the proof of a statement being regarded as judging some specific term to inhabit that type.
However, we have also seen some ways in which the ``logic'' resulting from this reading seems unfamiliar to a classical mathematician.
For instance, in \autoref{thm:ttac} we saw that the statement
\begin{quote}
  ``If for all $x:X$ there exists an $a:A(x)$ such that $P(x,a)$, then there exists a function $g:\prd{x:A} A(x)$ such that for all $x:X$ we have $P(x,g(x))$,''
\end{quote}
which looks like the classical \emph{axiom of choice}, is always true under this reading.  That's a good thing, but rather unexpected, since the axiom of choice is not generally thought to be a law of logic.

On the other hand, we can now show that corresponding statements looking like the classical \emph{law of double negation} and \emph{law of excluded middle} are incompatible with the univalence axiom.

\begin{thm}\label{thm:not-dneg}
  It is not the case that for all $A:\UU$ we have $\neg(\neg A) \to A$.
\end{thm}
\begin{proof}
  Recall that $\neg A \jdeq (A\to\emptyt)$.
  We also read ``it is not the case that \dots'' as the operator $\neg$.
  Thus, in order to prove this statement, it suffices to assume given some $f:\prd{A:\UU} (\neg\neg A \to A)$ and construct an element of \emptyt.

  The idea of the following proof is to observe that $f$, like any function, is automatically ``continuous'' with respect to paths in its domain.
  By univalence, this implies that $f$ is \emph{natural} with respect to equivalences of types.
  From this, and a fixed-point-free autoequivalence, we will be able to extract a contradiction.

  Recall that $\bool\defeq \unit+\unit$, and let $e:\eqv\bool\bool$ be the equivalence defined by $e(\inl(\ttt))\defeq\inr(\ttt)$ and $e(\inr(\ttt))\defeq\inl(\ttt)$, as in \autoref{thm:type-is-not-a-set}.
  Let $p:\bool=\bool$ be the path corresponding to $e$ by univalence, i.e.\ $p\defeq \ua(e)$.
  Then we have $f(\bool) : \neg\neg\bool \to\bool$ and
  \[\apd f p : \transfib{A\mapsto (\neg\neg A \to A)}{p}{f(\bool)} = f(\bool).\]
  Hence, for any $u:\neg\neg\bool$, we have
  \[\happly(\apd f p,u) : \transfib{A\mapsto (\neg\neg A \to A)}{p}{f(\bool)}(x) = f(\bool)(u).\]

  Now by~\eqref{eq:transport-arrow}, transporting $f(\bool):\neg\neg\bool\to\bool$ along $p$ in the type family ${A\mapsto (\neg\neg A \to A)}$ is equal to the function which transports its argument along $\opp p$ in the type family $A\mapsto \neg\neg A$, applies $f(\bool)$, then transports the result along $p$ in the type family $A\mapsto A$:
  \[ \transfib{A\mapsto (\neg\neg A \to A)}{p}{f(\bool)}(u) =
  \transfib{A\mapsto A}{p}{f(\bool) (\transfib{A\mapsto \neg\neg A}{\opp{p}}{u})}
  \]
  However, any two points $u,v:\neg\neg\bool$ are equal by function extensionality, since for any $x:\neg\bool$ we have $u(x):\emptyt$ and thus we can derive any conclusion, in particular $u(x)=v(x)$.
  Thus, we have $\transfib{A\mapsto \neg\neg A}{\opp{p}}{u} = u$, and so from $\happly(\apd f p,u)$ we obtain an equality
  \[ \transfib{A\mapsto A}{p}{f(\bool)(u)} = f(\bool)(u).\]
  Finally, as discussed in \S\ref{sec:compute-universe}, transporting in the type family $A\mapsto A$ along the path $p\jdeq \ua(e)$ is equivalent to applying the equivalence $e$; thus we have
  \begin{equation}
    e(f(\bool)(u)) = f(\bool)(u).\label{eq:fpaut}
  \end{equation}

  However, we can also prove that
  \begin{equation}
    \prd{x:\bool} \neg(e(x)=x).\label{eq:fpfaut}
  \end{equation}
  This follows from a case analysis on $x$: both cases are immediate from the definition of $e$ and the injectivity of $\inl$ and $\inr$ which we proved in \S\ref{sec:compute-coprod}.
  Thus, applying~\eqref{eq:fpfaut} to $f(\bool)(u)$ and~\eqref{eq:fpaut}, we obtain an element of $\emptyt$.
\end{proof}

\begin{cor}\label{thm:not-lem}
  It is not the case that for all $A:\UU$ we have $A+(\neg A)$.
\end{cor}
\begin{proof}
  Suppose we had $g:\prd{A:\UU} (A+(\neg A))$.
  We will show that then $\prd{A:\UU} (\neg\neg A \to A)$, so that we can apply \autoref{thm:not-dneg}.
  Thus, suppose $A:\UU$ and $u:\neg\neg A$; we want to construct an element of $A$.

  Now $g(A):A+(\neg A)$, so by case analysis, we may assume either $g(A)\jdeq \inl(a)$ for some $a:A$, or $g(A)\jdeq \inr(w)$ for some $w:\neg A$.
  In the first case, we have $a:A$, while in the second case we have $u(w):\emptyt$ and so we can obtain anything we wish (such as $A$).
  Thus, in both cases we have an element of $A$, as desired.
\end{proof}

In conclusion, although the propositions-as-types logic has many good properties (such as simplicity, constructivity, and computability), it leads to some conclusions that are hard to reconcile with conventional mathematical reasoning, and especially our homotopical point of view.  The combination of the axiom of choice with the negated law of excluded middle, for example, makes for a rather unusual ``logic".

\subsection{Mere propositions}
\label{subsec:hprops}

The good and bad things about propositions-as-types logic have a common cause: when types are viewed as propositions, they can contain more information than mere truth or falsity, and all ``logical'' constructions on them must respect this additional information.
This suggests that we could obtain a more conventional logic by restricting attention to types that do \emph{not} contain any more information than truth or falsity, and only regarding these as logical propositions,.

Such a type will be ``true'' if it is inhabited and ``false'' if it is not inhabited.
We want to avoid treating as logical propositions those types for which giving an element of them gives more information than simply knowing that the type is inhabited.
For instance, if we are given an element of \bool, then we receive more information than the mere fact that \bool contains some element.
Indeed, we receive exactly \emph{one bit} more information: we know \emph{which} element of \bool we were given.
By contrast, if we are given an element of \unit, then we receive no more information than the mere fact that \unit contains an element, since any two elements of \unit are equal to each other.
This suggests the following definition.

\begin{defn}
  A type $P$ is a \textbf{mere proposition} if for all $x,y:P$ we have $x=y$.
\end{defn}

Note that since we are still doing mathematics \emph{in} type theory, this is a definition \emph{in} type theory, which means it is a type --- or, rather, a type family.
Specifically, for any $P:\type$, the type $\isprop(P)$ is defined to be
\[ \isprop(P) \defeq \prd{x,y:P} (x=y). \]
Thus, to assert that ``$P$ is a mere proposition'' means to exhibit an inhabitant of $\isprop(P)$, which is a dependent function connecting any two elements of $P$ by a path.
The continuity/naturality of this function implies that not only are any two elements of $P$ equal, but $P$ contains no higher homotopy either.

\begin{lem}\label{thm:inhabprop-eqvunit}
  If $P$ is a mere proposition and $x_0:P$, then $\eqv P \unit$.
\end{lem}
\begin{proof}
  Define $f:P\to\unit$ by $f(x)\defeq \ttt$, and $g:\unit\to P$ by $g(u)\defeq x_0$.
  Then for any $u:\unit$ we have $\eqv{(f(g(u))=u)}{\unit}$, hence $f(g(u))=u$, while for any $x:P$ we have $g(f(x))=x$ since $P$ is a mere proposition.
\end{proof}

In homotopy theory, a space that is homotopy equivalent to \unit is said to be \emph{contractible}.
Thus, any mere proposition which is inhabited is contractible (see also \S\ref{sec:contractibility}).
On the other hand, the uninhabited type \emptyt is also (vacuously) a mere proposition.
In classical mathematics, at least, these are the only two possibilities.

Mere propositions are also called \emph{subterminal objects} (if thinking categorically), \emph{subsingletons} (if thinking set-theoretically), or \emph{h-propositions}.
In Chapter~\ref{cha:hlevels} we will learn to also call them \emph{$(-1)$-truncated types}.
The adjective ``mere'' emphasizes that although any type may be regarded as a proposition (which we prove by giving an inhabitant of it), a type that is a mere proposition cannot usefully be regarded as any \emph{more} than a proposition: there is no additional information contained in a witness of its truth.

Note that a type $A$ is a set if and only if for all $x,y:A$, the identity type $\id[A]xy$ is a mere proposition.
On the other hand, by copying (and simplifying) the proof of \autoref{thm:isset-is1type}, we see that every mere proposition is a set.
We have seen one other example so far: condition~\ref{item:be3} in \S\ref{sec:basics-equivalences} asserts that for any function $f$, the type $\isequiv (f)$ should be a mere proposition.


We can now give the proper formulation of the \emph{law of excluded middle} in homotopy type theory:
\begin{equation}
  \label{eq:lem}
  \mathsf{LEM}\;\defeq\;
  \prd{A:\UU} \Big(\isprop(A) \to (A + \neg A)\Big).
\end{equation}
Similarly, the \emph{law of double negation} is
\begin{equation}
  \label{eq:ldn}
  \mathsf{DN}\;\defeq\;
  \prd{A:\UU} \Big(\isprop(A) \to (\neg\neg A \to A)\Big).
\end{equation}
These formulations avoid the paradoxes of \autoref{thm:not-dneg} and \autoref{thm:not-lem}, since \bool is not a mere proposition.
Although they are not consequences of the basic type theory described in Chapter~\ref{cha:typetheory}, they may be consistently assumed as axioms.
For instance, we will assume them in \S\ref{sec:wellorderings}.

However, it can be surprising how far we can get without using such axioms.
Quite often, a simple reformulation of a definition or theorem enables us to avoid invoking excluded middle or double negation.
This is even more pronounced in \emph{homotopy} type theory.
For instance, none of the homotopy theory we will develop in Chapter~\ref{cha:homotopy} requires LEM or DN, despite the fact that classical homotopy theory (formulated using topological spaces or simplicial sets) makes heavy use of them (as well as the axiom of choice).


\subsection{Subsets}
\label{subsec:prop-subsets}

As another example of the usefulness of mere propositions, we discuss subsets (and more generally subtypes).
Suppose $P:A\to\type$ is a type family, with each type $P(x)$ regarded as a proposition.
Then $P$ itself is a \emph{predicate} on $A$, or a \emph{property} of elements of $A$.

In set theory, whenever we have a predicate on $P$ on a set $A$, we may form the subset $\setof{x\in A | P(x)}$.
In type theory, the obvious analogue is the $\Sigma$-type $\sm{x:A} P(x)$.
An inhabitant of $\sm{x:A} P(x)$ is, of course, a pair $(x,p)$ where $x:A$ and $p$ is a proof of $P(x)$.
However, for general $P$, an element $a:A$ might give rise to more than one distinct element of $\sm{x:A} P(x)$, if the proposition $P(a)$ has more than one distinct proof.
This is counter to the usual intuition of a \emph{subset}.
But if $P$ is a \emph{mere} proposition, then this cannot happen.

\begin{lem}
  Suppose $P:A\to\type$ is a type family such that $P(x)$ is a mere proposition for all $x:A$.
  If $u,v:\sm{x:A} P(x)$ are such that $\proj1(u) = \proj1(v)$, then $u=v$.
\end{lem}
\begin{proof}
  Suppose $p:\proj1(u) = \proj1(v)$.
  By \autoref{thm:path-sigma}, to show $u=v$ it suffices to show $\trans{p}{\proj2(u)} = \proj2(v)$.
  But $\trans{p}{\proj2(u)}$ and $\proj2(v)$ are both elements of $P(\proj1(v))$, which is a mere proposition; hence they are equal.
\end{proof}

For instance, recall that in \S\ref{sec:basics-equivalences} we defined
\[(\eqv A B) \;\defeq\; \sm{f:A\to B} \isequiv (f),\]
and we showed that each type $\isequiv (f)$ is a mere proposition.
It follows that if two equivalences have equal underlying functions, then they are equal as equivalences.

Henceforth, if $P:A\to \type$ is a family of mere propositions, we will allow ourselves to write $\setof{x:A | P(x)}$ as an alternative notation for $\sm{x:A} P(x)$.


\subsection{The logic of mere propositions}
\label{subsec:logic-hprop}

We mentioned in \S\ref{sec:types-vs-sets} that in contrast to type theory, which has only one basic notion (types), set-theoretic foundations have two basic notions: sets and propositions.
Thus, a classical mathematician is accustomed to manipulating these two kinds of objects separately.

It is possible to recover a similar dichotomy in type theory, with the role of the set-theoretic propositions being played by the types (and type families) that are \emph{mere} propositions.
In many cases, the logical connectives and quantifiers can be represented in this logic by simply restricting the corresponding type-former to the mere propositions.
Of course, this requires knowing that the type-former in question preserves mere propositions.

\begin{eg}
  If $A$ and $B$ are mere propositions, so is $A\times B$.
  This is easy to show using the characterization of paths in products, just like \autoref{thm:isset-prod} but simpler.
  Thus, the connective ``and'' preserves mere propositions.
\end{eg}

\begin{eg}\label{thm:isprop-forall}
  If $A$ is any type and $B:A\to \type$ is such that for all $x:A$, the type $B(x)$ is a mere proposition, then $\prd{x:A} B(x)$ is a mere proposition.
  The proof is just like \autoref{thm:isset-forall} but simpler: given $f,g:\prd{x:A} B(x)$, for any $x:A$ we have $f(x)=g(x)$ since $B(x)$ is a mere proposition.
  But then by function extensionality, we have $f=g$.

  In particular, if $B$ is a mere proposition, then so is $A\to B$ regardless of what $A$ is.
  In even more particular, since \emptyt is a mere proposition, so is $\neg A \jdeq (A\to\emptyt)$.
  Thus, the connectives ``implies'' and ``not'' preserve mere propositions, as does the quantifier ``for all''.
\end{eg}

On the other hand, some type formers do not preserve mere propositions.
Even if $A$ and $B$ are mere propositions, $A+B$ will not in general be.
For instance, \unit is a mere proposition, but $\bool\defeq\unit+\unit$ is not.
Logically speaking, $A+B$ is a ``purely constructive'' sort of ``or'': a witness of it contains the additional information of \emph{which} disjunct is true.
Sometimes this is very useful, but if we want a more classical sort of ``or'' that preserves mere propositions, we need a way to ``truncate'' this type into a mere proposition by forgetting this additional information.

The same issue arises with the $\Sigma$-type $\sm{x:A} P(x)$.
This is a purely constructive interpretation of ``there exists an $x:A$ such that $P(x)$'' which remembers the witness $x$, and hence is not generally a mere proposition even if each type $P(x)$ is.
(Recall that we observed in \S\ref{sec:prop-subsets} that $\sm{x:A} P(x)$ can also be regarded as ``the subset of those $x:A$ such that $P(x)$''.
The tension between these two interpretations is exactly the point.)


\subsection{Propositional truncation}
\label{subsec:prop-trunc}

The \emph{propositional truncation}, also called the \emph{$(-1)$-truncation}, \emph{bracket type}, or \emph{squash type}, is an additional type former which ``truncates'' or ``squashes'' a type down to a mere proposition, forgetting all information contained in inhabitants of that type other than their existence.

More precisely, for any type $A$, there is a type $\brck{A}$.
It has two constructors:
\begin{itemize}
\item For any $a:A$ we have $\bproj a : \brck A$.
  Thus, if $A$ is inhabited, so is $\brck A$.
\item For any $x,y:\brck A$, we have $x=y$.
  In other words, $\brck A$ is a mere proposition; usually we leave the witness of this fact nameless.
\end{itemize}
The induction principle of $\brck A$ says that:
\begin{itemize}
\item If $B$ is a mere proposition and we have $f:A\to B$, then there is an induced $g:\brck A \to B$ such that $g(\bproj a) \jdeq f(a)$ for all $a:A$.
\end{itemize}
Thus, $\brck A$, as a mere proposition, contains no more information than the inhabitedness of $A$, since any mere proposition which follows from the inhabitedness of $A$ already follows from $\brck A$.

With the propositional truncation, we can extend the ``logic of mere propositions'' to cover disjunction and the existential quantifier.
Specifically, $\brck{A+B}$ is a mere propositional version of ``$A$ or $B$'', which does not ``remember'' the information of which disjunct is true.

The induction principle of truncation implies that we can still do a case analysis on $\brck{A+B}$ \emph{when attempting to prove a mere proposition}.
That is, suppose we have an assumption $u:\brck{A+B}$ and we are trying to prove a mere proposition $Q$.
In other words, we are trying to define an element of $\brck{A+B} \to Q$.
Since $Q$ is a mere propositon, by the induction principle for propositional truncation, it suffices to construct a function $A+B\to Q$.
But now we can use case analysis on $A+B$.

Similarly, for a type family $P:A\to\type$, we can consider $\brck{\sm{x:A} P(x)}$, which is a mere propositional version of ``there exists an $x:A$ such that $P(x)$''.
As for disjunction, by combining the induction principles of truncation and $\Sigma$-types, if we have an assumption of type $\brck{\sm{x:A} P(x)}$, we may introduce new assumptions $x:A$ and $y:P(x)$ \emph{when attempting to prove a mere proposition}.
In other words, if we know that there exists some $x:A$ such that $P(x)$, but we don't have a particular such $x$ in hand, then we are free to make use of such an $x$ as long as we aren't trying to construct anything which might depend on the particular value of $x$.
Requiring the codomain to be a mere proposition expresses this independence of the result on the witness, since all possible inhabitants of such a type must be equal.

We can now properly formulate the \emph{axiom of choice} in homotopy type theory.
Assume a type $X$ and type families $A:X\to\type$ and $P:\prd{x:X} A(x)\to\type$, and moreover that
\begin{itemize}
\item $X$ is a set,
\item $A(x)$ is a set for all $x:X$, and
\item $P(x,a)$ is a mere proposition for all $x:X$ and $a:A(x)$.
\end{itemize}
The axiom of choice asserts that under these assumptions,
\begin{multline}\label{eq:ac}
  \left(\prd{x:X} \Brck{\sm{a:A(x)} P(x,a)}\right)
  \to\\
  \Brck{\sm{g:\prd{x:X} A(x)} \prd{x:X} P(x,g(x))}
\end{multline}
Note that the propositional truncation appears twice.
The truncation in the domain means we assume that for every $x$ there exists some $a:A(x)$ such that $P(x,a)$, but that these values are not chosen or specified in any known way.
The truncation in the codomain means we conclude that there exists some function $g$, but this function is not determined or specified in any known way.

As with LEM and DN,~\eqref{eq:ac} is not a consequence of our basic type theory, but it may consistently be assumed as an axiom.
Note the restriction to types that are sets; because of the continuity/functoriality of all functions, it is unreasonable to assert such a statement without some such restriction.

\begin{rmk}
  A word of caution about a common pitfall: although dependent function types preserve mere propositions (\autoref{thm:isprop-forall}), they do not commute with truncation: $\brck{\prd{x:A} P(x)}$ is not generally equivalent to $\prd{x:A} \brck{P(x)}$.
  Indeed, while the version of the ``axiom of choice'' which is provable in type theory is really just a statement about how $\Sigma$'s and $\Pi$'s commute, the proper axiom of choice~\eqref{eq:ac} is arguably a statement about how $\Pi$'s commute with truncation.
\end{rmk}


\subsection{When are propositions truncated?}
\label{subsec:when-trunc}

At first glance, it may seem that the truncated versions of $+$ and $\Sigma$ are actually closer to the informal mathematical meaning of ``or'' and ``there exists'' than the untruncated ones.
Certainly, they are closer to the \emph{precise} meaning of ``or'' and ``there exists'' in the first-order logic which underlies formal set theory, since the latter makes no attempt to remember any witnesses to the truth of propositions.
However, it may come as a surprise to realize that the practice of \emph{informal} mathematics is often more accurately described by the untruncated forms.

For example, consider a statement like ``every prime number is either $2$ or odd.''
The working mathematician feels no compunction about using this fact not only to prove \emph{theorems} about prime numbers, but also to perform \emph{constructions} on prime numbers, perhaps doing one thing in the case of $2$ and another in the case of an odd prime.
Since the end result of the construction is not merely the truth of some statement, but a piece of data which may depend on the parity of the prime number, from a type-theoretic perspective such a construction is naturally phrased using the induction principle for the coproduct type ``$(p=2)+(p\text{ is odd})$'', not its propositional truncation.

Admittedly, this is not an ideal example, since ``$p=2$'' and ``$p$ is odd'' are mutually exclusive, so that $(p=2)+(p\text{ is odd})$ is in fact already a mere proposition and hence equivalent to its truncation (see Exercises~\ref{ex:disjoint-or} and~\ref{ex:prop-eqvtrunc}).
More compelling examples come from the existential quantifier.
It is not uncommon to prove a theorem of the form ``there exists an $x$ such that \dots'' and then refer later on to ``the $x$ constructed in Theorem Y'' (note the definite article).
Moreover, when deriving further properties of this $x$, one may use phrases such as ``by the construction of $x$ in the proof of Theorem Y''.

A very common example is ``$A$ is isomorphic to $B$'', which strictly speaking means only that there exists \emph{some} isomorphism between $A$ and $B$.
But almost invariably, when proving such a statement one exhibits a specific isomorphism, or proves that some previously known map is an isomorphism, and it often matters later on what particular isomorphism was given.

Honest, set-theoretically trained mathematicians generally feel a twinge of guilt at such ``abuses of language''.
We may attempt to apologize for them, expunge them from final drafts, or weasel out of them with vague words like ``canonical''.
The problem is exacerbated by the fact that in formalized set theory, there is technically no way to ``construct'' objects at all --- we can only prove that an object with certain properties exists.
Untruncated logic in type theory thus captures some common practices of informal mathematics that the set theoretic reconstruction obscures.
(This is similar to how the univalence axiom validates the common, but formally unjustified, practice of identifying isomorphic objects.)

On the other hand, sometimes truncated logic is essential.
We have seen this in the statements of LEM and AC; some other examples will appear later on in the book.
Thus, we are faced with the problem: when writing informal type theory, what should we mean by the words ``or'' and ``there exists'' (along with common synonyms such as ``there is'' and ``we have'')?

A universal consensus may not be possible.
Perhaps depending on the sort of mathematics being done, one convention or the other may be more useful --- or, perhaps, the choice of convention may be irrelevant.
In this case, a remark at the beginning of a mathematical paper may suffice to inform the reader of the linguistic conventions in use therein.
However, even after one overall convention is chosen, the other sort of logic will usually arise at least occasionally, so we need a way to refer to it.
More generally, one may consider replacing the propositional truncation with another operation on types that behaves similarly, such as the double negation $A\mapsto \neg\neg A$, or the $n$-truncations to be considered in Chapter~\ref{cha:hlevels}.
As an experiment in exposition,  in what follows we will occasionally use \emph{adverbs} to denote the application of such ``modalities'' as propositional truncation.

For instance, if untruncated logic is the default convention, we may use the adverb \textbf{merely} to denote propositional truncation.
Thus the phrase
\begin{center}
  ``there merely exists an $x:A$ such that $P(x)$''
\end{center}
indicates the type $\brck{\sm{x:A} P(x)}$.
On the other hand, if truncated logic is the current default convention, we may use an adverb such as \textbf{purely} or \textbf{constructively} to indicate its absence, so that
\begin{center}
``there purely exists an $x:A$ such that $P(x)$''
\end{center}
would denote the type $\sm{x:A} P(x)$.
We may also use ``purely'' just to emphasize the absence of truncation, even when that is the default convention.

In this book we will continue using untruncated logic as the default convention, for a number of reasons.
\begin{enumerate}[label=(\arabic*)]
\item We want to encourage the newcomer to experiment with it, rather than sticking to truncated logic simply because it is more familiar.
\item Using truncated logic as the default in type theory suffers from the same sort of ``abuse of language'' problems as set-theoretic foundations, which untruncated logic avoids.
  For instance, our definition of ``$\eqv A B$'' as the type of equivalences between $A$ and $B$, rather than its propositional truncation, means that to prove a theorem of the form ``$\eqv A B$'' is literally to construct a particular such equivalence.
  This specific equivalence can then be referred to later on.
\item We want to emphasize that the notion of ``mere proposition'' is not a fundamental part of type theory.
  As we will see in Chapter~\ref{cha:hlevels}, mere propositions are just the second rung on an infinite ladder, and there are also many other modalities not lying on this ladder at all.
\item Many statements that classically are mere propositions are no longer so in homotopy type theory.
  Of course, foremost among these is equality.
\item On the other hand, one of the most interesting observations of homotopy type theory is that a surprising number of types are \emph{automatically} mere propositions, or can be slightly modified to become so, without the need for any truncation.
  (See \autoref{ex:isprop-isprop} and Chapters~\ref{cha:equivalences}, \ref{cha:hlevels}, \ref{cha:category-theory}, and~\ref{cha:set-math}.)
  Thus, although these types contain no data beyond a truth value, we can nevertheless use them to construct untruncated objects, since there is no need to use the induction principle of propositional truncation.
  This useful fact is more clumsy to express if propositional truncation is applied to all statements by default.
\item Finally, truncations are not very useful for most of the mathematics we will be doing in this book, so it is simpler to notate them explicitly when they occur.
\end{enumerate}

\section{Contractibility}
\label{sec:contractibility}

In \autoref{thm:inhabprop-eqvunit} we observed that a mere proposition which is inhabited must be equivalent to unit, and it is not hard to see that the converse also holds.
Another equivalent definition, which is also sometimes convenient, is the following.

\begin{defn}\label{defn:contractible}
  A type $A$ is \textbf{contractible}, or a \textbf{singleton}, if there is $a:A$, called the \textbf{center of contraction}, such that $a=x$ for all $x:A$.
\end{defn}

In other words, the type $\iscontr(A)$ is defined to be
\[ \iscontr(A) \defeq \sm{a:A} \prd{x:A}(a=x). \]

\begin{lem}\label{thm:contr-paths}
  For a type $A$, the following are logically equivalent.
  \begin{enumerate}
  \item $A$ is contractible.\label{item:contr}
  \item $A$ is a mere proposition, and there is a point $a:A$.\label{item:contr-inhabited-prop}
  \item $A$ is equivalent to \unit.\label{item:contr-eqv-unit}
  \end{enumerate}
\end{lem}
\begin{proof}
  If $A$ is contractible, then it certainly has a point $a:A$ (the center of contraction), while for any $x,y:A$ we have $x=a=y$; thus $A$ is a mere proposition.
  Conversely, if we have $a:A$ and $A$ is a mere proposition, then for any $x:A$ we have $x=a$; thus $A$ is contractible.
  And we showed~\ref{item:contr-inhabited-prop}$\Rightarrow$\ref{item:contr-eqv-unit} in \autoref{thm:inhabprop-eqvunit}, while the converse follows since \unit easily has properties~\ref{item:contr} and~\ref{item:contr-inhabited-prop}.
\end{proof}

\begin{lem}\label{thm:isprop-iscontr}
  For any type $A$, the type $\iscontr(A)$ is a mere proposition.
\end{lem}
\begin{proof}
  Suppose given $c,c':\iscontr(A)$.
  We may assume $c\jdeq(a,p)$ and $c'\jdeq(a',p')$ for $a,a':A$ and $p:\prd{x:A} (a=x)$ and $p':\prd{x:A} (a'=x)$.
  By the characterization of paths in $\Sigma$-types, to show $c=c'$ it suffices to exhibit $q:a=a'$ such that $\trans{q}{p}=p'$.

  We choose $q\defeq p(a')$.
  For the other equality, by function extensionality we must show that $(\trans q p)(x)=p'(x)$ for any $x:A$.
  For this, it will suffice to show that for any $x,y:A$ and $u:x=y$ we have $u= \opp{p(x)} \ct p(y)$, since then we would have $(\trans q p)(x) = \opp{p(x)} \ct p(y) = p'(x)$.
  But now we can invoke path induction to assume that $x\jdeq y$ and $u\jdeq \refl{x}$.
  In this case our goal is to show that $\refl x = \opp{p(x)} \ct p(x)$, which is just the inversion law for paths.
\end{proof}

\begin{cor}\label{thm:contr-contr}
  If $A$ is contractible, then so is $\iscontr(A)$.
\end{cor}
\begin{proof}
  By \autoref{thm:isprop-iscontr} and \autoref{thm:contr-paths}\ref{item:contr-inhabited-prop}.
\end{proof}

\begin{lem}\label{thm:contr-forall}
  If $P:A\to\type$ is a type family such that each $P(a)$ is contractible, then $\prd{x:A} P(x)$ is contractible.
\end{lem}
\begin{proof}
  By \autoref{thm:isprop-forall}, $\prd{x:A} P(x)$ is a mere proposition since each $P(x)$ is.
  But it also has an element, namely the function sending each $x:A$ to the center of contraction of $P(x)$.
  Thus by \autoref{thm:contr-paths}\ref{item:contr-inhabited-prop}, $\prd{x:A} P(x)$ is contractible.
\end{proof}

(In fact, the statement of \autoref{thm:contr-forall} is equivalent to the function extensionality axiom.
See Appendix~[?].)


%%%%%%%%%%%%%%%%%%%%%%%%%%%
\section*{Notes}
\label{sec:notes}

The definition of identity types and the elimination rule $J$ are due to Martin-L\"of (what is the best reference?).
Our identity types are generally called \emph{intensional}, by contrast with the \emph{extensional} case which would have an additional ``reflection rule'' saying that if $p:x=y$, then in fact $x\jdeq y$.
This reflection rule implies that all the higher groupoid structure collapses, so for nontrivial homotopy we must use the intensional version. 
One may argue, however, that homotopy type theory is more ``extensional'' than traditional extensional type theory, because of the function extensionality and univalence rules.  

The proofs of symmetry (inversion) and transitivity (concatenation) for equalities are well-known in type theory.
The fact that these make each type into a 1-groupoid (up to homotopy) is also folklore, and was exploited in~\cite{hs:gpd-typethy} to give the first homotopical semantics for type theory.  The general homotopical interpretation, with identity types as path spaces, is due to \cite{aw:hiit}.
For a construction of \emph{all} the higher operations and coherences of an $\infty$-groupoid in type theory, see~\cite{pll:wkom-type} and~\cite{bg:type-wkom}.

Operations such as $\transfib{P}{p}{-}$ and $\apfunc{f}$ and one good notion of equivalence were first studied extensively in type theory by Voevodsky, using the proof assistant Coq.
Subsequent researchers have found many other equivalent definitions of equivalence, which we will compare in Chapter~\ref{cha:equivalences}.

The ``computational'' interpretation of identity types, transport, and so on described in \S\ref{sec:computational} has been emphasized by~\cite{lh:canonicity}.
They also described a ``1-truncated'' type theory (see Chapter~\ref{cha:hlevels}) in which these rules really are computation steps (that is, definitional equalities which a computer can ``evaluate'').
The possibility of extending this to the full untruncated theory is a subject of current research.

The naive form of function extensionality which says that ``if two functions are pointwise equal, then they are equal'' is a common axiom in type theory.
Some stronger forms of function extensionality were considered in~\cite{garner:depprod}.
The version we have used, which identifies the identity types of function types up to equivalence, was first studied by Voevodsky, who also proved that it is implied by the naive version.

The univalence axiom is also due to Voevodsky.
It was originally motivated by semantic considerations; see~\cite{klv:ssetmodel} [and possibly an appendix, if we include one about semantics].

The simple conclusions in \S\S\ref{sec:compute-coprod}--\ref{sec:compute-nat} such as ``coproduct injections are injective and disjoint'' are well-known in type theory, and the construction of the function \encode is the usual way to prove them.
The more refined approach we have described, which characterizing the entire identity type of a positive type (up to equivalence), is a more recent development; see e.g.~\cite{ls:pi1s1}.

The type-theoretic axiom of choice~\eqref{eq:sigma-ump-map} was first observed to be true by Martin-L\"of [TODO: Is that true?].

The fact that it is possible to define sets in type theory using finitely many data, with all higher homotopies automatically taken care of as in \S\ref{sec:basics-sets}, was first observed by Voevodsky.
His original definition of $\isset$ was a bit more complicated than ours, but has the advantage of fitting neatly into an infinite hierarchy; see Chapter~\ref{cha:hlevels}.

\autoref{thm:not-dneg} and \autoref{thm:not-lem} can be traced back to a classical theorem of Hedberg, which we will prove in Chapter~\ref{cha:hlevels}.
The proof we have given of \autoref{thm:not-dneg} is due to Thierry Coquand.

Mere propositions were first defined in type theory by Voevodsky.
His original definition was slightly more complicated than ours, but fits into the more general framework of Chapter~\ref{cha:hlevels}.

The propositional truncation was introduced, in extensional type theory, by~\cite{ab:bracket-types}.
The intensional version was constructed by Voevodsky using an impredicative quantification, and later by Lumsdaine using higher inductive types (see Chapter~\ref{cha:hits}).



\section*{Exercises}
\label{basics:exercises}

\begin{ex}\label{ex:basics:concat}
  Show that the three obvious proofs of \autoref{lem:concat} are pairwise equal.
\end{ex}

\begin{ex}
  Show that the three equalities of proofs constructed in the previous exercise form a commutative triangle.
\end{ex}

\begin{ex}
  Give a fourth, different, proof of \autoref{lem:concat}, and prove that it is equal to the others.
\end{ex}

\begin{ex}
  Prove that the functions~\eqref{eq:ap-to-apd} and~\eqref{eq:apd-to-ap} are inverse equivalences, and that they take $\apfunc f(p)$ to $\apdfunc f (p)$ and vice versa.
\end{ex}

\begin{ex}\label{ex:ap-sigma}
  State and prove a generalization of \autoref{thm:ap-prod} from cartesian products to $\Sigma$-types.
\end{ex}

\begin{ex}
  State and prove an analogue of \autoref{thm:ap-prod} for coproducts.
\end{ex}

\begin{ex}
  Prove that coproducts have the expected universal property:
  \[ \eqv{(A+B \to X)}{(A\to X)\times (B\to X)} \]
  Can you generalize this to an equivalence involving dependent functions?
\end{ex}

\begin{ex}
  Prove that if $\eqv A B$ and $A$ is a set, then so is $B$.
\end{ex}

\begin{ex}\label{ex:isset-coprod}
  Prove that if $A$ and $B$ are sets, then so is $A+B$.
\end{ex}

\begin{ex}\label{ex:isset-sigma}
  Prove that if $A$ is a set and $B:A\to \type$ is a type family such that $B(x)$ is a set for all $x:A$, then $\sm{x:A} B(x)$ is a set.
\end{ex}

\begin{ex}\label{ex:prop-endocontr}
  Show that $A$ is a mere proposition if and only if $A\to A$ is contractible.
\end{ex}

\begin{ex}
  Show that if $A$ is a mere proposition, then so is $A+(\neg A)$.
  Thus, there is no need to insert a propositional truncation in~\eqref{eq:lem}.
\end{ex}

\begin{ex}\label{ex:disjoint-or}
  More generally, show that if $A$ and $B$ are mere propositions and $\neg(A\times B)$, then $A+B$ is also a mere proposition.
\end{ex}

\begin{ex}\label{ex:hprop-iff-equiv}
  Show that if $A$ and $B$ are mere propositions such that $A\to B$ and $B\to A$, then $\eqv A B$.
\end{ex}

\begin{ex}\label{ex:isprop-isprop}
  Show that for any type $A$, the types $\isprop(A)$ and $\isset(A)$ are mere propositions.
\end{ex}

\begin{ex}\label{ex:prop-eqvtrunc}
  Show that if $A$ is already a mere proposition, then $\eqv A{\brck{A}}$.
\end{ex}

\begin{ex}\label{ex:brck-qinv}
  Assuming that some type $\isequiv(f)$ satisfies conditions~\ref{item:be1}--\ref{item:be3} of \S\ref{sec:basics-equivalences}, show that the type $\brck{\qinv(f)}$ satisfies the same conditions and is equivalent to $\isequiv(f)$.
\end{ex}

\begin{ex}
  Show that it is not the case that for all $A:\type$ we have $\brck{A} \to A$.
  (However, there can be particular types for which $\brck{A}\to A$.
  \autoref{ex:brck-qinv} implies that $\qinv(f)$ is such.)
\end{ex}

\begin{ex}
  Show that the rules for the propositional truncation given in \S\ref{sec:prop-trunc} are sufficient to imply a dependent version of the induction principle: for any type family $B:\brck A \to \type$ such that each $B(x)$ is a mere proposition, if for every $a:A$ we have $B(\bproj a)$, then for every $x:\brck A$ we have $B(x)$.
\end{ex}


% Local Variables:
% TeX-master: "main"
% End:
