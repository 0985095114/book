\chapter{Set theory}
\label{cha:set-math}

Our conception of sets as types with particularly simple homotopical character, cf.\
\autoref{sec:basics-sets}, is quite different from the sets of Zermelo-Frankel set theory, which form a
cumulative hierarchy with an intricate nested membership structure.
For many mathematical purposes, the homotopy-theoretic sets are just as good as
the Zermelo-Fraenkel ones, but there are important differences.

We begin this chapter in \autoref{sec:piw-pretopos} by showing that the category $\uset$ has (most of) the usual properties of the category of sets.
In constructive, predicative, univalent foundations, it is a ``$\Pi\mathsf{W}$-pretopos''; whereas if we assume propositional resizing (\autoref{subsec:prop-subsets}) it is an elementary topos, and if we assume LEM and AC then it is a model of Lawvere's Elementary Theory of the Category of Sets.
This is sufficient to ensure that the sets in homotopy type theory behave mostly like sets as used by most mathematicians outside of set theory.

In the rest of the chapter, we investigate some subjects that traditionally belong to ``set theory''.
In \autoref{sec:cardinals,sec:ordinals,sec:wellorderings} we study cardinal and ordinal numbers.
These are traditionally defined in set theory using the global membership relation, but we will see that the univalence axiom enables an equally convenient, more ``structural'' approach.

Finally, in \autoref{sec:cumulative-hierarchy} we consider the possibility of constructing \emph{inside} of homotopy type theory a cumulative hierarchy of sets, equipped with a binary membership relation akin to that of Zermelo-Fraenkel set theory.
This combines higher inductive types with ideas from the field of Algebraic Set Theory.

In this chapter we will often use the traditional logical notation described in \autoref{subsec:prop-trunc}.

%\section{\texorpdfstring{$\set$}{Set} is a \texorpdfstring{$\Pi$}{Π}W-pretopos}
\section{The category of sets}
\label{sec:piw-pretopos}

Recall that in \autoref{cha:category-theory} we defined the category \uset to consist of all $0$-types (in some universe \UU) and maps between them, and observed that it is a category (not just a precategory).
We consider successively the levels of structure which \uset possesses.

\subsection{Limits and colimits}
\label{subsec:limits-sets}

Since sets are closed under products, the universal property of products in \autoref{thm:prod-ump} shows immediately that \uset has finite products.
In fact, infinite products follow just as easily from the equivalence
\[ \left(X\to \prd{a:A} B(a) \right) \simeq \left(\prd{a:A} (X\to B)\right) .\]
And we remarked already that the pullback of $f:A\to C$ and $g:B\to C$ can be defined as $\sm{a:A}{b:B} f(a)=g(b)$; this is a set if $A,B,C$ are and inherits the correct universal property.
Thus, \uset is a \emph{complete} category in the obvious sense.

Since sets are closed under $+$ and contain \emptyt, \uset has finite coproducts.
Similarly, since $\sm{a:A}B(a)$ is a set whenever $A$ and each $B(a)$ are, it yields a coproduct of the family $B$ in \uset.
Finally, we showed in \autoref{sec:pushouts} that pushouts exist in $n$-types, which includes \uset in particular.
Thus, \uset is also \emph{cocomplete}.

\subsection{Images}
\label{sec:image}

%We will show that $\uset$ is a $\Pi$W-pretopos. 
Next, we show that $\uset$ is a \define{regular category}, i.e.:
%
\begin{enumerate}
\item $\uset$ is finitely complete.\label{item:reg1}
\item The kernel pair $\proj1,\proj2: (\sm{x,y:A} f(x)= f(y)) \to A$ of any
      function $f : A \to B$ has a coequalizer.\label{item:reg2}
\item Pullbacks of regular epimorphisms are again regular epimorphisms.\label{item:reg3}
\end{enumerate}
%
Recall that a \define{regular epimorphism} is a morphism that is the coequalizer of \emph{some} pair of maps.
Thus in~\ref{item:reg3} the pullback of a coequalizer is required to again be a coequalizer, but not necessarily of the pulled-back pair.

The obvious candidate for the coequalizer of the kernel pair of $f:A\to B$ is its \emph{image}, as defined in \autoref{sec:image-factorization}.
Recall that we defined $\im(f)\defeq \sm{b:B} \brck{\hfib f b}$, with functions 
$\tilde{f}:A\to\im(f)$ and $i_f:\im(f)\to B$ defined by
\begin{align*}
  \tilde{f} & \defeq \lam{a} \Parens{f(a),\bproj{\pairr{a,\refl{f(a)}}}}\\
i_f & \defeq \proj1
\end{align*}
fitting into a diagram:
\begin{equation*}
  \xymatrix{
    **[l]{\sm{x,y:A} f(x)= f(y)}
    \ar@<0.25em>[r]^{\proj1}
    \ar@<-0.25em>[r]_{\proj2}
    &
    {A}
    \ar[r]^(0.4){\tilde{f}}
    \ar[rd]_{f}
    &
    {\im(f)}
    \ar@{..>}[d]^{i_f}
    \\ & &
    B
  }
\end{equation*}

Recall that a function $f:A\to B$ is called \emph{surjective} if $\fall{b:B}\brck{\hfib f b}$, or equivalently $\fall{b:B} \exis{a:A} f(a)=b$.
A function $f:A\to B$ between sets is called \emph{injective} if $\fall{a,a':A} f(a) = f(a') \Rightarrow a=a'$, or equivalently if each of its fibers is a mere proposition.
Since these are the $(-1)$-connected and $(-1)$-truncated maps in the sense of \autoref{cha:hlevels}, the general theory there implies that $\tilde f$ above is surjective and $i_f$ is injective, and that this factorization is stable under pullback.

We now identify surjectivity and injectivity with the appropriate category-theoretic notions.
First we observe that categorical monomorphisms and epimorphisms have a slightly stronger equivalent formulation.

\begin{lem}\label{thm:mono}
  For a morphism $f:\hom_A(a,b)$ in a category $A$, the following are equivalent.
  \begin{enumerate}
  \item $f$ is a \define{monomorphism}: for all $x:A$ and ${g,h:\hom_A(x,a)}$, if $f\circ g = f\circ h$ then $g=h$.\label{item:mono1}
  \item (If $A$ has products) the diagonal map $A\to A\times A$ is an isomorphism.\label{item:mono4}
  \item For all $x:A$ and $k:\hom_A(x,b)$, the type $\sm{h:\hom_A(x,a)} k = f\circ h$ is a mere proposition.\label{item:mono2}
  \item For all $x:A$ and ${g:\hom_A(x,a)}$, the type $\sm{h:\hom_A(x,a)} f\circ g = f\circ h$ is contractible.\label{item:mono3}
  \end{enumerate}
\end{lem}
\begin{proof}
  The equivalence of conditions~\ref{item:mono1} and~\ref{item:mono4} is standard category theory.
  Now consider the function $(f\circ -):\hom_A(x,a) \to \hom_A(x,b)$ between sets.
  Condition~\ref{item:mono1} says that it is injective, while~\ref{item:mono2} says that its fibers are mere propositions; hence they are equivalent.
  And~\ref{item:mono2} implies~\ref{item:mono3} by taking $k\defeq f\circ g$ and recalling that an inhabited mere proposition is contractible.
  Finally,~\ref{item:mono3} implies~\ref{item:mono1} since if $p:f\circ g= f\circ h$, then $(g,\refl{})$ and $(h,p)$ both inhabit the type in~\ref{item:mono3}, hence are equal and so $g=h$.
\end{proof}

\begin{lem}
  A function $f:A\to B$ between sets is injective if and only if it is a monomorphism in \uset.
\end{lem}
\begin{proof}
  Left to the reader.
\end{proof}

Of course, an \define{epimorphism} is a monomorphism in the opposite category.
We now show that in \uset, the epimorphisms are precisely the surjections, and also precisely the coequalizers (regular epimorphisms).

The coequalizer of a pair of maps $f,g:A\to B$ in $\uset$ is defined as the 0-truncation of a general (homotopy) coequalizer.
For clarity, we may call this the \define{set-coequalizer}.
It is convenient to express its universal property as follows.

\begin{lem}
Let $f,g:A\to B$ be functions between sets $A$ and $B$. The 
{set-co}equalizer $c_{f,g}:B\to Q$ has the property that, for any set $C$ and any $h:B\to C$ with $h\circ f = h\circ g$, the type
\begin{equation*}
\sm{k:Q\to C} k\circ c_{f,g} = h
\end{equation*}
is contractible.
\end{lem}

\begin{lem}\label{epis-surj}
For any function $f:A\to B$ between sets, the following are equivalent:
\begin{enumerate}
\item $f$ is an epimorphism.
\item Consider the pushout diagram
\begin{equation*}
  \xymatrix{
    {A}
    \ar[r]^{f}
    \ar[d]
    &
    {B}
    \ar[d]^{\iota}
    \\
    {\unit}
    \ar[r]_{t}
    &
    {C_f}
  }
\end{equation*}
in $\uset$ defining the mapping cone. Then the type $C_f$ is contractible.
\item $f$ is surjective.
\end{enumerate}
\end{lem}

\begin{proof}
Let $f:A\to B$ be a function between sets, and suppose it to be an epimorphism; we show $C_f$ is contractible.
The basic constructor of $C_f$ corresponding to
$\unit\to C_f$ gives us an element $t:C_f$. We have to show that
\begin{equation*}
\prd{x:C_f} x= t.
\end{equation*}
Note that $x= t$ is a mere proposition, hence we can use induction on
$C_f$: it suffices to find
\begin{align*}
I_0 & : \prd{b:B} \iota(b)= t\\
I_1 & : \prd{a:A} \opp{\alpha_1(a)} \ct I_0(f(a))=\refl{t}.
\end{align*}
where $\alpha_1:\prd{a:A} \iota(f(a))= t$ is a basic constructor
of $C_f$. Note that $\alpha_1$ is a homotopy from $\iota\circ f$ to
$\mathsf{const}_t\circ f$, so we find the elements
\begin{equation*}
\pairr{\iota,\refl{\iota\circ f}},\pairr{\mathsf{const}_t,\alpha_1}:
\sm{h:B\to C_f} \iota\circ f = h\circ f.
\end{equation*}
By the dual of \autoref{thm:mono}\ref{item:mono3}, there is a path
\begin{equation*}
\gamma:\pairr{\iota,\refl{\iota\circ f}}=\pairr{\mathsf{const}_t,\alpha_1}
\end{equation*}
Hence, we may define $I_0(b)\defeq \happly(\projpath1(\gamma),b):\iota(b)=t$.
We also have
\[\projpath2(\gamma) : \trans{\projpath1(\gamma)}{\refl{\iota\circ f}} = \alpha_1. \]
This transport involves precomposition with $f$, which commutes with $\happly$.
Thus, from transport in path types we obtain $I_0(f(a)) = \alpha_1(a)$ for any $a:A$, which gives us $I_1$.

Now suppose $C_f$ is contractible; we show $f$ is surjective.
We first construct a type family $P:C_f\to\prop$ by recursion on $C_f$, which is valid since \prop is a set.
On the point constructors, we define
\begin{align*}
P(t) & \defeq \unit\\
P(\iota(b)) & \defeq \brck{\hfiber{f}b}.
\end{align*}
To complete the construction of $P$, it remains to give a path $\brck{\hfiber{f}{f(a)}} =_\prop \unit$ for all $a:A$.
However, $\brck{\hfiber{f}{f(a)}}$ is inhabited by $(f(a),\refl{f(a)})$.
Since it is a mere proposition, this means it is contractible --- and thus equivalent, hence equal, to \unit.
This completes the definition of $P$.
Now, since $C_f$ is assumed to be contractible, it follows that $P(x)$ is equivalent to $P(t)$ for any $x:C_f$.
In particular, $P(\iota(b))\jdeq \brck{\hfiber{f}b}$ is equivalent to $P(t)\jdeq \unit$ for each $b:B$, and hence contractible.
Thus, $f$ is surjective.

Finally, suppose $f:A\to B$ to be surjective, and consider a set $C$ and two functions
$g,h:B\to C$ with the property that $g\circ f = h\circ f$. Since $f$ 
is assumed to be surjective,
we have an equivalence $B\simeq\im(f)$. We have the following equivalences
\begin{align*}
\prd{b:B} (g(b)= h(b))
& \simeq \prd{w:\im(f)} g(\proj1 w)= h(\proj1(w))\\
& \simeq \prd{b:B}{a:A}{p:f(a)= b} g(b)= h(b)\\
& \simeq \prd{a:A} g(f(a))= h(f(a)).
\end{align*}
By assumption, there is an element of the latter type.
\end{proof}

% \begin{rem}
% The above theorem is not true when we replace $\set$ by $\type$
% (replacing it also in the definition of $\mathsf{epi}$ and $\mathsf{epi}'$). 
% However, we do
% get the implications $\textit{ii.}\Rightarrow\textit{iii.}\Rightarrow
% \textit{iv.}$
% \end{rem}

\begin{thm}\label{thm:set_regular}\label{lem:images_are_coequalizers}
The category $\uset$ is regular. Moreover, surjective functions between sets are regular epimorphisms.
\end{thm}

\begin{proof}
It is a standard lemma in category theory that a category is regular as soon as it admits finite limits and a pullback-stable orthogonal 
factorization system $(\mathcal{E},\mathcal{M})$ with $\mathcal{M}$ the monomorphisms, in which case $\mathcal{E}$ conists automatically of 
the regular epimorphisms.
(See e.g.~\cite[A1.3.4]{elephant}.)
The existence of the factorization system was proved in \autoref{thm:orth-fact}.
\end{proof}

\begin{lem}\label{lem:pb_of_coeq_is_coeq}
Pullbacks of regular epis in \uset are regular epis.
\end{lem}
\begin{proof}
  We showed in \autoref{thm:stable-images} that pullbacks of $n$-connected functions are $n$-connected.
  By \autoref{lem:images_are_coequalizers}, it suffices to apply this when $n=-1$.
\end{proof}

\subsection{Quotients}\label{subsec:quotients}
Now that we know that $\uset$ is regular, to show that $\uset$ is exact, we need to show that every
equivalence relation is effective. In other words, given an equivalence
relation $R:A\to A\to\prop$, there is a coequalizer $c_R$ of the pair
$\proj1,\proj2:\sm{x,y:A} R(x,y)\to A$ and, moreover, the $\proj1$ and $\proj2$
form the kernel pair of $c_R$.

We have already seen, in \autoref{sec:set-quotients}, two general ways to construct the quotient of a set by an equivalence relation $R:A\to A\to\prop$.
The first can be described as the set-coequalizer of the two projections
\[\proj1,\proj2:\sm{x,y:A} R(x,y)\to A.\]
The important property of such a quotient is the following.

\begin{defn}
A relation $R:A\to A\to\prop$ is said to be \define{effective} if the square
\begin{equation*}
  \xymatrix{
    {\sm{x,y:A} R (x,y)}
    \ar[r]^(0.7){\proj1}
    \ar[d]_{\proj2}
    &
    {A}
    \ar[d]^{c_R}
    \\
    {A}
    \ar[r]_{c_R}
    &
    {A/R}
    }
\end{equation*}
is a pullback. 
\end{defn}

\begin{lem}\label{lem:sets_exact}
Suppose $\pairr{A,R}$ is an equivalence relation. Then there is an equivalence
\begin{equation*}
(c_R(x)= c_R(y))\simeq R(x,y)
\end{equation*}
for any $x,y:A$. In other words, equivalence relations are effective.
\end{lem}

\begin{proof}
We begin by extending $R$ to a relation $\tilde{R}:A/R\to A/R\to\prop$. After
the construction of $\tilde{R}$ we will show that $\tilde{R}$ is equivalent
to the identity type on $A/R$. We define $\tilde{R}$ by double induction on
$A/R$ (note that $\prop$ is a set by univalence for mere propositions). We
define $\tilde{R}(c_R(x),c_R(y)) \defeq R(x,y)$. For $r:R(x,x')$ and $s:R(y,y')$,
the transitivity and symmetry 
of $R$ gives an equivalence from $R(x,y)$ to $R(x',y')$. This completes the
definition of $\tilde{R}$. To finish the proof of the lemma, we need
to show that $\tilde{R}(w,w')\simeq w= w'$ for every $w,w':A/R$. We can
do this by showing that the type $\sm{w':A/R} \tilde{R}(w,w')$ is contractible for
each $w:A/R$. We do this by induction. Let $x:A$. We have the element
$\pair{c_R(x),\rho(x)}:\sm{w':A/R} \tilde{R}(c_R(x),w')$, where $\rho$ is
the reflexivity proof of $R$, hence we only
have to show that
\begin{equation*}
\prd{w':A/R}{r:\tilde{R}(c_R(x),w')} \pairr{w',r}=\pairr{c_R(x),\rho(x)},
\end{equation*}
which we do by induction on $w'$. Let $y:A$ and let $r:R(x,y)$. Then we have
the path $p_R(r)^{-1}:c_R(y)= c_R(x)$. We automatically get a path from
$p_R(r)^{-1}\cdot r=\rho(x)$, finishing the proof.
\end{proof}

The second construction of quotients, which uses impredicativity of mere propositions, is as the set of equivalence classes of $R$ (a subset
of its power set):
\[ A\sslash R \defeq \setof{ P:A\to\prop | P \text{ is an equivalence class of } R} \]
Note that if we regard $R$ as a function from $A$ to $A\to \prop$, then $A\sslash R$ is equivalent to $\im(R)$, as constructed in \autoref{sec:image}.
Now in \autoref{lem:images_are_coequalizers} we have shown that images are
coequalizers. In particular, we immediately get the coequalizer diagram
\begin{equation*}
  \xymatrix{
    **[l]{\sm{x,y:A} R (x)= R (y)}
    \ar@<0.25em>[r]^{\proj1}
    \ar@<-0.25em>[r]_{\proj2}
    &
    {A}
    \ar[r]
    &
    {A \sslash R.}
  }
\end{equation*}
We can use this to give an alternative proof that any equivalence relation is effective and that the two definitions of quotients agree.

\begin{thm}\label{prop:kernels_are_effective}
For any function $f:A\to B$ between any two sets, 
the relation $\ker(f):A\to A\to\type$ given by 
$\ker(f,x,y)\defeq f(x)= f(y)$ is effective. 
\end{thm}

\begin{proof}
We will use that $\im(f)$ is the coequalizer of $\proj1,\proj2:
\sm{x,y:A} f(x)= f(y)\to A$.
%we get this equivalence from~\autoref{prop:images_are_coequalizers}
Note that the canonical kernel pair of the function 
$c_f\defeq\lam{a} \pairr{f(a),\tau_1(\pairr{a,\refl{f(a)}})}$ consists
of the two projections
\begin{equation*}
\proj1,\proj2:\big(\sm{x,y:A} c_f(x)= c_f(y)\big)\to A.
\end{equation*}
For any $x,y:A$, we have equivalences
\begin{align*}
c_f(x)= c_f(y) & \simeq \sm{p:f(x)= f(y)} p\cdot\tau_1(\pairr{x,\refl{f(x)}})
=\tau_1(\pairr{y,\refl{f(x)}})\\ & \simeq f(x)= f(y),
\end{align*}
where the last equivalence holds because 
$\brck{\hfiber{f}b}$ is a mere for any $b:B$. 
Therefore, we get that
\begin{equation*}
\sm{x,y:A} c_f(x)= c_f(y)\simeq\sm{x,y:A} f(x)= f(y)
\end{equation*}
and hence we may conclude that $\ker f$ is an effective relation 
for any function $f$.
\end{proof}

\begin{thm}
Equivalence relations are effective and there is an equivalence $A/R \simeq A\sslash  R $. 
\end{thm}

\begin{proof}
We need to analyze the coequalizer diagram
\begin{equation*}
  \xymatrix{
    **[l]{\sm{x,y:A} R (x)= R (y)}
    \ar@<0.25em>[r]^{\proj1}
    \ar@<-0.25em>[r]_{\proj2}
    &
    {A}
    \ar[r]
    &
    {A \sslash R.}
  }
\end{equation*}
By the univalence axiom, the space $R (x)= R (y)$ is equivalent to the space of homotopies from $R (x)$ to $R (y)$, which is
equivalent to $\prd{z:A} R (x,z)\simeq R (y,z)$. Since $R $ is an equivalence relation, the latter space is equivalent to $R (x,y)$. To
summarize, we get that $(R (x)= R (y))\simeq R (x,y)$, so $R $ is effective since it is equivalent to an effective relation. Also,
the diagram
\begin{equation*}
  \xymatrix{
    **[l]{\sm{x,y:A} R(x, y)}
    \ar@<0.25em>[r]^{\proj1}
    \ar@<-0.25em>[r]_{\proj2}
    &
    {A}
    \ar[r]
    &
    {A \sslash R.}
  }
\end{equation*}
is a coequalizer diagram. Since coequalizers are unique up to equivalence, it follows that $A/R \simeq A\sslash  R $.
\end{proof}

We finish this section by mentioning a possible third construction of the quotient of a set $A$ by an equivalence relation $R$. One
considers the precategory with objects $A$ and hom-sets $R$. The type of objects of the Rezk completion (see \autoref{sec:rezk}) of this precategory will be the
quotient. The reader is invited to check the details of this construction.

\subsection{\texorpdfstring{$\uset$}{Set} is a \texorpdfstring{$\Pi\mathsf{W}$}{ΠW}-pretopos}
\label{subsec:piw}

The notion of a \emph{$\Pi\mathsf{W}$-pretopos} --- that is, a locally cartesian closed category with finite coproducts, effective equivalence relations, and $\mathsf{W}$-types --- is intended as a ``predicative" notion of topos, i.e.\ a category of ``predicative sets", which can serve the purpose for constructive mathematics that the usual category of sets does for classical mathematics.  

Typically, in constructive type theory, one resorts to an external construction of ``setoids" --- an exact completion --- to obtain a category with such closure properties.  In particular, the well-behaved quotients are required for many constructions in mathematics that usually involve (non-constructive) power sets.  It is noteworthy that HoTT  provides these constructions \emph{internally} (via HITs), without requiring such external constructions.  This represents a powerful advantage of our approach, as we shall see in subsequent examples.

\begin{thm} The category $\uset$ is a $\Pi\mathsf{W}$-pretopos.
\end{thm}
\begin{proof}
We have an initial object $0$ and finite, disjoint sums $A+B$.  These are stable under pullback, simply because pullback has a right adjoint.  Indeed, $\uset$ is locally cartesian closed, since for any map $f:A\to B$ between sets, the ``fibrant replacement" $\sm{a:A}f(a)=b$ is equivalent to $A$ (over $B$), and we have $\prod$-types for the replacement.
We've just shown that $\uset$ is regular (\autoref{thm:set_regular}) and that quotients are effective (\autoref{lem:sets_exact}). We thus have an LCC pretopos. Finally, since the $n$-types are closed under the formation of $\mathsf{W}$-types
(\autoref{sec:w-types}), we see that $\uset$ is a $\Pi\mathsf{W}$-pretopos. 
\end{proof}

One naturally wonders what, if anything, prevents $\uset$ from being an (elementary) topos?
In addition to the structure already mentioned, a topos has a
\emph{subobject classifier}: a pointed object classifying (equivalence classes of) monomorphisms.  (In fact, in the presence of a subobject
classifier, things become somewhat simpler: one merely needs cartesian closure in order to get the colimits).
In HoTT, univalence implies that the type $\prop \defeq \sm{X:\UU}\isprop(X)$ does classify monos (by an argument similar to \autoref{sec:object-classification}), but in general it is as large as the ambient universe $\UU$.
Thus, it is a ``set'' in the sense of being a $0$-type, but it is not ``small'' in the sense of being an object of $\UU$, hence not an object of the category \uset.
However, if we assume an appropriate form of propositional resizing (see \autoref{subsec:prop-subsets}), then we may find a small version of $\prop$, so that \uset becomes an elementary topos.

\begin{thm}\label{thm:settopos}
  If there is a type $\Omega:\UU$ of all mere propositions, then the category $\uset_\UU$ is then an elementary topos.
\end{thm}

A sufficient condition for this is the law of excluded middle, in the ``mere-propositional" form that we have called LEM; for then we have $\prop = \bool$, which \emph{is} small, and which then also classifies all mere propositions.
Moreover, in topos theory a well-known sufficient condition for LEM is the axiom of choice, which is of course often assumed as an axiom in classical set theory.
In the next section, we briefly investigate the relation between these conditions in our setting.


%%%%%%%%%%%%%%%%%%%%%%%%%%%%%%%%%%%%%%%%%%%%%%%%
\subsection{The axiom of choice implies excluded middle}
\label{subsec:emacinsets}

% In this section we prove a classic result that the axiom of choice implies excluded
% middle.

We begin with the following lemma.

\begin{lem}\label{prop:trunc_of_prop_is_set}
If $A$ is a mere proposition then its supension $\susp(A)$ is a set,
and $A$ is equivalent to $\id[\susp(A)]{\north}{\south}$.
\end{lem}

\begin{proof}
To show that $\susp(A)$ is a set, we define a
family $P:\susp(A)\to\susp(A)\to\type$ with the 
property that $P(x,y)$ is a mere proposition for each $x,y:\susp(A)$,
and which is equivalent to $\idtypevar{\susp(A)}$.
%
We make the following definitions:
\begin{align*}
P(\north,\north) & \defeq \unit &
P(\south,\north) & \defeq A\\
P(\north,\south) & \defeq A &
P(\south,\south) & \defeq \unit.
\end{align*}
We have to check that the definition preserves paths.
Given any $a : A$, there is a meridian $\merid(a) : \north = \south$,
so we should also have
%
\begin{equation*}
  P(\north, \north) = P(\north, \south) = P(\south, \north) = P(\south, \south).
\end{equation*}
%
But since $A$ is inhabited by $a$, it is equivalent to $\unit$, so we have
%
\begin{equation*}
  P(\north, \north) \simeq P(\north, \south) \simeq P(\south, \north) \simeq P(\south, \south).
\end{equation*}
%
The univalence axiom turns these into the desired equalities. Also, $P(x,y)$ is a mere
proposition for all $x, y : \susp(A)$, which is proved by induction on $x$ and $y$, and
using the fact that being a mere proposition is a mere proposition.

Note that $P$ is a reflexive relation.
Therefore we may apply \autoref{thm:h-set-refrel-in-paths-sets}, so it suffices to
construct $\tau : \prd{x,y:\susp(A)}P(x,y)\to(x=y)$. We do this by a double induction.
When $x$ is $\north$, we define $\tau(\north)$ by
%
\begin{equation*}
  \tau(\north,\north, \blank) \defeq \refl{\north}
  \qquad\text{and}\qquad
  \tau(\north,\south,a) \defeq \merid(a).
\end{equation*}
%
If $A$ is inhabited by $a$ then $\merid(a) : \north = \south$ so we also need $\trans{\merid(a)}{\tau(\north, \north)} =
\tau(\north, \south)$. This we get by function extensionality using the fact that, for all $x : A$,
%
\begin{multline*}
  \trans{\merid(a)}{\tau(\north,\north,x)} =
  \tau(\north,\north,x) \ct \opp{\merid(a)} \jdeq \\
  \refl{\north} \ct \merid(a) =
  \merid(a) =
  \merid(x) \jdeq
  \tau(\north, \south, x).
\end{multline*}
In a symmetric fashion we may define $\tau(\south)$ by
%
\begin{equation*}
  \tau(\south,\north, a) \defeq \opp{\merid(a)}
  \qquad\text{and}\qquad
  \tau(\south,\south, \blank) \defeq \refl{\south}.
\end{equation*}
%
To complete the construction of $\tau$, we need to check $\trans{\merid(a)}{\tau(\north)} = \tau(\south)$,
given any $a : A$. The verification proceeds much along the same lines by induction on the
second argument of $\tau$.

It remains to show that the mere propositions $A$ and $\id[\susp(A)]{\north}{\south}$ are
equivalent, but this is very easy so we leave it as exercise.
\end{proof}

\begin{thm}[Diaconescu]\label{thm:1surj_to_surj_to_pem}
  The axiom of choice implies the law of excluded middle.
\end{thm}

\begin{proof}
We use the equivalent form of the axiom of choice given in \autoref{thm:ac-epis-split}.
Consider a mere proposition $A$.
The function $f:\bool\to\susp(A)$ defined by
$f(\btrue) \defeq \south$ and $f(\bfalse) \defeq \north$
is surjective.
Indeed, we have
$\pairr{\bfalse,\refl{\north}} : \hfiber{f}{\north}$
and $\pairr{\btrue,\refl{\south}} :\hfiber{f}{\south}$.
Since $\brck{\hfiber{f}{x}}$ is a mere proposition, by induction the claimed surjectivity follows.

By \autoref{prop:trunc_of_prop_is_set} the supension $\susp(A)$
is a set, so by the axiom of choice there merely exists a
section $g: \susp(A) \to \bool$ of $f$.
As equality on $\bool$ is decidable we get
\begin{equation*}
 (g(f(\bfalse))= g(f(\btrue))) +
 \lnot (g(f(\bfalse))= g(f(\btrue))),
\end{equation*}
and, since $g$ is a section of $f$,
\begin{equation*}
(f(\bfalse) = f(\btrue)) +
\lnot (f(\bfalse) = f(\btrue)).
\end{equation*}
Now we see that it suffices for $A$ to be equivalent to
$f(\bfalse) = f(\btrue)$, which is the case by \autoref{prop:trunc_of_prop_is_set}.
\end{proof}

% This conclusion needs only LEM, see \autoref{ex:lemnm}.

% \begin{cor}\label{cor:ACtoLEM0}
%   If the axiom of choice AC holds then $\brck{A + \neg A}$ for every set $A$.
% \end{cor}

% \begin{proof}
%   There is a surjection
%   \[
%   A + \neg A \epi \brck{A} + \brck{\neg A} \epi
%   \brck{(\brck{A} + \brck{\neg A})} = \brck{A} \vee \brck{\neg A} = \brck{A} \vee \neg \brck{A} = \unit,
%   \]
%   %
%   where in the last step excluded middle is available as a consequence of the axiom of choice.
%   Again by the axiom of choice there merely exists a section of the surjection, but this
%   is none other than an inhabitant of $A + \neg A$. Therefore $\brck{A+\neg A}$.
% \end{proof}

\begin{thm}\label{thm:ETCS}
  If the axiom of choice holds then the category $\uset$ is a well-pointed boolean elementary topos with choice.  
\end{thm}

\begin{proof}
  Since AC implies LEM, we have a Boolean elementary topos with choice by \autoref{thm:settopos} and the remark following it.  We leave the proof of well-pointedness as
an exercise for the reader.
\end{proof}

\begin{rmk}
  The conditions on a category mentioned in the theorem are known as Lawvere's
  axioms for the \emph{Elementary Theory of the Category of
    Sets}~\cite{lawvere:etcs-long}.
\end{rmk}

\section{Cardinal numbers}
\label{sec:cardinals}

\begin{defn}
  The \define{type of cardinal numbers} is the 0-truncation of \set:
  \[ \card \defeq \pizero{\set} \]
  Thus, a \define{cardinal number}, or \define{cardinal}, is an inhabitant of $\card\jdeq \pizero\set$.
\end{defn}

\begin{rmk}
  Technically, there is a separate type ``\card'' associated to each universe ``\type'', but with these conventions we can state theorems beginning with ``for all cardinal numbers\dots''\ and give them exactly the same sort of meaning as those beginning ``for all types\dots''.
\end{rmk}

If $A$ is a set, we write $\cd{A}$ for its image under the canonical projection $\set \to \card$.
Of course, by definition, \card is a set.
It also inherits the structure of a semiring from \set.

\begin{defn}
  The operation of \define{cardinal addition}
  \[ \blank+\blank : \card \to \card \to \card \]
  is defined by induction on truncation:
  \[ \cd{A} + \cd{B} \defeq \cd{A+B} .\]
\end{defn}
\begin{proof}
  Since $\card\to\card$ is a set, to define $\alpha+\blank:\card\to\card$ for all $\alpha:\card$, by induction it suffices to assume that $\alpha$ is $\cd{A}$ for some $A:\set$.
  Now we want to define $\cd{A}+\blank :\card\to\card$, i.e.\ we want to define $\cd{A}+\beta :\card$ for all $\beta:\card$.
  However, since $\card$ is a set, by induction it suffices to assume that $\beta$ is $\cd{B}$ for some $B:\set$.
  But now we can define $\cd{A}+\cd{B}$ to be $\cd{A+B}$.
\end{proof}

\begin{defn}
  Similarly, the operation of \define{cardinal multiplication}
  \[ \blank\cdot\blank : \card \to \card \to \card \]
  is defined by induction on truncation:
  \[ \cd{A} \cdot \cd{B} \defeq \cd{A\times B} \]
\end{defn}

\begin{lem}\label{card:semiring}
  \card is a commutative semiring, i.e.\ for $\alpha,\beta,\gamma:\card$ we have the following.
  \begin{align*}
    (\alpha+\beta)+\gamma &= \alpha+(\beta+\gamma)\\
    \alpha+0 &= \alpha\\
    \alpha + \beta &= \beta + \alpha\\
    (\alpha \cdot \beta) \cdot \gamma &= \alpha \cdot (\beta\cdot\gamma)\\
    \alpha \cdot 1 &= \alpha\\
    \alpha\cdot\beta &= \beta\cdot\alpha\\
    \alpha\cdot(\beta+\gamma) &= \alpha\cdot\beta + \alpha\cdot\gamma
  \end{align*}
  where $0 \defeq \cd{\emptyset}$ and $1\defeq\cd{\unit}$.
\end{lem}
\begin{proof}
  We prove the commutativity of multiplication, $\alpha\cdot\beta = \beta\cdot\alpha$; the others are exactly analogous.
  Since \card is a set, the type $\alpha\cdot\beta = \beta\cdot\alpha$ is a mere proposition, and in particular a set.
  Thus, by induction it suffices to assume $\alpha$ and $\beta$ are of the form $\cd{A}$ and $\cd{B}$ respectively, for some $A,B:\set$.
  Now $\cd{A}\cdot \cd{B} \jdeq \cd{A\times B}$ and $\cd{B}\times\cd{A} \jdeq \cd{B\times A}$, so it suffices to show $A\times B = B\times A$.
  Finally, by univalence, it suffices to give an equivalence $A\times B \simeq B\times A$.
  But this is easy: take $(a,b) \mapsto (b,a)$ and its obvious inverse.
\end{proof}

\begin{defn}
  The operation of \define{cardinal exponentiation} is also defined by induction on truncation:
  \[ \cd{A}^{\cd{B}} \defeq \cd{B\to A}. \]
\end{defn}

\begin{lem}\label{card:exp}
  For $\alpha,\beta,\gamma:\card$ we have
  \begin{align*}
    \alpha^0 &= 1\\
    1^\alpha &= 1\\
    \alpha^1 &= \alpha\\
    \alpha^{\beta+\gamma} &= \alpha^\beta \cdot \alpha^\gamma\\
    \alpha^{\beta\cdot \gamma} &= (\alpha^{\beta})^\gamma\\
    (\alpha\cdot\beta)^\gamma &= \alpha^\gamma \cdot \beta^\gamma
  \end{align*}
\end{lem}
\begin{proof}
  Exactly like \autoref{card:semiring}.
\end{proof}

\begin{defn}
  The relation of \define{cardinal inequality}
  \[ \blank\le\blank : \card\to\card\to\prop \]
  is defined by induction on truncation:
  \[ \cd{A} \le \cd{B} \defeq \brck{\inj(A,B)} \]
  where $\inj(A,B)$ is the type of injections from $A$ to $B$.
  In other words, $\cd{A} \le \cd{B}$ means that there merely exists an injection from $A$ to $B$.
\end{defn}

\begin{lem}
  Cardinal inequality is a preorder, i.e.\ for $\alpha,\beta:\card$ we have
  \begin{gather*}
    \alpha \le \alpha\\
    (\alpha \le \beta) \to (\beta\le\gamma) \to (\alpha\le\gamma)
  \end{gather*}
\end{lem}
\begin{proof}
  As before, by induction on truncation.
  For instance, since $(\alpha \le \beta) \to (\beta\le\gamma) \to (\alpha\le\gamma)$ is a mere proposition, by induction on 0-truncation we may assume $\alpha$, $\beta$, and $\gamma$ are $\cd{A}$, $\cd{B}$, and $\cd{C}$ respectively.
  Now since $\cd{A} \le \cd{C}$ is a mere proposition, by induction on $(-1)$-truncation we may assume given injections $f:A\to B$ and $g:B\to C$.
  But then $g\circ f$ is an injection from $A$ to $C$, so $\cd{A} \le \cd{C}$ holds.
  Reflexivity is even easier.
\end{proof}

We may likewise show that cardinal inequality is compatible with the semiring operations.

\begin{lem}\label{thm:injsurj}
  Consider the following statements:
  \begin{enumerate}
  \item There is an injection $A\to B$.\label{item:cle-inj}
  \item There is a surjection $B\to A$.\label{item:cle-surj}
  \end{enumerate}
  Then, assuming excluded middle:
  \begin{itemize}
  \item Given $a_0:A$, we have~\ref{item:cle-inj}$\to$\ref{item:cle-surj}.
  \item Therefore, if $A$ is merely inhabited, we have~\ref{item:cle-inj} $\to$ merely \ref{item:cle-surj}.
  \item Assuming the axiom of choice, we have~\ref{item:cle-surj} $\to$ merely \ref{item:cle-inj}.
  \end{itemize}
\end{lem}
\begin{proof}
  If $f:A\to B$ is an injection, define $g:B\to A$ at $b:B$ as follows.
  Since $f$ is injective, the fiber of $f$ at $b$ is a mere proposition.
  Therefore, by excluded middle, either there is an $a:A$ with $f(a)=b$, or not.
  In the first case, define $g(b)\defeq a$; otherwise set $g(b)\defeq a_0$.
  Then for any $a:A$, we have $a = g(f(a))$, so $g$ is surjective.

  The second statement follows from this by induction on truncation.
  For the third, if $g:B\to A$ is surjective, then by the axiom of choice, there merely exists a function $f:A\to B$ with $g(f(a)) = a$ for all $a$.
  But then $f$ must be injective.
\end{proof}

\begin{thm}[Schroeder-Bernstein]
  Assuming excluded middle, for sets $A$ and $B$ we have
  \[ \inj(A,B) \to \inj(B,A) \to (A\cong B) \]
\end{thm}
\begin{proof}
  The usual ``back-and-forth'' argument applies without significant changes.
  Note that it actually constructs an isomorphism $A\cong B$ (assuming excluded middle so that we can decide whether a given element belongs to a cycle, an infinite chain, a chain beginning in $A$, or a chain beginning in $B$).
\end{proof}

\begin{cor}
  Assuming excluded middle, cardinal inequality is a partial order, i.e.\ for $\alpha,\beta:\card$ we have
  \[ (\alpha\le\beta) \to (\beta\le\alpha) \to (\alpha=\beta). \]
\end{cor}
\begin{proof}
  Since $\alpha=\beta$ is a mere proposition, by induction on truncation we may assume $\alpha$ and $\beta$ are $\cd{A}$ and $\cd{B}$, respectively, and that we have injections $f:A\to B$ and $g:B\to A$.
  But then the Schroeder--Bernstein theorem gives an isomorphism $A\simeq B$, hence an equality $\cd{A}=\cd{B}$.
\end{proof}

Finally, we can reproduce Cantor's theorem, showing that for every cardinal there is a greater one.

\begin{thm}[Cantor]
  For $A:\set$, there is no surjection $A \to (A\to \mathbf{2})$.
\end{thm}
\begin{proof}
  Suppose $f:A \to (A\to \mathbf{2})$ is any function, and define $g:A\to \mathbf{2}$ by $g(a) \defeq \neg f(a)(a)$.
  If $g = f(a_0)$, then $g(a_0) = f(a_0)(a_0)$ but $g(a_0) = \neg f(a_0)(a_0)$, a contradiction.
  Thus, $f$ is not surjective.
\end{proof}

\begin{cor}
  Assuming excluded middle, for any $\alpha:\card$, there is a cardinal $\beta$ such that $\alpha\le\beta$ and $\alpha\neq\beta$.
\end{cor}
\begin{proof}
  Let $\beta = 2^\alpha$.
  Now we want to show a mere propositon, so by induction we may assume $\alpha$ is $\cd{A}$, so that $\beta\jdeq \cd{A\to \mathbf{2}}$.
  Using excluded middle, we have a function $f:A\to (A\to \mathbf{2})$ defined by
  \[f(a)(a') \defeq
  \begin{cases}
    \top &\quad a=a'\\
    \bot &\quad a\neq a'.
  \end{cases}
  \]
  And if $f(a)=f(a')$, then $f(a')(a) = f(a)(a) = \top$, so $a=a'$; hence $f$ is injective.
  Thus, $\alpha \jdeq \cd{A} \le \cd{A\to \mathbf{2}} \jdeq 2^\alpha$.

  On the other hand, if $2^\alpha \le \alpha$, then we would have an injection $(A\to\mathbf{2})\to A$.
  By \autoref{thm:injsurj}, since we have $(\lam{x} \bot):A\to \mathbf{2}$ and excluded middle, there would then be a surjection $A \to (A\to \mathbf{2})$, contradicting Cantor's theorem.
\end{proof}

\section{Ordinal numbers}
\label{sec:ordinals}

\begin{defn}
  Let $A$ be a set and
  \[\blank<\blank:A\to A\to \prop\]
  a binary relation on $A$.
  We define by induction what it means for an element $a:A$ to be \define{accessible} by $<$:
  \begin{itemize}
  \item If $b$ is accessible for every $b<a$, then $a$ is accessible.
  \end{itemize}
  We write $\acc(a)$ to mean that $a$ is accessible.
\end{defn}

It may seem that such an inductive definition can never get off the ground, but of course if $a$ has the property that there are \emph{no} $b$ such that $b<a$, then $a$ is vacuously accessible.

\begin{lem}
  Accessibility is a mere property.
\end{lem}
\begin{proof}
  We claim that for any $a_1,a_2:A$, any $p:a_1=a_2$, and any $s_1:\acc(a_1)$ and $s_2:\acc(a_2)$, we have $\trans{p}{s_1}=s_2$.
  By induction, we may assume that $s_1$ is given by a function assigning to each $b_1<a_1$ a proof $s_1(b_1):\acc(b_1)$, and moreover that for any $b_2:A$, any $q:b_1=b_2$, and $t_2:\acc(b_2)$, we have $\trans{q}{s_1(b_1)}=t_2$.
  Similarly, we may assume that $s_2$ is given by a function assigning to each $b_2<a_2$ a proof $s_2(b_2):\acc(b_2)$, and that for any $b_1:A$, any $q:b_1=b_2$, and $t_1:\acc(b_1)$, we have $\trans{q}{t_1}=s_2(b_2)$.
  
  Now by function extensionality, to show $\trans{p}{s_1}=s_2$ it suffices to show that for any $b_2<a_2$ we have $\trans{p}{s_1}(b_2) = s_2(b_2)$.
  However, we can obtain this from the induction hypothesis for $s_2$ with $q\defeq \refl{b_2}$ and $t_1 \defeq \trans{p}{s_1}(b_2)$.
  This proves the claim.

  Finally, we instantiate the claim with $a_2\defeq a_1\defeq a$ and $p\defeq \refl{a}$.
  Thus, for any $a:A$ and $s_1,s_2:\acc(a)$, we have $s_1=s_2$, as desired.
\end{proof}

\begin{defn}
  A binary relation $<$ on a set $A$ is \define{well-founded} if every element of $A$ is accessible.
\end{defn}

\begin{lem}
  Well-foundedness is a mere property.
\end{lem}
\begin{proof}
  Well-foundedness of $<$ is the type $\prd{a:A} \acc(a)$, which is a mere proposition since each $\acc(a)$ is.
\end{proof}

\begin{eg}\label{thm:nat-wf}
  Perhaps the most familiar well-founded relation is the usual strict ordering on \nat.
  To show that this is well-founded, we must show that $n$ is accessible for each $n:\nat$.
  This is just the usual proof of ``strong induction'' from ordinary induction on \nat.

  Specifically, we prove by induction on $n:\nat$ that $k$ is accessible for all $k\le n$.
  The base case is just that $0$ is accessible, which is vacuously true since nothing is strictly less than $0$.
  For the inductive step, we assume that $k$ is accessible for all $k\le n$, which is to say for all $k<n+1$; hence by definition $n+1$ is also accessible.

  A different relation on \nat which is also well-founded is obtained by setting only $n < \suc(n)$ for all $n:\nat$.
  Well-foundedness of this relation is almost exactly the ordinary induction principle of \nat.
\end{eg}

\begin{eg}\label{thm:wtype-wf}
  Let $A:\set$ and $B : A \to \set$ be a family of sets.
  Recall from \autoref{sec:w-types} that the $W$-type $\wtype{a:A} B(a)$ is inductively generated by the single constructor
  \begin{itemize}
  \item $\supp : \prd{a:A} (B(a) \to \wtype{x:A} B(x)) \to \wtype{x:A} B(x)$
  \end{itemize}
  We define the relation $<$ on $\wtype{x:A} B(x)$ by recursion on its second argument:
  \begin{itemize}
  \item For any $a:A$ and $f:B(a) \to \wtype{x:A} B(x)$, we define $w<\supp(a,f)$ to mean that there merely exists a $b:B(a)$ such that $w = f(b)$.
  \end{itemize}
  Now we prove that every $w:\wtype{x:A} B(x)$ is accessible for this relation, using the usual induction principle for $\wtype{x:A}B(x)$.
  This means we assume given $a:A$ and $f:B(a) \to \wtype{x:A} B(x)$, and also a lifting $f' : \prd{b:B(a)} \acc(f(b))$.
  But then by definition of $<$, we have $\acc(w)$ for all $w<\supp(a,f)$; hence $\supp(a,f)$ is accessible.
\end{eg}

Well-foundedness allows us to define functions by recursion and prove statements by induction, such as for instance the following.

\begin{lem}\label{thm:wfrec}
  Suppose $B$ is a set and we have a function
  \[ g : \mathcal{P}B \to B \]
  Then if $<$ is a well-founded relation on $A$, there is a function $f:A\to B$ such that for all $a:A$ we have
  \begin{equation*}
    f(a) = g\Big(\setof{ f(a') | a'<a }\Big).
  \end{equation*}
\end{lem}
\begin{proof}
  We first define, for every $a:A$ and $s:\acc(a)$, an element $\bar f(a,s):B$.
  By induction, it suffices to assume that $s$ is a function assigning to each $a'<a$ a proof $s(a'):\acc(a')$, and that moreover for each such $a'$ we have an element $\bar f(a',s(a')):B$.
  In this case, we define
  \begin{equation*}
    \bar f(a,s) \defeq g\Big(\setof{ \bar f(a',s(a')) | a'<a }\Big).
  \end{equation*}

  Now since $<$ is well-founded, we have a function $w:\prd{a:A} \acc(a)$.
  Thus, we can define $f(a)\defeq \bar f (a,w(a))$.
\end{proof}

In classical logic, well-foundedness has a more well-known reformulation.

\begin{lem}\label{thm:wfmin}
  Assuming excluded middle, $<$ is well-founded if and only if every nonempty subset $B\subseteq A$ merely has a minimal element.
\end{lem}
\begin{proof}
  Suppose first $<$ is well-founded, and suppose $B\subseteq A$ is a subset with no minimal element.
  That is, for any $a:A$ with $a\in B$, there merely exists a $b:A$ with $b<a$ and $b\in B$.

  We claim that for any $a:A$ and $s:\acc(a)$, we have $a\notin B$.
  By induction, we may assume $s$ is a function assigning to each $a'<a$ a proof $s(a'):\acc(a)$, and that moreover for each such $a'$ we have $a'\notin B$.
  If $a\in B$, then by assumption, there would merely exist a $b<a$ with $b\in B$, which contradicts this assumption.
  Thus, $a\notin B$; this completes the induction.
  Since $<$ is well-founded, we have $a\notin B$ for all $a:A$, i.e. $B$ is empty.

  Now suppose each nonempty subset merely has a minimal element.
  Let $B = \setof{ a:A | \neg \acc(a) }$.
  Then if $B$ is nonempty, it merely has a minimal element.
  Thus there merely exists an $a:A$ with $a\in B$ such that for all $b<a$, we have $\acc(b)$.
  But then by definition (and induction on truncation), $a$ is merely accessible, and hence accessible, contradicting $a\in B$.
  Thus, $B$ is empty, so $<$ is well-founded.
\end{proof}

\begin{defn}
  A well-founded relation $<$ on a set $A$ is \define{extensional} if for any $a,b:A$, we have
  \[ \Big(\prd{c:A} (c<a) \leftrightarrow (c<b)\Big) \to (a=b). \]
\end{defn}

Note that since $A$ is a set, extensionality is a mere proposition.
This notion of ``extensionality'' is unrelated to function extensionality, and also unrelated to the extensionality of identity types.
Rather, it is a ``local'' counterpart of the axiom of extensionality in classical set theory.

\begin{thm}
  The type of extensional well-founded relations is a set.
\end{thm}
\begin{proof}
  By the univalence axiom, it suffices to show that if $(A,<)$ is extensional and well-founded and $f:(A,<) \cong (A,<)$, then $f=\idfunc[A]$.
  We prove by induction on $<$ that $f(a)=a$ for all $a:A$.
  The inductive hypothesis is that for all $a'<a$, we have $f(a')=a'$.

  Now since $A$ is extensional, to conclude $f(a)=a$ it is sufficient to show
  \[\prd{c:A}(c<f(a)) \leftrightarrow (c<a).\]
  However, since $f$ is an automorphism, we have $(c<a) \leftrightarrow (f(c)<f(a))$.
  But $c<a$ implies $f(c)=c$ by the induction hypothesis, so $(c<a) \to (c<f(a))$.
  On the other hand, if $c<f(a)$, then $f^{-1}(c)<a$, and so $c = f(f^{-1}(c)) = f^{-1}(c)$ by the induction hypothesis again; thus $c<a$.
  Therefore, we have $(c<a) \leftrightarrow (c<f(a))$ for any $c:A$, so $f(a)=a$.
\end{proof}

\begin{defn}\label{def:simulation}
  If $(A,<)$ and $(B,<)$ are extensional and well-founded, a \define{simulation} is a function $f:A\to B$ such that
  \begin{enumerate}
  \item if $a<a'$, then $f(a)<f(a')$, and\label{item:sim1}
  \item for all $a:A$ and $b:B$, if $b<f(a)$, then there merely exists an $a'<a$ with $f(a')=b$.\label{item:sim2}
  \end{enumerate}
\end{defn}

\begin{lem}
  Any simulation is injective.
\end{lem}
\begin{proof}
  We prove by double well-founded induction that for any $a,b:A$, if $f(a)=f(b)$ then $a=b$.
  The induction hypothesis for $a:A$ says that for any $a'<a$, and any $b:B$, if $f(a')=f(b)$ then $a=b$.
  The inner induction hypothesis for $b:A$ says that for any $b'<b$, if $f(a')=f(b')$ then $a'=b'$.

  Suppose $f(a)=f(b)$; we must show $a=b$.
  By extensionality, it suffices to show that for any $c:A$ we have $(c<a)\leftrightarrow (c<b)$.
  If $c<a$, then $f(c)<f(a)$ by \autoref{def:simulation}\ref{item:sim1}.
  Hence $f(c)<f(b)$, so by \autoref{def:simulation}\ref{item:sim2} there merely exists $c':A$ with $c'<b$ and $f(c)=f(c')$.
  By the induction hypothesis for $a$, we have $c=c'$, hence $c<b$.
  The dual argument is symmetrical.
\end{proof}

In particular, this implies that the word ``merely'' in \autoref{def:simulation}\ref{item:sim2} could be omitted without change of sense.

\begin{cor}
  If $f:A\to B$ is a simulation, then for all $a:A$ and $b:B$, if $b<f(a)$, there \emph{purely} exists an $a'<a$ with $f(a')=b$.
\end{cor}
\begin{proof}
  Since $f$ is injective, $\sm{a:A} (f(a)=b)$ is a mere proposition.
\end{proof}

We say that a subset $C :\mathcal{P}B$ is an \define{initial segment} if $c\in C$ and $b<c$ imply $b\in C$.
The image of a simulation must be an initial segment, while the inclusion of any initial segment is a simulation.
Thus, by univalence, every simulation $A\to B$ is \emph{equal} to the inclusion of some initial segment of $B$.

\begin{thm}
  For a set $A$, let $P(A)$ be the type of extensional well-founded relations on $A$.
  If $\mathord{<_A} : P(A)$ and $\mathord{<_B} : P(B)$ and $f:A\to B$, let $H_{\mathord{<_A}\mathord{<_B}}(f)$ be the mere proposition that $f$ is a simulation.
  Then $(P,H)$ is a standard notion of structure over \uset in the sense of \autoref{sec:sip}.
\end{thm}
\begin{proof}
  We leave it to the reader to verify that identities are simulations, and that composites of simulations are simulations.
  Thus, we have a notion of structure.
  For standardness, we must show that if $<$ and $\prec$ are two extensional well-founded relations on $A$, and $\idfunc[A]$ is a simulation in both directions, then $<$ and $\prec$ are equal.
  Since extensionality and well-foundedness are mere propositions, for this it suffices to have $\prd{a,b:A} (a<b) \leftrightarrow (a\prec b)$.
  But this follows from \autoref{def:simulation}\ref{item:sim1} for $\idfunc[A]$.
\end{proof}

\begin{cor}\label{thm:wfcat}
  There is a category whose objects are sets equipped with extensional well-founded relations, and whose morphisms are simulations.
\end{cor}

In fact, this category is a poset.

\begin{lem}
  For extensional and well-founded $(A,<)$ and $(B,<)$, there is at most one simulation $f:A\to B$.
\end{lem}
\begin{proof}
  Suppose $f,g:A\to B$ are simulations.
  Since being a simulation is a mere property, it suffices to show $\prd{a:A}(f(a)=g(a))$.
  By induction on $<$, we may suppose $f(a')=g(a')$ for all $a'<a$.
  And by extensionality of $B$, to have $f(a)=g(a)$ it suffices to have $\prd{b:B}(b<f(a)) \leftrightarrow (b<g(a))$.

  But since $f$ is a simulation, if $b<f(a)$, then we have $a'<a$ with $f(a')=b$.
  By the inductive hypothesis, we have also $g(a')=b$, hence $b<g(a)$.
  The dual argument is symmetrical.
\end{proof}

Thus, if $A$ and $B$ are equipped with extensional and well-founded relations, we may write $A\le B$ to mean there exists a simulation $f:A\to B$.
\autoref{thm:wfcat} implies that if $A\le B$ and $B\le A$, then $A=B$.

\begin{defn}
  An \define{ordinal} is a set $A$ with an extensional well-founded relation which is \emph{transitive}, i.e.\ $\prd{a,b,c:A}(a<b)\to (b<c) \to (a<c)$.
\end{defn}

\begin{eg}
  Of course, the usual strict order on \nat is transitive.
  It is easily seen to be extensional as well; thus it is an ordinal.
  As usual, we denote this ordinal by $\omega$.
\end{eg}

Let \ord denote the type of ordinals.
By the previous results, \ord is a set and has a natural partial order.
We now show that \ord also admits a well-founded relation.

If $A$ is an ordinal and $a:A$, let $\ordsl A a$ denote the initial segment $\setof{ b:A | b<a}$.
Note that if $\ordsl A a = \ordsl A b$ as ordinals, then that isomorphisms must respect their inclusions into $A$ (since simulations form a poset), and hence they are equal as subsets of $A$.
Therefore, since $A$ is extensional, $a=b$.
Thus the function $a\mapsto \ordsl A a$ is an injection $A\to \ord$.

\begin{defn}
  For ordinals $A$ and $B$, a simulation $f:A\to B$ is said to be \define{bounded} if there exists $b:B$ such that $A = \ordsl B b$.
\end{defn}

The remarks above imply that such a $b$ is unique when it exists, so that boundedness is a mere property.

We write $A<B$ if there exists a bounded simulation from $A$ to $B$.
Since simulations are unique, $A<B$ is also a mere proposition.

\begin{thm}\label{thm:ordord}
  $(\ord,<)$ is an ordinal.
\end{thm}

\begin{rmk}
  If universe levels were made explicit, this theorem would state that the set of ordinals in one universe is an ordinal in the next higher universe.
\end{rmk}

\begin{proof}
  Let $A$ be an ordinal; we first show that $\ordsl A a$ is accessible (in \ord) for all $a:A$.
  By induction, suppose $\ordsl A b$ is accessible for all $b:A$.
  By definition of accessibility, we must show that $B$ is accessible in \ord for all $B<\ordsl A a$.
  However, if $B<\ordsl A a$ then there is some $b<a$ such that $B = \ordsl{(\ordsl A a)}{b} = \ordsl A b$, which is accessible by the inductive hypothesis.
  Thus, $\ordsl A a$ is accessible for all $a:A$.

  Now to show that $A$ is accessible in \ord, by definition we must show $B$ is accessible for all $B<A$.
  But as before, $B<A$ means $B=\ordsl A a$ for some $a:A$, which is accessible as we just proved.
  Thus, \ord is well-founded.

  For extensionality, suppose $A$ and $B$ are ordinals such that $\prd{C:\ord} (C<A) \leftrightarrow (C<B)$.
  Then for every $a:A$, since $\ordsl A a<A$, we have $\ordsl A a<B$, hence there is $b:B$ with $\ordsl A a = \ordsl B b$.
  Define $f:A\to B$ to take each $a$ to the corresponding $b$; it is straightforward to verify that $f$ is an isomorphism.
  Thus $A\cong B$, hence $A=B$ by univalence.

  Finally, it is easy to see that $<$ is transitive.
\end{proof}

Treating \ord as an ordinal is often very convenient, but it has its pitfalls as well.
For instance, consider the following lemma, where we pay attention to how universes are used.

\begin{lem}\label{thm:ordsucc}
  Let \bbU be a universe.
  For any $A:\ord_\bbU$, there is a $B:\ord_\bbU$ such that $A<B$.
\end{lem}
\begin{proof}
  Let $B=A+\unit$, with the element $\star:\unit$ being greater than all elements of $A$.
  Then $B$ is an ordinal and it is easy to see that $A\cong \ordsl B \star$.
\end{proof}

This lemma illustrates a potential pitfall of the ``typically ambiguous'' style of using \UU to denote an arbitrary, unspecified universe.
Consider the following alternative proof of it.

\begin{proof}[Another putative proof of \autoref{thm:ordsucc}]
  Note that $C<A$ if and only if $C=\ordsl A a$ for some $a:A$.
  This gives an isomorphism $A \cong \ordsl \ord A$, so that $A<\ord$.
  Thus we may take $B\defeq\ord$.
\end{proof}

The second proof would be valid if we had stated \autoref{thm:ordsucc} in a typically ambiguous style.
But the resulting lemma would be less useful, because the second proof would constrain the second ``\ord'' in the lemma statement to refer to a higher universe level than the first one.
The first proof allows both universes to be the same.

Similar remarks apply to the next lemma, which could be proved in a less useful way by observing that $A\le \ord$ for any $A:\ord$.

\begin{lem}\label{thm:ordunion}
  Let \bbU be a universe.
  For any $X:\type_\bbU$ and $F:X\to \ord_\bbU$, there exists $B:\ord_\bbU$ such that $Fx\le B$ for all $x:X$.
\end{lem}
\begin{proof}
  Let $B$ be the quotient of the equivalence relation on $\sm{x:X} Fx$ defined as follows:
  \[ (x,y)  = (x',y')
  \;\defeq\;
  \Big(\ordsl{(Fx)}{y} \cong \ordsl{(Fx')}{y'}\Big).
  \]
  Define $(x,y)<(x',y')$ if $\ordsl{(Fx)}{y} < \ordsl{(Fx')}{y'}$.
  This clearly descends to the quotient, and can be seen to make $B$ into an ordinal.
  Moreover, for each $x:X$ the induced map $Fx\to B$ is a simulation.
\end{proof}



\section{Classical well-orderings}
\label{sec:wellorderings}

We now show the equivalence of our ordinals with the more familiar classical well-orderings.

\begin{lem}
  Assuming excluded middle, every ordinal is trichotomous:
  \[ \prd{a,b:A} (a<b) \vee (a=b) \vee (b<a). \]
\end{lem}
\begin{proof}
  By induction on $a$, we may assume that for every $a'<a$ and every $b':A$, we have $(a'<b') \vee (a'=b') \vee (b'<a')$.
  Now by induction on $b$, we may assume that for every $b'<b$, we have $(a<b') \vee (a=b') \vee (b'<a)$.

  By excluded middle, either there merely exists a $b'<b$ such that $a<b'$, or there merely exists a $b'<b$ such that $a=b'$, or for every $b'<b$ we have $b'<a$.
  In the first case, merely $a<b$ by transitivity, hence $a<b$ as it is a mere proposition.
  Similarly, in the second case, $a<b$ by transport.
  Thus, suppose $\prd{b':A}(b'<b)\to (b'<a)$.

  Now analogously, either there merely exists $a'<a$ such that $b<a'$, or there merely exists $a'<a$ such that $a'=b$, or for every $a'<a$ we have $a'<b$.
  In the first and second cases, $b<a$, so we may suppose $\prd{a':A}(a'<a)\to (a'<b)$.
  However, by extensionality, our two suppositions now imply $a=b$.
\end{proof}

\begin{lem}
  A well-founded relation contains no cycles, i.e.\
  \[ \prd{n:\mathbb{N}}{a:\mathbb{N}_n\to A} \neg\Big((a_0<a_1) \wedge \dots \wedge (a_{n-1}<a_n)\wedge (a_n<a_0)\Big). \]
\end{lem}
\begin{proof}
  We prove by induction on $a:A$ that there is no cycle containing $a$.
  Thus, suppose by induction that for all $a'<a$, there is no cycle containing $a'$.
  But in any cycle containing $a$, there is some element less than $a$ and contained in the same cycle.
\end{proof}

\begin{thm}\label{thm:wellorder}
  Assuming excluded middle, $(A,<)$ is an ordinal if and only if every nonempty subset $B\subseteq A$ has a least element.
\end{thm}
\begin{proof}
  If $A$ is an ordinal, then by \autoref{thm:wfmin} every nonempty subset merely has a minimal element.
  But trichotomy implies that any minimal element is a least element.
  Moreover, least elements are unique when they exist, so merely having one is as good as having one.

  Conversely, if every nonempty subset has a least element, then $A$ is well-founded by \autoref{thm:wfmin}.
  We also have trichotomy, since for any $a,b$ the set $\setof{a,b}$ merely has a least element, which must be either $a$ or $b$.
  This implies transitivity, since if $a<b$ and $a<c$, then either $a=c$ or $c<a$ would produce a cycle.
  Similarly, it implies extensionality, for if $\prd{c:A}(c<a)\leftrightarrow (c<b)$, then $a<b$ implies (letting $c$ be $a$) that $a<a$, which is a cycle, and similarly if $b<a$; hence $a=b$.
\end{proof}

In classical mathematics, the characterization of \autoref{thm:wellorder} is taken as the definition of a \emph{well-ordering}, with the \emph{ordinals} being a canonical set of representatives of isomorphism classes for well-orderings.
In our context, the structure identity principle means that there is no need to look for such representatives: any well-ordering is as good as any other.

We now move on to consider consequences of the axiom of choice.

\begin{thm}\label{thm:wop}
  Assuming excluded middle, the following are equivalent.
  \begin{enumerate}
  \item For every set $X$, there merely exists a function
    $ f: \mathcal{P}_+X \to X $
    such that $f(Y)\in Y$ for all $Y:\mathcal{P}X$.\label{item:wop1}
  \item Every set merely admits the structure of an ordinal.\label{item:wop2}
  \end{enumerate}
\end{thm}

Of course,~\ref{item:wop1} is a standard classical version of the axiom of choice.

\begin{proof}
  One direction is easy: suppose~\ref{item:wop2}.
  Since we aim to prove the mere proposition~\ref{item:wop1}, we may assume $A$ is an ordinal.
  But then we can define $f(B)$ to be the least element of $B$.

  Now suppose~\ref{item:wop1}.
  As before, since~\ref{item:wop2} is a mere proposition, we may assume given such an $f$.
  We extend $f$ to a function
  \[ \bar f:\mathcal{P}X \cong (\mathcal{P}_+ X) + \unit \longrightarrow X+\unit
  \]
  in the obvious way.
  Now for any ordinal $A$, we can define $g_A:A\to X+\unit$ by well-founded recursion:
  \[ g_A(a) \defeq 
    \bar f\Big(X \setminus \setof{ g_A(b) | \rule{0pt}{1em} (b<a) \wedge (g_A(b) \in X) }\Big)
  \]
  (regarding $X$ as a subset of $X+\unit$ in the obvious way).

  Let $A'$ be the preimage of $X$; then we claim the restriction $g_A':A' \to X$ is injective.
  For if $a,a':A$ with $a\neq a'$, then by trichotomy and without loss of generality, we may assume $a'<a$.
  Thus $g_A(a') \in \setof{ g_A(b) | b<a }$, so since $f(Y)\in Y$ for all $Y$ we have $g_A(a) \neq g_A(a')$.

  Moreover, $A'$ is an initial segment of $A$.
  For $g_A(a)$ lies in \unit if and only if $\setof{g_A(b)|b<a} = X$, and if this holds then it also holds for any $a'>a$.
  Thus, $A'$ is itself an ordinal.

  Finally, since \ord is an ordinal, we can take $A\defeq\ord$.
  Let $X'$ be the image of $g_\ord':\ord' \to X$; then the inverse of $g_\ord'$ yields an injection $H:X'\to \ord$.
  By \autoref{thm:ordunion}, there is an ordinal $C$ such that $Hx\le C$ for all $x:X'$.
  Then by \autoref{thm:ordsucc}, there is a further ordinal $D$ such that $C<D$, hence $Hx<D$ for all $x:X'$.
  Now we have
  \begin{align*}
    g_{\ord}(D) &= \bar f\Big( X \setminus \setof{ g_\ord(B) | \rule{0pt}{1em} B<D \wedge (g_\ord(B) \in X)} \Big)\\
    &=\bar f\Big( X \setminus \setof{ g_\ord(B) | \rule{0pt}{1em} B:\ord \wedge (g_\ord(B) \in X)} \Big)
  \end{align*}
  since if $B:\ord$ and $(g_\ord(B) \in X)$, then $B = Hx$ for some $x:X'$, hence $B<D$.
  Now if
  \[\setof{ g_\ord(B) | \rule{0pt}{1em} B:\ord \wedge (g_\ord(B) \in X)}\]
  is not all of $X$, then $g_\ord(D)$ would lie in $X$ but not in this subset, which would be a contradiction since $D$ is itself a potential value for $B$.
  So this set must be all of $X$, and hence $g_\ord'$ is surjective as well as injective.
  Thus, we can transport the ordinal structure on $\ord'$ to $X$.
\end{proof}

\begin{rmk}
  If we had given the wrong proof of \autoref{thm:ordsucc} or \autoref{thm:ordunion}, then the resulting proof of \autoref{thm:wop} would be invalid: there would be no way to consistently assign universe levels.
\end{rmk}

\begin{cor}
  Assuming the axiom of choice, the function $\ord\to\set$ (which forgets the order structure) is a surjection.
\end{cor}

Note that \ord is a set, while \set is a 1-type.
In general, there is no reason for a 1-type to admit any surjective function from a set.
Even the axiom of choice does not appear to imply that \emph{every} 1-type does so, but it readily implies that this is so for 1-types constructed out of \set, such as the types of objects of categories of structures as in \autoref{sec:sip}.
The following corollary also applies to such categories.

\begin{cor}
  Assuming AC, \uset admits a weak equivalence functor from a strict category.
\end{cor}
\begin{proof}
  Let $X_0\defeq \ord$, and for $A,B:X_0$ let $\hom_X(A,B) \defeq (A\to B)$.
  Then $X$ is a strict category, since \ord is a set, and the above surjection $X_0 \to \set$ extends to a weak equivalence functor $X\to \uset$.
\end{proof}

Now recall from \autoref{sec:cardinals} that we have a further surjection $\cd{-}:\set\to\card$, and hence a composite surjection $\ord\to\card$ which sends each ordinal to its cardinality.

\begin{thm}
  Assuming AC, the surjection $\ord\to\card$ has a section.
\end{thm}
\begin{proof}
  There is an easy and wrong proof of this: since \ord and \card are both sets, AC implies that any surjection between them \emph{merely} has a section.
  However, we actually have a canonical \emph{specified} section: because \ord is an ordinal, every nonempty subset of it has a uniquely specified least element.
  Thus, we can map each cardinal to the least element in the corresponding fiber.
\end{proof}

It is traditional in set theory to identify cardinals with their image in \ord: the least ordinal having that cardinality.

It follows that \card also canonically admits the structure of an ordinal: in fact, one isomorphic to \ord.
Specifically, we define by well-founded recursion a function $\aleph:\ord\to\ord$, such that $\aleph(A)$ is the least ordinal having cardinality greater than $\aleph({\ordsl A a})$ for all $a:A$.
Then (assuming AC) the image of $\aleph$ is exactly the image of \card.


\section{The cumulative hierarchy}
\label{sec:cumulative-hierarchy}

We can define a cumulative hierarchy $V$ of all sets in a given universe $\UU$ as a higher inductive type, in such a way that $V$ is again a set (in a larger universe $\UU'$), equipped with a binary ``membership" relation $x\in y$ which satisfies the usual laws of set theory.

\begin{defn}
  The \define{cumulative hierarchy $V$} relative to a type universe $\UU$ is the
  higher inductive type generated by the following constructors.
  %
  \begin{enumerate}
  \item For every $A : \UU$ and $f : A \to V$, there is an element $\vset(A, f)$ : V.
  \item For all $A, B : \UU$, $f : A \to V$ and $g : B \to V$ such that
    %
    \begin{equation} \label{eq:V-path}
      \big(\fall{a:A} \exis{b:B} \id[V]{f(a)}{g(b)}\big) \land \big(\fall{b:B} \exis{a:A} \id[V]{f(a)}{g(b)}\big)
    \end{equation}
    %
    there is a path $\id[V]{\vset(A,f)}{\vset(B,g)}$.
  \item The 0-truncation constructor: for all $x,y:V$ and $p,q:x=y$, we have $p=q$.
  \end{enumerate}
\end{defn}

In set-theoretic language, $\vset(A,f)$ can be understood as the set that is the image of $A$ under $f$, i.e.\ $\{ f(a) \mid a
\in A \}$, but of course we cannot write this in type theory. The hierarchy $V$ is
bootstrapped from the empty map $! : \emptyt \to V$, which gives the empty set as $\emptyset = \vset(\emptyt,!)$.
Then the singleton $\{\emptyset\}$ enters $V$ through $\unit \to V$, defined as $\ttt \mapsto \emptyset$, and so
on. The type $V$ lives in the same universe as the base universe $\UU$.

The second constructor of $V$ has a form unlike any we have seen before: it involves not only paths in $V$ (which in \autoref{sec:hittruncations} we claimed were slightly fishy) but truncations of sums of them.
It certainly does not fit the general scheme described in \autoref{sec:naturality}, and thus it may not be obvious what its induction principle should be.
Fortunately, like our first definition of the 0-truncation in \autoref{sec:hittruncations}, it can be re-expressed using auxiliary higher inductive types.

We leave it to the reader to work out the details (see \autoref{ex:cumhierhit}).
At the end of the day, the induction principle for $V$ (written in pattern-matching language) says that given $P:V\to \set$, in order to construct $h:\prd{x:V} P(x)$, it suffices to give the following.
\begin{enumerate}
\item For any $f:A\to V$, construct $h(\vset(A,f))$, assuming as given $h(f(a))$ for all $a:A$.
\item Verify that if $f : A \to V$ and $g : B \to V$ satisfy~\eqref{eq:V-path}, then $h(\vset(A,f)) = h(\vset(B,g))$, assuming inductively that $h(f(a)) = h(g(b))$ whenever $f(a)=g(b)$.
\end{enumerate}
The second clause checks that the map being defined must respect the paths introduced in \eqref{eq:V-path}.
As usual when we state higher induction principles using pattern-matching, it may seem tautologous, but is not.
The point is that ``$h(f(a))$'' is essentially a formal symbol which we cannot peek inside of, which $h(\vset(A,f))$ must be defined in terms of. Thus, in the second clause, we assume equality of these formal symbols when appropriate, and verify that the elements resulting from the construction of the first clause are also equal.  

Observe that, by induction, for each $v:V$ there merely exist $A:\UU$ and $f:A\to V$ such that $v=\vset(A,f)$.
Thus, it is reasonable to try to define the \define{membership relation} $x\in v$ on $V$ by setting:
%
% Note: "membership" rather than "elementhood", because "element" is taken.
%
\begin{equation*}
  (x \in \vset(A,f)) \defeq \exis{a : A} x = f(a).
\end{equation*}
%
To see that the definition is valid, we must use the recursion principle of $V$.  Thus, suppose we have a path $\vset(A, f) = \vset(B, g)$
constructed through~\eqref{eq:V-path}. If $x \in \vset(A,f)$ then there merely is $a : A$ such
that $x = f(a)$, but by~\eqref{eq:V-path} there merely is $b : B$ such that $f(a) = g(b)$, hence
$x = g(b)$ and $x \in \vset(B,f)$. The converse is symmetric.

The \define{subset relation} $x\subseteq y$ is defined on $V$ as usual by
%
\begin{equation*}
  (x \subseteq y) \defeq \fall{z : V} z \in x \Rightarrow z \in y.
\end{equation*}

A \define{class} may be taken to be a mere predicate on~$V$. We can say that a class $C : V \to \prop$ is a
  \define{$V$-set} if there merely exists $v\in V$ such that
%
\begin{equation*}
  \fall{x : V} C(x) \Leftrightarrow x \in v.
\end{equation*}

We may also use the conventional notation for classes:
\[
\setof{ x | C(x) } \defeq \lam{x}C(x)
\]
%
A class $C: V\to \prop$ will be called \define{$\UU$-small} if all of its values $C(x)$ lie in $\UU$, specifically $C: V\to \prop_{\UU}$.
Since $V$ lives in the same universe $\UU'$ as does the base universe $\UU$ from which it is built, the same is true for the identity types $v=_V w$ for any $v,w:V$. For a well-behaved theory in the absence of propositional resizing, therefore, it will be convenient to have a $\UU$-small ``resizing" of the identity relation, which we can define by induction as follows.

\begin{defn}\label{def:bisimulation}
  Define the \define{bisimulation} relation
  %
  \begin{equation*}
    \sim\ : V\times V \longrightarrow \prop_{\UU}
  \end{equation*}
  %
  by double induction over $V$, where for $\vset(A,f)$ and $\vset(B,g)$ we let:
\begin{align*}
\vset(A,f)  = \vset(B,g) &\defeq\\
&\big(\fall{a:A}\exis{b:B} f(a)  = g(b)\big) \land \big(\fall{b:B}\exis{a:A} f(a) = g(b)\big)
\end{align*}
\end{defn}
%
To verify that the definition is correct, we just need to check that it respects paths $\vset(A, f) = \vset(B, g)$ constructed through~\eqref{eq:V-path}, but this is obvious, and that $\prop_{\UU}$ is a set, which it is.  Note that $u = v$ is in $\propU$ by construction.

\begin{lem}\label{lem:BisimEqualsId}
For any $u,v:V$ we have $(u=_V v) = (u = v)$.
\end{lem}

\begin{proof}
It suffices to show that $(u = v)\to (u=_V v)$.
By induction on $u$ and $v$, we may assume they are $\vset(A,f)$ and $\vset(B,g)$ respectively.
Then by definition, $\vset(A,f)\sim\vset(B,g)$ implies $(\fall{a:A}\exis{b:B}f(a)  = g(b))$ and conversely.
But the induction hypothesis then tells us that $(\fall{a:A}\exis{b:B}f(a) = g(b))$ and conversely.
So by the path-constructor for $V$ we have $\vset(A,f) =_V \vset(B,g)$.
\end{proof}

Now we can use the resized identity relation to get the following useful principle.

\begin{lem}\label{lem:MonicSetPresent}
For every $u:V$ there is a given $A_u:\UU$ and monic $m_u: A \mono V$ such that $u = \vset(A_u, m_u)$.
\end{lem}

\begin{proof}
  Take any presentation $u = \vset(A,f)$ and factor $f:A\to V$ as
  %
  \begin{equation*}
    f = m_u\circ e_u : A \epi A_u \mono V.    
  \end{equation*}
  %
  Clearly $u = \vset(A_u, m_u)$ if only $A_u$ is still in $\UU$, which holds if the kernel of $e_u : A \epi A_u$ is in $\UU$.  But the kernel of $e_u : A \epi A_u$ is the pullback along $f : A\to V$ of the identity on $V$, which we just showed to be $\UU$-small, up to equivalence.  Now, this construction of the pair $(A_u, m_u)$ with $m_u :A_u \mono V$ and $u = \vset(A_u, m_u)$ from $u:V$ is unique up to equivalence over $V$, and hence up to identity by univalence.  Thus by the principle of unique choice \eqref{cor:UC} there is a map $c : V\to\sm{A:\UU}(A\to V)$ such that $c(u) = (A_u, m_u)$, with $m_u :A_u \mono V$ and $u = \vset(c(u))$, as claimed.
\end{proof}

\begin{defn}\label{def:TypeOfElements}
For $u:V$, the just constructed monic presentation $m_u: A_u \mono V$ such that $u = \vset(A_u, m_u)$ may be called the \define{type of members} of $u$ and denoted $m_u : [u] \mono V$, or even $[u] \mono V$.  We can think of $[u]$ as the ``subclass of $V$ consisting of members of $u$".
\end{defn}

\begin{thm}\label{thm:VisCST}
  The following hold for $(V, {\in})$:
  %
  \begin{enumerate}
  \item \emph{extensionality:}
    %
    \begin{equation*}
      \fall{x, y : V} x \subseteq y \land y \subseteq x \Leftrightarrow x = y.
    \end{equation*}
    %
     \item \emph{empty set:} for all $x:V$, we have $\neg (x\in \emptyset)$.
    %
    \item \emph{pairing:} for all $u, v:V$, the class $\setof{ x | x = u \vee x = v}$ is a $V$-set.
      %
    \item \emph{infinity:} there is a $v:V$ with $\emptyset\in v$ and $x\in v$ implies $x\cup \{x\}\in v$.
    %
  \item \emph{union:} for all $v:V$, the class $\setof{ x | \exis{u:V} x \in u \in v}$ is a $V$-set.
    %
    \item \emph{function set:} for all $u, v:V$, the class $\setof{ x | x : u\to v}$ is a $V$-set.
    %
   \item \emph{$\in$-induction:} if $C : V \to \prop$ is a class such that $C(a)$ holds whenever $C(x)$ for all $x\in a$, then $C(v)$ for all $v:V$.
   %
     \item \emph{replacement:} given any $r : V \to V$ and $a : V$, the class 
       %
       \begin{equation*}
         \setof{ x | \exis{y : V} y \in a \land x = r(y)}
       \end{equation*}
       %
       is a $V$-set.
  %
   \item \emph{separation:} given any $a : V$ and $\UU$-small $C : V \to \propU$, the class
     %
     \begin{equation*}
       \setof{ x | x \in a \land C(x)}
     \end{equation*}
     %
     is a $V$-set.
  \end{enumerate}
\end{thm}


\begin{proof}[Sketch of proof]
  \mbox{}
  %
  \begin{enumerate}
  \item Extensionality: if $\vset(A,f) \subseteq \vset(B, g)$ then $f(a) \in \vset(B, g)$
    for every $a : A$, therefore for every $a : A$ there merely exists $b : B$ such that
    $f(a) = g(b)$. The assumption $\vset(B, g) \subseteq \vset(A, f)$ gives the other half
    of~\eqref{eq:V-path}, therefore $\vset(A,f) = \vset(B,g)$.
    
  \item Empty set: suppose $x\in \emptyset = \vset(\emptyt,!)$.  Then $\exis{a:\emptyt}x=\, !a$, which is absurd.
  
  \item Pairing: given $u$ and $v$, let $w=\vset(1+1,[\lam{x}u, \lam{x}v])$.

  \item Infinity: take $w = \vset(\nat,I)$, where $I: \nat \to V$ is given by the recursion $I(0) = \emptyset$ and $I(n+1) = I(n)\cup \{I(n)\}$.
 
  \item Union: Take any $v:V$ and any presentation $f :A\to V$ with $v=\vset(A,f)$.  Then let $\tilde{A} \defeq \sm{a:A}[fa]$, where $[fa] \mono V$ is the type of members from \autoref{def:TypeOfElements}.  $\tilde{A}$ is plainly $\UU$-small and has a projection map $\pi:\tilde{A}\to A$.  We then let $w \defeq \vset(\tilde{A}, f\circ\pi)$.
  
  \item Function set: given $u, v:V$, take the types of elements $[u] \mono V$ and $[u] \mono V$, and the function type $[u]\to [v]$.  We want to define a map
  \[
 r: ([u]\to [v])\ \longrightarrow\ V
  \] 
   with ``$r(f) = \{ \pairr{x, f(x)} \mid x : [u] \}$", but in order for this to make sense we must first define the ordered pair $\pairr{x, y}$, and then we take the map $r': x \mapsto \pairr{x, f(x)}$, and then we can put $r(f)\defeq \vset([u], r')$.  But the ordered pair can be defined in terms of unordered pairing as usual.
   
  \item $\in$-induction: let $C : V \to \prop$ be a class such that $C(a)$ holds whenever $C(x)$ for all $x\in a$, and take any $v=\vset(B,g)$.  To show that $C(v)$ by induction, assume that $C(g(b))$ for all $b:B$.  For every $x\in v$ there merely exists some $b:B$ with $x = g(b)$, and so $C(x)$.  Thus $C(v)$.

  \item Replacement: the statement ``$C$ is a $V$-set'' is a mere proposition, so we may
    proceed by induction as follows. Supposing $x$ is $\vset(A, f)$, we claim that $w
    \defeq \vset(A, r \circ f)$ is the set we are looking for.  If $C(y)$ then there merely exists
    $z : V$ and $a : A$ such that $z = f(a)$ and $y = r(z)$, therefore $y \in w$.
    Conversely, if $y \in w$ then there merely exists $a : A$ such that $y = r(f(a))$, so
    if we take $z \defeq f(a)$ we see that $C(y)$ holds.

\item Let us say that a class $C: V\to\prop$ is \define{separable} if for any $a:V$ the class
  %
  \begin{equation*}
    a \cap C \defeq\setof{x | x\in a \wedge C(x)}
  \end{equation*}
  %
  is a $V$-set.
We need to show that any $\UU$-small  $C: V \to \propU$ is separable. Indeed, given $a=\vset(A,f)$, let $A' = \exis{x:A}C(fx)$, and take $f' = f\circ i$, where $i : A' \to A$ is the obvious inclusion.  Then we can take $a' = \vset(A',f')$ and we have $x\in a\wedge C(x) \Leftrightarrow x\in a'$ as claimed.  We needed the assumption that $C$ lands in $\UU$ in order for $A' = \exis{x:A}C(fx)$ to be in $\UU$.\qedhere
\end{enumerate}
\end{proof}

It is also convenient to have a strictly syntactic criterion of separability, so that one can read off from the expression for a class that it produces a $V$-set.  One such familiar condition is being ``$\Delta_0$", which means that the expression is built up from equality $x=_V y$ and membership $x\in y$, using only mere-propositional connectives $\neg$, $\land$, $\lor$, $\Rightarrow$ and quantifers $\forall$, $\exists$ over particular sets, i.e.\ of the form $\exists(x\in a)$ and $\forall(y\in b)$.  

\begin{cor}\label{cor:Delta0sep}
If the class $C: V \to \prop$ is $\Delta_0$ in the above sense, then it is separable.
\end{cor}

\begin{proof}
Recall that we have a $\UU$-small resizing $x = y$ of identity $x = y$. Since $x\in y$ is defined in terms of $x=y$, we also have a $\UU$-small resizing of membership
%
\begin{equation*}
  x\bin\vset(A,f) \defeq \exis{a:A} x = f(a).
\end{equation*}
%
Now, let $\Phi$ be a $\Delta_0$ expression for $C$, so that as classes $\Phi = C$ (strictly speaking, we should distinguish expressions from their meanings, but we will blur the difference). Let $\widetilde{\Phi}$ be the result of replacing all occurrences of $=$ and $\in$ by their resized equivalents $\sim$ and $\bin$.  Clearly then $\widetilde{\Phi}$ also expresses $C$, in the sense that for all $x:V$, $\widetilde{\Phi}(x) \Leftrightarrow C(x)$, and hence $\widetilde{\Phi}=C$ by univalence.  It now suffices to show that $\widetilde{\Phi}$ is $\UU$-small, for then it will be separable by the theorem.  

We show that  $\widetilde{\Phi}$ is $\UU$-small by induction on the construction of the expression.  The base cases are $x = y$ and $x\bin y$, which have already been resized into $\UU$.  It is also clear that $\UU$ is closed under the mere-propositional operations (and $(-1)$-truncation), so it just remains to check the bounded quantifiers $\exists(x\in a)$ and $\forall(y\in b)$.  By definition,  
\begin{align*}
\exists(x\in a) P(x)\ &\defeq \Brck {\sm{x:V}(x\bin a \land P(x))}\\
\forall(y\in b) P(x)\ &\defeq  \prd{x:V}(x\bin a \to P(x))
\end{align*} 
Let us consider $\brck {\sm{x:V}(x\bin a \land P(x))}$.  Although the body $(x\bin a \land P(x))$ is $\UU$-small since $P(x)$ is so by the induction hypothesis, the quantification over $V$ need not stay inside $\UU$.  However, in the present case we can replace this with a quantification over the type $[a]\mono V$ of members of $a$, and easily show that
\begin{equation*}
  \sm{x:V}(x\bin a \land P(x)) = \sm{x:[a]} P(x).
\end{equation*}
The righthand side does remain in $\UU$, since both $[a]$ and $P(x)$ are in $\UU$.  The case of $\prd{x:V}(x\bin a \to P(x))$ is analogous, using $\prd{x:V}(x\bin a \to P(x)) = \prd{x:[a]}P(x)$.
\end{proof}

We have shown that in type theory with a universe $\UU$, the cumulative hierarchy $V$ is a model of a ``constructive set theory" with many of the standard axioms, but (as far as we know) without the usual \emph{strong collection} or \emph{subset collection} axioms of the set theory CZF.  We do not expect these to hold, since they seem to be related to the interaction between the type-theoretic axiom of choice $\mathsf{AC}_\infty$ and the (setoid-like) definition of equality in the usual interpretation of CZF into type theory.  Indeed, our model $(V,\in)$ is based on a cumulative hierarchy $V$ which is a higher inductive type \emph{inside} the system, rather than being an \emph{external} construction, and so it is not surprising that it differs in some ways from prior interpretations.

Finally, consider the result of adding the axiom of choice for sets to our type theory, in the form  $\mathsf{AC}$ from \autoref{subsec:emacinsets} above.  This has the consequence that $\mathsf{LEM}$ then also holds, by \autoref{thm:1surj_to_surj_to_pem}, and so $\set$ is a topos with subobject classifier $\bool$, by \autoref{thm:settopos}.  In this case, we have $\prop = \bool:\UU$, and so \emph{all classes are separable}.  Thus we have shown:

\begin{lem}\label{lem:fullsep}
In type theory with $\mathsf{AC}$, the law of \define{(full) separation} holds for $V$: given \emph{any} class $C : V \to \prop$ and $a : V$, the class $a \cap C$ is a $V$-set.
\end{lem}

\begin{thm}\label{thm:zfc}
In type theory with $\mathsf{AC}$ and a universe $\UU$, the cumulative hierarchy $V$ is a model of Zermelo-Frankel set theory with choice, ZFC.
\end{thm}

\begin{proof}
We have all the axioms listed in \autoref{thm:VisCST}, plus full separation, so we just need to show that there are power sets $\mathcal{P}(a):V$ for all $a:V$.  But since we have $\mathsf{LEM}$ these are simply function types $\mathcal{P}(a) = (a\to\bool)$.
\end{proof}

%%%%%%%%%%%%%%%%%%%%%%%%%%%%%%%%%%%%%%%%%%%%%%%%%%%%%%%%%%%%%%%%%%%%%%
\sectionNotes

The basic properties one expects of the category of sets date back to the early days of elementary topos theory.
The \emph{Elementary theory of the category of sets} referred to in \autoref{subsec:emacinsets} was introduced by Lawvere in
\cite{lawvere:etcs-long}, as a category-theoretic axiomatization of set theory.
The notion of $\Pi W$-pretopos, regarded as a predicative version of an elementary topos, was introduced in~\cite{MoerdijkPalmgren2002}; see also~\cite{palmgren:cetcs}.

The treatment of the category of sets in \autoref{sec:piw-pretopos} roughly follows that in~\cite{RijkeSpitters}.
The fact that epimorphisms are surjective (\autoref{epis-surj}) is well known in classical mathematics, but is not as trivial as it may seem to prove \emph{predicatively}.
The proof in~\cite{Mines/R/R:1988} uses the power set operation (which is impredicative), although it can also be seen as a predicative proof of the weaker statement that a map in a universe $\UU_i$ is surjective if it is an epimorphism in the next universe $\UU_{i+1}$.
A predicative proof for setoids was given by Wilander~\cite{Wilander2010}. 
Our proof is similar to Wilander's, but avoids setoids by using pushouts and univalence.

The implication in \autoref{thm:1surj_to_surj_to_pem} from $\mathsf{AC}$ to $\mathsf{LEM}$ is an adaptation to homotopy type
theory of a theorem from topos theory due to Diaconescu~\cite{Diaconescu}; it was posed as a problem already by Bishop~\cite[Problem
2,p58]{Bishop1967}.

For the intuitionistic theory of ordinal numbers, see~\cite{taylor:ordinals} and also~\cite{JoyalMoerdijk1995}.

The idea of algebraic set theory, which informs our development in \autoref{sec:cumulative-hierarchy} of the cumulative hierarchy, is due to~\cite{JoyalMoerdijk1995}, but it derives from earlier work by~\cite{AczelCZF}.


%%%%%%%%%%%%%%%%%%%%%%%%%%%%%%%%%%%%%%%%%%%%%%%%%%%%%%%%%%%%%%%%%%%%%%
\sectionExercises

\begin{ex}
Following the pattern of $\uset$, we would like to make a category $\utype$ of all types and maps between them (in a given universe $\UU$).  In order for this to be a category in the sense of \autoref{sec:cats}, however, we must first declare $\hom(X,Y) \defeq \pizero{X\to Y}$, with composition defined by induction on truncation from ordinary composition $(Y\to Z) \to (X\to Y) \to (X\to Z)$.  This was defined as the \emph{homotopy precategory of types} in \autoref{ct:hoprecat}.  It is still not a category, however, but only a precategory (its type of objects $\UU$ is not even a $0$-type).  It becomes a category by Rezk completion (see \autoref{ct:hocat}), and its type of objects can be identified with $\trunc1\type$ by \autoref{ct:ex:hocat}.  Show that the resulting category $\utype$, unlike $\uset$, is not a pretopos.
\end{ex}

\begin{ex}
  Show that if every surjection has a section in the category $\uset$, then the Axiom of choice holds.
\end{ex}

\begin{ex}
Show that with $\mathsf{AC}$, the category $\uset$ is well-pointed, in the sense that the following statement holds: for any $f, g : A\to B$, if $f \neq g$ then there is a function $a : 1\to A$ such that $f(a) \neq g(a)$.  Show that the ``slice category" $\uset/\bool$ consisting of functions $A\to \bool$ and commutative triangles does not have this property. (Hint: the terminal object in $\uset/\bool$ is the identity function $\bool \to \bool$, so in this category, there are objects $X$ that have no elements $1\to X$.)
\end{ex}

\begin{ex}
  Prove that if $(A,<_A)$ and $(B,<_B)$ are well-founded, extensional, or ordinals, then so is $A+B$, with $<$ defined by
  \begin{alignat*}{2}
    (a<a') &\defeq (a<_A a') &\qquad\text{for }& a,a':A\\
    (b<b') &\defeq (b<_B b') &\qquad\text{for }& b,b':B\\
    (a<b) &\defeq \unit &\qquad\text{for }& (a:A),(b:B)\\
    (b<a) &\defeq \emptyset &\qquad\text{for } &(a:A),(b:B)
  \end{alignat*}
\end{ex}
% \begin{proof}
%   We first prove by induction on $<_A$ that every element of $A$ is accessible in $A+B$.
%   This is easy since the only elements less than $a:A$ in $A+B$ are also in $A$.
%   We then prove by induction on $<_B$ that every element of $B$ is accessible in $A+B$.
%   This is easy since we have already proven that every element of $A$ is accessible.
% \end{proof}

\begin{ex}
  Prove that if $(A,<_A)$ and $(B,<_B)$ are well-founded, extensional, or ordinals, then so is $A\times B$, with $<$ defined by
  \[ ((a,b) <(a',b')) \defeq (a<_A a') \vee ((a=a') \wedge (b<_B b')). \]
\end{ex}
% \begin{proof}
%   We prove by induction on $<_A$ that for every $a:A$, every element of the form $(a,b)$ is accessible in $A\times B$.
%   The induction hypothesis is that for all $a'<_A a$, every pair $(a',b)$ is accessible.
%   Inside this induction, we prove by induction on $<_B$ that for every $b:B$, the element $(a,b)$ is accessible.
%   The nested induction hypothesis is that for every $b'<_B b$, the element $(a,b')$ is accessible.
%   But now, if $(a',b')< (a,b)$, then either $a<_A a'$ in which case $(a',b')$ is accessible by the first induction hypothesis, or $a=a'$ and $b'<_B b$, in which case $(a,b')$ is accessible by the second induction hypothesis.
%   Thus, by definition of accessibility, $(a,b)$ is accessible.
%   This completes both inductions.
% \end{proof}

\begin{ex}
  Define the usual algebraic operations on ordinals, and prove that they satisfy the usual properties.
\end{ex}

\begin{ex}
  Show that if $(A,<)$ is well-founded and extensional and $A:\UU$, then there is a simulation $A\to V$, where $(V,\in)$ is the cumulative hierarchy from \autoref{sec:cumulative-hierarchy} built from the universe \UU.
\end{ex}

\begin{ex}\label{ex:cumhierhit}
  Given types $A$ and $B$, define a \define{bitotal relation} to be $R:A\to B\to \prop$ such that
  \[ \Big(\fall{a:A}\exis{b:B} R(a,b) \Big) \land \Big(\fall{b:B}\exis{a:A} R(a,b) \Big). \]
  For such $A,B,R$, let $A\sqcup^R B$ be the higher inductive type generated by
  \begin{itemize}
  \item $i:A\to A\sqcup^R B$
  \item $j:B\to A\sqcup^R B$
  \item For each $a:A$ and $b:B$ such that $R(a,b)$, a path $i(a)=j(b)$.
  \end{itemize}
  Show that the cumulative hierarchy $V$ can be defined by the following more straightforward list of constructors, and that the resulting induction principle is the one given in \autoref{sec:cumulative-hierarchy}.
  \begin{itemize}
  \item For every $A : \UU$ and $f : A \to V$, there is an element $\vset(A, f) : V$.
  \item For any $A,B:\UU$ and bitotal relation $R:A\to B\to \prop$, and any map $h:A\sqcup^R B \to V$, there is a path $\id{\vset(A,h\circ i)}{\vset(B,h\circ j)}$.
  \item The 0-truncation constructor.
  \end{itemize}
\end{ex}

\begin{ex}
   In the constructive set theory CZF, the \define{axiom of strong collection} has the form:
   \begin{multline*}
   \fall{x\in v}\exis{y} R(x,y) \Rightarrow \\
   \exis{w}\big[\big(\fall{x\in v}\exis{y\in w}R(x,y)\big)\land \big(\fall{y\in w}\exis{x\in v}R(x,y) \big)\big]
   \end{multline*}
   Does it hold in the cumulative hierarchy $V$?  (We do not know the answer to this.)
\end{ex}

% Local Variables:
% TeX-master: "main"
% End:
