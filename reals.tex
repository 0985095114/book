\newcommand{\rclim}{\mathsf{lim}} % HIT constructor for Cauchy reals
\newcommand{\rcrat}{\mathsf{rat}} % Embedding of rationals into Cauchy reals
\newcommand{\rceq}{\mathsf{eq}_{\RC}} % HIT path constructor
\newcommand{\CAP}{\mathcal{C}}    % The type of Cauchy approximations
\newcommand{\Qp}{\Q_{+}}
\newcommand{\apart}{\mathrel{\#}}  % apartness
\newcommand{\dcut}{\mathsf{isCut}}  % Dedekind cut


\chapter{Real Numbers}
\label{cha:real-numbers}

Any foundation of mathematics worthy of its name must eventually address the construction
of real numbers as understood by mathematical analysis, namely as a complete archimedean
ordered field. Completeness can be understood either in the sense of Cauchy or Dedekind.
In this chapter we shall look at both, Dedekind's construction of reals as cuts, and
Cauchy reals as completion of rational numbers by limits of Cauchy sequences.

It is worth pointing out that the total space of the universal cover of the circle, which
in \autoref{subsec:pi1s1-homotopy-theory} played a role similar to ``the real numbers'' in
classical algebraic topology, is \emph{not} the type of reals we are looking for. That
type is contractible, and thus equivalent to the singleton type, so it cannot be equipped
with a non-trivial algebraic structure.


\section{The Field of Rational Numbers}
\label{sec:field-rati-numb}

We first construct the rational numbers \Q, as the reals can then be seen as a completion
of~\Q. An expert will point out that \Q could be replaced by any approximate field, i.e.,
a subring of \Q in which arbitrarily precise approximate inverses exist. An example is the
ring of dyadic rationals, which are those of the form $n/2^k$. From an implementation
point of view an approximate field is more suitable, but we leave such finesse for those
who care about the digits of~$\pi$.

We constructed the integers \Z in \autoref{sec:set-quotients} as a quotient of $\N\times
\N$, and observed that this quotient is generated by an idempotent. In
\autoref{sec:free-algebras} we saw that \Z is the free group on \unit; we could similarly
show that it is the free commutative ring on \emptyt. The field of rationals \Q is
constructed along the same lines as well, namely as the quotient
%
\[ \Q \defeq (\Z \times \N)/{\approx} \]
%
where
\[ (u,a) \approx (v,b) \defeq (u (b + 1) = v (a + 1)). \]
%
In other words, a pair $(u, a)$ represents the rational number $u / (1 + a)$. There can be
no division by zero because we cunningly added one to the denominator~$a$. Here too we
have a canonical choice of representatives, namely fractions in lowest terms. Thus we may
apply \autoref{lem:quotient-when-canonical-representatives} to obtain a set \Q, which
again has a decidable equality.

We do not bother to write down the arithmetical operations on \Q as we trust our readers
know how to compute with fractions even in the case when one is added to the denominator.
Let us just record the conclusion that there is an entirely unproblematic construction of
the ordered field of rational number \Q, with a decidable equality and decidable order.
The field \Q can also be characterized as the initial one.

\section{Dedekind Reals}
\label{sec:dedekind-reals}

Let us first recall the basic idea of Dedekind's construction. We use two-sided Dedekind
cuts, as opposed to an often used one-sided version, because the symmetry makes
constructions more elegant, and it works constructively as well as classically. A
\emph{Dedekind cut} consists of a pair $(L, U)$ of subsets $L, U \subseteq \Q$, called the
\emph{lower} and \emph{upper cut} respectively, which are:
% 
\begin{enumerate}
\item \emph{inhabited}: there are $q \in L$ and $r \in U$,
\item \emph{rounded:} $q \in L \Leftrightarrow \exists r \in Q .\; q < r \in L$
  and $r \in U \Leftrightarrow \exists r \in Q .\; U \ni q < r$,
\item \emph{disjoint:} $\lnot (q \in L \land q \in U)$, and
\item \emph{located:} $q < r \Rightarrow q \in L \lor r \in U$.
\end{enumerate}
%
Reading the roundedness condition from left to right tells us that cuts are \emph{open},
and from right to left that they are \emph{lower}, respectively \emph{upper}, sets. The
locatedness condition states that there is no large gap between $L$ and $U$. Because cuts
are always open, they never include the ``point in between'', even when it is rational. A
typical Dedekind cut looks like this:
%
\begin{center}
  \begin{tikzpicture}
    \draw[<-),line width=1.5pt] (0,0) -- (3.99,0) node[anchor=south east]{$L\ $};
    \draw[(->,line width=1.5pt] (4.01, 0) node[anchor=south west]{$\ U$} -- (14, 0) ;
  \end{tikzpicture}
\end{center}
%
We might naively translate the informal definition into type theory by saying that a cut
is a pair of maps $L, U : \Q \to \prop$. But we saw in \autoref{subsec:prop-subsets} that
$\prop$ is an ambiguous notation for $\prop_{\UU_i}$ where~$\UU_i$ is a universe. Once we
use a particular $\UU_i$ to define cuts, the type of reals will reside in the next
universe $\UU_{i+1}$, a property of reals two levels higher in $\UU_{i+2}$, a property of
subsets of reals in $\UU_{i+3}$, etc. In principle we should be able to keep track of the
universe levels, especially with the help of a proof assistant, but doing so here would
just burden us with bureaucracy that we prefer to avoid. We shall therefore make a
simplifying assumption that a single type of propositions $\Omega$ is sufficient for all
our purposes.

In fact, the construction of the Dedekind reals is quite resilient to logical
manipulations. There are several ways in which we can make sense of using a single type
$\Omega$:
%
\begin{enumerate}

\item We could identify $\Omega$ with the ambiguous $\prop$ and track all the universes
  that appear in definitions.

\item We could assume impredicativity of mere propositions, cf.\
  \ref{subsec:prop-subsets}, which essentially collapses the $\prop_{\UU_i}$'s to the
  lowest level, which we call $\Omega$.

\item A classical mathematician who is not interested in the intricacies of type-theoretic
  universes or computation may simply assume the law of excluded middle~\eqref{eq:lem} for
  mere propositions so that $\Omega \jdeq \bool$. This not only eradicates questions about
  levels of $\prop$, but also turns everything we do into the standard classical
  construction of real numbers. We discuss this point further in
  \autoref{sec:intuitionistic-vs-classical-analysis}.

\item On the other end of the spectrum one might ask for a minimal requirement that makes
  the constructions work. The condition that a mere predicate be a Dedekind cuts is
  expressible using only conjunctions, disjunctions existential quantifiers over~\Q, which
  is a countable set. Thus we could take $\Omega$ to be the initial \emph{$\sigma$-frame},
  i.e., a lattice with countable joins in which binary meets distribute over countable
  joins. (The initial $\sigma$-frame cannot be the two-point lattice $\bool$ because
  $\bool$ is not closed under countable joins, unless we assume excluded middle.) This
  would lead to a construction of~$\Omega$ as a higher inductive-inductive type, but one
  experiment of this kind in \autoref{sec:cauchy-reals} is enough.
\end{enumerate}

In all of the above cases $\Omega$ is a set.
%
Without further ado, we translate the informal definition into type theory. We use the
logical notation from \autoref{defn:logical-notation}.

\begin{defn} \label{defn:dedekind-reals}
  A \emph{Dedekind cut} is a pair $(L, U)$ of mere propositions $L : \Q \to \Omega$ and $U
  : \Q \to \Omega$ which is:
  %
  \begin{enumerate}
  \item \emph{inhabited}: $\exis{q : \Q} L(q)$ and $\exis{r : Q} U(r)$,
  \item \emph{rounded:} for all $q, r : \Q$,
    %
    \begin{equation*}
      L(q) \Leftrightarrow \exis{r : \Q} (q < r) \land L(r)
      \quad\text{and}\quad
      U(r) \Leftrightarrow \exis{r : \Q} (q < r) \land L(r),
    \end{equation*}
  \item \emph{disjoint:} $\lnot (L(q) \land U(q))$ for for all $q : \Q$,
  \item \emph{located:} $(q < r) \Rightarrow L(q) \lor U(r)$ for all $q, r : \Q$.
  \end{enumerate}
  %
  We let $\dcut(L, U)$ denote the conjunction of these conditions. The type of
  \emph{Dedekind reals} is
  %
  \begin{equation*}
    \RD \defeq \setof{ (L, U) : (\Q \to \Omega) \times (\Q \to \Omega) | \dcut(L,U)}.
  \end{equation*}
\end{defn}

It is apparent that $\dcut(L, U)$ is a mere proposition, and since $\Q \to \Omega$ is a set
the Dedekind reals form a set too. 

There is an embedding $\Q \to \RD$ which associates with each rational $q : \Q$ the cut
$(L_q, U_q)$ where
%
\begin{equation*}
  L_q(r) \defeq (r < q)
  \qquad\text{and}\qquad
  U_q(r) \defeq (q < r).
\end{equation*}
%
We shall simply write $q$ for the cut $(L_q, U_q)$ associated with a real number.

The construction of the algebraic and order-theoretic structure of Dedekind reals proceeds
as usual in intuitionistic logic. Rather than dwelling on details we point out the
differences between the classical and intuitionistic setup. Writing $L_x$ and $U_x$ for
the lower and upper cut of a real number $x : \RD$, we define addition as%
%
\begin{align*}
  L_{x + y}(q) &\defeq \exis{r, s : \Q} L_x(r) \land L_y(s) \land q = r + s, \\
  U_{x + y}(q) &\defeq \exis{r, s : \Q} U_x(r) \land U_y(s) \land q = r + s
\end{align*}
%
and the additive inverse by
%
\begin{align*}
  L_{-x}(q) &\defeq \exis{r : \Q} U_x(r) \land q = - r, \\
  U_{-x}(q) &\defeq \exis{r : \Q} L_x(r) \land q = - r.
\end{align*}
%
With these operations $(\RD, 0, {+}, {-})$ is a commutative group. Multiplication is a bit
more cumbersome:
%
\begin{align*}
  L_{x \cdot y}(q) &\defeq
  \begin{aligned}[t]
    \exis{a, b, c, d : \Q} & L_x(a) \land U_x(b) \land L_y(c) \land U_y(d) \land {}\\
                           & \qquad q < \min (a \cdot c, a \cdot d, b \cdot c, b \cdot d),
  \end{aligned} \\
  U_{x \cdot y}(q) &\defeq
  \begin{aligned}[t]
    \exis{a, b, c, d : \Q} & L_x(a) \land U_x(b) \land L_y(c) \land U_y(d) \land {}\\
                           & \qquad \max (a \cdot c, a \cdot d, b \cdot c, b \cdot d) < q.
  \end{aligned}
\end{align*}
%
These formulas are related to multiplication of intervals in interval arithmetic, where
intervals $[a,b]$ and $[c,d]$ with rational endpoints multiply to the interval
%
\begin{equation*}
  [a,b] \cdot [c,d] =
  [\min(a c, a d, b c, b d), \max(a c, a d, b c, b d)].
\end{equation*}
%
For instance, the formula for the lower cut can be read as saying that $q < x \cdot y$
when there are intervals $[a,b]$ and $[c,d]$ containing $x$ and $y$, respectively, such
that $q$ is to the left of $[a,b] \cdot [c,d]$. It is generally useful to think of an
interval $[a,b]$ such that $L_x(a)$ and $U_x(b)$ as an approximation of~$x$, see
\autoref{ex:RD-interval-arithmetic}.

We now have a commutative ring with unit $(\RD, 0, 1, {+}, {-}, {\cdot})$. To treat
multiplicative inverses, we must first introduce order. Define $\leq$ and $<$ as
%
\begin{align*}
  (x \leq y) &\ \defeq \ \fall{q : \Q} L_x(q) \Rightarrow L_y(q), \\
  (x < y)    &\ \defeq \ \exis{q : \Q} U_x(q) \land L_y(q).
\end{align*}

\begin{lem} \label{dedekind-in-cut-as-le}
  For all $x : \RD$ and $q, r : \Q$, $L_x(q) \Leftrightarrow (q < x)$ and $U_x(r)
  \Leftrightarrow (x < r)$.
\end{lem}

\begin{proof}
  If $L_x(q)$ then by roundedness there merely is $r > q$ such that $L_x(r)$, and since
  $U_q(r)$ it follows that $q < x$. Conversely, if $q < x$ then there is $r : \Q$ such
  that $U_q(r)$ and $L_x(r)$, hence $L_x(q)$ because $L_x$ is a lower set. The other half
  of the proof is symmetric.
\end{proof}

The relation $\leq$ is a partial order, and $<$ is transitive and irreflexive. The
trichotomy law
%
\begin{equation*}
  (x < y) \lor (x = y) \lor (y < x)
\end{equation*}
%
is valid if we assume excluded middle, but without it we get the intuitionistic version
%
\begin{equation} \label{eq:RD-linear-order}
  (x < y) \Rightarrow (x < z) \lor (z < y).
\end{equation}
%
To see this, suppose $x < y$. Then there merely exists $q : \Q$ such that $U_x(q)$ and
$L_y(q)$. By roundedness there merely exist $r, s : \Q$ such that $r < q < s$, $U_x(r)$
and $L_y(s)$. Then, by locatedness $L_z(r)$ or $U_z(s)$. In the first case we get $x < z$
and in the second $z < y$. At first sight it might not be clear
what~\eqref{eq:RD-linear-order} has to do with the linear order. But if we take $x \jdeq
u - \epsilon$ and $y \jdeq u + \epsilon$ for $\epsilon > 0$, then we get
%
\begin{equation*}
  (u - \epsilon < z) \lor (z < u + \epsilon).
\end{equation*}
%
This is linear order ``up to a small numerical error'', i.e., since it is unreasonable to
expect that we can actually compute with infinite precision, we should not be surprised
that we can decide~$<$ only up to whatever finite precision we have computed.

Classically multiplicative inverses exist for all numbers which are different from zero.
However, without excluded middle, a stronger condition is required. Say that $x, y : \RD$
are \emph{apart} from each other, written $x \apart y$, when $(x < y) \lor (y < x)$:
%
\begin{equation*}
  (x \apart y) \defeq (x < y) \lor (y < x).
\end{equation*}
%
If $x \apart y$ then $\lnot (x = y)$, but the converse implies a little bit of excluded
middle known, see \autoref{ex:reals-apart-neq-MP}.

\begin{thm} \label{RD-inverse-apart-0}
  A real is invertible if, and only if, it is apart from $0$.
\end{thm}

\begin{proof}
  Suppose $x \cdot y = 1$. Then there merely exist $a, b, c, d : \Q$ such that
  $a < x < b$, $c < y < d$ and $0 < \min (a c, a d, b c, b d)$. From $0 < a c$ it follows
  that $0 < a < x < c$ or $a < x < c < 0$, hence $x \apart 0$.

  Conversely, if $x \apart 0$ then
  %
  \begin{align*}
    L_{x^{-1}}(q) &\defeq
    \exis{r : \Q} U_x(r) \land ((0 < r \land q r < 1) \lor (r < 0 \land 1 < q r))
    \\
    U_{x^{-1}}(q) &\defeq
    \exis{r : \Q} L_x(r) \land ((0 < r \land q r > 1) \lor (r < 0 \land 1 > q r))
  \end{align*}
  %
  defines the desired inverse. Indeed, $L_{x^{-1}}$ and $U_{x^{-1}}$ are bounded because
  $x \apart 0$.
\end{proof}

The archimedean axiom can be stated in several ways. We find it most illuminating in the
form which says that $\Q$ is dense in $\RD$. Let $\Qp = \setof{ q : \Q | q > 0 }$ be the
type of positive rational numbers.

\begin{lem} \label{RD-multi-located}
  Suppose $x : \RD$ and $q_0 < \cdots < q_{n+1}$ are rational numbers such that $q_0 < x <
  q_{n+1}$. Then there merely exist $i, j$ such that $q_i < x < q_j$ and $j - i \leq 2$.
\end{lem}

\begin{proof}
  We proceed by induction on~$n$. If $n = 0$ or $n = 1$ then we may take $i = 0$ and $j =
  n + 1$. For the induction step, consider $q_0 < \cdots < q_{n+3}$ such that $q_0 < x <
  q_{n+3}$. Because $x$ is located and $q_1 < q_{n+2}$, merely $q_1 < x$ or $x < q_{n+2}$.
  In the first case we apply the induction hypothesis to $q_1 < \cdots < q_{n+3}$ and in
  the second to $q_0 < \cdots < q_{n+2}$.
\end{proof}

\begin{thm}[Archimedean axiom] \label{RD-archimedean}
  %
  For every $x : \RD$ and $\epsilon : \Qp$ there merely exist $q, r : \Q$ such that $q < x
  < r$ and $r - q < \epsilon$.
\end{thm}

\begin{proof}
  Because $x$ is bounded there merely exist $s, t : \Q$ such that $s < x < t$. There
  exists $n : \N$ such that $2 (t - s)/(n + 1) < \epsilon$. We apply
  \autoref{RD-multi-located} to the sequence
  %
  \begin{equation*}
    q_k \defeq \frac{n + 1 - k}{n+1} \cdot s + \frac{k}{n+1} \cdot t
  \end{equation*}
  where $k = 0, \ldots, n+1$ to obtain $i, j$ such that $q_i < x < q_j$ and $j - i \leq
  2$. Because $q_j - q_i = (j - i) (t - s)/(n + 1) < \epsilon$ we may take $q \defeq q_i$
  and $r \defeq q_j$.
\end{proof}

There are two notions of completeness of real numbers, and the Dedekind reals enjoy both.

\subsection{Dedekind reals are Cauchy complete}
\label{sec:RD-cauchy-complete}

Recall that $x : \N \to \Q$ is a \emph{Cauchy sequence} when it satisfies
%
\begin{equation} \label{eq:cauchy-sequence}
  \prd{\epsilon : \Qp} \sm{n : \N} \prd{m, k \geq n} |x_m - x_k| < \epsilon.
\end{equation}
%
Note that we did \emph{not} truncate the inner existential because we actually want to
compute rates of convergence---an approximation without an error estimate carries little
useful information. By \autoref{thm:ttac}, \eqref{eq:cauchy-sequence} yields a function $M
: \Qp \to \N$, called the \emph{modulus of convergence}, such that $m, k \geq M(\epsilon)$
implies $|x_m - x_k| < \epsilon$. From this we get $|x_{M(\delta/2)} - x_{M(\epsilon/2)}|<
\delta + \epsilon$ for all $\epsilon : \Qp$. In fact, the map $(\epsilon \mapsto
x_{M(\epsilon/2)}) : \Qp \to \Q$ carries the same information about the limit as the
original Cauchy condition~\eqref{eq:cauchy-sequence}. We shall work with these
approximation functions rather than with Cauchy sequences.

\begin{defn}\label{defn:cauchy-approximation}
  A \textbf{Cauchy approximation} is a map $x : \Qp \to \RD$ which satisfies
  %
  \begin{equation}
    \label{eq:cauchy-approx}
    \fall{\delta, \epsilon :\Qp} |x_\delta - x_\epsilon| < \delta + \epsilon.
  \end{equation}
  %
  The \emph{limit} of a Cauchy approximation $x : \Qp \to \RD$ is a number $\ell : \RD$ such
  that
  % 
  \begin{equation*}
    \fall{\epsilon, \theta : \Qp} |x_\epsilon - \ell| < \epsilon + \theta.
  \end{equation*}
\end{defn}

\begin{thm} \label{RD-cauchy-complete}
  Every Cauchy approximation in $\RD$ has a limit.
\end{thm}

\begin{proof}
  Note that we are showing existence, not mere existence, of the limit.
  Given a Cauchy approximation $x : \Qp \to \RD$, define
  % 
  \begin{align*}
    L_y(q) &\defeq \exis{\epsilon, \theta : \Qp} L_{x_\epsilon}(q + \epsilon + \theta),\\
    U_y(q) &\defeq \exis{\epsilon, \theta : \Qp} U_{x_\epsilon}(q - \epsilon - \theta).
  \end{align*}
  %
  It is clear that $L_y$ and $U_y$ are bounded, rounded, and disjoint. To establish
  locatedness, consider any $q, r : \Q$ such that $q < r$. There is $\epsilon : \Qp$ such
  that $5 \epsilon < r - q$. Since $q + 2 \epsilon < r - 2 \epsilon$ merely
  $L_{x_\epsilon}(q + 2 \epsilon)$ or $U_{x_\epsilon}(r - 2 \epsilon)$. In the first case
  we have $L_y(q)$ and in the second $U_y(r)$.

  To show that $y$ is the limit of $x$, consider any $\epsilon, \theta : \Qp$. Because
  $\Q$ is dense in $\RD$ there merely exist $q, r : \Q$ such that
  %
  \begin{equation*}
    x_\epsilon - \epsilon - \theta/2 < q < x_\epsilon - \epsilon - \theta/4
    < x_\epsilon <
    x_\epsilon + \epsilon + \theta/4 < r < x_\epsilon + \epsilon + \theta/2,
  \end{equation*}
  % 
  and thus $q < y < r$. Now either $y < x_\epsilon + \theta/2$ or $x_\epsilon - \theta/2 < y$.
  In the first case we have
  %
  \begin{equation*}
    x_\epsilon - \epsilon - \theta/2 < q < y < x_\epsilon + \theta/2,
  \end{equation*}
  %
  and in the second
  %
  \begin{equation*}
    x_\epsilon - \theta/2 < y < r < x_\epsilon + \epsilon + \theta/2.
  \end{equation*}
  %
  In either case it follows that $|y - x_\epsilon| < \epsilon + \theta$.
\end{proof}

For sake of completeness we record the classic formulation as well.

\begin{cor}
  Suppose $x : \N \to \RD$ satisfies the Cauchy condition~\eqref{eq:cauchy-sequence}. Then
  there exists $y : \RD$ such that
  %
  \begin{equation*}
    \prd{\epsilon : \Qp} \sm{n : \N} \prd{m \geq n} |x_m - y| < \epsilon.
  \end{equation*}
\end{cor}

\begin{proof}
  By \autoref{thm:ttac} there is $M : \Qp \to \N$ such that $\bar{x}(\epsilon) \defeq
  x_{M(\epsilon/2)}$ is a Cauchy approximation. Let $y$ be its limit, which exists by
  \autoref{RD-cauchy-complete}. Given any $\epsilon : \Qp$, let $n \defeq M(\epsilon/4)$
  and observe that, for any $m \geq n$,
  %
  \begin{equation*}
    |x_m - y| \leq |x_m - x_n| + |x_n - y| =
    |x_m - x_n| + |\bar{x}(\epsilon/2) - y| <
    \epsilon/4 + \epsilon/2 + \epsilon/4 = \epsilon.
  \end{equation*}
\end{proof}

\subsection{Dedekind reals are Dedekind complete}
\label{sec:RD-dedekind-complete}

We obtained $\RD$ as the type of Dedekind cuts on $\Q$. What if we repeat the construction
and form Dedekind cuts on $\RD$, will we get even more real numbers? As it turns out, the
answer is negative. Or to say it more constructively, the Dedekind completion of $\RD$ is
equivalent ot $\RD$.

A \textbf{real-valued Dedekind cut} is a pair $(L, U)$ where $L : \RD \to \Omega$ is the
lower cut and $U : \RD \to \Omega$ the upper cut, which satisfy the conditions of
\autoref{defn:dedekind-reals}, except that we replace $\Q$ by $\RD$ throughout. For
example, the roundedness condition for~$L$ states that $L(x) = \exis{y : \RD} (x < y)
\land L(y)$.

\begin{thm} \label{RD-dedekind-complete}
  %
  The Dedekind reals are Dedekind complete: for every real-valued Dedekind cut $(L, U)$
  there is a unique $x : \RD$ such that $L(y) = (y < x)$ and $U(y) = (x < y)$ for all $y :
  \RD$.
\end{thm}

\begin{proof}
  Let $i : \Q \to \RD$ be the inclusion of rational numbers into the Dedekind reals. In
  this proof it is important to distinguish between $q : \Q$ and $i(q) : \RD$, so we
  refrain from abusing notation.
  %
  Define $L_x, U_x : \Q \to \Omega$ by $L_x \defeq L \circ i$ and $U_x \defeq U \circ i$.
  It is not hard to check that $x \defeq (L_x, U_x)$ is a cut. We only verify that $L(y) =
  (y < x)$ for every $y : \RD$. The proof that $U(y) = (x < y)$ is similar.

  Consider any $y : \RD$. If $L(y)$ then by roundedness of $L$ there merely is $z : \RD$
  such that $y < z$ and $L(z)$. Because $y < z$ there merely is $q : \Q$ such that $y <
  i(q) < z$, hence $L(i(q)) \jdeq L_x(q)$ and so $i(q) < x$. By transitivity of $<$ it
  follows that $y < x$. Conversely, suppose $y < x$. Then there merely exists $q : \Q$
  such that $y < i(q) < x$, from which we get $L_x(q) \jdeq L(i(q))$. Thus $L(y)$ by
  roundedness of $L$.
\end{proof}

\section{Cauchy Reals}
\label{sec:cauchy-reals}

The Cauchy reals are, by intent, the completion of \Q under limits of Cauchy sequences.
In the classical construction of the Cauchy reals, we consider the set $\mathcal{C}$ of all Cauchy sequences in \Q and then form a suitable quotient $\mathcal{C}/{\approx}$.
Then, to show that $\mathcal{C}/{\approx}$ is Cauchy complete, we consider a Cauchy sequence $x : \N \to \mathcal{C}/{\approx}$, lift it to a sequence of sequences $\bar{x} : \N \to \mathcal{C}$, and construct the limit of $x$ using $\bar{x}$. However, the lifting of~$x$ to $\bar{x}$ uses
the Axiom of Countable Choice (the instance of \autoref{eq:ac} where $X=\N$), which we do not have at our disposal.
Every construction of reals whose last step is a quotient suffers from this deficiency.
There are three common ways out of the conundrum in constructive mathematics:
%
\begin{enumerate}
\item Pretend that the reals are a setoid $(\mathcal{C}, {\approx})$, i.e., the type of
  Cauchy approximations $\mathcal{C}$ with a coincidence relation attached to it by
  administrative decree. A sequence of reals then simply \emph{is} a sequence of Cauchy
  approximations representing them.
\item Give in to temptation and accept the Axiom of Countable Choice. After all, the axiom
  is valid in most models of constructive mathematics based on a computational viewpoint,
  such as realizability models.
\item Declare the Cauchy reals unworthy and construct the Dedekind reals instead.
  Such a verdict is perfectly valid in certain contexts, such as in sheaf-theoretic models of constructive mathematics.
  However, as we saw in \autoref{sec:dedekind-reals}, the constructive Dedekind reals have their own problems.
\end{enumerate}

Using higher inductive types, however, there is a fourth solution, which we believe to be preferable to any of the above, and interesting even to a classical mathematician.
The idea is that the Cauchy real numbers should be the \emph{free complete metric space} generated by \Q.
In general, the construction of a free gadget of any sort requires applying the gadget operations repeatedly many times to the generators.
For instance, the elements of the free group on a set $X$ are not just binary products and inverses of elements of $X$, but words obtained by iterating the product and inverse constructions.
Thus, we might naturally expect the same to be true for Cauchy completion, with the relevant ``operation'' being ``take the limit of a Cauchy sequence''.
(In this case, the iteration would have to take place transfinitely, since even after infinitely many steps there will be new Cauchy sequences to take the limit of.)

The argument referred to above shows that if Countable Choice holds, then Cauchy completion is very special: when building the completion of a space, it suffices to stop applying the operation after \emph{one step}.
This may be regarded as analogous to the fact that free monoids and free groups can be given explicit descriptions in terms of (reduced) words.
However, we saw in \autoref{sec:free-algebras} that higher inductive types allow us to construct free gadgets \emph{directly}, whether or not there is also an explicit description available.
In this section we show that the same is true for the Cauchy reals (a similar technique would construct the Cauchy completion of any metric space).
Specifically, higher inductive types allow us to \emph{simultaneously} add limits of Cauchy sequences and quotient by the coincidence relation, so that we can avoid the problem of lifting a sequence of reals to a sequence of representatives.


\subsection{Construction of Cauchy Reals}
\label{sec:constr-cauchy-reals}

The construction of the Cauchy reals $\RC$ as a higher inductive type is a bit more subtle than that of the free algebraic structures considered in \autoref{sec:free-algebras}.
We intend to include a ``take the limit'' constructor whose input is a Cauchy sequence of reals, but the notion of ``Cauchy sequence of reals'' depends on having some way to measure the ``distance'' between real numbers.
In general, of course, the distance between two real numbers will be another real number, leading to a potentially problematic circularity.

However, what we actually need for the notion of Cauchy sequence (or Cauchy approximation) of reals is not the general notion of ``distance'', but a way to say that ``the distance between two real numbers is less than $\epsilon$'' for any $\epsilon:\Qp$.
This can be represented by a family of binary relations, which we will denote $\mathord{\sim_\epsilon} : \RC\to\RC\to \prop$.
The intended meaning of $x \sim_\epsilon y$ is $|x - y| < \epsilon$, but since we do not have notions of subtraction, absolute value, or inequality available yet (we are only just defining $\RC$, after all), we will have to define these relations $\sim_\epsilon$ at the same time as we define $\RC$ itself.
And since $\sim_\epsilon$ is a type family indexed by two copies of $\RC$, we cannot do this with an ordinary mutual (higher) inductive definition; instead we have to use a \emph{higher inductive-inductive definition}.

Recall from \autoref{sec:generalizations} that the ordinary notion of inductive-inductive definition allows us to define a type and a type family indexed by it by simultaneous induction.
Of course, the ``higher'' version of this allows both the type and the family to have path-constructors as well as point-constructors.
We will not attempt to formulate any general theory of higher inductive-inductive definitions, but hopefully the description we will give of $\RC$ and $\sim_\epsilon$ will make the idea transparent.

\begin{rmk}
  We might also consider a \emph{higher inductive-recursive definition}, in which $\sim_\epsilon$ is defined using the \emph{recursion} principle of $\RC$, simultaneously with the \emph{inductive} definition of $\RC$.
  We choose the inductive-inductive route instead for two reasons.
  Firstly, higher inductive-recursive definitions seem to be more difficult to justify in homotopical semantics.
  Secondly, and more importantly, the inductive-inductive definition yields a more powerful induction principle, which we will need in order to develop even the basic theory of Cauchy reals.
\end{rmk}

\begin{defn}\label{defn:cauchy-reals}
  Let $\RC$ and the relation $\mathord\sim:\Qp \times \RC \times \RC \to \type$ be the following higher inductive-inductive type family.
  The type $\RC$ of \textbf{Cauchy reals} is generated by the following constructors:
  \begin{itemize}
  \item \emph{rational points:} 
    for any $q : \Q$ there is a real $\rcrat(q)$.
  \item \emph{limit points}:
    for any $x : \Qp \to \RC$ such that
    %
    \begin{equation}
      \label{eq:RC-cauchy}
      \fall{\delta, \epsilon : \Qp} x_\delta \sim_{\delta + \epsilon} x_\epsilon
    \end{equation}
    %
    there is a point $\rclim(x) : \RC$. We call $x$ a \emph{Cauchy approximation}.
    %
  \item \emph{paths:}
    for $u, v : \RC$ such that
    %
    \begin{equation}
      \label{eq:RC-path}
      \fall{\epsilon : \Qp} u \sim_\epsilon v
    \end{equation}
    %
    then there is a path $\rceq(u, v) : \id[\RC]{u}{v}$.
  \end{itemize}
  Simultaneously, the type family $\mathord\sim:\RC\to\RC\to\Qp \to\type$ is generated by the following constructors:
  \begin{itemize}
  \item For $q,r,\epsilon:\Qp$, if $-\epsilon < q - r < \epsilon$, then $\rcrat(q) \sim_\epsilon \rcrat(r)$.
  \item For $q:\Q$ and $\delta,\epsilon,\theta:\Qp$ and $y$ a Cauchy approximation, if $\rcrat(q) \sim_{\epsilon - \delta - \theta} y_\delta$, then $\rcrat(q) \sim_{\epsilon} \rclim(y)$.
  \item For $r:\Q$ and $\delta,\epsilon,\theta:\Qp$ and $x$ a Cauchy approximation, if $x_\delta \sim_{\epsilon - \delta - \theta} \rcrat(r)$, then $\rclim(x) \sim_\epsilon \rcrat(r)$.
  \item For $\epsilon,\delta,\eta,\theta:\Qp$ and $x,y$ Cauchy approximations, if $x_\delta \sim_{\epsilon - \delta - \eta - \theta} y_\eta$, then $\rclim(x) \sim_\epsilon \rclim(y)$.
  \item The propositional truncation constructor: for $\epsilon:\Qp$ and $x,y:\RC$, and $\xi,\zeta:x\sim_{\epsilon} y$, we have $\xi=\zeta$.
  \end{itemize}
\end{defn}

The first constructor of $\RC$ says that any rational number can be regarded as a real number.
The second says that from any Cauchy approximation to a real number, we can obtain a new real number called its ``limit''.
And the third expresses the idea that if two Cauchy approximations coincide, then their limits are equal.
(Note that the notion of ``coincidence'' is \emph{simpler} than that of \autoref{defn:cauchy-approximation}, since we are free to use the relation $\sim$ between Cauchy reals themselves.
Moreover, it also handles equalites between limits and rational points.)

The first four constructors of $\sim$ specify when two rational numbers are close, when a rational is close to a limit, and when two limits are close.
In the case of two rational numbers, this is just the usual notion of $\epsilon$-closeness for rational numbers, whereas the other cases can be derived by noting that each approximant $x_\delta$ is supposed to be within $\delta$ of the limit $\rclim(x)$.
The mysterious $\theta$s are there to give extra ``wiggle room'', ensuring that $\sim$ behaves like $<$ and not like $\leq$.

We remind ourselves of proof-relevance: a real number obtained from $\rclim$ is represented not
just by a Cauchy approximation $x$, but also a proof $p$ of~\eqref{eq:RC-cauchy}, so we
should technically have written $\rclim(x,p)$ instead of just $\rclim(x)$.
A similar observation also applies to $\rceq$ and~\eqref{eq:RC-path}, but we shall write just
$\rceq : u = v$ instead of $\rceq(u, v, p) : u = v$. These abuses of notation are
mitigated by the fact that we are omitting mere propositions and information that is
readily guessed.
Likewise, the last constructor of $\mathord{\sim_\epsilon}$ justifies our leaving the other four nameless.

We are immediately able to populate $\RC$ with many real numbers. For suppose $q : \N \to
\Q$ is a traditional Cauchy sequence of rational numbers, and let $M : \Qp \to \N$ be its
modulus of convergence. Then $\rcrat \circ q \circ M : \Qp \to \RC$ is a Cauchy
approximation, using the first constructor of $\sim$ to produce the necessary witness.
Thus, $\rclim(\rcrat \circ q \circ m)$ is a real number. Various famous
real numbers $\sqrt{2}$, $\pi$, $e$, \dots are are all limits of such Cauchy sequences of
rationals.

\medskip

In order to do anything useful with $\RC$, of course, we need to give its induction principle.
As is the case whenever we inductively define two or more objects at once, the basic induction principle for $\RC$ and $\mathord\sim$ requires a simultaneous induction over both at once.
Thus, we should expect it to say that assuming two type families over $\RC$ and $\sim$, respectively, together with data corresponding to each constructor, there exist sections of both of these families.
However, since $\sim$ is indexed on on two copies of $\RC$, the precise dependencies of these families is a bit subtle.
The induction principle will apply to any pair of type families:
\begin{align*}
A&:\RC\to\type\\
B&:\prd{x,y:\RC}{a:A(x)}{b:A(y)}{\epsilon:\Qp} (x\sim_\epsilon y) \to \type.
\end{align*}
The type of $A$ is obvious, but the type of $B$ requires a little thought.
Since $B$ must depend on $\sim$, but $\sim$ in turn depends on two copies of $\RC$ and one copy of $\Qp$, it is fairly obvious that $B$ must also depend on the variables $x,y:\RC$ and $\epsilon:\Qp$ as well as an element of $(x\sim_\epsilon y)$.
What is slightly less obvious is that $B$ must also depend on two variables $a:A(x)$ and $b:A(y)$ of type $A$.

This may be more evident if we consider the non-dependent case (the recursion principle), where $A$ is a simple type.
In this case we would expect $B$ not to depend on $x,y:\RC$ or $x\sim_\epsilon y$.
But the recursion principle (along with its associated uniqueness principle) is supposed to say that $\RC$ with $\sim_\epsilon$ is an ``initial object'' in some category, so in this case the dependency structure of $A$ and $B$ should mirror that of $\RC$ and $\sim_\epsilon$: that is, we should have $B:A\to A\to \Qp \to \type$.
Combining this observation with the fact that, in the dependent case, $B$ must also depend on $x,y:\RC$ and $x\sim_\epsilon y$, leads inevitably to the type given above for $B$.

Now, given $A$ and $B$ as above, the hypotheses of the induction principle consist of the following data, one for each constructor of $\RC$ or $\sim$.
\begin{itemize}
\item For any $q : \Q$, an element $f_q:A(\rcrat(q))$.
\item For any Cauchy approximation $x$, and any $a:\prd{\epsilon:\Qp} A(x_\epsilon)$ such that
  \begin{equation}
    \fall{\delta, \epsilon : \Qp} B(x_\delta,x_\epsilon,a_\delta,a_\epsilon,\delta+\epsilon,\blank),\label{eq:depCauchyappx}
  \end{equation}
  an element $f_{x,a}:A(\rclim(x))$.  We call such $a$ a \emph{dependent Cauchy approximation} over $x$.
\item For $u, v : \RC$ and $h:\fall{\epsilon : \Qp} u \sim_\epsilon v$, and $a:A(u)$ and $b:A(v)$ such that $\fall{\epsilon:\Qp} B(u,v,a,b,\epsilon,h(\epsilon))$, a dependent path $\dpath{A}{\rceq(u,v)}{a}{b}$.
\item For $q,r:\Q$ and $\epsilon:\Qp$, if $-\epsilon < q - r < \epsilon$, an element of $B(\rcrat(q),\rcrat(r),f_q,f_r,\epsilon,\blank)$.
\item For $q:\Q$ and $\delta,\epsilon,\theta:\Qp$ and $y$ a Cauchy approximation, and $b$ a dependent Cauchy approximation over $y$, if $\rcrat(q) \sim_{\epsilon - \delta - \theta} y_\delta$, then
  \[B(\rcrat(q),y_\delta,f_q,b_\delta,\epsilon-\delta-\theta,\blank)
  \to
  B(\rcrat(q),\rclim(y),f_q,f_{y,b},\epsilon,\blank).\]
\item Similarly, for $r:\Q$ and $\delta,\epsilon,\theta:\Qp$ and $x$ a Cauchy approximation, and $a$ a dependent Cauchy approximation over $x$, if $x_\delta \sim_{\epsilon - \delta - \theta} \rcrat(r)$, then
  \[B(x_\delta,\rcrat(r),a_\delta,f_r,\epsilon-\delta-\theta,\blank) \to
  B(\rclim(y),\rcrat(q),f_{x,a},f_r,\epsilon,\blank).
  \]
\item For $\epsilon,\delta,\eta,\theta:\Qp$ and $x,y$ Cauchy approximations, and $a$ and $b$ dependent Cauchy approximations over $x$ and $y$ respectively, if $x_\delta \sim_{\epsilon - \delta - \eta - \theta} y_\eta$, then
  \[ B(x_\delta,y_\eta,a_\delta,b_\eta,{\epsilon - \delta - \eta - \theta},\blank)\to
  B(\rclim(x),\rclim(y),f_{x,a},f_{y,b},\epsilon,\blank).\]
\item For $\epsilon:\Qp$ and $x,y:\RC$ and $\xi,\zeta:x\sim_{\epsilon} y$, and $a:A(x)$ and $b:A(y)$, any two elements of $B(x,y,a,b,\epsilon,\xi)$ and $B(x,y,a,b,\epsilon,\zeta)$ are dependently equal over $\xi=\zeta$.
  Note that as usual, this is essentially equivalent to asking that $B$ is a mere relation.
\end{itemize}
Under these hypotheses, we deduce functions
\begin{align*}
  f&:\prd{x:\RC} A(x)\\
  g&:\prd{x,y:\RC}{\epsilon:\Qp}{\xi:x\sim_{\epsilon} y} B(x,y,f(x),f(y),\epsilon,\xi)
\end{align*}
which compute as expected:
\begin{align*}
  f(\rcrat(q)) &\defeq f_q\\
  f(\rclim(x)) &\defeq f_{x,(f,g)[x]}
\end{align*}
Here $(f,g)[x]$ denotes the result of applying $f$ and $g$ to a Cauchy approximation $x$ to obtain a dependent Cauchy approximation over $x$.
That is, we define $(f,g)[x]_\epsilon \defeq f(x_\epsilon) : A(x_\epsilon)$, and then for any $\epsilon,\delta:\Qp$ we have $g(x_\epsilon,x_\delta,\epsilon+\delta,\blank)$ to witness the fact that $(f,g)[x]$ is a dependent Cauchy approximation.

This induction principle is admittedly quite a mouthful.
To help make sense of it, we observe that it contains as special cases two separate induction principles for $\RC$ and for $\sim$.
Firstly, suppose given only a type family $A:\RC\to\type$, and define $B$ to be constant at \unit.
Then much of the required data becomes trivial, and we are left with:
\begin{itemize}
\item For any $q : \Q$, an element $f_q:A(\rcrat(q))$.
\item For any Cauchy approximation $x$, and any $a:\prd{\epsilon:\Qp} A(x_\epsilon)$, an element $f_{x,a}:A(\rclim(x))$.
\item For $u, v : \RC$ and $h:\fall{\epsilon : \Qp} u \sim_\epsilon v$, and $a:A(u)$ and $b:A(v)$, a dependent path $\dpath{A}{\rceq(u,v)}{a}{b}$.
\end{itemize}
Given these data, the induction principle yield a function $f:\prd{x:\RC} A(x)$ such that
\begin{align*}
  f(\rcrat(q)) &\defeq f_q\\
  f(\rclim(x)) &\defeq f_{x,f(x)}.
\end{align*}
We call this principle \emph{$\RC$-induction}; it says essentially that if we take $\sim_\epsilon$ as given, then $\RC$ is inductively generated by its constructors.

In particular, if $A$ is a mere property, the third hypothesis in $\RC$-induction is trivial.
Thus, we may prove mere properties of real numbers by simply proving them for rationals and for limits of Cauchy approximations.
Here is an example.

\begin{lem}
  For any $u:\RC$ and $\epsilon:\Qp$, we have $u\sim_\epsilon u$.
\end{lem}
\begin{proof}
  Define $A(u) \defeq \fall{\epsilon:\Qp} (u\sim_\epsilon u)$.
  Since this is a mere proposition (by the last constructor of $\sim$), by $\RC$-induction, it suffices to prove it when $u$ is $\rcrat(q)$ and when $u$ is $\rclim(x)$.
  In the first case, we obviously have $|q-q|<\epsilon$ for any $\epsilon$, hence $\rcrat(q) \sim_\epsilon \rcrat(q)$ by the first constructor of $\sim$.
  %
  And in the second case, we may assume inductively that $x_\delta \sim_\epsilon x_\delta$ for all $\delta,\epsilon:\Qp$.
  Then in particular, we have $x_{\epsilon/4} \sim_{\epsilon/4} x_{\epsilon/4}$, whence $\rclim(x) \sim_{\epsilon} \rclim(x)$ by the fourth constructor of $\sim$.
\end{proof}

\begin{thm}\label{thm:Cauchy-reals-are-a-set}
  $\RC$ is a set.
\end{thm}
\begin{proof}
  We have just shown that the mere relation $P(u,v) \defeq \fall{\epsilon:\Qp} (u\sim_\epsilon u)$ is reflexive.
  Since it implies identity, by the path-constructor of $\RC$, the result follows from \autoref{thm:h-set-refrel-in-paths-sets}.
\end{proof}

We can also show that although $\RC$ may not be a quotient of the set of Cauchy sequences of \emph{rationals}, it is nevertheless a quotient of the set of Cauchy sequences of \emph{reals}.
(Of course, this is not a valid \emph{definition} of $\RC$, but it is a useful property.)
We define the type of Cauchy approximations to be
% 
\begin{equation*}
  \CAP \defeq
  \setof{ x : \Qp \to \RC |
    \fall{\epsilon, \delta : \Qp} x_\delta \sim_{\delta + \epsilon} x_\epsilon
  }.
\end{equation*}
The second constructor of $\RC$ gives a function $\rclim:\CAP\to\RC$.

\begin{lem} \label{RC-lim-onto}
  Every real merely is a limit point: $\fall{u : \RC} \exis{x : \CAP} u = \rclim(x)$.
  In other words, $\rclim:\CAP\to\RC$ is surjective.
\end{lem}
\begin{proof}
  By $\RC$-induction, we may divide into cases on $u$.
  Of course, if $u$ is a limit $\rclim(x)$, the statement is trivial.
  So suppose $u$ is a rational point $\rcrat(q)$; we claim $u$ is equal to $\rclim(\lam{\blank} \rcrat(q))$.
  By the path-constructor of $\RC$, it suffices to show $\rcrat(q) \sim_\epsilon \rclim(\lam{\blank} \rcrat(q))$ for all $\epsilon:\Qp$.
  And by the second constructor of $\sim$, for this it suffices to find $\delta,\theta:\Qp$ such that $\rcrat(q)\sim_{\epsilon-\delta-\theta} \rcrat(q)$.
  But by the first constructor of $\sim$, we may take, e.g., $\delta \defeq \epsilon/3$ and $\theta \defeq \epsilon/3$ so that $\delta+\theta <\epsilon$.
\end{proof}

% 

\begin{lem} \label{RC-lim-factor}
  If $A$ is a set and $f : \CAP \to A$ respects coincidence of Cauchy approximations, in the sense that
  %
  \begin{equation*}
    \fall{x, y : \CAP} \rclim(x) = \rclim(y) \Rightarrow f(x) = f(y),
  \end{equation*}
  %
  then $f$ factors uniquely through $\rclim : \CAP \to \RC$.
\end{lem}
\begin{proof}
  Since $\rclim$ is surjective, by \autoref{lem:images_are_coequalizers}, $\RC$ is the quotient of $\CAP$ by the kernel of $\rclim$.
  But this is exactly the statement of the lemma.
\end{proof}

For the second special case of the induction principle, suppose instead that we are given only $C:\RC\to\RC\to\Qp\to\prop$, and define $A(\blank)\defeq\unit$ and $B(u,v,\blank,\blank,\epsilon,\blank)\defeq C(u,v,\epsilon)$.
Then the required data reduces to:
\begin{itemize}
\item For $q,r:\Q$ and $\epsilon:\Qp$, if $-\epsilon < q - r < \epsilon$, then $C(\rcrat(q),\rcrat(r),\epsilon)$.
\item For $q:\Q$ and $\delta,\epsilon,\theta:\Qp$ and $y$ a Cauchy approximation, if $\rcrat(q) \sim_{\epsilon - \delta - \theta} y_\delta$ and
  $C(\rcrat(q),y_\delta,\epsilon-\delta-\theta)$,
  then $C(\rcrat(q),\rclim(y),\epsilon)$.
\item Similarly, for $r:\Q$ and $\delta,\epsilon,\theta:\Qp$ and $x$ a Cauchy approximation, if $x_\delta \sim_{\epsilon - \delta - \theta} \rcrat(r)$ and
  $C(x_\delta,\rcrat(r),\epsilon-\delta-\theta)$,
  then $C(\rclim(y),\rcrat(q),\epsilon)$.
\item For $\epsilon,\delta,\eta,\theta:\Qp$ and $x,y$ Cauchy approximations, if $x_\delta \sim_{\epsilon - \delta - \eta - \theta} y_\eta$ and
  $C(x_\delta,y_\eta,{\epsilon - \delta - \eta - \theta})$,
  then $C(\rclim(x),\rclim(y),\epsilon)$.
\end{itemize}
The resulting conclusion is $\fall{u,v:\RC}{\epsilon:\Qp} (u\sim_\epsilon v) \to C(u,v,\epsilon)$.
We call this principle \emph{$\sim$-induction}; it says essentially that if we take $\RC$ as given, then $\sim_\epsilon$ is inductively generated (as a family of types) by \emph{its} constructors.
For example, we can use this to show that $\sim$ is symmetric.

\begin{lem}\label{thm:RCsim-symmetric}
  For any $u,v:\RC$ and $\epsilon:\Qp$, we have $(u\sim_\epsilon v) = (v\sim_\epsilon u)$.
\end{lem}
\begin{proof}
  Since both are mere propositions, by symmetry it suffices to show one implication.
  Thus, let $C(u,v,\epsilon) \defeq (v\sim_\epsilon u)$.
  By $\sim$-induction, we may reduce to the case that $u\sim_\epsilon v$ is derived from one of the four interesting constructors of $\sim$.
  In the first case when $u$ and $v$ are both rational, the result is trivial (we can apply the first constructor again).
  In the other three cases, the inductive hypothesis (together with commutativity of addition in $\Q$) yields exactly the input to another of the constructors of $\sim$ (the second and third constructors switch, while the fourth stays put).
\end{proof}

The general induction principle, which we may call \emph{$(\RC,\sim)$-induction}, is therefore a sort of joint $\RC$-induction and $\sim$-induction.
Consider, for instance, its non-dependent version, which we call \emph{$(\RC,\sim)$-recursion}, which is the one that we will have the most use for.
Ordinary $\RC$-recursion tells us that to define a function $f : \RC \to A$ it suffices to:
\begin{enumerate}
\item for every $q : \Q$ construct $f(\rcrat(q)) : A$,
\item for every Cauchy approximation $x : \Qp \to \RC$, construct $f(x) : A$,
  assuming that $f(x_\epsilon)$ has already been defined for all $\epsilon : \Qp$,
\item prove $f(u) = f(v)$ for all $u, v : \RC$ satisfying $\fall{\epsilon:\Qp} u\sim_\epsilon v$.\label{item:rcrec3}
\end{enumerate}
However, it is generally quite difficult to show~\ref{item:rcrec3} without knowing something about how $f$ acts on $\epsilon$-close Cauchy reals.
The enhanced principle of $(\RC,\sim)$-recursion remedies this deficiency, allowing us to specify an \emph{arbitrary} ``way in which $f$ acts on $\epsilon$-close Cauchy reals'', which we can then prove to be the case by a simultaneous induction with the definition of $f$.
This is the family of relations $B:A\to A\to \Qp\to\prop$.
\newcommand{\bsim}{\frown}
\newcommand{\bbsim}{\smile}
If we write the mere property $B(a,b,\epsilon)$ as ``$a \bsim_\epsilon b$'', then defining a function $f:\RC\to A$ by $(\RC,\sim)$-recursion requires the following cases.
\begin{itemize}
\item For every $q : \Q$, construct $f(\rcrat(q)) : A$.
\item For every Cauchy approximation $x : \Qp \to \RC$, construct $f(x) : A$, assuming inductively that $f(x_\epsilon)$ has already been defined for all $\epsilon : \Qp$ and form a ``Cauchy approximation with respect to $\bsim$'', i.e.\ that $\fall{\epsilon,\delta:\Qp} (f(x_\epsilon) \bsim_{\epsilon+\delta} f(x_\delta))$.
\item Prove that the relations $\bsim$ are \emph{separated}, i.e.\ that $(\fall{\epsilon:\Qp} a\bsim_\epsilon b) \to (a=b)$ for any $a,b:A$.
\item Prove that if $-\epsilon<|q-r|<\epsilon$ for $q,r:\Q$, then $f(\rcrat(q))\bsim_\epsilon f(\rcrat(r))$.
\item For any $q:\Q$ and any Cauchy approximation $y$, prove that $f(\rcrat(q)) \bsim_\epsilon f(\rclim(y))$, assuming inductively that $\rcrat(q)\sim_{\epsilon-\delta-\theta} y_\delta$ and $f(\rcrat(q)) \bsim_{\epsilon-\delta-\theta} f(y_\delta)$ for some $\epsilon,\delta,\theta:\Qp$, and that $\epsilon\mapsto f(x_\epsilon)$ is a Cauchy approximation with respect to $\bsim$.
\item For any Cauchy approximation $x$ and any $r:\Q$, prove that $f(\rclim(x)) \bsim_\epsilon f(\rcrat(r))$, assuming inductively that $x_\delta \sim_{\epsilon-\delta-\theta} \rcrat(r)$ and $f(x_\delta) \bsim_{\epsilon-\delta-\theta} f(\rcrat(r))$ for some $\epsilon,\delta,\theta:\Qp$, and that $\epsilon\mapsto f(x_\epsilon)$ is a Cauchy approximation with respect to $\bsim$.
\item For any Cauchy approximations $x,y$, prove that $f(\rclim(x)) \bsim_\epsilon f(\rclim(y))$, assuming inductively that $x_\delta \sim_{\epsilon-\delta-\eta-\theta} y_\eta$ and $f(x_\delta) \bsim_{\epsilon-\delta-\eta-\theta} f(y_\eta)$ for some $\epsilon,\delta,\eta,\theta:\Qp$, and that $\epsilon\mapsto f(x_\epsilon)$ and $\epsilon\mapsto f(y_\epsilon)$ are Cauchy approximations with respect to $\bsim$.
\end{itemize}
Note that in the last four proofs, we are free to use the specific definitions of $f(\rcrat(q))$ and $f(\rclim(x))$ given in the first two data.
However, the proof of separatedness must apply to \emph{any} two elements of $A$, without any relation to $f$: it is a sort of ``admissibility'' condition on the family of relations $\bsim$.
Thus, we often verify it first, immediately after defining $\bsim$, before going on to define $f(\rcrat(q))$ and $f(\rclim(x))$.

Under the above hypotheses, $(\RC,\sim)$-recursion yields a function $f:\RC\to A$ such that $f(\rcrat(q))$ and $f(\rclim(x))$ are judgmentally equal to the definitions given for them in the first two clauses.
Moreover, we may also conclude
\begin{equation}
  \fall{u,v:\RC}{\epsilon:\Qp} (u\sim_\epsilon v) \to (f(u) \bsim_\epsilon f(v)).\label{eq:RC-sim-recursion-extra}
\end{equation}

As a paradigmatic example, $(\RC,\sim)$-recursion allows us to extend functions defined on $\Q$ to all of $\RC$, as long as they are sufficiently continuous.

\begin{defn}\label{defn:lipschitz}
  A function $f:\Q\to\RC$ is \textbf{Lipschitz} if there exists $L:\Qp$ (the \textbf{Lipschitz constant}) such that
  \[ |q - r|<\epsilon \Rightarrow (f(q) \sim_{L\epsilon} f(r)) \]
  for all $\epsilon:\Q$ and $q,r:\Q$.
  %
  Similarly, $g:\RC\to\RC$ is \textbf{Lipschitz} if there exists $L:\Qp$ such that
  \[ (u\sim_\epsilon v) \Rightarrow (g(u) \sim_{L\epsilon} g(v)) \]
  for all $\epsilon:\Q$ and $u,v:\RC$..
\end{defn}

In particular, note that by the first constructor of $\sim$, if $f:\Q\to\Q$ is Lipschitz in the obvious sense, then so is the composite $\Q\xrightarrow{f} \Q \hookrightarrow \RC$.

\begin{lem}\label{RC-extend-Q-Lipschitz}
  Suppose $f : \Q \to \RC$ is Lipschitz with constant $L : \Qp$.
  Then there exists a Lipschitz map $\bar{f} : \RC \to \RC$, also with constant $L$, such that $\bar{f}(\rcrat(q)) \jdeq f(q)$ for all $q:\Q$.
\end{lem}

\begin{proof}
  % Uniqueness follows directly from \autoref{RC-Lipschitz-eq}.
  We define $\bar{f}$ by $(\RC,\sim)$-recursion, with codomain $A\defeq \RC$.
  We choose the relation $\mathord\bsim: \RC \to \RC \to \Qp \to \prop$ to be
  \begin{align*}
    (a \bsim_\epsilon b) &\defeq (a \sim_{L\epsilon} b).
  \end{align*}
  For $q : \Q$, we define
  %
  \begin{equation*}
    \bar{f}(\rcrat(q)) \defeq \rcrat(f(q)).
  \end{equation*}
  %
  For a Cauchy approximation $x : \Qp \to \RC$, we define
  % 
  \begin{equation*}
    \bar{f}(\rclim(x)) \defeq \rclim(\lamu{\epsilon : \Qp} \bar{f}(x_{\epsilon/L})).
  \end{equation*}
  %
  For this to make sense, we must verify that $y \defeq \lamu{\epsilon : \Qp} \bar{f}(x_{\epsilon/L})$ is a Cauchy approximation.
  However, the inductive hypothesis for this step is that for any $\delta,\epsilon:\Qp$ we have $\bar{f}(x_\delta) \bsim_{\delta+\epsilon} \bar{f}(x_\epsilon)$, i.e.\ $\bar{f}(x_\delta) \sim_{L\delta+L\epsilon} \bar{f}(x_\epsilon)$.
  Thus we have
  \[y_\delta \jdeq f(x_{\delta/L}) \sim_{\delta + \epsilon} f(x_{\epsilon/L})   \jdeq y_\epsilon. \]

  For the proof of separatedness, we simply observe that $\fall{\epsilon:\Qp} a\bsim_\epsilon b$ means $\fall{\epsilon:\Qp} a\sim_{L\epsilon} b$, which implies $\fall{\epsilon:\Qp}a\sim_\epsilon b$ and thus $a=b$.

  To complete the $(\RC,\sim)$-recursion, it remains to verify the four conditions on $\bsim$.
  This basically amounts to proving that $\bar f$ is Lipschitz for all the four constructors of $\sim$.
  \begin{enumerate}
  \item When $u$ is $\rcrat(q)$ and $v$ is $\rcrat(r)$ with $-\epsilon < |q-r| <\epsilon$, the assumption that $f$ is Lipschitz yields $f(q) \sim_{L\epsilon} f(r)$, hence $\bar{f}(\rcrat(q)) \bsim_\epsilon \bar{f}(\rcrat(r))$ by definition.
  \item When $u$ is $\rclim(x)$ and $v$ is $\rcrat(q)$ with $x_\eta \sim_{\epsilon - \eta - \theta} \rcrat(q)$, then the
      induction hypothesis is $\bar{f}(x_\eta) \sim_{L \epsilon - L \eta - L \theta} \rcrat(f(q))$, which proves $\bar{f}(\rclim(x)) \sim_{L \epsilon}
      \bar{f}(\rcrat(q))$ by the third constructor of $\sim$.
  \item The symmetric case when $u$ is rational and $v$ is a limit is essentially identical.
  \item When $u$ is $\rclim(x)$ and $v$ is $\rclim(y)$, with $\delta, \eta, \theta : \Qp$ such that $x_\delta \sim_{\epsilon - \delta - \eta - \theta} y_\eta$,
      the induction hypothesis is $\bar{f}(x_\delta) \sim_{L \epsilon - L \delta - L \eta - L \theta} \bar{f}(y_\eta)$, which proves $\bar{f}(\rclim(x)) \sim_{L
        \epsilon} \bar{f}(\rclim(y))$ by the fourth constructor of $\sim$.
  \end{enumerate}
  This completes the $(\RC,\sim)$-recursion, and hence the construction of $\bar f$.
  The desired equality $\bar f(\rcrat(q))\jdeq f(q)$ is exactly the first computation rule for $(\RC,\sim)$-recursion, and the additional condition~\eqref{eq:RC-sim-recursion-extra} says exactly that $\bar f$ is Lipschitz with constant $L$.
\end{proof}

At this point we have gone about as far as we can without a better characterization of $\sim$.
We have specified, in the constructors of $\sim$, the conditions under which we want Cauchy reals of the two different forms to be $\epsilon$-close.
However, how do we know that in the resulting inductive-inductive type family, these are the \emph{only} witnesses to this fact?
We have seen that inductive type families (such as identity types, see \autoref{sec:identity-systems}) and higher inductive types have a tendency to contain ``more than was put into them'', so this is not an idle question.

In order to characterize $\sim$ more precisely, we will define a family of relations $\approx_\epsilon$ on $\RC$ \emph{recursively}, so that they will compute on constructors, and prove that this family is equivalent to $\sim_\epsilon$.

% More specifically, we will define $\mathord\approx:\RC\to\RC\to\Qp\to\prop$, together with some of its basic properties, by double $(\RC,\sim)$-recursion.
% Roughly, the $\RC$-part of the definition will be the family of relations $\approx_\epsilon$, together with the fact that they are \emph{rounded}.
% The $\sim$-part of the definition will be that these relations satisfy the ``triangle inequality'' with respect to $\sim_\epsilon$.
% These two facts will be what enable us to show that $\approx_\epsilon$ respects the path constructor of $\RC$, allowing the recursion to go through.

\begin{thm}\label{defn:RC-approx}
  There is a family of mere relations $\mathord\approx:\RC\to\RC\to\Qp\to\prop$ such that
  \begin{align}
    (\rcrat(q) \approx_\epsilon \rcrat(r))  &\defeq
    (-\epsilon < q - r < \epsilon)\label{eq:RCappx1}\\
    (\rcrat(q) \approx_\epsilon \rclim(y)) &\defeq
    \exis{\delta, \theta : \Qp} \rcrat(q) \approx_{\epsilon - \delta - \theta} y_\delta\label{eq:RCappx2}\\
    (\rclim(x) \approx_\epsilon \rcrat(r)) &\defeq
    \exis{\delta, \theta : \Qp} x_\delta \approx_{\epsilon - \delta - \theta} \rcrat(r)\label{eq:RCappx3}\\
    (\rclim(x) \approx_\epsilon \rclim(y)) &\defeq
    \exis{\delta, \eta, \theta : \Qp} x_\delta \approx_{\epsilon - \delta - \eta - \theta} y_\eta.\label{eq:RCappx4}
  \end{align}
  Moreover, we have
  \begin{gather}
    (u \approx_\epsilon v) \leftrightarrow \exis{\theta:\Qp} (u \approx_{\epsilon-\theta} v) \label{RC-sim-rounded}\\
    (u \approx_\epsilon v) \to (v\sim_\delta w) \to (u\approx_\epsilon w)\label{eq:RC-sim-rtri}\\ 
    (u \sim_\epsilon v) \to (v\approx_\delta w) \to (u\approx_\epsilon w)\label{eq:RC-sim-ltri}.
  \end{gather}
\end{thm}

The additional conditions~\eqref{RC-sim-rounded}--\eqref{eq:RC-sim-ltri} turn out to be required in order to make the inductive definition go through.
Condition~\eqref{RC-sim-rounded} is called being \textbf{rounded}.
Reading it from right to left gives \emph{monotonicity} of $\approx$,
%
\begin{equation*}
  (\delta < \epsilon) \land (u \approx_\delta v) \Rightarrow (u \approx_\epsilon v)
\end{equation*}
%
while reading it left to right to \emph{openness} of $\approx$,
%
\begin{equation*}
  (u \approx_\epsilon v) \Rightarrow \exis{\epsilon : \Qp} (\delta < \epsilon) \land (u \approx_\delta v).
\end{equation*}
%
Conditions~\eqref{eq:RC-sim-rtri} and~\eqref{eq:RC-sim-ltri} are forms of the triangle inequality, which say that $\approx$ is a ``module'' over $\sim$ on both sides.

\newcommand{\hapx}{\diamondsuit\approx}
\newcommand{\hapname}{\diamondsuit}
\newcommand{\hapxb}{\heartsuit\approx}
\newcommand{\hapbname}{\heartsuit}
\newcommand{\tap}[1]{\bullet\approx_{#1}\triangle}
\newcommand{\tapname}{\triangle}
\newcommand{\tapb}[1]{\bullet\approx_{#1}\square}
\newcommand{\tapbname}{\square}
\begin{proof}
  We will define $\mathord\approx:\RC\to\RC\to\Qp\to\prop$ by double $(\RC,\sim)$-recursion.
  First we will apply $(\RC,\sim)$-recursion with codomain the subset of $\RC\to\Qp\to\prop$ consisting of those families of predicates which are rounded and satisfy the one appropriate form of the triangle inequality.
  Thinking of these predicates as half of a binary relation, we will write them as $(u,\epsilon) \mapsto (\hapx_\epsilon u)$, with the symbol $\hapname$ referring to the whole relation.
  Now we can write $A$ precisely as
  \begin{multline*}
    A \defeq\; \Bigg\{ \hapname :\RC\to\Qp\to\prop \;\bigg|\; \\
    \Big(\fall{u:\RC}{\epsilon:\Qp}
    \big((\hapx_\epsilon u) \leftrightarrow \exis{\theta:\Qp} (\hapx_{\epsilon-\theta} u)\big)\Big)  \\
    \land \Big(\fall{u,v:\RC}{\eta,\epsilon:\Qp} (u\sim_\epsilon v) \to\\
    \big((\hapx_\eta u) \to (\hapx_{\eta+\epsilon} v) \big) \land \big((\hapx_\eta v) \to (\hapx_{\eta+\epsilon} u) \big)\Big)\Bigg\}
  \end{multline*}
  As usual with subsets, we will use the same notation for an inhabitant of $A$ and its first component $\hapname$.
  As the family of relations required for $(\RC,\sim)$-recursion, we consider the following, which will ensure the other form of the triangle inequality:
  \begin{equation*}
    (\hapname \bsim_\epsilon \hapbname )
    \defeq \fall{u:\RC}{\eta:\Qp} ((\hapx_\eta u) \to (\hapxb_{\epsilon+\eta} u)) \land ((\hapxb_\eta u) \to (\hapx_{\epsilon+\eta} u)).
  \end{equation*}
  We observe that these relations are separated.
  For assuming $\fall{\epsilon:\Qp} (\hapname \bsim_\epsilon \hapbname)$, to show $\hapname = \hapbname$ it suffices to show $(\hapx_\epsilon u) \leftrightarrow (\hapxb_\epsilon u)$ for all $u:\RC$.
  But $\hapx_\epsilon u$ implies $\hapx_{\epsilon-\theta} u$ for some $\theta$, by roundedness, which together with $\hapname \bsim_\epsilon \hapbname$ implies $\hapxb_\epsilon u$; and the converse is identical.

  Now the first two data the recursion principle requires are the following.
  \begin{itemize}
  \item For any $q:\Q$, we must give an element of $A$, which we denote $(\rcrat(q)\approx_{(-)} -)$.
  \item For any Cauchy approximation $x$, if we assume defined a function $\Qp \to A$, which we will denote by $\epsilon \mapsto (x_\epsilon \approx_{(-)} -)$, with the property that 
    % \[ \fall{u,v:\RC}{\delta,\epsilon,\eta:\Qp} (x_\delta \approx_\eta u) \to (u\sim_{\delta+\epsilon} v) \to (x_\epsilon \approx_{\eta+\delta+\epsilon} v) \]
    \begin{equation}
      \fall{u:\RC}{\delta,\epsilon,\eta:\Qp} (x_\delta \approx_\eta u) \to (x_\epsilon \approx_{\eta+\delta+\epsilon} u),\label{eq:appxrec2}
    \end{equation}
    we must give an element of $A$, which we will denote $(\rclim(x)\approx_{(-)} -)$.
  \end{itemize}
  In both cases, we give the required definition by using a nested $(\RC,\sim)$-recursion, with codomain the subset of $\Qp\to\prop$ consisting of rounded families of mere propositions.
  Thinking of these propositions as zero halves of a binary relation, we will write them as $\epsilon \mapsto (\tap{\epsilon})$, with the symbol $\tapname$ referring to the whole family.
  Now we can write the codomain of these inner recursions precisely as
  \begin{equation*}
    C \defeq \bigg\{ \tapname :\Qp\to\prop \;\;\Big|\;\;
    \fall{\epsilon:\Qp} \Big((\tap\epsilon) \leftrightarrow \exis{\theta:\Qp} (\tap{\epsilon-\theta})\Big)\bigg\}
  \end{equation*}
  We take the required family of relations to be the remnant of the triangle inequality:
  \begin{equation*}
    (\tapname \bbsim_\epsilon \tapbname) \defeq
    \fall{\eta:\Qp} ((\tap\eta) \to (\tapb{\epsilon+\eta})) \land ((\tapb\eta) \to (\tap{\epsilon+\eta})).
  \end{equation*}
  These relations are separated by the same argument as for $\bsim$, using roundedness of all elements of $C$.

  Note that if such an inner recursion succeds, it will yield a family of predicates $\hapname : \RC\to\Qp\to \prop$ which are rounded (since their image in $\Qp\to\prop$ lies in $C$) and satisfy
  \[ \fall{u,v:\RC}{\epsilon:\Qp} (u\sim_\epsilon v) \to \big((\hapx_{(-)} u) \bbsim_\epsilon (\hapx_{(-)} u)\big). \]
  Expanding out the definition of $\bbsim$, this yields precisely the third condition for $\hapname$ to belong to $A$; thus it is exactly what we need.

  It is at this point that we can give the definitions~\eqref{eq:RCappx1}--\eqref{eq:RCappx4}, as the first two clauses of each of the two inner recursions, corresponding to rational points and limits.
  In each case, we must verify that the relation is rounded and hence lies in $C$.
  In the rational-rational case~\eqref{eq:RCappx1} this is clear, while in the other cases it follows from an inductive hypothesis.
  (In~\eqref{eq:RCappx2} the relevant inductive hypothesis is that $(\rcrat(q) \approx_{(-)} y_\delta) : C$, while in~\eqref{eq:RCappx3} and~\eqref{eq:RCappx4} it is that $(x_\delta \approx_{(-)} -) : A$.)

  The remaining data of the sub-recursions consist of showing that~\eqref{eq:RCappx1}--\eqref{eq:RCappx4} satisfy the triangle inequality on the right with respect to the constructors of $\sim$.
  There are eight cases --- four in each sub-recursion --- corresponding to the eight possible ways that $u$, $v$, and $w$ in~\eqref{eq:RC-sim-rtri} can be chosen to be rational points or limits.
  First we consider the cases when $u$ is $\rcrat(q)$.
  \begin{enumerate}
  \item Assuming $\rcrat(q)\approx_\phi \rcrat(r)$ and $-\epsilon<|r-s|<\epsilon$, we must show $\rcrat(q)\approx_{\phi+\epsilon} \rcrat(s)$.
    But by definition of $\approx$, this reduces to the triangle inequality for rational numbers.
  \item We assume $\phi,\epsilon,\delta,\theta:\Qp$ such that $\rcrat(q)\approx_\phi \rcrat(r)$ and $\rcrat(r) \sim_{\epsilon-\delta-\theta} y_\delta$, and inductively that
    \begin{equation}
      \fall{\psi:\Qp}(\rcrat(q) \approx_{\psi} \rcrat(r)) \to (\rcrat(q) \approx_{\psi+\epsilon-\delta-\theta} y_\delta).\label{eq:RCappx-rtri-rrl1}
    \end{equation}
    We assume also that $\psi,\delta\mapsto (\rcrat(q) \approx_{\psi} y_\delta)$ is a Cauchy approximation with respect to $\bbsim$, i.e.\
    \begin{equation}
      \fall{\psi,\xi,\zeta:\Qp} (\rcrat(q) \approx_{\psi} y_\xi) \to (\rcrat(q) \approx_{\psi+\xi+\zeta} y_\zeta),\label{eq:RCappx-rtri-rrl2}
    \end{equation}
    although we will not need this assumption in this case.
    Indeed,~\eqref{eq:RCappx-rtri-rrl1} with $\psi\defeq \phi$ yields immediately $\rcrat(q) \approx_{\phi+\epsilon-\delta-\theta} y_\delta$, and hence $\rcrat(q) \approx_{\phi+\epsilon} \rclim(y)$ by definition of $\approx$.
  \item We assume $\phi,\epsilon,\delta,\theta:\Qp$ such that $\rcrat(q)\approx_\phi \rclim(y)$ and $y_\delta \sim_{\epsilon-\delta-\theta} \rcrat(r)$, and inductively that
    \begin{gather}
      \fall{\psi:\Qp}(\rcrat(q) \approx_{\psi} y_\delta) \to (\rcrat(q) \approx_{\psi+\epsilon-\delta-\theta} \rcrat(r)).\label{eq:RCappx-rtri-rlr1}\\
      \fall{\psi,\xi,\zeta:\Qp} (\rcrat(q) \approx_{\psi} y_\xi) \to (\rcrat(q) \approx_{\psi+\xi+\zeta} y_\zeta).\label{eq:RCappx-rtri-rlr2}
    \end{gather}
    Now by definition, $\rcrat(q)\approx_\phi \rclim(y)$ means that we have $\xi,\zeta:\Qp$ with $\rcrat(q) \approx_{\phi-\xi-\zeta} y_\xi$.
    By assumption~\eqref{eq:RCappx-rtri-rlr2}, therefore, we have also $\rcrat(q) \approx_{\phi+\delta-\zeta} y_\delta$.
    But now by assumption~\eqref{eq:RCappx-rtri-rlr1}, we have $\rcrat(q) \approx_{\phi+\epsilon-\zeta-\theta} \rcrat(r)$, and hence $\rcrat(q) \approx_{\phi+\epsilon} \rcrat(r)$ by monotonicity, as desired.
  \item We assume $\phi,\epsilon,\delta,\theta,\eta:\Qp$ such that $\rcrat(q)\approx_\phi \rclim(y)$ and $y_\delta \sim_{\epsilon-\delta-\eta-\theta} z_\eta$, and inductively that 
    \begin{gather}
      \fall{\psi:\Qp}(\rcrat(q) \approx_{\psi} y_\delta) \to (\rcrat(q) \approx_{\psi+\epsilon-\delta-\eta-\theta} z_\eta)\label{eq:RCappx-rtri-rll1}\\
      \fall{\psi,\xi,\zeta:\Qp} (\rcrat(q) \approx_{\psi} y_\xi) \to (\rcrat(q) \approx_{\psi+\xi+\zeta} y_\zeta)\label{eq:RCappx-rtri-rll2}\\
      \fall{\psi,\xi,\zeta:\Qp} (\rcrat(q) \approx_{\psi} z_\xi) \to (\rcrat(q) \approx_{\psi+\xi+\zeta} z_\zeta)\label{eq:RCappx-rtri-rll3}
    \end{gather}
    Again, $\rcrat(q)\approx_\phi \rclim(y)$ means we have $\xi,\zeta:\Qp$ with $\rcrat(q) \approx_{\phi-\xi-\zeta} y_\xi$, while~\eqref{eq:RCappx-rtri-rll2} then implies $\rcrat(q) \approx_{\phi+\delta-\zeta} y_\delta$ and~\eqref{eq:RCappx-rtri-rll1} implies $\rcrat(q) \approx_{\phi+\epsilon-\eta-\zeta} z_\eta$.
    But by definition of $\approx$, this implies $\rcrat(q) \approx_{\phi+\epsilon} \rclim(z)$ as desired.
  \end{enumerate}
  Now we move on to the cases when $u$ is $\rclim(x)$, with $x$ a Cauchy approximation.
  In this case, the ambient induction hypothesis of the definition of $(\rclim(x) \approx_{(-)} -) : A$ is that we have $(x_\delta \approx_{(-)} -): A$, so that in addition to being rounded they satisfy the triangle inequality on the right.
  \begin{enumerate}\setcounter{enumi}{4}
  \item Assuming $\rclim(x)\approx_\phi \rcrat(r)$ and $-\epsilon<|r-s|<\epsilon$, we must show $\rclim(x)\approx_{\phi+\epsilon} \rcrat(s)$.
    But by definition of $\approx$, the former means $x_\delta \approx_{\phi-\delta-\theta} \rcrat(r)$, so that above-mentioned triangle inequality implies $x_\delta \approx_{\epsilon+\phi-\delta-\theta} \rcrat(s)$, hence $\rclim(x)\approx_{\phi+\epsilon} \rcrat(s)$ as desired.
  \item We assume $\phi,\epsilon,\delta,\theta:\Qp$ such that $\rclim(x)\approx_\phi \rcrat(r)$ and $\rcrat(r) \sim_{\epsilon-\delta-\theta} y_\delta$, and two unneeded inductive hypotheses.
    % \begin{gather}
    %   \fall{\psi:\Qp}(\rclim(x) \approx_{\psi} \rcrat(r)) \to (\rclim(x) \approx_{\psi+\epsilon-\delta-\theta} y_\delta)\label{eq:RCappx-rtri-lrl1}\\
    %   \fall{\psi,\xi,\zeta:\Qp} (\rclim(x) \approx_{\psi} y_\xi) \to (\rclim(x) \approx_{\psi+\xi+\zeta} y_\zeta).\label{eq:RCappx-rtri-lrl2}
    % \end{gather}
    By definition, we have $\eta,\xi:\Qp$ such that $x_\eta \approx_{\phi-\eta-\xi} \rcrat(r)$, so the inductive triangle inequality gives $x_\eta \approx_{\phi+\epsilon-\eta-\delta-\theta-\xi} y_\delta$.
    The definition of $\approx$ then immediately yields $\rclim(x) \approx_{\phi+\epsilon} \rclim(y)$.
  \item We assume $\phi,\epsilon,\delta,\theta:\Qp$ such that $\rclim(x)\approx_\phi \rclim(y)$ and $y_\delta \sim_{\epsilon-\delta-\theta} \rcrat(r)$, and two unneeded inductive hypotheses.
    % \begin{gather}
    %   \fall{\psi:\Qp}(\rclim(x) \approx_{\psi} \rcrat(r)) \to (\rclim(x) \approx_{\psi+\epsilon-\delta-\theta} y_\delta)\label{eq:RCappx-rtri-llr1}\\
    %   \fall{\psi,\xi,\zeta:\Qp} (\rclim(x) \approx_{\psi} y_\xi) \to (\rclim(x) \approx_{\psi+\xi+\zeta} y_\zeta).\label{eq:RCappx-rtri-llr2}
    % \end{gather}
    By definition, then, we have $\xi,\zeta,\psi:\Qp$ with $x_\xi \approx_{\phi-\xi-\zeta-\psi} y_\zeta$.
    Since $y$ is a Cauchy approximation, we have $y_\zeta \sim_{\zeta+\delta} y_\delta$, so the inductive triangle inequality gives $x_\xi \approx_{\phi+\delta-\xi-\psi} y_\delta$ and then $x_\xi \sim_{\phi+\epsilon-\xi-\psi-\theta} \rcrat(r)$.
    The definition of $\approx$ then gives $\rclim(x) \approx_{\phi+\epsilon}\rcrat(r)$, as desired.
  \item Finally, we assume $\phi,\epsilon,\delta,\eta,\theta:\Qp$ such that $\rclim(x)\approx_\phi \rclim(y)$ and $y_\delta \sim_{\epsilon-\delta-\eta-\theta} z_\eta$.
    Then as before we have $\xi,\zeta,\psi:\Qp$ with $x_\xi \approx_{\phi-\xi-\zeta-\psi} y_\zeta$, and two applications of the triangle inequality suffices as before.
  \end{enumerate}

  This completes the two inner recursions, and thus the definitions of the families of relations $(\rcrat(q)\approx_{(-)}-)$ and $(\rclim(x)\approx_{(-)}-)$.
  Since all are elements of $A$, they are rounded and satisfy the triangle inequality on the right with respect to $\sim$.
% , and satisfy~\eqref{eq:appxrec2}.
  What remains is to verify the conditions relating to $\bsim$, which is to say that these relations satisfy the triangle inequality on the \emph{left} with respect to the constructors of $\sim$.
  The four cases correspond to the four choices of rational or limit points for $u$ and $v$ in~\eqref{eq:RC-sim-ltri}, and since they are all mere propositions, we may apply $\RC$-induction and assume that $w$ is also either rational or a limit.
  This yields another eight cases, whose proofs are essentially identical to those just given; so we will not subject the reader to them.
\end{proof}

We can now prove:

\begin{thm}\label{thm:RC-sim-characterization}
  For any $u,v:\RC$ and $\epsilon:\Qp$ we have $(u\sim_\epsilon v) = (u\approx_\epsilon v)$.
\end{thm}
\begin{proof}
  Since both are mere propositions, it suffices to prove bidirectional implication.
  For the left-to-right direction, we use $\sim$-induction with $C(u,v,\epsilon)\defeq (u\approx_\epsilon v)$.
  Thus, it suffices to consider the four constructors of $\sim$.
  In each case, $u$ and $v$ are specialized to either rational points or limits, so that the definition of $\approx$ evaluates, and the inductive hypothesis always applies.

  For the right-to-left direction, we use $\RC$-induction to assume that $u$ and $v$ are rational points or limits, allowing $\approx$ to evaluate.
  But now the definitions of $\approx$, and the inductive hypotheses, supply exactly the data required for the relevant constructors of $\sim$.
\end{proof}

Stretching a point, one might call $\approx$ a fibration of ``codes'' for $\sim$, with the two directions of the above proof being \encode and \decode respectively.
By the definition of $\approx$, from \autoref{thm:RC-sim-characterization} we get equivalences
\begin{align*}
  (\rcrat(q) \sim_\epsilon \rcrat(r))  &=
  (-\epsilon < q - r < \epsilon)\\
  (\rcrat(q) \sim_\epsilon \rclim(y)) &=
  \exis{\delta, \theta : \Qp} \rcrat(q) \sim_{\epsilon - \delta - \theta} y_\delta\\
  (\rclim(x) \sim_\epsilon \rcrat(r)) &=
  \exis{\delta, \theta : \Qp} x_\delta \sim_{\epsilon - \delta - \theta} \rcrat(r)\\
  (\rclim(x) \sim_\epsilon \rclim(y)) &=
  \exis{\delta, \eta, \theta : \Qp} x_\delta \sim_{\epsilon - \delta - \eta - \theta} y_\eta.
\end{align*}
Our proof also provides the following additional information.

\begin{cor}
  $\sim$ is rounded and satisfies the triangle inequality:
    \begin{gather}
      \eqvspaced{
        (u \sim_\epsilon v)
      }{
        \exis{\theta : \Qp} u \sim_{\epsilon - \theta} v
      }\\
      (u\sim_\epsilon v) \to (v\sim_\delta w) \to (u\sim_{\epsilon+\delta} w)\label{item:RC-sim-triangle}
    \end{gather}
\end{cor}
% \begin{proof}
%   The construction of $\approx$ showed simultaneously that it is rounded, and satisfies ``triangle inequalities'' such as
%   \[ (u\approx_\epsilon v) \to (v\sim_\delta w) \to (u\approx_{\epsilon+\delta} w). \]
%   Thus, both properties follow from \autoref{thm:RC-sim-characterization}.
% \end{proof}

With the triangle inequality in hand, we can show that ``limits'' of Cauchy approximations actually behave like limits.

\begin{lem}\label{thm:RC-sim-lim}
  For any $u:\RC$, any Cauchy approximation $y$, and any $\epsilon,\delta,\theta:\Qp$, if $u\sim_\epsilon y_\delta$ then $u\sim_{\epsilon+\delta+\theta} \rclim(y)$.
\end{lem}
\begin{proof}
  We use $\RC$-induction on $u$.
  If $u$ is $\rcrat(q)$, then this is exactly the second constructor of $\sim$.
  Now suppose $u$ is $\rclim(x)$, and that each $x_\eta$ has the property that for any $y,\epsilon,\delta,\theta$, if $x_\eta\sim_\epsilon y_\delta$ then $x_\eta \sim_{\epsilon+\delta+\theta} \rclim(y)$.
  In particular, taking $y\defeq x$ and $\delta\defeq\eta$ in this assumption, we conclude that $x_\eta \sim_{\eta+\theta} \rclim(x)$ for any $\eta,\theta:\Qp$, and in particular $x_\eta \sim_{2\eta} \rclim(x)$.

  Now let $y,\epsilon,\delta,\theta$ be arbitrary and assume $\rclim(x) \sim_\epsilon y_\delta$.
  Then by the triangle inequality, for any $\eta$ we have $x_\eta \sim_{\epsilon+2\eta} y_\delta$.
  Hence the fourth constructor of $\sim$ yields $\rclim(x) \sim_{\epsilon+4\eta+\delta} \rclim(y)$.
  Thus it suffices to take $\eta \defeq \theta/4$.
\end{proof}

\begin{lem}\label{thm:RC-sim-lim-term}
  For any Cauchy approximation $y$ and any $\delta,\eta:\Qp$ we have $y_\delta \sim_{\delta+\eta} \rclim(y)$.
\end{lem}
\begin{proof}
  Take $u\defeq y_\delta$ and $\epsilon\defeq\theta\defeq \eta/2$ in the previous lemma.
\end{proof}

\begin{rmk}
  We might have expected to have $y_\delta \sim_{\delta} \rclim(y)$, but this fails in examples.
  For instance, consider $x$ defined by $x_\epsilon \defeq \epsilon$.
  Its limit is clearly $0$, but we do not have $|\epsilon - 0 |<\epsilon$, only $\le$.
\end{rmk}

As an application, \autoref{thm:RC-sim-lim-term} enables us to show that the extensions of Lipschitz functions from \autoref{RC-extend-Q-Lipschitz} are unique.

\begin{lem} \label{RC-Lipschitz-eq}
  Suppose $f,g : \RC \to \RC$ are Lipschitz functions, as in \autoref{defn:lipschitz}.
  If $f(\rcrat(q)) = g(\rcrat(q))$ for all $q : \Qp$, then $f = g$.
\end{lem}

\begin{proof}
  We show that $f(u) = g(u)$ for all $u : \RC$ by $\RC$-induction on $u$. When $u$ is
  $\rcrat(q)$ this holds by assumption. When $u$ is $\rclim(x)$ we prove $f(\rclim(x)) =
  g(\rclim(x))$ using~\eqref{eq:RC-path}. For any $\epsilon : \Qp$, using \autoref{thm:RC-sim-lim-term} we have
  %
  \begin{equation*}
    \rclim(x) \sim_{\epsilon/(2 L)}
    x_{\epsilon/(4 L)} =
    x_{\epsilon/(4 L)} \sim_{\epsilon/(2 L)}
    \rclim(x).
  \end{equation*}
  %
  Now, by the Lipschitz condition and the induction hypothesis
  %
  \begin{equation*}
    f(\rclim(x)) \sim_{\epsilon/2}
    f(x_{\epsilon/(4L)}) =
    g(x_{\epsilon/(4L)}) \sim_{\epsilon/2}
    g(\rclim(x)),
  \end{equation*}
  %
  therefore $f(\rclim(x)) \sim_\epsilon g(\rclim(x))$ by the triangle inequality, as required.
\end{proof}


\subsection{Additive structure}
\label{sec:additive-structure-order}

It is tedious to define a function on $\RC$ with bare hands, but we can use (the
multivariate variant of) \autoref{RC-Lipschitz-eq,RC-extend-Q-Lipschitz} to define the
additive and the lattice structure of the reals.

Addition and subtraction are Lipschitz functions on rationals, so they extend to reals and
the extensions satisfy the same equations. This shows that $(\RC, {+}, {-}, 0)$ is a
commutative group.

\begin{thm} \label{RC-commutative-group}
  The Cauchy reals form a commutative group.
\end{thm}

\subsection{Order}
\label{sec:order}

Next, $\min$ and $\max$ are Lipschitz as well, so they can be extended to give $\RC$ the
structure of a lattice, which interacts as expected with the commutative group structure.
The mere binary relation $\leq$, defined for $u, v : \RC$ by
%
\begin{equation*}
  (u \leq v) \defeq \max(u,v) = v
\end{equation*}
%
is a partial order on $\RC$, for which $\min$ and $\max$ are the infimum and supremum. The
strict order $<$ is defined constructively as
%
\begin{equation*}
  (u < v) \defeq
  \exis{q, r : \Q} (q < r) \land (x \leq \rcrat(q)) \land (\rcrat(r) \leq y).
\end{equation*}
%
That is, a witness for $u < v$ is a pair or rational numbers $q < r$ such that $x \leq
\rcrat(q)$ and $\rcrat(r) \leq v$. The strict order is irreflexive and transitive, but we
have to be careful about linearity. We shall address this in \autoref{sec:RC-order}, for
now we focus on expressing $u \sim_\epsilon v$ with standard concepts. Note that the
absolute value $|{-}|$ is a Lipschitz function on $\Q$, which therefore extends to $\RC$
and has the expected properties.

We are now able to show that the auxiliary relation $\sim$ is what we think it is.

\begin{thm} \label{RC-sim-eqv-le}
  $\eqv{(u \sim_\epsilon v)}{(|u - v| < \rcrat(\epsilon))}$
  for all $u, v : \RC$ and $\epsilon : \Qp$.
\end{thm}

\begin{proof}
  TODO.
\end{proof}

TODO: Prove or state relevant properties of $<$.

\subsection{The ring structure}
\label{sec:ring-structure}

TODO: Get multiplication working.

\subsection{The field structure}
\label{sec:field-structure}

TODO: Introduce apartness and inverses.

\subsection{Completeness}
\label{sec:completeness-RC}

TODO: State traditional Cauchy completion


%%%%%%%%%%%%%%%%%%%%%%%%%%%%%%%%%%%%%%%%%%%%%%%%%%

\section{Comparison of Cauchy and Dedekind Reals}
\label{sec:comp-cacuhy-dedek}

Cauchy embeds in Dedekind, we do not expect them to be the same (sheaf model over the
reals) Note, we do not use univalence, hence we have models in sSh(X).

They are the same if we have CAC for Sigma-predicates
Also, $Sigma = 2^N/\sim$.


\subsection{Intuitionstic vs.\ Classical Analysis}
\label{sec:intuitionistic-vs-classical-analysis}

TODO: Example theorems: mean-value theorem.


\section*{Notes}
\label{sec:reals-notes}

\section*{Exercises}
\label{sec:reals-exercises}

\begin{ex} \label{ex:RD-extended-reals}
  %
  Suppose we remove the boundedness condition in \autoref{defn:dedekind-reals} of Dedekind
  reals. Then we obtain the \emph{extended reals} which contain $-\infty \defeq
  (\emptyset, \Q)$ and $\infty \defeq (\Q, \emptyset)$. Which definitions of arithmetical
  operations on cuts still make sense for extended reals? What algebraic structure do we
  get?
\end{ex}

\begin{ex} \label{ex:RD-lower-cuts}
  %
  By considering one-sided cuts we obtain \emph{lower} and \emph{upper} Dedekind reals,
  respectively. For example, a lower real is given by a predicate $L : \Q \to \Omega$
  which is
  %
  \begin{enumerate}
  \item \emph{bounded:} $\exis{q : \Q} L(q)$ and
  \item \emph{rounded:} $L(q) = \exis{r : \Q} q < r \land L(r)$.
  \end{enumerate}
  %
  (We could also require $\exis{r : \Q} \lnot L(r)$ to exclude the cut $\infty \defeq
  \Q$.) Which arithmetical operations can you define on the lower reals? In particular,
  what happens with the additive inverse?
\end{ex}

\begin{ex} \label{ex:RD-interval-arithmetic}
  %
  Suppose we remove the locatedness condition in \autoref{defn:dedekind-reals} of Dedekind
  reals. Then we obtain the \emph{interval domain} $\mathbb{I}$ because cuts are allowed
  to have ``gaps'', which are just intervals. Define the partial order $\sqsubseteq$ on
  $\mathbb{I}$ by
  %
  \begin{equation*}
    ((L, U) \sqsubseteq (L', U'))
    \defeq
    (\fall{q : \Q} L(q) \Rightarrow L'(q)) \land
    (\fall{q : \Q} U(q) \Rightarrow U'(q)).
  \end{equation*}
  %
  What are the maximal elements of $\mathbb{I}$ with respect to $\mathbb{I}$? Define the
  ``endpoint'' operations which assign to an element of the interval domain its lower and
  upper endpoints. Are the endpoints reals, lower reals, or upper reals (see
  \autoref{ex:RD-lower-cuts})? Which definitions of arithmetical operations on cuts still
  make sense for the interval domain?
\end{ex}

\begin{ex}
  Show that $\leq$ is the negation of $<$ on $\RD$. Is $<$ the negation of $\leq$?
\end{ex}

\begin{ex} \label{ex:reals-non-constant-into-Z}
  \mbox{}
  %
  \begin{enumerate}
  \item Assuming excluded middle, construct a non-constant map $\RD \to \Z$.
  \item Suppose $f : \RD \to \Z$ is a map such that $f(0) = 0$ and $f(x) \neq 0$ for all
    $x > 0$. Derive from this the following limited form of excluded middle, known as the
    \emph{Limited Principle of Omniscience (LPO)}: for every $\alpha : \N \to \bool$,
    %
    \begin{equation*}
      (\sm{n : \N} \alpha(n) = \btrue) + (\prd{n : \N} \alpha(n) = \bfalse).
    \end{equation*}
    %
  \end{enumerate}
\end{ex}

\begin{ex} \label{ex:reals-apart-neq-MP}
  \emph{Markov principle} says that for all $f : \nat \to \bool$,
  %
  \begin{equation*}
    (\lnot \lnot \exis{n : \nat} f(n) = \btrue)
    \Rightarrow
    \exis{n : \nat} f(n) = \btrue.
  \end{equation*}
  %
  This is a particular instance of the law of double negation~\eqref{eq:ldn}. Show that
  $\fall{x, y: \RD} x \neq y \Rightarrow x \apart y$ implies Markov principle. Does the
  converse holds as well?
\end{ex}

%%% Local Variables: 
%%% mode: latex
%%% TeX-master: "main"
%%% End: 
