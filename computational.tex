\newcommand\ua{\ensuremath{\mathsf{ua}}}
\newcommand{\fcomp}{\circ}
\newcommand{\code}{\ensuremath{\mathsf{code}}}
\newcommand{\encode}{\ensuremath{\mathsf{encode}}\xspace}
\newcommand{\decode}{\ensuremath{\mathsf{decode}}\xspace}
\newcommand{\pairpath}{\ensuremath{\mathsf{pair}^{\mathord{=}}}\xspace}
% \newcommand{\projpath}[1]{\proj{#1}^{\mathord{=}}}
\newcommand{\projpath}[1]{\ensuremath{\apfunc{\proj{#1}}}\xspace}

\section{The computational behavior of type formers}
\label{sec:computational}

In Chapter~\ref{cha:typetheory}, we introduced many ways to form new types: cartesian products, disjoint unions, dependent products, dependent sums, etc.
In the previous sections of this chapter, we have seen that \emph{all} types in homotopy type theory behave like spaces or higher groupoids.
Our goal in this section is to make explicit how this higher structure behaves, for particular types defined as in Chapter~\ref{cha:typetheory}.

It turns out that for many types $A$, the equality types $\id[A]xy$ can be characterized, up to equivalence, in terms of whatever data was used to construct $A$.
For instance, if $A$ is a cartesian product $B\times C$, and $x\jdeq (b,c)$ and $y\jdeq(b',c')$, then we have an equivalence
\begin{equation}\label{eq:prodeqv}
  \eqv{\Big((b,c)=(b',c')\Big)}{\Big((b=b')\times (c=c')\Big)}.
\end{equation}
In more traditional language, two ordered pairs are equal just when their components are equal (but the equivalence~\eqref{eq:prodeqv} says rather more than this).
The higher structure of the identity types can also be expressed in terms of these equivalences; for instance, concatenating two equalities between pairs corresponds to pairwise concatenation.

Similarly, when a type family $P:A\to\type$ is built up fiberwise using the type forming rules from Chapter~\ref{cha:typetheory}, the operation $\transfib{P}{p}{-}$ can be characterized, up to homotopy, in terms of the corresponding operations on the data that went into $P$.
For instance, if $P(x) \jdeq B(x)\times C(x)$, then we have
\[\transfib{P}{p}{(b,c)} = \left(\transfib{B}{p}{b},\transfib{C}{p}{c}\right).\]

Finally, the type forming rules are also functorial, and if a function $f$ is built from this functoriality, then the operations $\apfunc f$ and $\apdfunc f$ can be computed based on the corresponding ones on the data going into $f$.
For instance, if $g:B\to B'$ and $h:C\to C'$ and we define $f:B\times C \to B'\times C'$ by $f(b,c)\defeq (g(b),h(c))$, then modulo the equivalence~\eqref{eq:prodeqv}, we can identify $\apfunc f$ with ``$(\apfunc g,\apfunc h)$''.

In this section, we will state and prove theorems of this sort for all the basic type forming rules.

\begin{rmk}\label{rmk:computational-hope}
  In the type theory we are working with, identity types are defined simultaneously for all types by the inductive $J$-rule.
  The characterizations for particular types to be discussed in this chapter are then theorems which we have to discover and prove.
  An alternative presentation of type theory might take these characterizations as \emph{definitions} of the identity types (by induction over the construction of types), with the inductive $J$-rule then being provable.
  While such a type theory has not yet been made precise except in very simple cases, it is still helpful to think of the rules to be presented in this section as ``computation'' rules for ``evaluating'' identity types, transport, and function application.
  It is important to note, though, that not \emph{all} identity types can be ``defined'' by induction over the construction of types; counterexamples include most nontrivial higher inductive types (see Chapters~\ref{cha:hits} and~\ref{cha:homotopy}).
\end{rmk}

\subsection{Cartesian product types}
\label{sec:compute-cartprod}

Given types $A$ and $B$, consider the cartesian product type $A \times B$.  
For any elements $x,y:A\times B$ and a path $p:\id[A\times B]{x}{y}$, by functoriality we can extract paths $\ap{\proj1}p:\id[A]{\proj1(x)}{\proj1(y)}$ and $\ap{\proj2}p:\id[B]{\proj2(x)}{\proj2(y)}$.
Thus, we have a function
\begin{equation}
  (\id[A\times B]{x}{y}) \;\to\; (\id[A]{\proj1(x)}{\proj1(y)}) \times (\id[B]{\proj2(x)}{\proj2(y)}).\label{eq:path-prod}
\end{equation}

\begin{thm}\label{thm:path-prod}
  For any $x$ and $y$, the function~\eqref{eq:path-prod} is an equivalence.
\end{thm}

Read logically, this says that two pairs are equal if they are equal
componentwise.  Read category-theoretically, this says that the
morphisms in a product groupoid are pairs of morphisms.  Read
homotopy-theoretically, this says that the paths in a product
space are pairs of paths.

\begin{proof}
  We need a function in the other direction:
  \begin{equation}
    (\id[A]{\proj1(x)}{\proj1(y)}) \times (\id[B]{\proj2(x)}{\proj2(y)}) \;\to\; (\id[A\times B]{x}{y}) .\label{eq:path-prod-inverse}
  \end{equation}
  By the induction rule for cartesian products, we may assume that $x$ and $y$ are both pairs, i.e.\ $x\jdeq (a,b)$ and $y\jdeq (a',b')$ for some $a,a':A$ and $b,b':B$.
  In this case, what we want is a function
  \begin{equation*}
    (\id[A]{a}{a'}) \times (\id[B]{b}{b'}) \;\to\; \big(\id[A\times B]{(a,b)}{(a',b')}\big).
  \end{equation*}
  Now by induction for the cartesian product in its domain, we may assume given $p:a=a'$ and $q:b=b'$.
  And by two path inductions, we may assume that $a\jdeq a'$ and $b\jdeq b'$ and both $p$ and $q$ are reflexivity.
  But in this case, we have $(a,b)\jdeq(a',b')$ and so we can take the output to also be reflexivity.

  It remains to prove that~\eqref{eq:path-prod-inverse} is quasi-inverse to~\eqref{eq:path-prod}.
  This is a simple sequence of inductions, but they have to be done in the right order.

  If we start with $r:\id[A\times B]{x}{y}$, then we first do a path induction on $r$ in order to assume that $x\jdeq y$ and $r$ is reflexivity.
  In this case, since $\apfunc{\proj1}$ and $\apfunc{\proj2}$ are defined by path induction,~\eqref{eq:path-prod} takes $r\jdeq \refl{x}$ to the pair $(\refl{\proj1x},\refl{\proj2x})$.
  Now by induction on $x$, we may assume $x\jdeq (a,b)$, so that this is $(\refl a, \refl b)$.
  Thus,~\eqref{eq:path-prod-inverse} takes it by definition to $\refl{(a,b)}$, which (under our current assumptions) is $r$.
  
  In the other direction, if we start with $s:(\id[A]{\proj1(x)}{\proj1(y)}) \times (\id[B]{\proj2(x)}{\proj2(y)})$, then we first do induction on $x$ and $y$ to assume that they are pairs $(a,b)$ and $(a',b')$, and then induction on $s:(\id[A]{a}{a'}) \times (\id[B]{b}{b'})$ to reduce it to a pair $(p,q)$ where $p:a=a'$ and $q:b=b'$.
  Now by induction on $p$ and $q$, we may assume they are reflexivities $\refl a$ and $\refl b$, in which case~\eqref{eq:path-prod-inverse} yields $\refl{(a,b)}$ and then~\eqref{eq:path-prod} returns us to $(\refl a,\refl b)\jdeq (p,q)\jdeq s$.
\end{proof}

In particular, we have shown that~\eqref{eq:path-prod} has an inverse~\eqref{eq:path-prod-inverse}, which we may denote by
\[
\pairpath : (\id{\proj{1}(x)}{\proj{1}(y)}) \times (\id{\proj{1}(x)}{\proj{1}(y)}) \;\to\; {(\id x y)}
\]
Note that a special case of this yields the $\eta$-equivalence rule for products: $z = (\proj1(z),\proj2(z))$.

It can be helpful to view \pairpath as a \emph{constructor} or \emph{introduction rule} for $\id x y$, analogous to the ``pairing'' constructor of $A\times B$ itself, which introduces the pair $(a,b)$ given $a:A$ and $b:B$.
From this perspective, the two components of~\eqref{eq:path-prod}:
\begin{align*}
  \projpath{1} &: (\id{x}{y}) \to (\id{\proj{1}(x)}{\proj{1} (y)})\\
  \projpath{2} &: (\id{x}{y}) \to (\id{\proj{2}(x)}{\proj{2} (y)})
\end{align*}
are \emph{elimination} rules.
Similarly, the two homotopies which witness~\eqref{eq:path-prod-inverse} as quasi-inverse to~\eqref{eq:path-prod} consist respectively, of \emph{computation} or \emph{$\beta$-reduction} rules:
\begin{align*}
  {\projpath{1}{(\pairpath(p, q)})}
  &= %_{(\id{\proj{1} x}{\proj{1} y})}
  {p} \qquad\text{for } p:\id{\proj{1} x}{\proj{1} y} \\
  {\projpath{2}{(\pairpath(p,q)})}
  &= %_{(\id{\proj{2} x}{\proj{2} y})}
  {q} \qquad\text{for } q:\id{\proj{2} x}{\proj{2} y}
\end{align*}
and an \emph{$\eta$-equivalence} rule:
\[
\id{r}{\pairpath(\projpath{1} (r), \projpath{2} (r)) }
\qquad\text{for } r : \id[A \times B] x y
\]

We can also characterize the reflexivity, inverses, and composition of $A\times B$ componentwise:
\begin{align*}
  {\refl{(z : A \times B)}}
  &= {\pairpath (\refl{\proj{1} z},\refl{\proj{2} z})} \\
  {\opp{p}}
  &= {\pairpath \big(\opp{\projpath{1} (p)},\, \opp{\projpath{2} (p)}\big)} \\
  {{p \ct q}}
  &= {\pairpath \big({\projpath{1} (p)} \ct {\projpath{1} (q)},\,{\projpath{2} (p)} \ct {\projpath{2} (q)}\big)}
\end{align*}
The same is true for the rest of the higher groupoid structure considered in \S\ref{sec:equality}.
All of these equations can be derived by using path induction on the given paths and then returning reflexivity.  

We now consider transport in a pointwise product of type families.
Given type families $ A, B : Z \to \type$, we abusively write $A\times B:Z\to \type$ for the type family defined by $(A\times B)(z) \defeq A(z) \times B(z)$.
Now given $p : \id[Z]{z}{w}$ and $x : A(z) \times B(z)$, we can transport $x$ along $p$ to obtain an element of $A(w)\times B(w)$.

\begin{thm}\label{thm:trans-prod}
  In the above situation, we have
  \[
  \id[A(y) \times B(y)]
  {\transfib{A\times B}px}
  {(\transfib{A}{p}{\proj{1}x}, \transfib{B}{p}{\proj{2}x})}
  \]
\end{thm}
\begin{proof}
  By path induction, we may assume $p$ is reflexivity, in which case we have
  \begin{align*}
    \transfib{A\times B}px&\jdeq x\\
    \transfib{A}{p}{\proj{1}x}&\jdeq \proj1x\\
    \transfib{A}{p}{\proj{2}x}&\jdeq \proj2x.
  \end{align*}
  Thus, it remains to show $x = (\proj1 x, \proj2x)$.
  But this is $\eta$-equivalence, which as we remarked above follows from \autoref{thm:path-prod}.
\end{proof}

Finally, we consider the functoriality of $\apfunc{}$ under cartesian products.
Suppose given types $A,B,A',B'$ and functions $g:A\to A'$ and $h:B\to B'$; then we can define a function $f:A\times B\to A'\times B'$ by $f(x) \defeq (g(\proj1x),h(\proj2x))$.

\begin{thm}\label{thm:ap-prod}
  In the above situation, given $x,y:A\times B$ and $p:\proj1x=\proj1y$ and $q:\proj2x=\proj2y$, we have
  \[ \id[(f(x)=f(y))]{\ap{f}{\pairpath(p,q)}} {\pairpath(\ap{g}{p},\ap{h}{q})}. \]
\end{thm}
\begin{proof}
  Note first that the above equation is well-typed.
  On the one hand, since $\pairpath(p,q):x=y$ we have $\ap{f}{\pairpath(p,q)}:f(x)=f(y)$.
  On the other hand, since $\proj1(f(x))\jdeq g(\proj1x)$ and $\proj2(f(x))\jdeq h(\proj2x)$, we also have $\pairpath(\ap{g}{p},\ap{h}{q}):f(x)=f(y)$.

  Now, by induction, we may assume $x\jdeq(a,b)$ and $y\jdeq(a',b')$, in which case we have $p:a=a'$ and $q:b=b'$.
  Thus, by path induction, we may assume $p$ and $q$ are reflexivity, in which case the desired equation holds judgmentally.
\end{proof}


\subsection{$\Sigma$-types}
\label{sec:compute-sigma}

Let $A$ be a type and $B:A\to\type$ a type family.
Recall that the $\Sigma$-type, or dependent pair type, $\sm{x:A} B(x)$ is a generalization of the cartesian product type.
Thus, we expect its higher groupoid structure to also be a generalization of the previous section.
In particular, its paths should be pairs of paths, but it takes a little thought to give the correct types of these paths.

Suppose that we have a path $p:w=w'$ in $\sm{x:A}P(x)$.
Then we get $\ap{\proj{1}}{p}:\proj{1}(w)=\proj{1}(w')$.
However, we cannot directly ask whether $\proj{2}(w)$ is identical to $\proj{2}(w')$ since they don't have to be in the same type.
But we can transport $\proj{2}(w)$ along the path $\ap{\proj{1}}{p}$, and this does give us a term of the same type as $\proj{2}(w')$.
By path induction, we do in fact obtain a path $\trans{\ap{\proj{1}}{p}}{\proj{2}(w)}=\proj{2}(w')$.

The next theorem states that we can also reverse this process.
Since it is a direct generalization of \autoref{thm:path-prod}, we will be more concise.

\begin{thm}\label{thm:path-sigma}
Suppose that $P:A\to\type$ is a type family over a type $A$ and let $w,w^\prime:\sm{x:A}P(x)$. Then there is an equivalence
\begin{equation*}
\eqvspaced{(w=w')}{\sm{p:\proj{1}(w)=\proj{1}(w')}\trans{p}{\proj{2}(w)}=\proj{2}(w^\prime)}.
\end{equation*}
\end{thm}

\begin{proof}
We define for any $w,w':\sm{x:A}P(x)$, a function
\begin{equation*}
f: (w=w') \;\to\; \sm{p:\proj{1}(w)=\proj{1}(w')}\trans{p}{\proj{2}(w)}=\proj{2}(w^\prime)
\end{equation*}
by path induction, with
\begin{equation*}
f(w,w,\refl{w})\defeq(\refl{\proj{1}(w)},\refl{\proj{2}(w)}).
\end{equation*}
We want to show that $f$ is an equivalence.
We can find a section $g(w,w')$ of $f(w,w')$ by finding a function $G(w,w')$ of type
\begin{equation*}
\prd{p:\proj 1(w)=\proj 1(w')}{q:\trans p{\proj 2(w)}=\proj 2(w')}\mathsf{hFiber}(f(w,w'),(p,q))
\end{equation*}
for each $w,w':\sm{x:A}P(x)$. To do this we will first use the induction principle of dependent sums, i.e.\ it is enough to find a function $G((x,u),(y,v))$ of type
\begin{equation*}
\prd{p:x=y}{q:\trans{p}{u}=v}\hfiber{f((x,u),(y,v))}{(p,q)}
\end{equation*}
This is obvious by induction on $p$ followed by induction on $q$ and we get
\begin{equation*}
G((x,u),(x,u),(\refl{x},\refl{u}))\defeq(\refl{(x,u)},\refl{(\refl{x},\refl{u})}).
\end{equation*}
This gives the function $g(w,w')\defeq\proj{1}\circ G(w,w')$ which is a section of $f(w,w')$ by construction. To show that there is a path $g(w,w',f(w,w',\alpha))=\alpha$ for any $\alpha:w=w'$ we use path induction on $\alpha$ and destruction on $w$. The result follows immediately.
\end{proof}

As for cartesian products, we have $\eta$-equivalence as a special case.

\begin{cor}\label{thm:eta-sigma}
  For $z:\sm{x:A} P(x)$, we have $z = (\proj1(z),\proj2(z))$.
\end{cor}
\begin{proof}
  We have $\refl{\proj1(z)} : \proj1(z) = \proj1(\proj1(z),\proj2(z))$, so by \autoref{thm:path-sigma} it will suffice to exhibit a path $\trans{(\refl{\proj1(z)})}{\proj2(z)} = \proj2(\proj1(z),\proj2(z))$.
  But both sides are judgmentally equal to $\proj2(z)$.
\end{proof}

As we did for binary cartesian products, we can also compute the action of transport on $\Sigma$-types.

\begin{thm}
Suppose we have a type family $P:A\to\type$ and a type family $Q:\big(\sm{x:A} P(x)\big)\to\type$. Then we get the type family
\begin{equation*}
x:A\vdash \sm{u:P(x)} Q(x,u):\type
\end{equation*}
For any path $p:x=y$ and any $(u,z):\sum(u:P(x)),\ Q(x,u)$ we have
\begin{equation*}
\trans{p}{u,z}=(\trans{p}{u},\trans{\mathsf{lift}(u,p)}{z}).
\end{equation*}
The path $\mathsf{lift}(u,p)$ is defined in \autoref{thm:path_lifting}.
\end{thm}

\begin{proof}
Immediate by path induction.
\end{proof}

We leave it to the reader to state and prove a generalization of \autoref{thm:ap-prod} (see \autoref{ex:ap-sigma}).


\subsection{The unit type}
\label{sec:compute-unit}

Trivial cases are sometimes important, so we mention briefly the case of the unit type \unit.

\begin{thm}\label{thm:path-unit}
  For any $x,y:\unit$, we have $\eqv{(x=y)}{\unit}$.
\end{thm}
\begin{proof}
  A function $(x=y)\to\unit$ is easy to define by sending everything to \ttt.
  Conversely, for any $x,y:\unit$ we may assume by induction that $x\jdeq \ttt\jdeq y$.
  In this case we have $\refl{\ttt}:x=y$, yielding a constant function $\unit\to(x=y)$.

  To show that these are inverses, consider first an element $u:\unit$.
  We may assume that $u\jdeq\ttt$, but this is also the result of the composite $\unit \to (x=y)\to\unit$.

  On the other hand, suppose given $p:x=y$.
  By path induction, we may assume $x\jdeq y$ and $p$ is $\refl x$.
  We may then assume that $x$ is \ttt, in which case the composite $(x=y) \to \unit\to(x=y)$ takes $p$ to $\refl x$, i.e.\ to $p$.
\end{proof}

In particular, any two elements of $\unit$ are equal.
We leave it to the reader to formulate this equivalence in terms of introduction, elimination, $\beta$, and $\eta$ rules.
The transport lemma for \unit is simply the transport lemma for constant type families.


\subsection{$\Pi$-types}
\label{sec:compute-pi}

Given a type $A$ and a type family $B : A \to \type$, consider the dependent function type $\prd{x:A}B(x)$.
A path in $\prd{x:A} B(x)$ is given by a pointwise path:
\begin{equation}
  \eqvspaced{(\id{f}{g})}{\prd{x:A} (\id[B(x)]{f(x)}{g(x)})}\label{eq:path-forall}
\end{equation}
From a traditional perspective, this says that two functions which are equal at each point are equal as functions.
From a topological perspective, it says that a path in a function space is the same as a continuous homotopy.
And from a categorical perspective, it says that an isomorphism in a functor category is a natural family of isomorphisms.

Unlike the case in the previous sections, the basic type theory presented in Chapter~\ref{cha:typetheory} is insufficient to prove~\eqref{eq:path-forall}.
All we can say is that there is a function
\begin{equation}
  \happly : {(\id{f}{g})}\;\to\; {\prd{x:A} (\id[B(x)]{f(x)}{g(x)})}\label{eq:happly}
\end{equation}
which is easily defined by path induction.
For the moment, therefore, we will assume:

\begin{axiom}[Function extensionality]
  For any $A$, $B$, $f$, and $g$, the function~\eqref{eq:happly} is an equivalence.
\end{axiom}

We will see in later chapters that this axiom follows either from univalence (see \S\ref{sec:compute-universe} and Chapter~\ref{cha:univalence}) or from an interval type (see \S\ref{sec:interval}).
Moreover, it should be more naturally built into a computational version of homotopy type theory, as envisioned in \autoref{rmk:computational-hope}.

In particular,~\eqref{eq:happly} has a quasi-inverse
\[
\funext : \big(\prd{x:A} (\id{f(x)}{g(x)})\big) \to {(\id{f}{g})}
\]
which is also called ``function extensionaility''.
As we did with $\pairpath$ in \S\ref{sec:compute-cartprod}, we can regard $\funext$ as an \emph{introduction rule} for the type $\id f g$.
From this point of view, $\happly$ is the \emph{elimination rule}, while the homotopies witnessing $\funext$ as quasi-inverse to $\happly$ become a $\beta$-reduction rule
\[
\id{\happly({\funext{(h)}},x)}{h(x)} \qquad\text{for }h:\prd{x:A} (\id{f(x)}{g(x)})
\]
and an $\eta$-equivalence rule:
\[
\id{p}{\funext (x \mapsto \happly(p,{x}))} \qquad\text{for } p: (\id f g)
\]
%% FIXME: where do the rules for \alpha[\delta] go in this style?

We can also compute the identity, inverses, and composition in $\Pi$-types; they are simply given by pointwise operations.
\begin{align*}
\refl{f} &= \funext(x \mapsto \refl{f(x)}) \\
\opp{\alpha} &= \funext (x \mapsto \opp{\happly (\alpha)(x)})  \\
{\alpha} \ct \beta &= \funext (x \mapsto {\happly({\alpha})(x) \ct \happly({\beta})(x)})
\end{align*}

Since the non-dependent function type $A\to B$ is a special case of the dependent function type $\prd{x:A} B(x)$ when $B$ is independent of $x$, all of the above applies essentially verbatim in the non-dependent case as well.
The rules for transport, however, are somewhat simpler in the non-dependent case.
Given a type $X$, a path $p:\id[X]{x_1}{x_2}$, type families $A,B:X\to \type$, and a function $f : A(x_1) \to B(x_1)$,  we have
\begin{align}\label{eq:transport-arrow}
  \transfib{A\to B}{p}{f} &=
  \; \Big(x \mapsto \transfib{B}{p}{f(\transfib{A}{\opp p}{x})}\Big)
\end{align}
where $A\to B$ denotes abusively the type family $X\to \type$ defined by
\[(A\to B)(x) \defeq (A(x)\to B(x)).\]
In other words, when we transport a function $f:A(x_1)\to B(x_1)$ along a path $p:x_1=x_2$, we obtain the function $A(x_2)\to B(x_2)$ which transports its argument backwards along $p$ (in the type family $A$), applies $f$, and then transports the result forwards along $p$ (in the type family $B$).
This can be proven easily by path induction.

Transporting dependent functions is similar, but more complicated.
Suppose given $X$ and $p$ as before, type families $A:X\to \type$ and $B:\prd{x:X} (A(x)\to\type)$, and also a dependent function $f : \prd{a:A(x_1)} B(x_1,a)$.
Then
\begin{align*}
  \transfib{\Pi_A(B)}{p}{f} =
  \;\Big(x \mapsto \transfib{\hat{B}}{\opp{(\mathsf{lift}(x,\opp{p}))}}{f (\transfib{A}{\opp p}x)}\Big)
\end{align*}
where $\Pi_A(B)$ and $\hat{B}$ denote respectively the type families
\[
\begin{array}{rclcl}
\Pi_A(B) &\defeq& \big(x\mapsto \prd{a:A(x)} B(x,a) \big) &:& X\to \type\\
\hat{B} &\defeq& \big(w \mapsto B(\proj1w,\proj2w) \big) &:& \sm{x:X} A(x) \to \type.
\end{array}
\]
If these formulas look a bit intimidating, don't worry about the details.
The basic idea is just the same as for the non-dependent function type: we transport the argument backwards, apply the function, and then transport the result forwards again.


\subsection{The Universe}
\label{sec:compute-universe}

Given two types $A$ and $B$, we may consider them as elements of some universe type \type, and thereby form the identity type $\id[\type]AB$.
As mentioned in the introduction, \emph{univalence} is the identification of $\id[\type]AB$ with the type $(\eqv AB)$ of equivalences from $A$ to $B$, which we described in \S\ref{sec:basics-equivalences}.
We perform this identification by way of the following canonical function.

\begin{lem}
  For types $A,B:\type$, there is a function
  \begin{equation}\label{eq:uidtoeqv}
    \mathsf{idtoeqv} : (\id[\type]AB) \to (\eqv A B).
  \end{equation}
\end{lem}
\begin{proof}
  We could construct this directly by induction on equality, but the following description is more convenient.
  Note that the identity function $\idfunc[\type]:\type\to\type$ may be regarded as a type family indexed by the universe \type; it assigns to each type $X:\type$ the type $X$ itself.
  Thus, given a path $p:A =_\type B$, we have a transport function $\transf{p}:A \to B$.
  We claim that $\transf{p}$ is an equivalence.
  But by induction, it suffices to assume that $p$ is $\refl A$, in which case $\transf{p} \jdeq \idfunc[A]$, which is an equivalence by \autoref{eg:idequiv}.
  Thus, we can define $\mathsf{idtoeqv}(p)$ to be $\transf{p}$ (together with the above proof that it is an equivalence).
\end{proof}

\begin{axiom}[Univalence]
  For any $A,B:\type$, the function~\eqref{eq:uidtoeqv} is an equivalence.
\end{axiom}

Thus, we have
\[
\eqv{(\id[\type]{A}{B})}{(\eqv A B)}
\]
As with function extensionality, the univalence axiom should eventually be built into a more computational version of homotopy type theory.

\begin{rmk}
  It is important for the univalence axiom that we defined $\eqv AB$ using a ``good'' version of $\isequiv$ as described in \S\ref{sec:basics-equivalences}, rather than (say) as $\sm{f:A\to B} \qinv(f)$.
\end{rmk}

In particular, univalence means that \emph{equivalent types may be identified}.
As we did in \S\ref{sec:compute-cartprod} and \S\ref{sec:compute-pi}, it is useful to break this equivalence into:

\begin{itemize}
\item An introduction rule for {(\id[\type]{A}{B})}:
  \[
  \ua : ({\eqv A B}) \to (\id[\type]{A}{B})
  \]
\item The elimination rule, which is $\mathsf{idtoeqv}$:
  \[
  \mathsf{idtoeqv} \jdeq \transfibf{X \mapsto X} : (\id{A}{B}) \to (\eqv A B)
  \]
\item $\beta$-reduction: 
  \[
  \begin{array}{l}
  \transfib{X \mapsto X}{\ua(f)}{x} = f(x)
  \end{array}
  \]
\item $\eta$-equivalence: For any $p : \id A B$
  \[
  \id{p}{\ua(\transfibf{X \mapsto X}(p))}
  \]
\end{itemize}

We can also identify the reflexivity, inverses, and concatenation of equalities in the universe with the corresponding operations on equivalences:
\begin{align*}
  \refl{A} &= \ua(\idfunc[A]) \\
  \opp{\ua(f)} &= \ua(f^{-1}) \\ 
  \ua(f) \ct \ua(g) &= \ua(g\circ f)
\end{align*}
Proving these formally requires knowing how to invert and compose equivalences, which in turn requires having an actual definition of equivalences.
Thus, we postpone them until Chapter~\ref{cha:equivalences}.


\subsection{Identity Type}
\label{sec:compute-paths}

When we know what \id[A]{}{} is, \id[ {\id[A]{}{}} ]{}{} follows:

Transport, when $A$ is non-dependent:
\[
\begin{array}{l}
\transport{x_0:A_0 \mapsto \id[A] {a_1(x_0)}{a_2(x_0)}} {\alpha_0}{\alpha} = 
\opp{(\map{a_1}{\alpha_0})} \ct \alpha \ct \map{a_2}{\alpha_0}
\end{array}
\]

Useful special cases:
\[
\begin{array}{l}
\transport{x:A \mapsto \id[A] {a}{x}} {\alpha_0} \: {\alpha} = \alpha \ct \alpha_0 \\
\transport{x:A \mapsto \id[A] {x_0}{a}} {\alpha_0} \: {\alpha} = \opp {\alpha_0} \ct \alpha \\
\end{array}
\]

Transport, when $A$ is dependent:
\[
\begin{array}{l}
\transport{x_0:A_0 \mapsto \id[A(x_0)] {a_1(x_0)}{a_2(x_0)}} {\alpha_0} \: {\alpha} = \\
\opp{(\map{a_1}{\alpha_0})} \ct \mapdep{(\transport{A}{\alpha_0})}{\alpha} \ct \map{a_2}{\alpha_0}
\end{array}
\]

\subsection{Coproducts}
\label{sec:compute-coprod}

So far, most of the type formers we have considered have been what are called \emph{negative}.
Intuitively, this means that their elements are determined by their behavior under the elimination rules: a (dependent) pair is determined by its projections, and a (dependent) function is determined by its values.
The identity types of negative types can almost always be characterized straightforwardly, along with all of their higher structure, as we have done in \S\S\ref{sec:compute-cartprod}--\ref{sec:compute-pi}.
The universe is not exactly a negative type, but its identity types behave similarly: we have a straightforward characterization (univalence) and a description of the higher structure.
Identity types themselves, of course, are a special case.

We now consider our first example of a \emph{positive} type former.
A positive type is one which is ``presented'' by certain constructors, with the universal property of a presentation being expressed by its elimination rule.
(Categorically speaking, a positive type has a ``mapping out'' universal property, while a negative type has a ``mapping in'' universal property.)
Because computing with presentations is, in general, an uncomputable problem, for positive types we cannot always expect a straightforward characterization of the identity type.
However, in many particular cases, a characterization or partial characterization does exist, and can be obtained by the general method we describe here and in later sections.

(Technically, our chosen presentation of cartesian products and $\Sigma$-types is also positive.
However, because these types also admit a negative presentation which differs only slightly, their identity types have a direct characterization that does not require the method to be described here.)

Consider the coproduct type $A+B$, which is ``presented'' by the injections $\inl:A\to A+B$ and $\inr:B\to A+B$.
Intuitively, we expect that $A+B$ contains exact copies of $A$ and $B$ disjointly, so that we should have
\begin{gather}
  \eqv{(\inl(a_1)=\inl(a_2))}{(a_1=a_2)} \label{eq:inlinj}\\
  \eqv{(\inr(b_1)=\inr(b_2))}{(b_1=b_2)}\\
  \eqv{(\inl(a)= \inr(b))}{\emptyt} \label{eq:inlrdj}
\end{gather}
We prove this as follows.
Fix an element $a_0:A$; our goal is to characterize the type family
\[(x\mapsto (\inl(a)=x)) : A+B \to \type.\]
In order to do this, we will define a type family $\code:A+B\to\type$ and show that $\prd{x:A+B} (\eqv{(\inl(a)=x)}{\code(x)})$.
Since we want to conclude~\eqref{eq:inlinj} from this, we should have $\code(\inl(a)) = (a_0=a)$, and since we also want to conclude~\eqref{eq:inlrdj}, we should have $\code (\inr(b)) = \emptyt$.
The essential insight is that we can use the elimination property of $A+B$ to \emph{define} $\code:A+B\to\type$ by these two equations:
\begin{align*}
  \code(\inl(a)) &\defeq (a_0=a)\\
  \code(\inr(b)) &\defeq \emptyt
\end{align*}
We can now show:

\begin{thm}\label{thm:path-coprod}
  For all $x:A+B$ we have $\eqv{(\inl(a)=x)}{\code(x)}$.
\end{thm}
\begin{proof}
  The key to the following proof is that we do it for all points $x$ together, enabling us to use the elimination principle for the coproduct.
  We first define a function
  \[ \encode : \prd{x:A+B}{p:\inl(a_0)=x} \code(x) \]
  by transporting reflexivity along $p$:
  \[ \encode(x,p) \defeq \transfib{\code}{p}{\refl{a_0}}. \]
  Note that $\refl{a_0} : \code(\inl(a_0))$, since $\code(\inl(a_0))\jdeq (a_0=a_0)$ by definition of \code.
  Next, we define a function
  \[ \decode : \prd{x:A+B}{c:\code(x)} (\inl(a_0)=x) \]
  To define $\decode(x,c)$, we may first use the elimination principle of $A+B$ to divide into cases based on whether $x$ is of the form $\inl(a)$ or the form $\inr(b)$.

  In the first case, where $x\jdeq \inl(a)$, then $\code(x)\jdeq (a_0=a)$, so that $c$ is an identification between $a_0$ and $a$.
  Thus, $\ap{\inl}{c}:(\inl(a_0)=\inl(a))$ so we can define this to be $\decode(\inl(a),c)$.

  In the second case, where $x\jdeq \inr(b)$, then $\code(x)\jdeq \emptyt$, so that $c$ inhabits the empty type.
  Thus, the elimination rule of $\emptyt$ yields a value for $\decode(\inr(b),c)$.

  This completes the definition of \decode; we now show that $\encode(x,-)$ and $\decode(x,-)$ are quasi-inverses for all $x$.
  On the one hand, suppose given $x:A+B$ and $p:\inl(a_0)=x$; we want to show $\decode(x,\encode(x,p))=p$.
  But now by path induction (in the Paulin-Mohring style), we may assume that $x\jdeq\inl(a_0)$ and $p\jdeq \refl{\inl(a_0)}$.
  In this case we have
  \begin{align*}
    \decode(x,\encode(x,p))
    &\jdeq \decode(\inl(a_0),\encode(\inl(a_0),\refl{\inl(a_0)}))\\
    &\jdeq \decode(\inl(a_0),\transfib{\code}{\refl{\inl(a_0)}}{\refl{a_0}})\\
    &\jdeq \decode(\inl(a_0),\refl{a_0})\\
    &\jdeq \ap{\inl}{\refl{a_0}}\\
    &\jdeq \refl{\inl(a_0)}\\
    &\jdeq p.
  \end{align*}
  On the other hand, suppose given $x:A+B$ and $c:\code(x)$; we want to show $\encode(x,\decode(x,c))=c$.
  We may again divide into cases based on $x$.
  If $x\jdeq\inl(a)$, then $c:a_0=a$ and $\decode(x,c)\jdeq \ap{\inl}{c}$, so that
  \begin{align*}
    \encode(x,\decode(x,c))
    &\jdeq \transfib{\code}{\ap{\inl}{c}}{\refl{a_0}}\\
    &= \transfib{a\mapsto (a_0=a)}{c}{\refl{a_0}}\\
    &= \refl{a_0} \ct c\\
    &= c.
  \end{align*}
  In the penultimate line, we have used \autoref{thm:path-paths}.
  (Is the lemma we used in the second line stated anywhere?)
\end{proof}

Of course, there is a corresponding theorem if we fix $b_0:B$ instead of $a_0:A$.

In particular, for any $a:A$ we have the function
\[ \encode(a,-) : (\inl(a_0)=\inl(a)) \to (a_0=a).\]
Traditionally, with identity types viewed as propositions, this is stated as ``$\inl$ is injective''.
The homotopical version gives more information: the types $\inl(a_0)=\inl(a)$ and $a_0=a$ are actually equivalent.
Similarly, for any $b:B$ we have
\[ \encode(b,-) : (\inl(a_0)=\inr(b)) \to \emptyt \]
which can be stated as ``$\inl(a_0)$ is not equal to $\inr(b)$'', i.e.\ the images of \inl and \inr are disjoint.

As usual, we can also characterize the action of transport in coproduct types.
Given a type $X$, a path $p:\id[X]{x_1}{x_2}$, and type families $A,B:X\to\type$, we have
\begin{align*}
  \transfib{A+B}{p}{\inl(a)} &= \inl (\transfib{A}{p}{a})\\
  \transfib{A+B}{p}{\inr(b)} &= \inr (\transfib{B}{p}{b})
\end{align*}
where as usual, $A+B$ denotes abusively the type family $x\mapsto A(x)+B(x)$.
The proof is easy using path induction.


\subsection{Natural numbers}
\label{sec:compute-nat}

The same technique we used for coproducts applies to the natural numbers, which are also a positive type.
In this case the codes for identities are a type family
\[\code:\N\to\N\to\type,\]
defined by double recursion over \N as follows:
\begin{align*}
  \code(0,0) &= \unit\\
  \code(\suc(m),0) &= \emptyt\\
  \code(0,\suc(n)) &= \emptyt\\
  \code(\suc(m),\suc(n)) &= \code(m,n).
\end{align*}
We also define by recursion a dependent function $r:\prd{n:\N} \code(n,n)$, with
\begin{align*}
  r(0) &= \ttt\\
  r(\suc(n)) &= r(n).
\end{align*}

\begin{thm}\label{thm:path-nat}
  For all $m,n:\N$ we have $\eqv{(m=n)}{\code(m,n)}$.
\end{thm}
\begin{proof}
  We define
  \[ \encode : \prd{m,n:\N} (m=n) \to \code(m,n) \]
  by transporting, $\encode(m,n,p) \defeq \transfib{\code(m,-)}{p}{r(m)}$.
  And we define
  \[ \decode : \prd{m,n:\N} \code(m,n) \to (m=n) \]
  by double induction on $m,n$.
  When $m$ and $n$ are both $0$, we need a function $\unit \to (0=0)$, which we define to send everything to $\refl{0}$.
  When $m$ is a successor and $n$ is $0$ or vice versa, the domain $\code(m,n)$ is \emptyt, so the eliminator for \emptyt suffices.
  And when both are successors, we can define $\decode(\suc(m),\suc(n))$ to be the composite
  \[ \code(\suc(m),\suc(n))\jdeq\code(m,n) \xrightarrow{\decode(m,n)} (m=n) \xrightarrow{\apfunc{\suc}} (\suc(m)=\suc(n)). \]
  Next we show that $\encode(m,n)$ and $\decode(m,n)$ are quasi-inverses for all $m,n$.

  On one hand, if we start with $p:m=n$, then by induction on $p$ it suffices to show $\decode(n,n,\encode(n,n,\refl{n}))=\refl{n}$.
  But $\encode(n,n,\refl{n}) \jdeq r(n)$, so it suffices to show that $\decode(n,n,r(n)) =\refl{n}$.
  We can prove this by induction on $n$.
  If $n\jdeq 0$, then $\decode(0,0,r(0)) =\refl{0}$ by definition of \decode.
  And in the case of a successor, by the inductive hypothesis we have $\decode(n,n,r(n)) = \refl{n}$, so it suffices to observe that $\apfunc{\suc}(\refl{n}) \jdeq \refl{\suc(n)}$.

  On the other hand, if we start with $c:\code(m,n)$, then we proceed by double induction on $m$ and $n$.
  If both are $0$, then $\decode(0,0,c) \jdeq \refl{0}$, while $\encode(0,0,\refl{0})\jdeq r(0) \jdeq \ttt$.
  Thus, it suffices to recall from \S\ref{sec:compute-unit} that every inhabitant of $\unit$ is equal to \ttt.
  If $m$ is $0$ but $n$ is a successor, or vice versa, then $c:\emptyt$, so we are done.
  And in the case of two successors, we have
  \begin{multline*}
    \encode(\suc(m),\suc(n),\decode(\suc(m),\suc(n),c))\\
    \begin{split}
    &= \encode(\suc(m),\suc(n),\apfunc{\suc}(\decode(m,n,c)))\\
    &= \transfib{\code(\suc(m),-)}{\apfunc{\suc}(\decode(m,n,c))}{r(\suc(m))}\\
    &= \transfib{\code(\suc(m),\suc(-))}{\decode(m,n,c)}{r(\suc(m))}\\
    &= \transfib{\code(m,-)}{\decode(m,n,c)}{r(m)}\\
    &= \encode(m,n,\decode(m,n,c))\\
    &= c
  \end{split}
  \end{multline*}
  using the inductive hypothesis.
\end{proof}

In particular, we have
\[ \encode(\suc(m),0) : (\suc(m)=0) \to \emptyt \]
which shows that ``$0$ is not the successor of any natural number''.
We also have the composite
\[ (\suc(m)=\suc(n)) \xrightarrow{\encode} \code(\suc(m),\suc(n)) \jdeq \code(m,n) \xrightarrow{\decode} (m=n) \]
which shows that the function $\suc$ is injective.

We will study more general positive types in Chapters~\ref{cha:induction} and~\ref{cha:hits}.
In Chapter~\ref{cha:homotopy} we will see that the same technique used here to characterize the identity types of coproducts and \nat can also be used to calculate homotopy groups of spheres.

% \subsection{Higher Inductives}
% \label{sec:compute-hits}

% \newcommand{\sone}{\mathsf{S^1}}

% Consider a higher inductive type such as $\sone$.  The definition of the
% higher inductive type does not immediately characterize
% \id[\sone]{x}{y}---which is good, because the calculation of higher
% homotopy groups can be a significant theorem, so we don't want it to be
% baked into the definitions.  However, we will often be able to prove a
% theorem characterizing the loop space, which follows the above form.
% For example, the proof in Chapter~\ref{cha:homotopy} that the fundamental
% group of the circle is the integers plays this role:

% \begin{itemize}
% \item An introduction rule for \id[\sone]{\mathsf{base}}{\mathsf{base}}:
%   \[
%   \mathsf{loopToThe} : \mathbb{Z} \to \id{\mathsf{base}}{\mathsf{base}}
%   \]
% \item An elimination rule:
%   \[
%   \mathsf{encode} : \id{\mathsf{base}}{\mathsf{base}} \to \mathbb{Z}
%   \]
% \item With $\beta$ and $\eta$ rules stating that these are mutually inverse.
% \end{itemize}

% It's less clear that you want to think about identity, inverses, and
% composition as being defined through this encoding (rather than thinking
% of them as constructors), but you can:

% \[
% \begin{array}{l}
% \refl{\mathsf{base}} = \mathsf{loopToThe} \: 0 \\
% \opp{\alpha} = \mathsf{loopToThe} \: (- (\mathsf{encode} \: \alpha)) \\
% \alpha \ct \beta = \mathsf{loopToThe} \: ((\mathsf{encode} \: \alpha) + (\mathsf{encode} \: \beta)) \\
% \end{array}
% \]

% This changes the representation of the group structure
% from the identity type to an explicit representation, as the free group
% on one generator (the additive group on the integers).  

% FIXME: say something about map



% Local Variables:
% TeX-master: "main"
% End:
