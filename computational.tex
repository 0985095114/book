\newcommand\ua[1]{\ensuremath{\mathsf{ua}} \: #1}
\newcommand{\fcomp}{\circ}

\section{The identity structure of specific types}
\label{sec:computational}

In Chapter~\ref{cha:introduction} we introduced many ways to form new types: cartesian products, disjoint unions, dependent products, dependent sums, etc.
In the previous sections of this chapter, we have seen that \emph{all} types in homotopy type theory behave like spaces or higher groupoids.
Our goal in this section is to make explicit how this higher structure behaves, for particular types defined as in Chapter~\ref{cha:typetheory}.

It turns out that for many types $A$, the equality types $\id[A]xy$ can be characterized, up to equivalence, in terms of whatever data was used to construct $A$.
For instance, if $A$ is a cartesian product $B\times C$, and $x\jdeq (b,c)$ and $y\jdeq(b',c')$, then we have an equivalence
\begin{equation}\label{eq:prodeqv}
  \eqv{\Big((b,c)=(b',c')\Big)}{\Big((b=b')\times (c=c')\Big)}.
\end{equation}
In more traditional language, two ordered pairs are equal just when their components are equal (but the equivalence~\eqref{eq:prodeqv} says rather more than this).
The higher structure of the identity types can also be expressed in terms of these equivalences; for instance, concatenating two equalities between pairs corresponds to pairwise concatenation.

Similarly, when a dependent type $P:A\to\type$ is built up fiberwise using the type forming rules from Chapter~\ref{cha:typetheory}, the operation $\transfib{P}{p}{-}$ can be characterized, up to homotopy, in terms of the corresponding operations on the data that went into $P$.
For instance, if $P(x) \jdeq B(x)\times C(x)$, then we have
\[\transfib{P}{p}{(b,c)} = \left(\transfib{B}{p}{b},\transfib{C}{p}{c}\right).\]

Finally, the type forming rules are also functorial, and if a function $f$ is built from this functoriality, then the operations $\apfunc f$ and $\apdfunc f$ can be computed based on the corresponding ones on the data going into $f$.
For instance, if $g:B\to B'$ and $h:C\to C'$ and we define $f:B\times C \to B'\times C'$ by $f(b,c)\defeq (g(b),h(c))$, then modulo the equivalence~\eqref{eq:prodeqv}, we can identify $\apfunc f$ with ``$(\apfunc g,\apfunc h)$''.

In this section, we will state and prove theorems of this sort for all the basic type forming rules.

\begin{rmk}\label{rmk:computational-hope}
  In the type theory we are working with, identity types are defined simultaneously for all types by the inductive $J$-rule.
  The characterizations for particular types to be discussed in this chapter are then theorems which we have to discover and prove.
  An alternative presentation of type theory might take these characterizations as \emph{definitions} of the identity types (by induction over the construction of types), with the inductive $J$-rule then being provable.
  While such a type theory has not yet been made precise except in very simple cases, it is still helpful to think of the rules to be presented in this section as ``computation'' rules for ``evaluating'' identity types, transport, and function application.
\end{rmk}

\subsection{Cartesian product types}
\label{sec:compute-cartprod}

Given types $A$ and $B$, consider the cartesian product type $A \times B$.  
For any elements $x,y:A\times B$ and a path $p:\id[A\times B]{x}{y}$, by functoriality we can extract paths $\ap{\proj1}p:\id[A]{\proj1(x)}{\proj1(y)}$ and $\ap{\proj2}p:\id[B]{\proj2(x)}{\proj2(y)}$.
Thus, we have a function
\begin{equation}
  (\id[A\times B]{x}{y}) \;\to\; (\id[A]{\proj1(x)}{\proj1(y)}) \times (\id[B]{\proj2(x)}{\proj2(y)}).\label{eq:path-prod}
\end{equation}

\begin{thm}\label{thm:path-prod}
  For any $x$ and $y$, the function~\eqref{eq:path-prod} is an equivalence.
\end{thm}

Read logically, this says that two pairs are equal if they are equal
componentwise.  Read category-theoretically, this says that the
morphisms in a product groupoid are pairs of morphisms.  Read
homotopy-theoretically, this says that the paths in a product
space are pairs of paths.

\begin{proof}
  We need a function in the other direction:
  \begin{equation}
    (\id[A]{\proj1(x)}{\proj1(y)}) \times (\id[B]{\proj2(x)}{\proj2(y)}) \;\to\; (\id[A\times B]{x}{y}) .\label{eq:path-prod-inverse}
  \end{equation}
  By the induction rule for cartesian products, we may assume that $x$ and $y$ are both pairs, i.e.\ $x\jdeq (a,b)$ and $y\jdeq (a',b')$ for some $a,a':A$ and $b,b':B$.
  In this case, what we want is a function
  \begin{equation*}
    (\id[A]{a}{a'}) \times (\id[B]{b}{b'}) \;\to\; \big(\id[A\times B]{(a,b)}{(a',b')}\big).
  \end{equation*}
  Now by induction for the cartesian product in its domain, we may assume given $p:a=a'$ and $q:b=b'$.
  And by two path inductions, we may assume that $a\jdeq a'$ and $b\jdeq b'$ and both $p$ and $q$ are reflexivity.
  But in this case, we have $(a,b)\jdeq(a',b')$ and so we can take the output to also be reflexivity.

  It remains to prove that~\eqref{eq:path-prod-inverse} is quasi-inverse to~\eqref{eq:path-prod}.
  This is a simple sequence of inductions, but they have to be done in the right order.

  If we start with $r:\id[A\times B]{x}{y}$, then we first do a path induction on $r$ in order to assume that $x\jdeq y$ and $r$ is reflexivity.
  In this case, since $\apfunc{\proj1}$ and $\apfunc{\proj2}$ are defined by path induction,~\eqref{eq:path-prod} takes $r\jdeq \refl{x}$ to the pair $(\refl{\proj1x},\refl{\proj2x})$.
  Now by induction on $x$, we may assume $x\jdeq (a,b)$, so that this is $(\refl a, \refl b)$.
  Thus,~\eqref{eq:path-prod-inverse} takes it by definition to $\refl{(a,b)}$, which (under our current assumptions) is $r$.
  
  In the other direction, if we start with $s:(\id[A]{\proj1(x)}{\proj1(y)}) \times (\id[B]{\proj2(x)}{\proj2(y)})$, then we first do induction on $x$ and $y$ to assume that they are pairs $(a,b)$ and $(a',b')$, and then induction on $s:(\id[A]{a}{a'}) \times (\id[B]{b}{b'})$ to reduce it to a pair $(p,q)$ where $p:a=a'$ and $q:b=b'$.
  Now by induction on $p$ and $q$, we may assume they are reflexivities $\refl a$ and $\refl b$, in which case~\eqref{eq:path-prod-inverse} yields $\refl{(a,b)}$ and then~\eqref{eq:path-prod} returns us to $(\refl a,\refl b)\jdeq (p,q)\jdeq s$.
\end{proof}

From a programming perspective, it's useful to unpack this equivalence into the following data:

\newcommand{\pairpath}{\mathsf{pair}^{\mathord{=}}}
\newcommand{\projpath}[1]{\proj{#1}^{\mathord{=}}}

\begin{itemize}
\item An introduction rule for $(\id[A \times B]{x}{y})$ (this is~\eqref{eq:path-prod-inverse})
  \[
  \pairpath : (\id{\proj{1} x}{\proj{1} y}) \times (\id{\proj{1} x}{\proj{1} y}) \to {(\id x y)}
  \]
\item Elimination rules (these are the two components of~\eqref{eq:path-prod}):
  \begin{align*}
    \projpath{1} &: (\id{x}{y}) \to (\id{\proj{1} x}{\proj{1} y})\\
    \projpath{2} &: (\id{x}{y}) \to (\id{\proj{2} x}{\proj{2} y})
  \end{align*}
\item $\beta$-reduction:
  \begin{align*}
    {\projpath{1}{(\pairpath(p, q)})}
    &=_{(\id{\proj{1} x}{\proj{1} y})}
    {p} \\
    {\projpath{2}{(\pairpath(p,q)})}
    &=_{(\id{\proj{2} x}{\proj{2} y})}
    {q}
  \end{align*}
\item $\eta$-equivalence: For any $r : \id[A \times B] x y$
  \[
  \id{r}{\pairpath(\projpath{1} (r), \projpath{2} (r)) }
  \]
\end{itemize}
Moreover, reflexivity, inverses, and composition are defined componentwise:
\begin{align*}
  {\refl{(z : A \times B)}}
  &= {\pairpath (\refl{\proj{1} z},\refl{\proj{2} z})} \\
  {\opp{p}}
  &= {\pairpath \big(\opp{(\projpath{1} p)},\, \opp{(\projpath{2} p)}\big)} \\
  {{p \ct q}}
  &= {\pairpath \big({\projpath{1} p} \ct {\projpath{1} q},\,{\projpath{2} p} \ct {\projpath{2} q}\big)}
\end{align*}
The same is true for all the higher groupoid structure considered in \S\ref{sec:equality}.
All of these equations can be derived by using path induction on the given paths and then returning reflexivity.  

We now consider transport in a product of dependent types.
Given dependent types $ A, B : Z \to \type$, we abusively write $A\times B:Z\to \type$ for the dependent type defined by $(A\times B)(z) \defeq A(z) \times B(z)$.
Now given $p : \id[Z]{z}{w}$ and $x : A(z) \times B(z)$, we can transport $x$ along $p$ to obtain an element of $A(w)\times B(w)$.

\begin{thm}\label{thm:trans-prod}
  In the above situation, we have
  \[
  \id[A(y) \times B(y)]
  {\transfib{A\times B}px}
  {(\transfib{A}{p}{\proj{1}x}, \transfib{B}{p}{\proj{2}x})}
  \]
\end{thm}
\begin{proof}
  By path induction, we may assume $p$ is reflexivity, in which case we have
  \begin{align*}
    \transfib{A\times B}px&\jdeq x\\
    \transfib{A}{p}{\proj{1}x}&\jdeq \proj1x\\
    \transfib{A}{p}{\proj{2}x}&\jdeq \proj2x.
  \end{align*}
  Thus, it remains to show $x = (\proj1 x, \proj2x)$, which follows by induction on $x$.
\end{proof}

Finally, we consider the functoriality of $\apfunc{}$ under cartesian products.
Suppose given types $A,B,A',B'$ and functions $g:A\to A'$ and $h:B\to B'$; then we can define a function $f:A\times B\to A'\times B'$ by $f(x) \defeq (g(\proj1x),h(\proj2x))$.

\begin{thm}\label{thm:ap-prod}
  In the above situation, given $x,y:A\times B$ and $p:\proj1x=\proj1y$ and $q:\proj2x=\proj2y$, we have
  \[ \id[(f(x)=f(y))]{\ap{f}{\pairpath(p,q)}} {\pairpath(\ap{g}{p},\ap{h}{q})}. \]
\end{thm}
\begin{proof}
  Note first that the above equation is well-typed.
  On the one hand, since $\pairpath(p,q):x=y$ we have $\ap{f}{\pairpath(p,q)}:f(x)=f(y)$.
  On the other hand, since $\proj1(f(x))\jdeq g(\proj1x)$ and $\proj2(f(x))\jdeq h(\proj2x)$, we also have $\pairpath(\ap{g}{p},\ap{h}{q}):f(x)=f(y)$.

  Now, by induction, we may assume $x\jdeq(a,b)$ and $y\jdeq(a',b')$, in which case we have $p:a=a'$ and $q:b=b'$.
  Thus, by path induction, we may assume $p$ and $q$ are reflexivity, in which case the desired equation holds judgmentally.
\end{proof}


\subsection{$\Sigma$-types}
\label{sec:compute-sigma}

Let $A$ be a type and $B:A\to\type$ a dependent type.
Recall that the $\Sigma$-type, or pair type, $\sm{x:A} B(x)$ is a generalization of the cartesian product type.
Thus, we expect its higher groupoid structure to also be a generalization of the previous section.
In particular, its paths should be pairs of paths, but it takes a little thought to give the correct types of these paths.

Suppose that we have a path $p:w=w'$ in $\sm{x:A}P(x)$.
Then we get $\ap{\proj{1}}{p}:\proj{1}(w)=\proj{1}(w')$.
However, we cannot directly ask whether $\proj{2}(w)$ is identical to $\proj{2}(w')$ since they don't have to be in the same type.
But we can transport $\proj{2}(w)$ along the path $\ap{\proj{1}}{p}$, and this does give us a term of the same type as $\proj{2}(w')$.
By path induction, we do in fact obtain a path $\trans{\ap{\proj{1}}{p}}{\proj{2}(w)}=\proj{2}(w')$.

The next theorem states that we can also reverse this process.
Since it is a direct generalization of \autoref{thm:path-prod}, we will be more concise.

\begin{thm}\label{thm:path-sigma}
Suppose that $P:A\to\type$ is a dependent type over a type $A$ and let $w,w^\prime:\sm{x:A}P(x)$. Then there is an equivalence
\begin{equation*}
\eqvspaced{(w=w')}{\sm{p:\proj{1}(w)=\proj{1}(w')}\trans{p}{\proj{2}(w)}=\proj{2}(w^\prime)}.
\end{equation*}
\end{thm}

\begin{proof}
We define for any $w,w':\sm{x:A}P(x)$ a function $f(w,w')$ of type
\begin{equation*}
(w=w')\to\sm{p:\proj{1}(w)=\proj{1}(w')}\trans{p}{\proj{2}(w)}=\proj{2}(w^\prime)
\end{equation*}
by path induction, with 
\begin{equation*}
f(w,w,\refl{w})\defeq(\refl{\proj{1}(w)},\refl{\proj{2}(w)}).
\end{equation*}
We want to show that $f$ is an isomorphism, so we can find a section $g(w,w')$ of $f(w,w')$ by finding a function $G(w,w')$ of type
\begin{equation*}
\prd{p:\proj 1(w)=\proj 1(w')}{q:\trans p{\proj 2(w)}=\proj 2(w')}\mathsf{hFiber}(f(w,w'),(p,q))
\end{equation*}
for each $w,w':\sm{x:A}P(x)$. To do this we will first use the induction principle of dependent sums, i.e.\ it is enough to find a function $G((x,u),(y,v))$ of type
\begin{equation*}
\prd{p:x=y}{q:\trans{p}{u}=v}\hfiber{f((x,u),(y,v))}{(p,q)}
\end{equation*}
This is obvious by induction on $p$ followed by induction on $q$ and we get
\begin{equation*}
G((x,u),(x,u),(\refl{x},\refl{u}))\defeq(\refl{(x,u)},\refl{(\refl{x},\refl{u})}).
\end{equation*}
This gives the function $g(w,w')\defeq\proj{1}\circ G(w,w')$ which is a section of $f(w,w')$ by construction. To show that there is a path $g(w,w',f(w,w',\alpha))=\alpha$ for any $\alpha:w=w'$ we use path induction on $\alpha$ and destruction on $w$. The result follows immediately.
\end{proof}

As for binary cartesian products, we can also compute the action of transport on $\Sigma$-types.

\begin{thm}
Suppose we have a dependent type $P:A\to\type$ and a dependent type $Q:\big(\sm{x:A} P(x)\big)\to\type$. Then we get the dependent type
\begin{equation*}
x:A\vdash \sm{u:P(x)} Q(x,u):\type
\end{equation*}
For any path $p:x=y$ and any $(u,z):\sum(u:P(x)),\ Q(x,u)$ we have
\begin{equation*}
\trans{p}{u,z}=(\trans{p}{u},\trans{\mathsf{lift}(u,p)}{z}).
\end{equation*}
The path $\mathsf{lift}(u,p)$ is defined in \autoref{thm:path_lifting}.
\end{thm}

\begin{proof}
Immediate by path induction.
\end{proof}

We leave it to the reader to state and prove generalizations of the other results from \S\ref{sec:compute-cartprod} (see \autoref{ex:ap-sigma}).


\subsection{The unit type}
\label{sec:compute-unit}

Trivial cases are sometimes important, so we mention briefly the case of the unit type \unit.

\begin{thm}\label{thm:path-unit}
  For any $x,y:\unit$, we have $\eqv{(x=y)}{\unit}$.
\end{thm}
\begin{proof}
  A function $(x=y)\to\unit$ is easy to define by sending everything to \ttt.
  Conversely, for any $x,y:\unit$ we may assume by induction that both $x$ and $y$ are \ttt, hence equal by definition.
  This gives an element of $x=y$, hence a constant function $\unit\to(x=y)$.

  To show that these are inverses, consider first an element $u:\unit$.
  We may assume that $u\jdeq\ttt$, but this is also the result of the composite $\unit \to (x=y)\to\unit$.

  On the other hand, suppose given $p:x=y$.
  By path induction, we may assume $x\jdeq y$ and $p$ is $\refl x$.
  We may then assume that $x$ is \ttt, in which case the composite $(x=y) \to \unit\to(x=y)$ computes to $\refl x$, i.e.\ to $p$.
\end{proof}

In particular, any two elements of $\unit$ are equal.
We leave it to the reader to formulate this equivalence in terms of introduction, elimination, $\beta$, and $\eta$ rules.
The transport lemma for \unit is simply the transport lemma for constant type families.


\subsection{$\Pi$-types}
\label{sec:compute-pi}

Given a type $A$ and a dependent type $B : A \to \type$, consider the dependent function type $\prd{x:A}B(x)$.
A path in $\prd{x:A} B(x)$ is given by a pointwise path:
\begin{equation}
  \eqvspaced{(\id{f}{g})}{\prd{x:A} (\id[B(x)]{f(x)}{g(x)})}\label{eq:path-forall}
\end{equation}
From a traditional perspective, this says that two functions which are equal at each point are equal as functions.
From a topological perspective, it says that a path in a function space is the same as a continuous homotopy.
And from a categorical perspective, it says that an isomorphism in a functor category is a natural family of isomorphisms.

Unlike the case in the previous sections, the basic type theory presented in Chapter~\ref{cha:typetheory} is insufficient to prove~\eqref{eq:path-forall}.
All we can say is that there is a function
\begin{equation}
  \happly : {(\id{f}{g})}\;\to\; {\prd{x:A} (\id[B(x)]{f(x)}{g(x)})}\label{eq:happly}
\end{equation}
which is easily defined by path induction.
For the moment, therefore, we will assume:

\begin{axiom}[Function extensionality]
  For any $A$, $B$, $f$, and $g$, the function~\eqref{eq:happly} is an equivalence.
\end{axiom}

We will see in later chapters that this axiom follows either from univalence (see \S\ref{sec:compute-universe} and Chapter~\ref{cha:univalence}) or from an interval type (see \S\ref{sec:interval}).
Moreover, it should be more naturally built into a computational version of homotopy type theory, as envisioned in \autoref{rmk:computational-hope}.

It is useful to break the equivalence~\eqref{eq:path-forall} into:

\begin{itemize}
\item An introduction rule for {(\id{f}{g})}, \emph{function extensionality}
  \[
  \funext : \big(\prd{x:A} (\id{f(x)}{g(x)})\big) \to {(\id{f}{g})}
  \]
\item An elimination rule: for all $f,g : \prd{x:A} B(x)$
  \[
  \happly : (\id{f}{g}) \to \prd{\alpha : \id{x}{y}} (\id{\transfib{B}{\alpha}{f(x)}}{g(y)})
  \]
  Note that this has a more general type than~\eqref{eq:happly}; the latter is recoverable when $\alpha$ is reflexivity.
\item $\beta$-reduction: for any $h:\prd{x:A} (\id{f(x)}{g(x)})$,
  \[
  \begin{array}{l}
  \id{\happly({\funext{(h)}},{\refl{x}})}{h(x)}
  \end{array}
  \]
\item $\eta$-equivalence: For any $p: (\id f g)$,
  \[
  \id{p}{\funext (x \mapsto \happly(p,\refl{x}))}
  \]
\end{itemize}
%% FIXME: where do the rules for \alpha[\delta] go in this style?

We can also compute the identity, inverses, and composition in $\Pi$-types; they are simply given by pointwise operations.
\begin{align*}
\refl{f} &= \funext(x \mapsto \refl{f(x)}) \\
\opp{\alpha} &= \funext (x \mapsto \opp{(\happly \: {\alpha} \: {\refl x})})  \\
{\alpha} \ct \beta &= \funext (x \mapsto {(\happly \: {\alpha} \: {\refl x}) \ct (\happly \: {\beta} \: {\refl x})})
\end{align*}

Since the non-dependent function type $A\to B$ is a special case of the dependent function type $\prd{x:A} B(x)$ when $B$ is independent of $x$, all of the above applies essentially verbating in the non-dependent case as well.
The rules for transport, however, are somewhat simpler in the non-dependent case.
Given a type $X$, a path $p:\id[X]{x_1}{x_2}$, dependent types $A,B:X\to \type$, and a function $f : A(x_1) \to B(x_1)$,  we have
\begin{align*}
  \transfib{A\to B}{p}{f} &=
  \; \Big(x \mapsto \transfib{B}{p}{f(\transfib{A}{\opp p}{x})}\Big)
\end{align*}
where $A\to B$ denotes abusively the dependent type $X\to \type$ defined by
\[(A\to B)(x) \defeq (A(x)\to B(x)).\]
In other words, when we transport a function $f:A(x_1)\to B(x_1)$ along a path $p:x_1=x_2$, we obtain the function $A(x_2)\to B(x_2)$ which transports its argument backwards along $p$ (in the dependent type $A$), applies $f$, and then transports the result forwards along $p$ (in the dependent type $B$).
This can be proven easily by path induction.

Transporting dependent functions is similar, but more complicated.
Suppose given $X$ and $p$ as before, a dependent type $A:X\to \type$, and a doubly dependent type $B:\prd{x:X} (A(x)\to\type)$, and also a dependent function $f : \prd{a:A(x_1)} B(x_1,a)$.
Then
\begin{align*}
  \transfib{\Pi_A(B)}{p}{f} =
  \;\Big(x \mapsto \transfib{\hat{B}}{\opp{(\mathsf{lift}(x,\opp{p}))}}{f (\transfib{A}{\opp p}x)}\Big)
\end{align*}
where $\Pi_A(B)$ and $\hat{B}$ denote the dependent types
\[
\begin{array}{rclcl}
\Pi_A(B) &\defeq& \big(x\mapsto \prd{a:A(x)} B(x,a) \big) &:& X\to \type\\
\hat{B} &\defeq& \big(w \mapsto B(\proj1w,\proj2w) \big) &:& \sm{x:X} A(x) \to \type.
\end{array}
\]
If these formulas look a bit intimidating, don't worry about the details.
The basic idea is just the same as for the non-dependent function type: we transport the argument backwards, apply the function, and then transport the result forwards again.


\subsection{The Universe}
\label{sec:compute-universe}

A path in $\type$ is given by univalence

\[
\eqv{(\id[\type]{A}{B})}{\eqv A B}
\]

Again it is useful to break this into 

\begin{itemize}
\item An introduction rule for {(\id[\type]{A}{B})}:
  \[
  \ua{} : {\eqv A B} \to (\id[\type]{A}{B})
  \]
\item The elimination rule is transport at $X:\type \mapsto X$:
  \[
  \transport{X \mapsto X}{} : \id{A}{B} \to (A \to B)
  \]
\item $\beta$-reduction: 
  \[
  \begin{array}{l}
  \id{\transport{X \mapsto X} \: (\ua {(f, fIsEquiv)})}{f}
  \end{array}
  \]
\item $\eta$-equivalence: For any $\alpha : \id A B$
  \[
  \id{\alpha}{\ua {(\transport{X \mapsto X}(\alpha) , \beta)}}
  \]
  where $\mathsf{transportIsEquiv} : \isequiv{\transport{X \mapsto X}(\alpha)}$ can be
  defined by doing path induction on $\alpha$, at which point it
  suffices to show that the identity function is an equivalence.  
\end{itemize}

Identity, inverses, and composition: (FIXME: pick a specific definition
of equivalence?)
\[
\begin{array}{l}
\refl{A} = \ua{\idfunc} \\
\opp{\alpha} = \ua {(\transport {X \mapsto X} {\opp \alpha}, \mathsf{transportIsEquiv} \: (\opp \alpha))} \\ 
{\alpha} \ct \beta = ? \\
\end{array}
\]

\subsection{Identity Type}
\label{sec:compute-paths}

When we know what \id[A]{}{} is, \id[ {\id[A]{}{}} ]{}{} follows:

Transport, when $A$ is non-dependent:
\[
\begin{array}{l}
\transport{x_0:A_0 \mapsto \id[A] {a_1(x_0)}{a_2(x_0)}} {\alpha_0}{\alpha} = 
\opp{(\map{a_1}{\alpha_0})} \ct \alpha \ct \map{a_2}{\alpha_0}
\end{array}
\]

Useful special cases:
\[
\begin{array}{l}
\transport{x:A \mapsto \id[A] {a}{x}} {\alpha_0} \: {\alpha} = \alpha \ct \alpha_0 \\
\transport{x:A \mapsto \id[A] {x_0}{a}} {\alpha_0} \: {\alpha} = \opp {\alpha_0} \ct \alpha \\
\end{array}
\]

Transport, when $A$ is dependent:
\[
\begin{array}{l}
\transport{x_0:A_0 \mapsto \id[A(x_0)] {a_1(x_0)}{a_2(x_0)}} {\alpha_0} \: {\alpha} = \\
\opp{(\map{a_1}{\alpha_0})} \ct \mapdep{(\transport{A}{\alpha_0})}{\alpha} \ct \map{a_2}{\alpha_0}
\end{array}
\]

\subsection{Coproducts}
\label{sec:compute-coprod}

\subsection{Natural numbers}
\label{sec:compute-nat}



% \subsection{Higher Inductives}
% \label{sec:compute-hits}

% \newcommand{\sone}{\mathsf{S^1}}

% Consider a higher inductive type such as $\sone$.  The definition of the
% higher inductive type does not immediately characterize
% \id[\sone]{x}{y}---which is good, because the calculation of higher
% homotopy groups can be a significant theorem, so we don't want it to be
% baked into the definitions.  However, we will often be able to prove a
% theorem characterizing the loop space, which follows the above form.
% For example, the proof in Chapter~\ref{cha:homotopy} that the fundamental
% group of the circle is the integers plays this role:

% \begin{itemize}
% \item An introduction rule for \id[\sone]{\mathsf{base}}{\mathsf{base}}:
%   \[
%   \mathsf{loopToThe} : \mathbb{Z} \to \id{\mathsf{base}}{\mathsf{base}}
%   \]
% \item An elimination rule:
%   \[
%   \mathsf{encode} : \id{\mathsf{base}}{\mathsf{base}} \to \mathbb{Z}
%   \]
% \item With $\beta$ and $\eta$ rules stating that these are mutually inverse.
% \end{itemize}

% It's less clear that you want to think about identity, inverses, and
% composition as being defined through this encoding (rather than thinking
% of them as constructors), but you can:

% \[
% \begin{array}{l}
% \refl{\mathsf{base}} = \mathsf{loopToThe} \: 0 \\
% \opp{\alpha} = \mathsf{loopToThe} \: (- (\mathsf{encode} \: \alpha)) \\
% \alpha \ct \beta = \mathsf{loopToThe} \: ((\mathsf{encode} \: \alpha) + (\mathsf{encode} \: \beta)) \\
% \end{array}
% \]

% This changes the representation of the group structure
% from the identity type to an explicit representation, as the free group
% on one generator (the additive group on the integers).  

% FIXME: say something about map



% Local Variables:
% TeX-master: "main"
% End:
