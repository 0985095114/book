

% Projection and extension for truncations
\newcommand{\extendsmb}{\mathsf{ext}}
\newcommand{\extend}[1]{\extendsmb(#1)}


\newbox\pbbox
\setbox\pbbox=\hbox{\xy \POS(65,0)\ar@{-} (0,0) \ar@{-} (65,65)\endxy}
\def\pb{\save[]+<3.5mm,-3.5mm>*{\copy\pbbox} \restore}


\newcommand{\comp}[2]{\ensuremath{{#2} \circ {#1}}}

%
% % Type-theoretic:
\newcommand{\contr}{\ensuremath{\mathsf{contr}}}
% \newcommand{\Eq}{\mathsf{Eq}}
 \newcommand{\HLevel}{\mathsf{HLevel}}
 \newcommand{\hProp}{\mathsf{hProp}}
% \newcommand{\inl}{\mathsf{inl}}
% \newcommand{\inr}{\mathsf{inr}}
 \newcommand{\ishProp}{\mathsf{ishProp}}
 \newcommand{\isofhlevel}{\mathsf{isofhlevel}}
 \newcommand{\istype}[1]{\mathsf{is}\mbox{-}{#1}\mbox{-}\mathsf{type}}
 \newcommand{\isweq}{\mathsf{isweq}}
% \newcommand{\refl}{\mathsf{refl}}
% \renewcommand{\S}{\mathsf{S}}
% \newcommand{\sfSigma}{\mathsf{\Sigma}}
 \newcommand{\Type}{\mathsf{Type}}
% \newcommand{\U}{\mathsf{U}}

% Other:
% \newcommand{\oftype}{\! : \!}
%  \newcommand*{\into}{\ensuremath{\lhook\joinrel\relbar\joinrel\rightarrow}}

% \makeatletter
% \renewcommand{\paragraph}{\@startsection{paragraph}{4}{0mm}{-0.5\baselineskip}{-1ex}{\bf}}
% \makeatother

\chapter{$n$-types}
\label{cha:hlevels}

\section{Definition and First Properties}

In homotopy theory one often talks about $n$-types i.e. spaces which homotopy groups vanish above $n$. In this section we introduce a corresponding notion for type theory. 

\begin{defn}\label{def:hlevel}
  Define the predicate $\istype{n} : \Type \to \Type$ for $n \geq -2$ by recursion as follows:

\[ \istype{n}(X) := \begin{cases}
                         \iscontr(X) & \text{ if } n = -2 , \\
                         \prod\limits_{x,y : X}\istype{n'}(\idtype[X]{x}{y}) & \text{ if } n = S(n') .
                        \end{cases}
\]
%
%  \begin{itemize}
%   \item $\isofhlevel(0)(X) := \iscontr(X)$, %if  $X$ is of h-level $0$, if it is contractible.
%   \item $\isofhlevel(S (n))(X) := \prod\limits_{x,y : X}\isofhlevel(n)(\idtype[X]{x}{y})$ %is of h-level $n+1$, if for all $x, y : X$ the type $(x =_X y)$ is of h-level $n$.
%  \end{itemize}
Given a number $n \geq -2$ and a type $X$, we say that $X$ is an $n$-type, if the type $\istype{n}(X)$ is inhabited.
\end{defn}

There is an important remark that we should make about this definition.

\begin{rmk}
 The number $n$ in \autoref{def:hlevel} ranges over all integers greater than or equal to $-2$. One should note however that the induction principle for type $\Nat$ applies in this case.  Alternatively, we could have defined a predicate $\istype{(k\mbox{-}2)}$ for $k : \Nat$. This justifies the correctness of the above definition and allows us to use the type-theoretic induction principle for $\Nat$ in the computation of $n$-types (for which we can take $n = -2$ as the base case).
 
 We shall also note that we could alternatively declare $n$ as a variable of the meta-theory. We choose not do so, precisely because we
wish to use the type-theoretic induction principle for $\Nat$ in the remainder of the section.
\end{rmk}

As in homotopy theory, a type does not have to be an $n$-type for some number $n$; in the section on higher inductive types we will see that one can describe the types (such as, for example, the $2$-sphere) that are not an $n$-type for any value of $n$.

We start by showing the preservation of $n$-types under certain operations and various logical constructors.
For the remainder of the section, let $X, Y : \Type$ and $n : \Nat$.
\begin{thm}\label{thm:h-level-retracts}
 Let $p \colon X \to Y$ be a retraction and suppose that $X$ is an $n$-type. Then $Y$ is also an $n$-type.
\end{thm}

Before giving a proof of the theorem we gather some useful facts about retractions:

\begin{lem}\label{lem:path-retract}
 Let $p \colon X \to Y$ be a retraction with section $s \colon Y \to X$.
 We denote by $\epsilon_y \defeq \epsilon(y) : p(s(y)) = y$ the equalities thus given.
 \begin{enumerate}
  \item \label{enum:section-paths-1}
      If $e : s(y) = x$ in $X$, then we have $e' :  y = p(x)$ in $Y$
        by the composition
     \[e' \defeq y \stackrel{\opp{\epsilon_y}}{=} p(s(y)) \stackrel{\map{p}{e}}{=} p(x).\]
  \item \label{enum:section-paths-2}
      Given $e : s(y) = s(y')$, we have $\vec e : y = y'$.
     \begin{proof}
        By \ref{enum:section-paths-1} we obtain $e' : y = p(s(y'))$.
        Postcomposing with $\epsilon_{y'}$ we get
           \[ \overrightarrow{e} : y \stackrel{e'}{=} p(s(y)) \stackrel{\epsilon_{y'}}{=} y'  . \]
     \end{proof}
  \item \label{enum:section-paths-3}
        If in \ref{enum:section-paths-2} we consider $e = \refl{s(y)}$, then $e' \ct \epsilon_{y}$
        is homotopic to the identity path, that is,
          \[ \overrightarrow{\refl{s(y)}} = \refl{y}  . \]
  \item Given $e : y = y'$ in $Y$, we have $\overrightarrow{\map{s}{e}} = e$.
      \begin{proof}
       This follows by induction on $e$ and \ref{enum:section-paths-3}.
      \end{proof}
 \end{enumerate}

\end{lem}


\begin{proof}[Proof of \autoref{thm:h-level-retracts}]
 We proceed by induction with respect to $n$.

 For $n=-2$, we know that $X$ is contractible and let $x_0 : X$ be the center of contraction.
 We claim that $y_0 \defeq p(x_0) : Y$ is the center of contraction for $Y$.
 Let $y : Y$, we need a path $y = y_0$. We have $\epsilon_y : p(s(y)) = y$ and $\contr_{s(y)} : s(y) = x_0$.
 By composing we obtain a path
 \[\xymatrix { y \ar@{=}[r]^{\opp{\epsilon_y}} & p(s(y)) \ar@{=}[rr]^{\map{p}{\contr_{s(y)}}} && p(x_0) \equiv y_0 } . \]
 For the inductive step, let $y, y' : Y$. We show that $\idtype {y}{y'}$ is an $n$-type.
 By assumption we know that $\idtype{s(y)}{s(y')}$ is an $n$-type.
 We have two maps
 \[\xymatrix{\idtype{y}{y'} \ar@<1.5ex>[rr]^{\map{s}{\_}} & & \idtype{s(y)}{s(y')} \ar@<1.2ex>[ll]^{\overrightarrow{(\_)}}}\]
  By \autoref{lem:path-retract} we have $\comp{\map{s}{\_}}{\overrightarrow{(\_)}} = 1$,
   that is, $\idtype {y}{y'}$ is a retract of $\idtype{s(y)}{s(y')}$.
    Thus, by the induction hypothesis, $\idtype {y}{y'}$ is $n$-type, as required.
\end{proof}

As an immediate corollary we obtain the stability of $n$-types under weak equivalence:

\begin{cor}\label{cor:preservation-hlevels-weq}
 If $\eqv{X}{Y}$ and $X$ is an $n$-type, then so is $Y$.
\end{cor}

\begin{thm}\label{thm:hlevel-cumulative}
 The hierarchy of $n$-types is cumulative in the following sense:
   given a number $n \geq -2$, if $X$ is an $n$-type, then it is also an $(n\mbox{\rm{+}}1)$-type.
\end{thm}

\begin{proof}
 We proceed by induction with respect to $n$.

 For $n = -2$, we need to show that a contractible type, say, $A$, has contractible path spaces.
       Let $a_0: A$ be the center of contraction of $A$, and let $x, y : A$. We show that $\idtype[A]{x}{y}$
       is contractible.
       By contractibility of $A$ we have a path $\contr_x \ct \opp{\contr_y} : x = y$, which we choose as
       the center of contraction for $\idtype{x}{y}$.
       Given any $p : x = y$, we need to show $p = \contr_x \ct \opp{\contr_y}$.
           By identity elimination, it suffices to show that
        $\refl{x} = \contr_x \ct \opp{\contr_x}$, which is trivial.

 For the inductive step, we need to show that $\idtype[X]{x}{y}$ is an $(n\mbox{+}1)$-type, provided
          that $X$ is an $(n\mbox{+}1)$-type. Applying the induction hypothesis to $\idtype[X]{x}{y}$
         yields the desired result.
\end{proof}

An equivalent criterion for being an $n$-type can be expressed using loop--spaces:

\begin{thm}\label{thm:hlevel-loops}
 Let $X : \type$ be a type, and let $n\geq -1$.
  If for all $x : X$, the type $\Omega(X, x) := \idtype[X]{x}{x}$ is an $n$-type,
       then $X$ is an $(n\mbox{\rm{+}}1)$-type.
\end{thm}

Note that the converse of this implication is always true.

Before proving the theorem, we prove an auxiliary lemma:

\begin{lem}\label{lem:hlevel-if-inhab-hlevel}
 Given $n \geq -1$ and $X : \type$. If, given any inhabitant of $X$ it follows that $X$ is
   an $n$-type, then $X$ is an $n$-type.
\end{lem}

\begin{proof}
  Let $f \colon X \to \istype{n}(X)$ be the given map. For any $x, x' : X$ we need to show that
     $\idtype{x}{x'}$ is an $(n \mbox{-} 1)$-type. But this is precisely given by
    $f(x)(x)(x') : \istype{(n-1)}(\idtype{x}{x'}) $.
\end{proof}


\begin{proof}[Proof of \autoref{thm:hlevel-loops}]
 In order to show that $X$ is an $(n\mbox{+}1)$-type, we need to show that for any given $x, x' : X$,
   the type $\idtype{x}{x'}$ is an $n$-type.
  Following \autoref{lem:hlevel-if-inhab-hlevel} it suffices to give a map
   \[ \idtype{x}{x'} \to \istype{n}(\idtype{x}{x'})  .\]
  By the elimination rule and the assumption that $\Omega(X, x)$ is an $n$-type we obtain this map.
\end{proof}

\section{Preservation under constructors}

\begin{thm}
 Let $n \geq -2$, and let $A : \type$ and $B \colon A \to \type$.
 If $A$ is an $n$-type and for all $a : A$, $B(a)$ is an $n$-type, then so is $\sum\limits_{x : A} B(x)$.
\end{thm}

\begin{proof}
 We proceed by induction with respect to $n$.

 For $n = -2$, we choose the center of contraction for $\sum\limits_{x : A} B(x)$ to be the pair
       $(a_0, b_0)$, where $a_0 : A$ is the center of contraction of $A$ and $b_0 : B(a_0)$ is the center of contraction of $B(a_0)$.
       Given any other element $(a,b)$ of $\sum\limits_{x : A} B(x)$, we provide a path $\id{(a, b)}{(a_0,b_0)}$
       by contractibility of $A$ and $B(a_0)$, respectively.

 For the inductive step, suppose that $A$ is an $(n\mbox{+}1)$-type and
         for any $a : A$, $B(a)$ is an $(n \mbox{+} 1)$-type. We show that $\sum\limits_{x : A} B(x)$ is an $(n \mbox{+} 1)$-type:
      fix $(a_1, b_1)$ and $(a_2,b_2)$ in $\sum\limits_{x : A} B(x)$,
     we show that $\idtype[\sum\limits_{x : A} B(x)]{(a_1, b_1)}{(a_2,b_2)}$ is an $n$-type.
      By \autoref{lemma_from_another_chapter} we have
      \[ \eqv{\idtype[\sum\limits_{x : A} B(x)]{(a_1, b_1)}{(a_2,b_2)}}{\sum_{p : \idtype{a_1}{a_2}} \idtype[B(a_2)]{\trans{p}{b_1}}{b_2}} \]
   and by preservation of $n$-types under weak equivalences (\autoref{cor:preservation-hlevels-weq})
   it suffices to prove that the latter is an $n$-type. This follows from the
   induction hypothesis.
\end{proof}


\begin{thm}\label{thm:hlevel-prod}
 Let $n$ be a natural number, and let $A : \type$ and $B \colon A \to \type$.
     If for all $a : A$, $B(a)$ is an $n$-type, then so is $\prod\limits_{x : A} B(x)$.
\end{thm}

TODO : link to definition of weak and strong function extensionality, where
weak Functional Extensionality is defined as
\[ \prod_{a : A} \iscontr (B(a)) \to \iscontr (\prod_{a : A} B(a)), \]
and strong Functional Extensionality as
\[ \isweq (\idtype{f}{g} \to \prod_x \idtype{f(x)}{g(x)}). \]

\begin{proof}
 We proceed by induction with respect to $n$. For $n = -2$ the result is precisely the statement of weak function extensionality.

 For the inductive step, let $f, g : \prod\limits_{a:A}B(a)$. We need to show that
      $\idtype{f}{g}$ is an $n$-type. By (strong) function extensionality and closure of $n$-types
      under weak equivalences it suffices to show that
      $\prod\limits_{a : A} \idtype[B(a)]{f(a)}{g(a)}$ is an $n$-type. But this follows from the
      induction hypothesis.
\end{proof}

\noindent
As a special case of the above theorem, we obtain that the function space $A \to B$ is an $n$-type
provided that the target space $B$ is an $n$-type.

\section{Propositions and sets}

\begin{defn}\label{defn:h-prop}
 A type $A$ is said to be an \emph{h-proposition} (an \emph{h-prop} for short), if it is $-1$-type.
  Thus we define the predicate
 \[\ishProp (A) := \istype{(-1)}(A)  .\]
\end{defn}

\begin{egs}
 \begin{itemize}
  \item Trivially, the empty type $\emptyset$ is an h-proposition.
  \item For any type $X$, its negation $\neg X := X \to \emptyset$ is an h-proposition. This follows from the previous example by the preservation of $n$-types under dependent products.
 \end{itemize}
\end{egs}

\begin{thm}\label{thm:hprop-inhab-contr}
 If a type $A$ is an h-proposition and is inhabited, then $A$ is contractible.
\end{thm}

\begin{proof}
 Let $a_0$ be an inhabitant of $A$. We choose $a_0$ as center of contraction and thus have to
  prove that for any $a : A$, $a = a_0$.
  Since $A$ is an h-prop, we know that $\idtype[A]{a}{a_0}$ is contractible.
  We choose its center of contraction in order to provide the desired equality.
\end{proof}

\begin{rmk}\label{rem:is-contr-implies-isaprop-and-element}
 The converse of \autoref{thm:hprop-inhab-contr} is also true; this follows directly from cumulativity of $n$-type hierarchy (\autoref{thm:hlevel-cumulative}).
\end{rmk}

We next turn towards giving some less obvious examples of h-propositions:

\begin{thm}\label{thm:isaprop-isofhlevel}
 For any $n \geq -2$ and any type $X$, the type $\istype{n}(X)$ is an h-proposition.
\end{thm}

The theorem will be proven by induction. We start with a lemma, which contains the crucial observation needed in the proof of the base case.

\begin{lem}\label{lem:contr-contr-is-contr}
  Let $X$ be a type. If $X$ is contractible, then so is $\iscontr(X)$.
\end{lem}

\begin{proof}
 Let $c : \iscontr(X)$. We claim that $c$ is the center of contraction of $\iscontr (X)$.
  Given any $c' : \iscontr (X)$, we need to show that
       \[ c = c' : \sum_{x:X}\prod_{x':X}\idtype{x}{x'}  . \]

 We use the equivalence between paths in total spaces and pairs of paths in the base space and the fiber.

 TODO : link to appropriate theorem

 The path $e : \pi_1(c) = \pi_1 (c')$ in the base space is given by contractibility of $X$.
 It remains to show:
 \[ \trans{e}{\pi_2(c)} = \pi_2 (c') : \prod_{x' : X} \idtype[X]{\pi_1 (c')} {x'}  .  \]
 The type $X$ is contractible and thus $\idtype[X]{\pi_1 (c')} {x'}$ is contractible as well by cumulativity.
From presevation of $n$-types under products it follows that the type $\prod_{x' : X} \idtype[X]{\pi_1 (c')} {x'}$
  is contractible as well. Its center of contraction gives the desired equality.
\end{proof}

\begin{proof}[Proof of \autoref{thm:isaprop-isofhlevel}]
  We proceed by induction with respect to $n$.

 For the base case, we need to show that for any type $X$, the type $\iscontr(X)$ is an
        h-proposition. By applying \autoref{lem:hlevel-if-inhab-hlevel}, it suffices to show that
        if $X$ is contractible, then $\iscontr(X)$ is an h-proposition.
        But since $X$ is contractible, the type $\iscontr(X)$ is contractible by \autoref{lem:contr-contr-is-contr},
          thus in particular it is an h-proposition by cumulativity.

For the inductive step we need to show
\[\prod_{X : \Type} \ishProp (\istype{n}(X)) \to \prod_{X : \Type} \ishProp (\istype{(n+1)}(X)) \]
To show the conclusion of this implication, we need to show that for any type $X$, the type
    \[\prod_{x, x' : X}\istype{n}(x = x')\]
is an h-proposition. By \autoref{thm:hlevel-prod} it suffices to show that for any $x, x' : X$, the type $\istype{n}(x =_X x')$ is an h-proposition.
But this follows from the induction hypothesis applied to the type $(x =_X x')$.
\end{proof}

As an immediate corollary we obtain:

\begin{cor}
 For any function $f \colon A \to B$ between two types, the type $\isweq(f)$ is an h-proposition.
\end{cor}

\begin{thm}
 A type $X$ is an h-proposition if and only if we have $\prod\limits_{x, y : X} (x =_X y)$.
\end{thm}

\begin{proof}
 The `only if' direction is trivial. Conversely, suppose that $X$ is an h-prop and let $x, y : X$. We need to show that $\idtype{x}{y}$ is contractible. For that it is enough to show that $X$ is contractible. Choosing $x : X$ as the center of contraction and applying the assumption, gives the desired result.
\end{proof}

\begin{defn}\label{defn:h-set}
 A type $X$ is an {\em h-set}, if it is a $0$-type.
\end{defn}

\begin{defn}
 Let $X : \type$. We say that $X$
 \begin{itemize}
  \item has {\em uniqueness of identity proofs} (UIP, for short), if for all $x, y : X$ and $p, q : x =_X y$ we have $p = q$.
  \item satisfies {\em Axiom K}, if for all $x : X$ and $p : (x =_A x)$ we have $p = \refl{}(x)$.
 \end{itemize}
\end{defn}

\begin{thm}\label{thm:h-set-uip-K}
 For a type $X$ the following are equivalent:
 \begin{enumerate}
  \item\label{enum:set-set} $X$ is an h-set.
  \item\label{enum:set-uip} $X$ has UIP.
  \item\label{enum:set-k} $X$ satisfies Axiom K.
 \end{enumerate}
\end{thm}

\begin{proof} (\ref{enum:set-set} $\Rightarrow$ \ref{enum:set-uip}) is clear by definition. (\ref{enum:set-uip} $\Rightarrow$ \ref{enum:set-k}) follows by instantiating UIP with $y := x$ and $q := \refl{x}$.
Finally, we need to show that if $X$ satisfies Axiom K, then $X$ is an h-set. Let $x, y : X$. We need to show that for any $p, q : (\id{x}{y})$ we have $\id{p}{q}$. But elimination on $q$ reduces the problem precisely to Axiom K.
\end{proof}

\begin{defn}
 A type $X$ has {\em decidable equality}, if for all $x, y : X$ we have
 \[(x =_X y) + \neg (x =_X y).\]
\end{defn}

The following theorem is due to Hedberg~\cite{hedberg1998coherence}, for more information and generalizations
see~\cite{krausgeneralizations}.
\begin{thm}
 Let $X : \type$. If $X$ has decidable equality, then $X$ is an h-set.
\end{thm}

\begin{proof}
Fix $x : X$ and assume that for all $y : Y$ we have
  \[\alpha(y) : (x = y) + \neg (x = y). \]
  For any $y : Y$ we define
      $  \eta_y : (x = y) \to (x = y) $
    by
    \[ \eta_y(p) := \begin{cases}
                     q & \text{ if } \alpha(y) = \inl(q) \\
                     \mathsf{elim}(f(p)) & \text{ if } \alpha(y) = \inr(f).
                    \end{cases}
\]

TODO : explain elimination $\mathsf{elim}$ on False

\noindent
 By case analysis we see that for any $p, p' : x = y$ we have
 \begin{equation}\eta_y(p) = \eta_y(p') . \label{eq:eqdec-eta}\end{equation}

 \noindent
 Next, we prove that for any $y : Y$, the map $\eta_y$ has a left inverse $\theta_y$.
   Once we have this left inverse $\theta_y$, we obtain the desired result as follows: given
     $p, p' : x = y$, we have
   \[ p = \theta_y (\eta_y(p)) \stackrel{\ref{eq:eqdec-eta}}{=} \theta_y (\eta_y(p')) = p' . \]
 Define $\theta_y$ by
     $\theta_y(q) \defeq \opp{\eta_x(\refl{}(x))} \ct q$.
  Finally, we prove $\theta_y(\eta_y(p)) = p$ by path induction on $p$:
   \[\theta_x(\eta_x(\refl{}(x))) = \opp{\eta_x(\refl{}(x))} \ct \eta_x(\refl{}(x)) = \refl{}(x). \]

\end{proof}

\begin{thm}\label{prop:nat-is-set}
 The type $\Nat$ of natural numbers is an h-set.
\end{thm}

\begin{proof}
 We will show that $\Nat$ has decidable equality, that is for all $x, y : \Nat$, we have $(x =_X y) + \neg (x =_X y)$.
 We proceed by induction on $x$ and case analysis on $y$.

 If $x = 0$ and $y = 0$, we take $\inl(\refl{}(0))$. If $x = 0$ and $y = S(n)$,
 then by discrimination on the constructors we get $\neg (0 = S (n))$.

 For the inductive step let $x = S (n)$. If $y = 0$, we use discrimination on the constructors again.
 If $y = S (m)$, using the induction hypothesis we may determine whether $(m = n)$ or $\neg(m = n)$ and proceed accordingly.
\end{proof}

\section{$n$-types form an $(n\mbox{+}1)$-type}

In this section we show that the type of $n$-types is itself an $(n\mbox{+}1)$-type. Formally, we may define the type:
 \[n\mbox{-}\type \defeq \sum_{X : \type} \ \istype{n}(X) \]

In particular, we define $\prop \defeq (-1)\mbox{-}\type$ and $\set \defeq 0\mbox{-}\type$.

\begin{thm}\label{thm:hleveln-of-hlevelSn}
 For any $n \geq -2$ the type $n\mbox{-}\type$ is an $(n\mbox{+}1)$-type.
\end{thm}

We start with a lemma that can be seen as a version of {\em Univalence for propositions on the universe}.

\begin{lem}\label{lem:sig-id-h-prop}
 Let $P : \type \to \prop$ and let $(X, p), (X', p') : \sum\limits_{X : \type} P(X)$. Then:
 \[ \idtype{(X, p)}{(X', p')}  \simeq (\eqv{X}{X'})\]
\end{lem}

\begin{proof}
 We have:
 \begin{equation*}\begin{split}
 \idtype{(X, p)}{(X', p')} & \simeq \sum_{z : \idtype{X}{X'}} \idtype{\trans{z}{p}}{p'} \\
  & \simeq \idtype{X}{X'} \\
  & \simeq (\eqv{X}{X'}),
 \end{split}
 \end{equation*}
 where the first equivalence follows from \ref{lem:paths-in-total-space}; the second is an immediate consequence of $P(X')$ being an h-prop; and the third is an application of the Univalence Axiom.
\end{proof}

\begin{proof}[Proof of \autoref{thm:hleveln-of-hlevelSn}]
 Let $(X, p), (X', p') : n\mbox{-}\type$. We need to show that $\idtype{(X, p)}{(X', p')}$ is an $n$-type. Since for any $n \geq -2$ and any
type $X$ the type: $\istype{n}(X)$ is an h-prop, we may apply \autoref{lem:sig-id-h-prop} with $P := \istype{n}$. Next, we observe that it
suffices to show that $X \rightarrow X'$ is an $n$-type because the projection:
 \[(\eqv{X}{X'}) \hookrightarrow (X \rightarrow X').\]
 is an inclusion (i.e.~its fibers are h-propositions). Since $n$-types are preserved under the arrow type, this reduces to an assumption that $X'$ is an $n$-type.
\end{proof} 

\section{Truncations}
\label{sec:truncations}

Recall the notion of $n$-types from Chapter~\ref{chap:hlevels}.  ``Being
an $n$-type'' is a predicate on types.  % Given an integer $n\ge-2$, we
% recall that $\typeleU{n}$ (or just $\typele{n}$) is the type of all
% $n$-truncated types (in the universe $\type$). For instance
% $\typelep{-2}$ is contractible (there is only one $(-2)$-truncated type,
% the point), $\typelep{-1}$ is also known as $\prop$, and $\typele{0}$ is
% also known as $\set$. 
This predicate is accompanied by an operation of $n$-truncation that takes a
type $A$ and makes the best approximation of $A$ as an $n$-type. This operation
of $n$-truncation equates (``kills'') all morphisms of level higher than $n$ in
$A$.  In this section, we explain the basic properties of truncations and we
will construct truncations using higher inductive types in section
\ref{sec:hittruncations}.

Given a type $A:\type$ and an integer $n\ge-2$, its \emph{$n$-truncation} is a
type $\trunc n A$ together with a map $\tprojf n:A\to\trunc nA$, such that
$\trunc n A$ is $n$-truncated and for every family of $n$-types $P$ over $\trunc
nA$ and $d:\prod_{a:A}P(\tproj na)$, we can define a section $\extend{d}$ of $P$
by $\extend{d}(\tproj na)\defeq d(a)$.

In particular, if $E$ is some $n$-truncated type, we can consider the constant
family of types equal to $E$ for every point of $A$ and we get that every map
$f:A\to{}E$ can be extended to a map $\extend{f}:\trunc nA\to{}E$ defined by
$\extend{f}(\tproj na)\defeq f(a)$.

We also have the nondependent $\eta$-rule which says that if two maps
$g,g':\trunc nA\to{}E$ are such that $g(\tproj na)=g'(\tproj na)$ for every
$a:A$, then $g(x)=g'(x)$ for all $x:\trunc nA$ and in particular $g=g'$.

\begin{lem}
  [Universal property of truncations]

  Let $n\ge-2$, $A:\type$ and $B:\typele{n}$. The following map is an
  equivalence:
  \[\function{(\trunc nA\to{}B)}{A\to{}B}{g}{g\circ\tprojf n}\]
%  In particular, $\typele{n}$ is a reflective subuniverse of $\type$. TODO move somewhere else
\end{lem}

\begin{proof}
  Given that $B$ is $n$-truncated, any $f:A\to{}B$ can be extended to a map
  $\extend{f}:\trunc nA\to{}B$.

  The map $\extend{f}\circ\tprojf n$ is equal to $f$ because for every $a:A$ we
  have $\extend{f}(\tproj na)=f(a)$ by definition and the map
  $\extend{g\circ\tprojf n}$ is equal to $g$ because they both send $\tproj na$
  to $g(\tproj na)$.
\end{proof}


%%% Local Variables: 
%%% mode: latex
%%% TeX-master: "main"
%%% End: 
