\newcommand{\rclim}{\mathsf{lim}} % HIT constructor for Cauchy reals

\chapter{Real Numbers}
\label{cha:real-numbers}

Any foundation of mathematics worthy of its name must eventual address the construction of
real numbers. HoTT can draw experience from constructions of reals in constructive
mathematics, which is not surprising as HoTT \emph{is} a kind of constructive mathematics.

From a classical point of view it is perhaps surprising that there are many kinds of real
numbers. We have encountered the \textbf{homotopical reals} as the universal covering
space of of the circle in \autoref{pi1_S1}. However, there is no algebraic structure on
the homotopical reals, as they are equivalent to the singleton space and so we cannot even
exhibit $0 \neq 1$. What is needed is a set of reals which forms a complete archimedean
ordered field. There are several notions of completeness, of which we shall have a closer
look at those of Cauchy and Dedekind. Since we are in a constructive setting, the field
structure will involve an apartness relation. And we cannot assume the Axiom of Countable
Choice, which further complicates matters. But let us proceed one step at a time.

\section{The Field of Rational Numbers}
\label{sec:field-rati-numb}

We first construct the rational numbers \Q, as the reals can then be seen as a completion
of \Q. An expert will point out that \Q could be replaced by any approximate field, i.e.,
a subring of \Q in which arbitrarily precise approximate inverses exist. An example is the
ring of dyadic rationals, which are those of the form $n/2^k$. From an
implementation point of view an approximate field is more suitable, but we leave such
finesse for those who care about the digits of~$\pi$.

Starting from the natural numbers \N, we first construct the integers in a familiar
fashion, namely as a quotient
%
\[ \Z \defeq (\N \times \N)/{\sim} \]
%
where
%
\[ (a,b) \sim (c,d) \defeq (a + d = b + c). \]
%
In other words, a pair $(a,b)$ represents the integer $a - b$. As $\sim$ is a decidable
relation, \Z has decidable equality and so is a set. The arithmetical operations are
induced by those on \N, e.g., if $u, v \in \Z$ are represented by $(a,b)$ and $(c,d)$,
respectively, then $u + v$ is represented by $(a + c, b + d)$, $-u$ is represented by $(b,
a)$, and $u v$ is represented by $(a c + b d, a d + b c)$. The integers \Z then form the
free commutative ring with unit. In fact, using the results from
\autoref{sec:free-algebras} we could have constructed \Z as the free commutative ring with
unit.

The field of rationals \Q is constructed along the same lines as well, namely as the
quotient
%
\[ \Q \defeq (\Z \times \N)/{\approx} \]
%
where
\[ (u,a) \approx (v,b) \defeq (u (b + 1) = v (a + 1)). \]
%
In other words, a pair $(u, a)$ represents the rational number $u / (1 + a)$. There can be
no division by zero because we cunningly added one to the natural denominator~$a$. Again
we obtain a set with a decidable equality whose operations are induced by those on \Z. We
do not bother to write these down as we trust our readers know how to compute with
fractions even in the case when one is added to the denominator. Let us just record that
there is an entirely unproblematic construction of the ordered field of rational number
\Q, with a dedicable equality and decidable order. The field \Q can also be characterized
as the initial one.

\begin{rmk}
  We could have constructed \Z and \Q without appeal to quotients. The integers can be
  seen as a sum of two copies of \N which correspond to the negative and non-negative
  integers, respectively. The rationals form a decidable subset of $\Z \times \N$ of
  fractions in lowest terms. Such a construction makes arithmetic quite annoying because
  all results must always be expressed as fractions in lowest terms.
\end{rmk}


\section{Cauchy Reals}
\label{sec:cauchy-reals}

The \textbf{Cauchy reals \RC} are the result of completing \Q with limits of Cauchy
sequences. The construction works well in the presence of the Axiom of Countable Choice,
which we do not have at our disposal. Nevertheless, the higher inductive types come to the
rescue and make the construction possible. This is a novel observation about an old
construction.

To steer around the lack of Countable Choice we first replace Cauchy sequences with rapid
Cauchy sequences.

\begin{defn}\label{defn:rapid-cauchy}
  A \textbf{rapid Cauchy sequence} $q : \N \to \Q$ is one which satisfies
  %
  \[ \prd{m,n:\N} |q_m - q_n| \leq 2^{-\min(m,n)}. \]
  %
  \textbf{Coincidence} of sequences is the relation defined by, for $q, r : \N \to \Q$,
  %
  \[ q \approx r \defeq
     \prd{m,n:\N} |q_m - r_n| \leq 2^{-\min(m,n) - 1}. \]
\end{defn}

\noindent
Rapidity guarantees that the $n$-th term of a Cauchy sequence is at most $2^{-n}$ from the
limit. From a computational point of view we require that each term contributes at least
one bit of information about the limit.

It is easy to check that coincidence is an equivalence relation, so we are tempted to
define the real numbers as the quotient
%
\[ \RC \defeq \mathcal{C}/{\approx} \tag{bad attempt} \]
%
where $\mathcal{C}$ is the type of rapid Cauchy sequences,
%
\[ \mathcal{C} \defeq
   \sm{q : \N \to \Q} \prd{m,n:\N} |q_m - q_n| \leq 2^{-\min(m,n)}.
\]
%
Unfortunately, we cannot show that this yields a field in which every (rapid) Cauchy
sequence has a limit, for a rather bureaucratic reason: lifting a sequence of real numbers
$x : \N \to \RC$ to a sequence of representatives $\N \to \mathcal{C}$ requires a kind of
choice which is not available to us. But without the representative Cauchy sequences we
have no way of constructing the Cauchy sequence which reprsents the limit. Every
construction of reals whose last step is a quotient suffers from this deficiency. There
are three common ways out of the conundrum:
%
\begin{enumerate}
\item Pretend that the reals are a setoid $(\mathcal{C}, {\approx})$, i.e., the type of
  Cauchy sequences $\mathcal{C}$ with a conincidence relation attached to it by
  administrative decree.
\item Give in to temptation and accept the Axiom of Countable Choice.
\item Declare the Cauchy reals unworthy and construct the Dedekind reals instead.
\end{enumerate}
%
We shall take a novel, forth way: combine the construction of Cauchy sequences and the
quotient into a single higher-inductive construction. For the completion to work, we need
to complete the rationals with limits of Cauchy sequences, but at the same time we must
identify coincident Cauchy sequences. This requires a simultaneous definition of the
higher-inductive type of Cauchy reals and the coincidence relation.

\begin{defn}
  The \textbf{Cauchy reals \RC} is the higher-inductive type given by the following data:
  % 
  \begin{itemize}
  \item points: for $x : \N \to \Q + \RC$ such that \[\prd{m, n :
      \N} x_m \sim_{\min(m,n)} x_n\] there is a point $\rclim(x) : \RC$,
  \item paths: for $x, y : \N \to \Q + \RC$
    such that \[\prd{m, n : \N} x_m \sim_{\min(m,n) + 1} y_n\] there
    is a path $\id[\RC]{\rclim(x)}{\rclim(y)}$.
  \end{itemize}
  % 
  Simultaneously, the relation $u \sim_k v$ is defined, for $u, v : \Q + \RC$ and $k : \N$ by
  % 
  \begin{align*}
    \inl(q) \sim_k \inl(r) &\defeq |q - r| \leq 2^{-k} \\
    \inl(q) \sim_k \inr(y) &\defeq \inl(q) \sim_{k+1} y_{k+1} \\
    \inr(x) \sim_k \inl(r) &\defeq x_{k+1} \sim_{k+1} \inl(r) \\
    \inr(x) \sim_k \inr(y) &\defeq x_{k+1} \sim_{k+1} y_{k+1}
  \end{align*}  
\end{defn}


Outline:
\begin{enumerate}
  \item performing the Cauchy construction on the rational once is problematic
  \item performing any construction ending with a quotient is probablematic
  \item But Hits work
  \item we get the minimal cmplete field.
  \item We obtain an hset, becase approx is a Prop
\end{enumerate}

\section{Dedekind Reals}
\label{sec:dedekind-reals}

\begin{enumerate}
\item Recall two-side cuts definition
\item discuss issues of size and impredicativity
\item replace Prop with the sigma-frame (Simpson), using HITs
\item Dedekind complete field, completeness wrt sigma-frames
\item This is a subset of an hSet
\end{enumerate}

\section{Comparison of Cacuhy and Dedekind Reals}
\label{sec:comp-cacuhy-dedek}

Cauchy embeds in Dedekind, we do not expect them to be the same (sheaf model over the
reals) Note, we do not use univalence, hence we have models in sSh(X).

They are the same if we have CAC for Sigma-predicates
Also, $Sigma = 2^N/\sim$.


\section{Reals as a Topological Space}
\label{sec:reals-as-topological}

Heine-Borel... Locales, or formal spaces, would be an alternative. The usual arguments
apply, but we will not go into them.

We obtain constructive mathematics, without countable choice. Little investigated.
See Richman. Note, we are still predicative.

With classical logic, we have Sigma=bool and fan-theorem, Heine-Borel.


%%% Local Variables: 
%%% mode: latex
%%% TeX-master: "main"
%%% End: 
