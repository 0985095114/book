\newcommand{\vset}{\mathsf{set}}  % point constructor for cummulative hierarchy V

\chapter{Set Theory}
\label{cha:set-math}

Our conception of sets as types with simple homotopy structure, cf.\
\autoref{sec:basics-sets}, is at odds with the Zermelo-Frankel sets which form a
cumulative hierarchy. For many purposes the homotopy-theoretic sets are just as good as
the Zermelo-Fraenkel ones, but there are also important differences. We address these in
the present chapter. 

This chapter has three parts. First, closely following~\cite{RijkeSpitters}, we show that constructively $\set$ has pleasant categorical
structure, namely it is a $\Pi W$-pretopos. We then concentrate on the distinction between sets and classes, for which Algebraic Set
Theory is the natural framework. Finally, we show that the theory of ordinals and cardinals from classical set theory can
naturally be developed by considering sets with isomorphisms as equalities. This equality is, of course, precisely the one we obtain from
the univalence axiom.

\section{\texorpdfstring{$\set$}{Set} is a \texorpdfstring{$\Pi$}{Π}W-pretopos}
\label{sec:piw-pretopos}

We will show that $\set$ is a $\Pi$W-pretopos. We start by proving that it is a regular category, namely that:
%
\begin{enumerate}
\item $\set$ is finitely complete.
\item The kernel pair $\pi_1,\pi_2: (\sm{x,y:A} f(x)= f(y)) \to A$ of any
      function $f : A \to B$ has a coequalizer.
\item Pullbacks of coequalizers are again coequalizers.
\end{enumerate}
%
The obvious candidate for the coequalizer of the kernel pair of a function is its image.
To prove that this really is so we can use unique choice to find the factorization in the
diagram
%
\begin{equation*}
\begin{tikzpicture}
\matrix (m) [std] {\sm{x,y:A} f(x)= f(y) & A & \mathsf{im}(f) \\ & & X \\};
\draw[ar] ([yshift=.5ex]m-1-1.east) -- node[above] {$\pi_1$} ([yshift=.5ex]m-1-2.west);
\draw[ar] ([yshift=-.5ex]m-1-1.east) -- node[below] {$\pi_2$} ([yshift=-.5ex]m-1-2.west);
\draw[ar] (m-1-2) -- node[above] {$\tilde{f}$} (m-1-3);
\draw[ar] (m-1-2) -- node[auto,swap] {$g$} (m-2-3);
\draw[ar,densely dotted] (m-1-3) -- (m-2-3);
\end{tikzpicture}
\end{equation*}
where $g$ coequalizes $\pi_1$ and $\pi_2$. The result 
that the image of a function $f$ is the coequalizer of its kernel pair is a consequence of the principle of unique choice.

\subsection{The image of a function}
\label{sec:image}

The truncations allow us to define a stable factorization
system.\footnote{\url{http://uf-ias-2012.wikispaces.com/file/view/images.pdf/401765624/images.pdf}}
In fact, this generalizes to reflective subuniverses; see~\autoref{subsec:reflective-subuniverses}.
We will only consider the factorization system given by $(-1)$-truncation.
This defines a stable orthogonal factorization system $(\mathcal{E},
\mathcal{M})$ on $\set$: every map factors as an surjection followed by a 
monomorphism and this factorization is stable under pullback.
We first define these notions:
\begin{defn}
Let $f:A\to B$ be a function between $\set$s. It is \emph{surjective} if
\begin{equation*}
\mathsf{surj}(f)\defeq\prd{b:B} \|\hfiber{f}b\|.
\end{equation*}
A monomorphism is a functions with contractible homotopy fibers over images of points in $A$, i.e.\ we define
\note{Move to \autoref{sec:mono-epi}?}
\begin{equation*}
\mathsf{inj}(f)\defeq\prd{a:A} \mathsf{isContr}(\hfiber{f}{f(a)}).
\end{equation*}
\end{defn}

We remark that $f:A\to B$ is a monomorphism if and only if the function
\begin{equation*}
\lambda a.\pairr{a,a,\refl{f(a)}}:A\to A\times_B A
\end{equation*}
is an equivalence.

The injective functions are of course the monomorphisms of
$\set$, which we can also define via a pullback diagram, but we will not do that here.

\begin{defn}
Let $f:A\to B$ be a function between $\set$s. Define the image 
\begin{equation*}
\im(f)\defeq\sm{b:B} \|\hfiber{f}b\|
\end{equation*}
We define the functions $\tilde{f}:A\to\im(f)$ and $i_f:\im(f)\to B$ by
\begin{align*}
\tilde{f} & \defeq \lambda a.\pairr{f(a),\pi(a,\refl{f(a)})}\\
i_f & \defeq \pi_1.
\end{align*}
\end{defn}

\begin{lem}
Then $\tilde{f}$ is surjective and $i_f$ is mono.
\end{lem}

We use the insights from~\autoref{subsec:reflective-subuniverses} and look at coequalizers
in $\set$. The coequalizer is defined using the truncated pushout. Let $D$ be the diagram $f,g:A\to B$.
The general, non-truncated, pushout of $D$ does not have the quotient-like properties as it would
have in the category $\set$. For instance, the coequalizer of
$\unit\rightrightarrows\unit$ is the circle. Nevertheless,
the $0$-truncation of the circle is again $\unit$, which is the
expected coequalizer in $\set$. Thus we get

\begin{defn}
Let $A$ and $B$ be sets and let $f,g:A\to B$.
We denote the set-coequalizer of $f$ and $g$ by $c_{f,g}:B\to B/_{f,g}$. A
\emph{regular epimorphism} is a function between sets which is a coequalizer.
\end{defn}

Recall from \autoref{defn:homotopy} the notation $f \sim g$ for homotopy between functions.

\begin{lem}
Let $f,g:A\to B$ be functions between sets $A$ and $B$. The 
{set-co}equalizer $c_{f,g}:B\to B/_{f,g}$ has the property that
\begin{equation*}
\prd{C:\set}{h:B\to C}{H:h\circ f\sim h\circ g}
\mathsf{isContr}\big(\sm{k:B/_{f,g}\to C} k\circ c_{f,g}\sim h\big).
\end{equation*}
\end{lem}

As explained in~\autoref{subsec:reflective-subuniverses}, this is a general phenomenon, the 0-truncated colimits behave as expected.

To prove that $\set$ is a regular category, we will first connect epis and surjective functions. 
We use the pinciple of unique choice; see~\autoref{cor:UC}. 

\begin{defn}
Let $f:A\to B$ be a function between sets. Define
\begin{align*}
\mathsf{epi}(f) & \defeq \prd{X:\set}{g,h:B\to X}
(g\circ f\sim h\circ f)\to (g\sim h)\\
\mathsf{epi}'(f) & \defeq\prd{X:\set}{g:B\to X}
\mathsf{isContr}\big(\sm{h:B\to X} g\circ f\sim h\circ f\big).
\end{align*}
\end{defn}

Note that $\mathsf{epi}(f)$ is the usual notion of epimorphism while the
notion $\mathsf{epi}'(f)$ seems to be stronger. The latter seems to be the right notion for general types.

\begin{lem}\label{epis-surj}
For any function $f:A\to B$ between sets, the following are equivalent:
\begin{enumerate}
\item $f$ is an epimorphism, i.e.\ we have $\mathsf{epi}(f)$.
\item We have $\mathsf{epi}'(f)$.
\item Consider the pushout diagram
\begin{equation*}
\begin{tikzpicture}
\matrix (m) [std] {A & B \\ \mathsf{unit} & C_f \\};
\draw[ar] (m-1-1) -- node[above] {$f$} (m-1-2);
\draw[ar] (m-1-2) -- node[right] {$\iota$} (m-2-2);
\draw[ar] (m-1-1) -- (m-2-1);
\draw[ar] (m-2-1) -- node[below] {$t$} (m-2-2);
\end{tikzpicture}
\end{equation*}
in $\set$ defining the mapping cone. The type $C_f$ is contractible.
\item $f$ is surjective.
\end{enumerate}
\end{lem}

\begin{proof}
Let $f:A\to B$ be a function between sets.
To show that $\mathsf{epi}(f)\to\mathsf{epi}'(f)$, let $H:\mathsf{epi}(f)$,
let $X$ be a set and let $g:B\to X$ be a function. Then we find
\begin{equation*}
\pairr{g,\refl{g\circ f}}:\sm{h:B\to X} g\circ f\sim h\circ f.
\end{equation*}
Thus, to show that $\sm{h:B\to X} g\circ f\sim h\circ f$ 
is contractible we need to show that
\begin{equation*}
\prd{h:B\to X}{K:g\circ f\sim h\circ f}\sm{L:g\sim h} K\sim L\circ f.
\end{equation*}
Note that $K\sim L\circ f$ is contractible since $X$ is assumed to be a set,
hence it suffices to show that
\begin{equation*}
\prd{h:B\to X} (g\circ f\sim h\circ f)\to (g\sim h).
\end{equation*}
This follows at once from the assumption that $f$ is an epimorphism.

To show that $\mathsf{epi}'(f)\to\mathsf{isContr}(C_f)$, suppose that
$H:\mathsf{epi}'(f)$. The basic constructor of $C_f$ corresponding to
$\mathsf{unit}\to C_f$ gives us an element $t:C_f$. We have to show that
\begin{equation*}
\prd{x:C_f} x= t.
\end{equation*}
Note that $x= t$ is a proposition, hence we can use induction on
$C_f$: it suffices to find
\begin{align*}
I_0 & : \prd{b:B} \iota(b)= t\\
I_1 & : \prd{a:A} \alpha_1(a)\cdot I_0(f(a))=\refl{t}.
\end{align*}
where $\alpha_1:\prd{a:A} \iota(f(a))= t$ is a basic constructor
of $C_f$. Note that $\alpha_1$ is a homotopy from $\iota\circ f$ to
$\mathsf{const}_t\circ f$, so we find the terms
\begin{equation*}
\pairr{\iota,\refl{\iota\circ f}},\pairr{\mathsf{const}_t,\alpha_1}:
\sm{h:B\to C_f} \iota\circ f\sim h\circ f.
\end{equation*}
By the assumption $H:\mathsf{epi}'(f)$, it follows that there is a path
\begin{equation*}
\gamma:\pairr{\iota,\refl{\iota\circ f}}=\pairr{\mathsf{const}_t,\alpha_1}
\end{equation*}
Hence we get $I_0$ from $A:\iota=\mathsf{const}_t$. Moreover,
we get $ R:A\cdot\refl{\iota\circ f}=\alpha_1$.
Note that $(A\cdot\refl{\iota\circ f})(a)= I_0(f(a))$, so it follows
that $I_0(f(a))= \alpha_1(a)$ for all $a:A$, which gives us $I_1$.

To show that $\mathsf{isContr}(C_f)\to\mathsf{surj}(f)$,
let $H:\mathsf{isContr}(C_f)$. Using the univalence axiom, we construct
a dependent type $P:C_f\to\prop$. Note that $\prop$ is
a proposition, so we can use induction on $C_f$. We define
\begin{align*}
P(t) & \defeq \mathsf{unit}\\
P(\iota(b)) & \defeq \|\hfiber{f}b\|.
\end{align*}
For $a:A$ the type $\|\hfiber{f}{f(a)}\|$ is canonically
equivalent to $\mathsf{unit}$, which finishes the construction of $P$.
Since $C_f$ is assumed to be contractible it follows that $P(x)$ is
equivalent to $P(t)$ for any $x:C_f$. In particular we find that
$\|\hfiber{f}b\|$ is contractible for each $b:B$, showing
that $f$ is surjective.

To show that $\mathsf{surj}(f)\to\mathsf{epi}(f)$,
let $f:A\to B$ be a surjective function and consider a set $C$ and two functions
$g,h:B\to C$ with the property that $g\circ f\sim h\circ f$. Since $f$ 
is assumed to be surjective,
we have an equivalence $B\simeq\mathsf{im}(f)$. We have the following equivalences
\begin{align*}
\prd{b:B} g(b)= h(b) 
& \simeq \prd{w:\mathsf{im}(f)} g(\pi_1 w)= h(\pi_1(w))\\
& \simeq \prd{b:B}{a:A}{p:f(a)= b} g(b)= h(b)\\
& \simeq \prd{a:A} g(f(a))= h(f(a)).
\end{align*}
By assumption, there is an element of the latter type.
\end{proof}
The proof that epis are surjective in~\cite{Mines/R/R:1988} uses the power set operation. 
This proof can be made predicative by using a large power set and typical ambiguity.
A predicative proof for setoids was given by Wilander~\cite{Wilander2010}. 
The proof above is similar, but avoids setoids by using the pushout and the
univalence axiom. 
% \begin{rem}
% The above theorem is not true when we replace $\set$ by $\type$
% (replacing it also in the definition of $\mathsf{epi}$ and $\mathsf{epi}'$). 
% However, we do
% get the implications $\textit{ii.}\Rightarrow\textit{iii.}\Rightarrow
% \textit{iv.}$
% \end{rem}

\begin{lem}\label{lem:images_are_coequalizers}
Surjective functions between sets are regular epimorphisms.
\end{lem}

\begin{proof}
Note that it suffices to show that for any function $f:A\to B$, the diagram
\begin{equation*}
\begin{tikzpicture}
\matrix (m) [std] {\sm{x,y:A} f(x)= f(y) & A & \mathsf{im}(f) \\};
\draw[ar] ([yshift=.5ex]m-1-1.east) -- node[above] {$\pi_1$} ([yshift=.5ex]m-1-2.west);
\draw[ar] ([yshift=-.5ex]m-1-1.east) -- node[below] {$\pi_2$} ([yshift=-.5ex]m-1-2.west);
\draw[ar] (m-1-2) -- node[above] {$\tilde{f}$} (m-1-3);
\end{tikzpicture}
\end{equation*}
is a coequalizer diagram.

The first thing we have to verify is that there is a homotopy 
$H:\tilde{f}\circ\pi_1\sim\tilde{f}\circ\pi_2$. Let $\pairr{x,y,p}$ be an element
of $\sm{x,y:A} f(x)= f(y)$. 
Then we have $\tilde{f}(\pi_1(\pairr{x,y,p}))= \pairr{f(x),u}$, 
where $u$ belongs to the contractible type $\|\hfiber{f}{f(x)}\|$. 
Similarly, we have a path 
$\tilde{f}(\pi_2(\pairr{x,y,p}))= \pairr{f(y),v}$,
where $v$ belongs to the contractible type $\|\hfiber{f}{f(y)}\|$.
Since we have $p:f(x)= f(y)$ and since
$\|\hfiber{f}{f(y)}\|$ is contractible, 
it follows that we get a path from $\tilde{f}(\pi_1(\pairr{x,y,p}))$ to
$\tilde{f}(\pi_2(\pairr{x,y,p}))$, which gives us our homotopy.

Now suppose that $g:A\to X$ is a function for which there is a homotopy 
$K:g\circ\pi_1\sim g\circ\pi_2$. We have to show that the space
\begin{equation*}
\sm{h:\mathsf{im}(f)\to X}{L:h\circ\tilde{f}\sim g} h\circ H\sim K
\end{equation*}
is contractible. We will apply unique choice to define a 
function from $\mathsf{im}(f)$ to $X$. Let $R',R:\mathsf{im}(f)\to
X\to\type$ be the relations defined by 
\begin{align*}
R'(w,x) & \defeq\prd{a:A} (\tilde{f}(a)= w)\to (g(a)= x)\\
R(w,x) & \defeq \|R'(w,x)\|.
\end{align*}
Then $\mathsf{atMostOne}(R(w))$ is inhabited for every $w:\mathsf{im}(f)$. 
To see this, note that the type $\mathsf{atMostOne}(R(w))$ is a
proposition. Therefore, there is an equivalence
\begin{equation*}
\big(\prd{w:\mathsf{im}(f)} \mathsf{atMostOne}(R(w))\big)
\simeq\prd{a:A} \mathsf{atMostOne}(R(\tilde{f}(a))).
\end{equation*}
We can simplify this even further to
\begin{equation*}
\prd{a:A} \mathsf{atMostOne}(R^\prime(\tilde{f}(a))).
\end{equation*}
This follows from the assumption that $g(a)= g(a^\prime)$ 
whenever $f(a)= f(a^\prime)$. Let $a:A$, $x,x^\prime:X$,
$u:R^\prime(\tilde{f}(a),x)$ and $u^\prime:R^\prime(\tilde{f}(a),x^\prime)$. 
Then there are the paths $u(a,\refl{\tilde{f}(a)}):g(a)=
x$ and $u^\prime(a,\refl{\tilde{f}(a)}):g(a)= x^\prime$, 
showing that $x= x^\prime$. 

Also, $\|\sm{x:X} R(w,x)\|$ is inhabited for every $w:\mathsf{im}(f)$. Indeed, the space
\begin{equation*}
\prd{w:\mathsf{im}(f)} \big\| \sm{x:X} R(w,x)\big\|
\end{equation*}
is equivalent to the space
\begin{equation*}
\prd{a:A} \big\| \sm{x:X} R(\tilde{f}(a),x)\big\|.
\end{equation*}
The space $\prd{a:A}\sm{x:X} R^\prime(\tilde{f}(a),x)$ is inhabited by
\begin{equation*}
\lambda a.\pairr{g(a),\lambda a^\prime,p.K(\pairr{a^\prime,a,p^{-1}})}
\end{equation*}
This shows that the hypotheses of the principle of unique choice are satisfied, so we get an element of
\begin{equation*}
\sm{h:\mathsf{im}(f)\to X}\prd{w:\mathsf{im}(f)} R(w,h(w)).
\end{equation*}
An immediate consequence of the way we constructed our function $h:\mathsf{im}(f)\to X$ is that $h\circ\tilde{f}\sim g$. The result follows
now from the observation that the space
\begin{equation*}
\sm{h^\prime:\mathsf{im}(f)\to X} h^\prime\circ\tilde{f}\sim h\circ\tilde{f}
\end{equation*}
is contractible because $\tilde{f}$ is an epimorphism. 
\end{proof}

\begin{lem}\label{lem:pb_of_coeq_is_coeq}
Pullbacks of surjective functions are surjective. Consequently,
pullbacks of coequalizers are coequalizers.
\end{lem}

\begin{proof}
Consider a pullback diagram
\begin{equation*}
\begin{tikzpicture}
\matrix (m) [std] {A & B \\ C & D \\};
\draw[ar] (m-1-1) -- (m-1-2);
\draw[ar] (m-1-2) -- node[right] {$g$} (m-2-2);
\draw[ar] (m-1-1) -- node[left]  {$f$} (m-2-1);
\draw[ar] (m-2-1) -- node[below] {$h$} (m-2-2);
\end{tikzpicture}
\end{equation*}
and assume that $g$ is surjective. Applying the pasting lemma of pullbacks
with the morphism $c:\mathsf{unit}\to C$, we obtain an
equivalence $\hfiber{f}c\simeq\hfiber{g}{h(c)}$ for any
$c:C$. This equivalence gives that $f$ is surjective.
\end{proof}

\begin{thm}\label{thm:set_regular}
The category $\set$ is regular.
\end{thm}

\begin{proof}
$\set$ has all limits, so it is finitely complete. 
\autoref{lem:images_are_coequalizers} gives
that the kernel pair of each function has a coequalizer.
\autoref{lem:pb_of_coeq_is_coeq} shows that
coequalizers are stable under pullbacks.
\end{proof}


\subsection{Quotients}\label{subsec:quotients}
Now that we know that $\set$ is regular, to show that $\set$ is exact, we need to show that every
equivalence relation is effective. In other words, given an equivalence
relation $R:A\to A\to\prop$, there is a coequalizer $c_R$ of the pair
$\pi_1,\pi_2:\sm{x,y:A} R(x,y)\to A$ and, moreover, the $\pi_1$ and $\pi_2$
for the kernel pair of $c_R$.

We have already seen, in \autoref{sec:set-quotients}, two general ways to construct the quotient of a set by an equivalence relation $R:A\to A\to\prop$.
The first can be described as the set-coequalizer of the two projections
\[\pi_1,\pi_2:\sm{x,y:A} R(x,y)\to A.\]
The important property of such a quotient is the following.

\begin{defn}
A relation $R:A\to A\to\prop$ is said to be \emph{effective} if the square
\begin{equation*}
\begin{tikzpicture}
\matrix (m) [std] {\sm{x,y:A} R (x,y) & A \\ A & A/R \\};
\draw[ar] (m-1-1) -- node[above] {$\pi_1$} (m-1-2);
\draw[ar] (m-1-2) -- node[right] {$c_{R }$} (m-2-2);
\draw[ar] (m-1-1) -- node[left]  {$\pi_2$} (m-2-1);
\draw[ar] (m-2-1) -- node[below] {$c_R $} (m-2-2);
\end{tikzpicture}
\end{equation*}
is a pullback. 
\end{defn}

\begin{lem}\label{lem:sets_exact}
Suppose $\pairr{A,R}$ is an equivalence relation. Then there is an equivalence
\begin{equation*}
(c_R(x)= c_R(y))\simeq R(x,y)
\end{equation*}
for any $x,y:A$. In other words, equivalence relations are effective.
\end{lem}

\begin{proof}
We begin by extending $R$ to a relation $\tilde{R}:A/R\to A/R\to\prop$. After
the construction of $\tilde{R}$ we will show that $\tilde{R}$ is equivalent
to the identity type on $A/R$. We define $\tilde{R}$ by double induction on
$A/R$ (note that $\prop$ is a set by univalence for propositions). We
define $\tilde{R}(c_R(x),c_R(y)) \defeq R(x,y)$. For $r:R(x,x')$ and $s:R(y,y')$,
the transitivity and symmetry 
of $R$ gives an equivalence from $R(x,y)$ to $R(x',y')$. This completes the
definition of $\tilde{R}$. To finish the proof of the proposition, we need
to show that $\tilde{R}(w,w')\simeq w= w'$ for every $w,w':A/R$. We can
do this by showing that the type $\sm{w':A/R} \tilde{R}(w,w')$ is contractible for
each $w:A/R$. We do this by induction. Let $x:A$. We have the element
$\pair{c_R(x),\rho(x)}:\sm{w':A/R} \tilde{R}(c_R(x),w')$, where $\rho$ is
the reflexivity proof of $R$, hence we only
have to show that
\begin{equation*}
\prd{w':A/R}{r:\tilde{R}(c_R(x),w')} \pairr{w',r}=\pairr{c_R(x),\rho(x)},
\end{equation*}
which we do by induction on $w'$. Let $y:A$ and let $r:R(x,y)$. Then we have
the path $p_R(r)^{-1}:c_R(y)= c_R(x)$. We automatically get a path from
$p_R(r)^{-1}\cdot r=\rho(x)$, finishing the proof.
\end{proof}

The second construction of quotients, which uses impredicativity of mere propositions, was as the set of equivalence classes of $R$ (a subset of its power set):
\[ A\sslash R \defeq \setof{ P:A\to\prop | P \text{ is an equivalence class of } R} \]
Note that if we regard $R$ as a function from $A$ to $A\to \prop$, then $A\sslash R$ is equivalent to $\im(R)$, as constructed in \autoref{sec:image}.
Now in \ref{lem:images_are_coequalizers} we have shown that images are
coequalizers. In particular, we immediately get the coequalizer diagram
\begin{equation*}
\begin{tikzpicture}
\matrix (m) [std] {\sm{x,y:A} R (x)= R (y) & A & A\sslash  R.  \\};
\draw[ar] ([yshift=.5ex]m-1-1.east) -- node[above] {$\pi_1$} ([yshift=.5ex]m-1-2.west);
\draw[ar] ([yshift=-.5ex]m-1-1.east) -- node[below] {$\pi_2$} ([yshift=-.5ex]m-1-2.west);
\draw[ar] (m-1-2) -- (m-1-3);
\end{tikzpicture}
\end{equation*}
We can use this to give an alternative proof that any equivalence relation is effective and that the two definitions of quotients agree.

\begin{thm}\label{prop:kernels_are_effective}
For any function $f:A\to B$ between any two sets, 
the relation $\ker(f):A\to A\to\type$ given by 
$\ker(f,x,y)\defeq f(x)= f(y)$ is effective. 
\end{thm}

\begin{proof}
We will use that $\im(f)$ is the coequalizer of $\pi_1,\pi_2:
\sm{x,y:A} f(x)= f(y)\to A$; 
%we get this equivalence from proposition~\ref{prop:images_are_coequalizers}
. Note that the canonical kernel pair of the function 
$c_f\defeq\lambda a.\pairr{f(a),\tau_1(\pairr{a,\refl{f(a)}})}$ consists 
of the two projections
\begin{equation*}
\pi_1,\pi_2:\big(\sm{x,y:A} c_f(x)= c_f(y)\big)\to A.
\end{equation*}
For any $x,y:A$, we have equivalences
\begin{align*}
c_f(x)= c_f(y) & \simeq \sm{p:f(x)= f(y)} p\cdot\tau_1(\pairr{x,\refl{f(x)}})
=\tau_1(\pairr{y,\refl{f(x)}})\\ & \simeq f(x)= f(y),
\end{align*}
where the last equivalence holds because 
$\|\hfiber{f}b\|$ is a proposition for any $b:B$. 
Therefore, we get that
\begin{equation*}
\sm{x,y:A} c_f(x)= c_f(y)\simeq\sm{x,y:A} f(x)= f(y)
\end{equation*}
and hence we may conclude that $\ker f$ is an effective relation 
for any function $f$.
\end{proof}

\begin{thm}
Equivalence relations are effective and there is an equivalence $A/R \simeq A\sslash  R $. 
\end{thm}

\begin{proof}
We need to analyze the coequalizer diagram
\begin{equation*}
\begin{tikzpicture}
\matrix (m) [std] {\sm{x,y:A} R (x)= R (y) & A & A\sslash  R. \\};
\draw[ar] ([yshift=.5ex]m-1-1.east) -- node[above] {$\pi_1$} ([yshift=.5ex]m-1-2.west);
\draw[ar] ([yshift=-.5ex]m-1-1.east) -- node[below] {$\pi_2$} ([yshift=-.5ex]m-1-2.west);
\draw[ar] (m-1-2) -- (m-1-3);
\end{tikzpicture}
\end{equation*}
By the univalence axiom, the space $R (x)= R (y)$ is equivalent to the space of homotopies from $R (x)$ to $R (y)$, which is
equivalent to $\prd{z:A} R (x,z)\simeq R (y,z)$. Since $R $ is an equivalence relation, the latter space is equivalent to $R (x,y)$. To
summarize, we get that $(R (x)= R (y))\simeq R (x,y)$, so $R $ is effective since it is equivalent to an effective relation. Also,
the diagram
\begin{equation*}
\begin{tikzpicture}
\matrix (m) [std] {\sm{x,y:A} R (x,y) & A & A\sslash  R. \\};
\draw[ar] ([yshift=.5ex]m-1-1.east) -- node[above] {$\pi_1$} ([yshift=.5ex]m-1-2.west);
\draw[ar] ([yshift=-.5ex]m-1-1.east) -- node[below] {$\pi_2$} ([yshift=-.5ex]m-1-2.west);
\draw[ar] (m-1-2) -- (m-1-3);
\end{tikzpicture}
\end{equation*}
is a coequalizer diagram. Since coequalizers are unique up to equivalence, it follows that $A/R \simeq A\sslash  R $.
\end{proof}

\subsection{Predicativity}\label{sec:resizing}
One may wonder about the predicative interpretation of the quotient constructions above.
One could argue that the HIT construction is predicative by considering the interpretation of this
quotient in the setoid model~\cite{Altenkirch1999,coquand2012constructive}. 
In this model the quotient does not raise the universe level. 
A similar observation holds for constructions that can be carried out in the groupoid 
model~\cite{hofmann1998groupoid}. These observations should suffice for the set-level higher inductive 
types we use in the present paper.

Voevodsky~\cite{Universe-poly} has proposed to allow some impredicativity by adding the following resizing rule:
\[a:U_n, p:(\mathsf{isprop}\ a)\vdash (\mathsf{rr}\ a\ p): U_0,\]
where $U_n$ is any universe and $U_0$ is the lowest universe.
Moreover, when lifted to $U_n$, $(\mathsf{rr}\ a\ p)= a$.
Hence, $\mathsf{rr}$ is an equivalence between $\prop_n$ and $\prop_0$. 
Here we assume that $U_0$ is a subuniverse of $U_n$. 
Another axiom resizes each $\prop_n$ to the lowest universe.

The following replacement axiom is derivable from the resizing axiom; see~\cite{Universe-poly}.
\begin{lem}\label{lem:replacement}
Let $U$ be a universe and $X:U$, if $f:X\twoheadrightarrow Y$ is a surjection, $Y$ is a set then there exists a $Z:U$ which is
equivalent to $Y$.
\end{lem}
\begin{proof}
Define $Z\defeq X\sslash \ker f$ using the type of equivalence classes; see~\ref{def:VVquotient}. 
Then $Z:U$, since $\prop_U$ is in the lowest universe. Finally, we observe that $Z\defeq X/\ker f\simeq\im f\simeq Y$.
\end{proof}

\subsection{The object classifier}\label{subsec:object-classification}
In topos theory we have a small subobject classifier classifying the monomorphisms.
In HoTT without resizing rules, $\prop$ is large. It is a reflective subuniverse see \autoref{subsec:reflective-subuniverses}.
In higher topos theory one considers not only subobject classifiers, but also \emph{object classifiers} 
which classify more general classes of maps.
In this section we will establish the existence of an internal analogue of such object classifiers, 
they will always sit a higher universe. Moreover, we will find an $n$-object classifier for
every $n:\mathbb{N}_{\geq-2}$, where the $n$-object classifier will classify the
functions with $n$-truncated homotopy fibers. 

In addition to the size issue of the object classifers, we will see that the
$n$-object classifier will generally not be of (homotopy) level $n$, but of level
$n+1$. This observation should be regarded in contrast to the theory of 
predicative toposes, where a universal small
map is required to exist. Such a universal small map is suggested to be a map
between sets, 
but it seems that within homotopy type theory we cannot expect
such a map to exist. The main reason is that the universal small map of sets
will in general be a map of groupoids; a universal small map of groupoids will
in general be a map of $2$-groupoids, etc.

\begin{thm}\label{thm:nobject_classifier_appetizer}
For any type $B$ there is an equivalence
\begin{equation*}
\chi:\big(\sm{A:\type} A\to B\big)\simeq B\to\type.
\end{equation*}
Likewise, there is an equivalence
\begin{equation*}
\chi_n:\big(\sm{A:\type}{f:A\to B} \mathsf{isTrunc}(n,f)\big)\simeq B\to \typele{n}
\end{equation*}
for every $n:\mathbb{N}_{\geq-2}$. Here $\mathsf{isTrunc}(n,f)$ denotes that all fibers of $f$ are $n$-truncated
\end{thm}

\begin{proof}
We begin by constructing the first equivalence, i.e.\ we have to construct functions
\begin{align*}
\chi & : \big(\sm{A:\type} A\to B\big)\to B\to\type\\
\psi & : (B\to\type)\to\big(\sm{A:\type} A\to B\big).
\end{align*}
The function $\chi$ is defined by $\chi(f,b)\defeq\hfiber{f}b$. The
function $\psi$ is defined by $\psi(P)\defeq\pairr{(\sm{b:B} P(b)),\pi_1}$. Now
we have to verify that $\chi\circ\psi\sim\idfunc{}$ and that $\psi\circ\chi
\sim\idfunc{}$:
\begin{enumerate}
\item Let $P$ be a dependent type over $B$. It is a basic fact that 
$\hfiber{\pi_1}{b}\simeq P(b)$ and therefore it follows immediately
that $P\sim\chi(\psi(P))$.
\item Let $f:A\to B$ be a function. We have to find a path
\begin{equation*}
\pairr{(\sm{b:B} \hfiber{f}b),\pi_1}=\pairr{A,f}
\end{equation*}
First note that we have the basic equivalence
$e:\sm{b:B} \hfiber{f}b\simeq A$ with $e(b,a,p)\defeq a$ and $e^{-1}(a)
\defeq(f(a),a,\refl{f(a)})$. It also follows that
$e\cdot\pi_1=\pi_1\circ e^{-1}$. From this, we immediately read off
that $(e\cdot\pi_1)(a)= f(a)$ for each $a:A$. This completes the proof
of the first of the asserted equivalences.
\end{enumerate}
To find the second set of equivalences, note that if we restrict $\chi$ to
functions with $n$-truncated homotopy fibers we get an $n$-truncated dependent
type. Likewise, if we restrict $\psi$ to $n$-truncated dependent types we get
a function with $n$-truncated homotopy fibers.
\end{proof}

\begin{defn}
Define
\begin{equation*}
\pointed{\type}\defeq\sm{A:\type} A\qquad\text{and}\qquad \pointed{\typele{n}}\defeq
\sm{A:\typele{n}} A.
\end{equation*}
Thus, $\pointed{\type}$ stands for the \emph{pointed types} and $\pointed{\typele{n}}$ stands for
the pointed $n$-types.
\end{defn}

The following theorem states that we have an object classifier.
\begin{thm}\label{thm:nobject_classifier}
Let $f:A\to B$ be a function. Then the diagram
\begin{equation*}
\begin{tikzpicture}
\matrix (m) [std] {A & \pointed{\type} \\ B & \type \\};
\draw[ar] (m-1-1) -- node[above] {$\vartheta_f$} (m-1-2);
\draw[ar] (m-1-2) -- node[right] {$\pi_1$} (m-2-2);
\draw[ar] (m-1-1) -- node[left]  {$f$} (m-2-1);
\draw[ar] (m-2-1) -- node[below] {$\chi_f$} (m-2-2);
\end{tikzpicture}
\end{equation*}
is a pullback diagram. Here, the function $\vartheta_f$ is defined by
\begin{equation*}
\lambda a.\pairr{\hfiber{f}{f(a)},\pairr{a,\refl{f(a)}}}.
\end{equation*}
A similar statement holds when we replace $\type$ by $\typele{n}$.
\end{thm}
\begin{proof}
Note that we have the equivalences
\begin{align*}
A & \simeq \sm{b:B} \hfiber{f}b\\
 & \simeq \sm{b:B}{X:\type}{p:\hfiber{f}b= X} X\\
 & \simeq \sm{b:B}{X:\type}{x:X} \hfiber{f}b= X\\
 & \equiv B\times_{\type}\pointed{\type}.
\end{align*}
which gives us a composite equivalence $e:A\simeq B\times_\type\pointed{\type}$. 
We may display the action of this composite equivalence step by step by
\begin{align*}
a & \mapsto \pairr{f(a),\pairr{a,\refl{f(a)}}}\\
 & \mapsto \pairr{f(a),\hfiber{f}{f(a)},\refl{\hfiber{f}{f(a)}},\pairr{a,\refl{f(a)}}}\\
 & \mapsto \pairr{f(a),\hfiber{f}{f(a)},\pairr{a,\refl{f(a)}},\refl{\hfiber{f}{f(a)}}}
\end{align*}
Therefore, we get homotopies $f\sim\pi_1\circ e$ and $\vartheta_f\sim \pi_2\circ e$. 
\end{proof}

\begin{lem}\label{lem:subobject}
The type $\pointed{\prop}$ is contractible.
\end{lem}
\begin{proof}
Suppose that $\pairr{P,u}$ is an element of $\sm{P:\prop} P$. Then we have $u:P$ and hence there is a term of type $\mathsf{isContr}(P)$. It
follows that $P\simeq\unit$ and therefore we get from the univalence axiom that there is a path
$\pairr{P,u}=\pairr{\unit,\mathsf{tt}}$.
\end{proof}

If we use the resizing rules we can replace the large type $\sm{P:\prop} P$ with the small type $\unit$. In this way we would obtain
the usual notion of a subobject classifier. Without the resizing rules we do obtain a large subobject classifier.

\subsection{\texorpdfstring{$\set$}{Set} is a \texorpdfstring{$\Pi$}{Π}W-pretopos}
\label{subsec:piw}
\begin{thm} The category $\set$ is a $\Pi$W-pretopos.
\end{thm}
\begin{proof}
We have an initial object and disjoint sums. We have finite limits and finite sums (\autoref{thm:set_regular}). 
Sums are stable under pullback. So, $\set$ is lextensive. 
$\set$ is locally Cartesian closed. This follows from the preparations we made in
\autoref{sec:compute-pi}, using the
fact that the existence of $\prod$-types gives local Cartesian closure; e.g.~\citep[Prop.~1.9.8]{jacobs1999categorical}.
The category $\set$ is regular (\autoref{thm:set_regular}) and quotients are effective (\autoref{lem:sets_exact}). We thus have an
exact category, since it is also lextensive, we have a pretopos. It has $\prod$-types (\autoref{thm:hlevel-prod}) and
W-types
(\autoref{sec:w-types}), so we have a $\Pi$W-pretopos. 
\end{proof}

One wonders what prevents $\set$ from being a topos. We the lack impredicativity to define the subobject classifier. If we
assume the resizing rules from \autoref{sec:resizing}, then $\prop$ becomes small. We have seen in
\autoref{thm:nobject_classifier} that it satisfies the properties of a subobject classifier and hence we actually
obtain a topos: $\set$ is also (locally) cartesian closed and has all finite limits and colimits.

\subsection{All surjectives split implies decidability of propositions}\label{sec:surj-split}
In this subsection we derive that if all surjective functions between sets have
a section, then for all propositions $A$ we can decide whether $A$ or $\neg A$
holds. This is an adaptation of an old, famous theorem by
Diaconescu and Bishop to homotopy type theory.

Recall the notion of \emph{suspension} from \autoref{sec:suspension}.
In the case where $A$ is a proposition, it turns out that the
suspension $\susp(A)$ is equivalent to the quotient $\mathsf{bool}/R$,
where $R$ is the equivalence relation we get by setting
$R(\mathsf{true},\mathsf{false})\defeq A$. This shall become clear in the proofs
below.

\begin{lem}\label{prop:trunc_of_prop_is_set}
The suspension of a proposition is a set and the path space 
$0=1$ in $\susp(A)$ is equivalent to $A$, for any proposition $A$. 
\end{lem}

\begin{proof}
Let $A$ be a proposition. Using the univalence axiom, we will define a 
dependent type $P:\susp(A)\to\susp(A)\to\type$ with the 
property that $P(x,y)$ is a proposition for each $x,y:\susp(A)$ 
and which turns out to be equivalent to the dependent type 
$\mathsf{Id}_{\susp(A)}$.

We make the following definitions:
\begin{align*}
P(0,0) & \equiv \unit & P(1,0) & \equiv A\\
P(0,1) & \equiv A & P(1,1) & \equiv \unit.
\end{align*}
To show that this gives a dependent type we need to verify that there 
is an equivalence $P(0,0)\simeq P(0,1)$ and an eqivalence 
$P(1,0)\simeq P(1,1)$ for every $a:A$. Since $A$ is assumed to 
be a proposition, this is indeed the case.

To finish the proof we need to show that $P(x,y)\simeq x=y$ 
for every $x,y:\susp(A)$. We can find such a fiberwise equivalence 
by finding a transformation
\begin{equation*}
\tau(y):\prd{x:\susp(A)} P(x,y)\to x=y
\end{equation*}
which induces an equivalence 
\begin{equation*}
\big(\sm{x:A} P(x,y)\big)\simeq\big(\sm{x:\susp(A)} x=y\big)
\end{equation*}
of total spaces, for every $y:\susp(A)$. The latter type 
is contractible, so it suffices to find the mentioned transformation 
and proof that the total space on the left is contractible. We may do 
the first of these tasks by applying the induction principle for 
$\susp(A)$. We make the definitions
\begin{align*}
\tau(0,0) & \defeq \lambda t.\refl{0} & \tau(1,0) &\defeq \lambda a.\alpha(a)^{-1}\\
\tau(0,1) & \defeq \alpha & \tau(1,1) &\defeq \lambda t.\refl{1}.
\end{align*}
To find a path $\alpha(a)\cdot\tau(0,0)=\tau(0,1)$ for $a:A$, 
note that because $A$ is a proposition there are paths
\begin{equation*}
(\alpha(a)\cdot\tau(0,0))(x)=\tau(0,0,x)\bullet\alpha(a)^{-1}
=\alpha(a)^{-1}= \alpha(x)^{-1}. 
\end{equation*}
Thus we get the requested path from function extensionality. Likewise, 
we obtain a path $\tau(0)=\tau(1)$ for every $a:A$, 
which finishes the construction of $\tau$.

The last thing to do to finish the proof is to show that 
$\sm{x:A} P(x,y)$ is contractible for every $y:\susp(A)$. 
Since $\mathsf{isContr}(X)$ is a mere proposition for every type $X$, 
we only need to show that $\sm{x:A} P(x,0)$ and 
$\sm{x:\susp(A)} P(x,1)$ are contractible. 
They are clearly equivalent, so we only verify the contractibility 
of the latter. The obvious candidate for the center of contraction 
is $\pairr{1,\mathsf{tt}}$, thus we need to show that there is a 
function $c$ of type
\begin{equation*}
\prd{x:\susp(A)}{u:P(x,1)} \langle{x,u}\rangle=\langle{1,\mathsf{tt}}\rangle.
\end{equation*}
We do this again by induction on $x$. To define $c(0)$, note that 
for any $a:A$, we have the path $\alpha(a):0= 1$. Since 
$\unit$ is contractible we will automatically get a path 
$\alpha(a)\cdot u=\mathsf{tt}$. The definition of $c(1)$ is obvious. 
Then we get, for $a:A$, a path 
$(\alpha(a)\cdot c(0))(\mathsf{tt})= c(1,\mathsf{tt})$ 
because $\susp(A)$ is contractible when $A$ is contractible. 
This gives us a path $\alpha(a)\cdot c(0)= c(1)$ and finishes 
the proof by induction that $\sm{x:A} P(x,1)$ is contractible.
\end{proof}

\begin{defn}
Suppose $A$ is a type. We define
\begin{equation*}
\mathsf{isDecidable}(A)\equiv  A+\neg A
\end{equation*}
\end{defn}

\begin{thm}[Bishop/Diaconescu]\label{thm:1surj_to_surj_to_pem}
Suppose that all surjections between sets split. Then for all $A:\prop$,
$\mathsf{isDecidable}(A)$. 
\end{thm}

\begin{proof}
Consider the function $\pi:\mathsf{bool}\to\susp(A)$ defined by 
$\pi(\mathsf{true})\equiv 1$ and $\pi(\mathsf{false})\equiv 0$. 
The first thing we will do is show that $\pi$ is surjective, 
i.e.\ that there is a function of type
\begin{equation*}
\prd{x:\susp(A)} \|\hfiber\pi{x}\|.
\end{equation*}
Note that we have the terms 
$\langle{\mathsf{false},\refl{0}}\rangle:\hfiber\pi0$ 
and $\langle{\mathsf{true},\refl{1}}\rangle:\hfiber\pi1$. 
Since $\|\hfiber\pi{x}\|$ is a proposition for each 
$x:\susp(A)$, this gives us the proof by induction of 
the surjectivity of $\pi$.

By proposition~\ref{prop:trunc_of_prop_is_set} the suspension of 
a proposition is always a set, so our assumption gives us a 
section $g:\susp(A)\to\mathsf{bool}$ of $\pi$. 
Since $\mathsf{bool}$ has decidable equality, there is an element of
\begin{equation*}
\mathsf{isDecidable}(g(\pi(\mathsf{true}))= g(\pi(\mathsf{false})))
\end{equation*}
and since $g$ is a section of $\pi$ it follows that there is an element of
\begin{equation*}
\mathsf{isDecidable}(\pi(\mathsf{true})=\pi(\mathsf{false}))
\end{equation*}
Now we see that it is enough to show that 
$\pi(\mathsf{true})= \pi(\mathsf{false})$ is equivalent to $A$. 
This we also obtained in proposition~\ref{prop:trunc_of_prop_is_set}.
\end{proof}

\section{Algebraic Set Theory}
\note{Steve Awodey}
We should obtain a model of CST following Awodey, Forsell, Warren. 
We may want to use the replacement axiom~\ref{lem:replacement}.
We should consider as classes only those large sets with small diagonal.

The universe is naturally a groupoid, but it may be possible to distort it to be a set.

It seems unlikely that the axiom of collection and the axiom of multiple choice are derivable in HoTT,
simply because none of its axioms seem applicable.  In the constructive set theory CZF~\cite{aczel2001notes}, unlike in classical Zermelo
set theory, the collection axiom is \emph{stronger} than the replacement axiom. 
The replacement axiom \emph{is} derivable from the resizing rules; see section~\ref{sec:resizing}.
In line with Voevodsky's proposal to add resizing rules to homotopy type theory, one could also consider its extension with the
collection axiom.

\subsection{Class of stable maps}
\begin{defn}\label{defn:small_maps}
A class $\smal:\prd{X,Y:\type} (X\to Y)\to\prop$ is \emph{stable} of if it satisfies:
\begin{description}
\item[pullback stability] If we have a pullback diagram
\begin{equation*}
\begin{tikzpicture}
\matrix (m) [std] {X & A \\ Y & B \\};
\draw[ar] (m-1-1) -- (m-1-2);
\draw[ar] (m-1-2) -- node[right] {$f$} (m-2-2);
\draw[ar] (m-1-1) -- node[left]  {$g$} (m-2-1);
\draw[ar] (m-2-1) -- (m-2-2);
\end{tikzpicture}
\end{equation*}
then $\smal(f)\to\smal(g)$.
\item[Descent] If the bottom arrow in the above diagram is surjective, then $\smal(g)\to\smal(f)$. 
\item[Sum] If $\smal(f),\smal(g)$, then $\smal(f+g)$.
\end{description}
A class of stable class of maps is \emph{locally full}~\citep[3.2]{MoerdijkPalmgren2002} if for all $g:X\to Y$ and $f:Y\to Z$ such
that $\smal(f)$: $\smal(g)$ iff $\smal(fg)$.

A class of maps $\smal$ is called a \emph{class of small maps} if it is stable, locally full and for each $X$, 
$\smal_X$ --- the small over maps over $X$ --- forms a $\Pi$W-pretopos; see~\citep[3.3]{MoerdijkPalmgren2002}.
\end{defn}

\begin{lem}
The class of set-fibered maps is a class of small maps.
\end{lem}
\begin{proof}
The class of maps with set-fibers is stable. It even has dependent sums.

We claim that it is locally full: If $g$ has set-fibers, then $fg$ has set-fibers, as sets are
closed under $\Sigma$-types. Conversely, fix $y\in Y$, then $\hfiber g y$ is the pullback
\[\sm {x:\hfiber{fg}{f(y)}} g(x)=y.\]

By the use of the object classifier, \autoref{thm:nobject_classifier}, we see that the type $\smal_X$ is equivalent
to the sets in context $X$. Now, sets in any context form a $\Pi$W-pretopos; see \autoref{subsec:piw}.
\end{proof}

\section{Excluded Middle and the Axiom of Choice}
\label{sec:excl-middle-axiom}

\textbf{TODO:} Discuss excluded middle and choice, state that the appropriate versions are
consistent, so me may assume them. Prove that choice implies excluded middle?
See~\ref{thm:1surj_to_surj_to_pem}

\section{Sets and Classes}
\label{sec:sets-classes}

\textbf{TODO:} Discuss the role of universes as classes-of-classes-of-classes, or
alternatively as inaccessible sets.
This could be part of AST(?)

\section{Powersets}
\label{sec:powersets}

\textbf{TODO:} Point out that we get \textbf{powerclasses} when we consider $X \to \prop$, that the
powerset axiom is about powerclasses being sets, that this is equivalent to $\prop$ being
a set, so resizing comes into play.

For a set $X$, we speak of predicates $Y:X\to \prop$ equivalently as \textbf{subsets} of $X$, and sometimes write $x\in Y$ to mean $Y(x)$.
We will also use the set-builder notation for such subsets:
\[ \setof{x:X | P } \defeq \lambda x.P \]
Univalence for \prop, plus function extensionality, implies that such subsets are \emph{extensional} in the usual sense of set theory:
\[ (Y_1 = Y_2) \leftrightarrow \Big(\prd{x:X} (x\in Y_1) \leftrightarrow (x\in Y_2)\Big) \]
Note that both sides of this equivalence are mere propositions.

We define the \textbf{power set} of $X$ to be
\begin{align*}
  \mathcal{P} X &\defeq X\to \prop
\end{align*}
Likewise, we define the subset of merely inhabited subsets in $X$ to be
\begin{align*}
  \mathcal{P}_+ X &\defeq \setof{P:X\to \prop | \brck{(\exists x:X), P(x)}}
\end{align*}
Assuming excluded middle, we have $\mathcal{P}_+ X \cong \setof{P:X\to \prop | P \neq (\lambda x.\bot)}$ and also $\mathcal{P} X \cong \mathcal{P}_+ X + \unit$.

For $Y:\mathcal{P}X$, we write $X\setminus Y \defeq \setof{x:X | x\notin Y}$.
Similarly, we have unions $Y_1 \cup Y_2$, intersections $Y_1 \cap Y_2$, and so on.

\section{The cumulative hierarchy}
\label{sec:cumulative-hierarchy}

Construction of $V$ as a HIT, mimicking Peter Aczel's construction of a type-theoretic
model of CZF.

\marginpar{Is this a valid HIT?}

\begin{defn}
  The \emph{cummulative hierachy $V$} relative to a type universe $\UU$ is the
  higher-inductive type with:
  %
  \begin{enumerate}
  \item \emph{points:} for every $A : \UU$ and $f : A \to V$ there is a point $\vset(A, f)$,
  \item \emph{paths:} for all $A, B : \UU$, $f : A \to V$ and $g : B \to V$ such that
    %
    \begin{equation} \label{eq:V-path}
      (\fall{a:A} \exis{b:B} \id[V]{f(a)}{g(b)}) \land (\fall{b:B} \exis{a:A} \id[V]{f(a)}{g(b)})
    \end{equation}
    %
    there is a path $\id[V]{\vset(A,f)}{\vset(B,g)}$.
  \end{enumerate}
\end{defn}

\noindent
%
In set-theoretic language the element $\vset(A,f)$ could be understood as $\{ f(a) \mid a
\in A \}$, but of course we cannot write this in type theory. The hierarchy $V$ is
bootstraped with the empty map $o : \emptyt \to V$, which corresponds to the empty set.
Then the singleton $\unit \to V$, defined as $\ttt \mapsto \vset(\emptyt, o)$ enters $V$, and so
on. The type $V$ lives in the same universe as the base universe $U$.

\begin{thm} \label{V-is-set}
  The cummulative hierarchy $V$ is $0$-truncated.
\end{thm}

\begin{proof}
  A confusing way of stating the therem is ``$V$ is a set.'' Because~\eqref{eq:V-path} is
  a mere proposition which is obviously reflexive, therefore
  \autoref{thm:h-set-refrel-in-paths-sets} applies.
\end{proof}

The elementhood relation $\in$ is defined on $V$ by
%
\begin{equation*}
  (x \in \vset(A,f)) \defeq \exis{a : A} x = f(a).
\end{equation*}
%
To see that the definition is valid, suppose we have a path $\vset(A, f) = \vset(B, g)$
constructed through~\eqref{eq:V-path}. If $x \in \vset(A,f)$ then there is $a : A$ such
that $x = f(a)$, but by~\eqref{eq:V-path} there is $b : B$ such that $f(a) = g(b)$, hence
$x = g(b)$ and $x \in \vset(B,f)$. The converse is symmetric.

Each $v \in V$ determines its \emph{type of elements} $\setof{ x : V | x \in y }$.

The subset relation $\subseteq$ is defined on $V$ by
%
\begin{equation*}
  (x \subseteq y) \defeq \fall{z : V} z \in x \Leftrightarrow z \in y.
\end{equation*}

A \emph{class} is a mere predicate on~$V$. We say that \emph{$C : V \to \prop$ is a
  $V$-set} if there merely exists $x \in V$ such that
%
\begin{equation*}
  \fall{y : V} C(y) \Leftrightarrow y \in x.
\end{equation*}

\begin{thm}
  The following hold for $(V, {\in})$:
  %
  \begin{enumerate}
  \item \emph{extensionality:}
    %
    \begin{equation*}
      \fall{x, y : V} x \subseteq y \land y \subseteq x \Leftrightarrow x = y.
    \end{equation*}
  \item \emph{replacement:} given any $r : V \to V$ and $x : V$, the class $C(y) \defeq
    \exis{z : V} z \in x \land y = r(z)$ is a $V$-set. If $C(y)$ then there merely exists
    $z : V$ and $a : A$ such that $z = f(a)$ and $y = r(z)$, therefore $y \in w$.
    Conversely, if $y \in w$ then there merely exists $a : A$ such that $y = r(f(a))$, so
    if we take $z \defeq f(a)$ we see that $C(y)$ holds.
  \end{enumerate}
\end{thm}


\begin{proof}
  \mbox{}
  %
  \begin{enumerate}
  \item Extensionality: if $\vset(A,f) \subseteq \vset(B, g)$ then $f(a) \in \vset(B, g)$
    for every $a : A$, therefore for every $a : A$ there merely exists $b : B$ such that
    $f(a) = g(b)$. The assumption $\vset(B, g) \subseteq \vset(A, f)$ gives the other half
    of~\eqref{eq:V-path}, therefore $\vset(A,f) = \vset(B,g)$.

  \item Replacement: the statement ``$C$ is a $V$-set'' is a mere proposition, so we may
    proceed by induction as follows. Supposing $x$ is $\vset(A, f)$, we claim that $w
    \defeq \vset(A, r \circ f)$ is the set we are looking for.

  \end{enumerate}
\end{proof}

\textbf{TODO:} What sort of set theory did we get?


\newcommand{\cd}[1]{\left|#1\right|}
\newcommand{\inj}{\ensuremath{\mathsf{inj}}}

\section{Cardinal numbers}
\label{sec:cardinals}

\begin{defn}
  The \textbf{type of cardinal numbers} is the 0-truncation of \set:
  \[ \card \defeq \pizero{\set} \]
  Thus, a \textbf{cardinal number}, or \textbf{cardinal}, is an inhabitant of $\card\jdeq \pizero\set$.
\end{defn}

\begin{rmk}
  As in some previous sections, here we are being universe polymorphic and typically ambiguous.
  Technically, there is a separate type ``\card'' associated to each universe ``\type'', but with these conventions we can state theorems beginning with ``for all cardinal numbers\dots''\ and give them exactly the same sort of meaning as those beginning ``for all types\dots''.
\end{rmk}

If $A$ is a set, we write $\cd{A}$ for its image under the canonical projection $\set \to \card$.
Of course, by definition, \card is a set.
It also inherits the structure of a semiring from \set.

\begin{defn}
  The operation of \textbf{cardinal addition}
  \[ \blank+\blank : \card \to \card \to \card \]
  is defined by induction on truncation:
  \[ \cd{A} + \cd{B} \defeq \cd{A+B} .\]
\end{defn}
\begin{proof}
  Since $\card\to\card$ is a set, to define $\alpha+\blank:\card\to\card$ for all $\alpha:\card$, by induction it suffices to assume that $\alpha$ is $\cd{A}$ for some $A:\set$.
  Now we want to define $\cd{A}+\blank :\card\to\card$, i.e.\ we want to define $\cd{A}+\beta :\card$ for all $\beta:\card$.
  However, since $\card$ is a set, by induction it suffices to assume that $\beta$ is $\cd{B}$ for some $B:\set$.
  But now we can define $\cd{A}+\cd{B}$ to be $\cd{A+B}$.
\end{proof}

\begin{defn}
  Similarly, the operation of \textbf{cardinal multiplication}
  \[ \blank\cdot\blank : \card \to \card \to \card \]
  is defined by induction on truncation:
  \[ \cd{A} \cdot \cd{B} \defeq \cd{A\times B} \]
\end{defn}

\begin{lem}\label{card:semiring}
  \card is a commutative semiring, i.e.\ for $\alpha,\beta,\gamma:\card$ we have the following.
  \begin{align*}
    (\alpha+\beta)+\gamma &= \alpha+(\beta+\gamma)\\
    \alpha+0 &= \alpha\\
    \alpha + \beta &= \beta + \alpha\\
    (\alpha \cdot \beta) \cdot \gamma &= \alpha \cdot (\beta\cdot\gamma)\\
    \alpha \cdot 1 &= \alpha\\
    \alpha\cdot\beta &= \beta\cdot\alpha\\
    \alpha\cdot(\beta+\gamma) &= \alpha\cdot\beta + \alpha\cdot\gamma
  \end{align*}
  where $0 \defeq \cd{\emptyset}$ and $1\defeq\cd{\unit}$.
\end{lem}
\begin{proof}
  We prove the commutativity of multiplication, $\alpha\cdot\beta = \beta\cdot\alpha$; the others are exactly analogous.
  Since \card is a set, the type $\alpha\cdot\beta = \beta\cdot\alpha$ is a mere proposition, and in particular a set.
  Thus, by induction it suffices to assume $\alpha$ and $\beta$ are of the form $\cd{A}$ and $\cd{B}$ respectively, for some $A,B:\set$.
  Now $\cd{A}\cdot \cd{B} \jdeq \cd{A\times B}$ and $\cd{B}\times\cd{A} \jdeq \cd{B\times A}$, so it suffices to show $A\times B = B\times A$.
  Finally, by univalence, it suffices to give an equivalence $A\times B \simeq B\times A$.
  But this is easy: take $(a,b) \mapsto (b,a)$ and its obvious inverse.
\end{proof}

\begin{defn}
  The operation of \textbf{cardinal exponentiation} is also defined by induction on truncation:
  \[ \cd{A}^{\cd{B}} \defeq \cd{B\to A}. \]
\end{defn}

\begin{lem}\label{card:exp}
  For $\alpha,\beta,\gamma:\card$ we have
  \begin{align*}
    \alpha^0 &= 1\\
    1^\alpha &= 1\\
    \alpha^1 &= \alpha\\
    \alpha^{\beta+\gamma} &= \alpha^\beta \cdot \alpha^\gamma\\
    \alpha^{\beta\cdot \gamma} &= (\alpha^{\beta})^\gamma\\
    (\alpha\cdot\beta)^\gamma &= \alpha^\gamma \cdot \beta^\gamma
  \end{align*}
\end{lem}
\begin{proof}
  Exactly like \autoref{card:semiring}.
\end{proof}

\begin{defn}
  The relation of \textbf{cardinal inequality}
  \[ \blank\le\blank : \card\to\card\to\prop \]
  is defined by induction on truncation:
  \[ \cd{A} \le \cd{B} \defeq \brck{\inj(A,B)} \]
  where $\inj(A,B)$ is the type of injections from $A$ to $B$.
  In other words, $\cd{A} \le \cd{B}$ means that there merely exists an injection from $A$ to $B$.
\end{defn}

\begin{lem}
  Cardinal inequality is a preorder, i.e.\ for $\alpha,\beta:\card$ we have
  \begin{gather*}
    \alpha \le \alpha\\
    (\alpha \le \beta) \to (\beta\le\gamma) \to (\alpha\le\gamma)
  \end{gather*}
\end{lem}
\begin{proof}
  As before, by induction on truncation.
  For instance, since $(\alpha \le \beta) \to (\beta\le\gamma) \to (\alpha\le\gamma)$ is a mere proposition, by induction on 0-truncation we may assume $\alpha$, $\beta$, and $\gamma$ are $\cd{A}$, $\cd{B}$, and $\cd{C}$ respectively.
  Now since $\cd{A} \le \cd{C}$ is a mere proposition, by induction on $(-1)$-truncation we may assume given injections $f:A\to B$ and $g:B\to C$.
  But then $g\circ f$ is an injection from $A$ to $C$, so $\cd{A} \le \cd{C}$ holds.
  Reflexivity is even easier.
\end{proof}

We may likewise show that cardinal inequality is compatible with the semiring operations.

\begin{lem}\label{thm:injsurj}
  Consider the following statements:
  \begin{enumerate}
  \item There is an injection $A\to B$.\label{item:cle-inj}
  \item There is a surjection $B\to A$.\label{item:cle-surj}
  \end{enumerate}
  Then, assuming excluded middle:
  \begin{itemize}
  \item Given $a_0:A$, we have~\ref{item:cle-inj}$\to$\ref{item:cle-surj}.
  \item Therefore, if $A$ is merely inhabited, we have~\ref{item:cle-inj} $\to$ merely \ref{item:cle-surj}.
  \item Assuming the axiom of choice, we have~\ref{item:cle-surj} $\to$ merely \ref{item:cle-inj}.
  \end{itemize}
\end{lem}
\begin{proof}
  If $f:A\to B$ is an injection, define $g:B\to A$ at $b:B$ as follows.
  Since $f$ is injective, the fiber of $f$ at $b$ is a mere proposition.
  Therefore, by excluded middle, either there is an $a:A$ with $f(a)=b$, or not.
  In the first case, define $g(b)\defeq a$; otherwise set $g(b)\defeq a_0$.
  Then for any $a:A$, we have $a = g(f(a))$, so $g$ is surjective.

  The second statement follows from this by induction on truncation.
  For the third, if $g:B\to A$ is surjective, then by the axiom of choice, there merely exists a function $f:A\to B$ with $g(f(a)) = a$ for all $a$.
  But then $f$ must be injective.
\end{proof}

\begin{thm}[Schroeder-Bernstein]
  Assuming excluded middle, for sets $A$ and $B$ we have
  \[ \inj(A,B) \to \inj(B,A) \to (A\cong B) \]
\end{thm}
\begin{proof}
  The usual ``back-and-forth'' argument applies without significant changes.
  Note that it actually constructs an isomorphism $A\cong B$ (assuming excluded middle so that we can decide whether a given element belongs to a cycle, an infinite chain, a chain beginning in $A$, or a chain beginning in $B$).
\end{proof}

\begin{cor}
  Assuming excluded middle, cardinal inequality is a partial order, i.e.\ for $\alpha,\beta:\card$ we have
  \[ (\alpha\le\beta) \to (\beta\le\alpha) \to (\alpha=\beta). \]
\end{cor}
\begin{proof}
  Since $\alpha=\beta$ is a mere proposition, by induction on truncation we may assume $\alpha$ and $\beta$ are $\cd{A}$ and $\cd{B}$, respectively, and that we have injections $f:A\to B$ and $g:B\to A$.
  But then the Schroeder-Bernstein theorem gives an isomorphism $A\simeq B$, hence an equality $\cd{A}=\cd{B}$.
\end{proof}

Finally, we can reproduce Cantor's theorem, showing that for every cardinal there is a greater one.

\begin{thm}[Cantor]
  For $A:\set$, there is no surjection $A \to (A\to \mathbf{2})$.
\end{thm}
\begin{proof}
  Suppose $f:A \to (A\to \mathbf{2})$ is any function, and define $g:A\to \mathbf{2}$ by $g(a) \defeq \neg f(a)(a)$.
  If $g = f(a_0)$, then $g(a_0) = f(a_0)(a_0)$ but $g(a_0) = \neg f(a_0)(a_0)$, a contradiction.
  Thus, $f$ is not surjective.
\end{proof}

\begin{cor}
  Assuming excluded middle, for any $\alpha:\card$, there is a cardinal $\beta$ such that $\alpha\le\beta$ and $\alpha\neq\beta$.
\end{cor}
\begin{proof}
  Let $\beta = 2^\alpha$.
  Now we want to show a mere propositon, so by induction we may assume $\alpha$ is $\cd{A}$, so that $\beta\jdeq \cd{A\to \mathbf{2}}$.
  Using excluded middle, we have a function $f:A\to (A\to \mathbf{2})$ defined by
  \[f(a)(a') \defeq
  \begin{cases}
    \top &\quad a=a'\\
    \bot &\quad a\neq a'.
  \end{cases}
  \]
  And if $f(a)=f(a')$, then $f(a')(a) = f(a)(a) = \top$, so $a=a'$; hence $f$ is injective.
  Thus, $\alpha \jdeq \cd{A} \le \cd{A\to \mathbf{2}} \jdeq 2^\alpha$.

  On the other hand, if $2^\alpha \le \alpha$, then we would have an injection $(A\to\mathbf{2})\to A$.
  By \autoref{thm:injsurj}, since we have $(\lam{x} \bot):A\to \mathbf{2}$ and excluded middle, there would then be a surjection $A \to (A\to \mathbf{2})$, contradicting Cantor's theorem.
\end{proof}

\section{Ordinal numbers}
\label{sec:ordinals}

\newcommand{\acc}{\ensuremath{\mathsf{acc}}}

\begin{defn}
  Let $A$ be a set and
  \[\blank<\blank:A\to A\to \prop\]
  a binary relation on $A$.
  We define by induction what it means for an element $a:A$ to be \textbf{accessible} by $<$:
  \begin{itemize}
  \item If $b$ is accessible for every $b<a$, then $a$ is accessible.
  \end{itemize}
  We write $\acc(a)$ to mean that $a$ is accessible.
\end{defn}

It may seem that such an inductive definition can never get off the ground, but of course if $a$ has the property that there are \emph{no} $b$ such that $b<a$, then $a$ is vacuously accessible.

\begin{lem}
  Accessibility is a mere property.
\end{lem}
\begin{proof}
  We claim that for any $a_1,a_2:A$, any $p:a_1=a_2$, and any $s_1:\acc(a_1)$ and $s_2:\acc(a_2)$, we have $\trans{p}{s_1}=s_2$.
  By induction, we may assume that $s_1$ is given by a function assigning to each $b_1<a_1$ a proof $s_1(b_1):\acc(b_1)$, and moreover that for any $b_2:A$, any $q:b_1=b_2$, and $t_2:\acc(b_2)$, we have $\trans{q}{s_1(b_1)}=t_2$.
  Similarly, we may assume that $s_2$ is given by a function assigning to each $b_2<a_2$ a proof $s_2(b_2):\acc(b_2)$, and that for any $b_1:A$, any $q:b_1=b_2$, and $t_1:\acc(b_1)$, we have $\trans{q}{t_1}=s_2(b_2)$.
  
  Now by function extensionality, to show $\trans{p}{s_1}=s_2$ it suffices to show that for any $b_2<a_2$ we have $\trans{p}{s_1}(b_2) = s_2(b_2)$.
  However, we can obtain this from the induction hypothesis for $s_2$ with $q\defeq \refl{b_2}$ and $t_1 \defeq \trans{p}{s_1}(b_2)$.
  This proves the claim.

  Finally, we instantiate the claim with $a_2\defeq a_1\defeq a$ and $p\defeq \refl{a}$.
  Thus, for any $a:A$ and $s_1,s_2:\acc(a)$, we have $s_1=s_2$, as desired.
\end{proof}

\begin{defn}
  A binary relation $<$ on a set $A$ is \textbf{well-founded} if every element of $A$ is accessible.
\end{defn}

\begin{lem}
  Well-foundedness is a mere property.
\end{lem}
\begin{proof}
  Well-foundedness of $<$ is the type $\prd{a:A} \acc(a)$, which is a mere proposition since each $\acc(a)$ is.
\end{proof}

\begin{eg}\label{thm:nat-wf}
  Perhaps the most familiar well-founded relation is the usual strict ordering on \nat.
  To show that this is well-founded, we must show that $n$ is accessible for each $n:\nat$.
  This is just the usual proof of ``strong induction'' from ordinary induction on \nat.

  Specifically, we prove by induction on $n:\nat$ that $k$ is accessible for all $k\le n$.
  The base case is just that $0$ is accessible, which is vacuously true since nothing is strictly less than $0$.
  For the inductive step, we assume that $k$ is accessible for all $k\le n$, which is to say for all $k<n+1$; hence by definition $n+1$ is also accessible.

  A different relation on \nat which is also well-founded is obtained by setting only $n < \suc(n)$ for all $n:\nat$.
  Well-foundedness of this relation is almost exactly the ordinary induction principle of \nat.
\end{eg}

\begin{eg}\label{thm:wtype-wf}
  Let $A:\set$ and $B : A \to \set$ be a family of sets.
  Recall from \autoref{sec:w-types} that the $W$-type $\wtype{a:A} B(a)$ is inductively generated by the single constructor
  \begin{itemize}
  \item $\supp : \prd{a:A} (B(a) \to \wtype{x:A} B(x)) \to \wtype{x:A} B(x)$
  \end{itemize}
  We define the relation $<$ on $\wtype{x:A} B(x)$ by recursion on its second argument:
  \begin{itemize}
  \item For any $a:A$ and $f:B(a) \to \wtype{x:A} B(x)$, we define $w<\supp(a,f)$ to mean that there merely exists a $b:B(a)$ such that $w = f(b)$.
  \end{itemize}
  Now we prove that every $w:\wtype{x:A} B(x)$ is accessible for this relation, using the usual induction principle for $\wtype{x:A}B(x)$.
  This means we assume given $a:A$ and $f:B(a) \to \wtype{x:A} B(x)$, and also a lifting $f' : \prd{b:B(a)} \acc(f(b))$.
  But then by definition of $<$, we have $\acc(w)$ for all $w<\supp(a,f)$; hence $\supp(a,f)$ is accessible.
\end{eg}

Well-foundedness allows us to define functions by recursion and prove statements by induction, such as for instance the following.

\begin{lem}\label{thm:wfrec}
  Suppose $B$ is a set and we have a function
  \[ g : \mathcal{P}B \to B \]
  Then if $<$ is a well-founded relation on $A$, there is a function $f:A\to B$ such that for all $a:A$ we have
  \begin{equation*}
    f(a) = g\Big(\setof{ f(a') | a'<a }\Big).
  \end{equation*}
\end{lem}
\begin{proof}
  We first define, for every $a:A$ and $s:\acc(a)$, an element $\bar f(a,s):B$.
  By induction, it suffices to assume that $s$ is a function assigning to each $a'<a$ a proof $s(a'):\acc(a')$, and that moreover for each such $a'$ we have an element $\bar f(a',s(a')):B$.
  In this case, we define
  \begin{equation*}
    \bar f(a,s) \defeq g\Big(\setof{ \bar f(a',s(a')) | a'<a }\Big).
  \end{equation*}

  Now since $<$ is well-founded, we have a function $w:\prd{a:A} \acc(a)$.
  Thus, we can define $f(a)\defeq \bar f (a,w(a))$.
\end{proof}

In classical logic, well-foundedness has a more well-known reformulation.

\begin{lem}\label{thm:wfmin}
  Assuming excluded middle, $<$ is well-founded if and only if every nonempty subset $B\subseteq A$ merely has a minimal element.
\end{lem}
\begin{proof}
  Suppose first $<$ is well-founded, and suppose $B\subseteq A$ is a subset with no minimal element.
  That is, for any $a:A$ with $a\in B$, there merely exists a $b:A$ with $b<a$ and $b\in B$.

  We claim that for any $a:A$ and $s:\acc(a)$, we have $a\notin B$.
  By induction, we may assume $s$ is a function assigning to each $a'<a$ a proof $s(a'):\acc(a)$, and that moreover for each such $a'$ we have $a'\notin B$.
  If $a\in B$, then by assumption, there would merely exist a $b<a$ with $b\in B$, which contradicts this assumption.
  Thus, $a\notin B$; this completes the induction.
  Since $<$ is well-founded, we have $a\notin B$ for all $a:A$, i.e. $B$ is empty.

  Now suppose each nonempty subset merely has a minimal element.
  Let $B = \setof{ a:A | \neg \acc(a) }$.
  Then if $B$ is nonempty, it merely has a minimal element.
  Thus there merely exists an $a:A$ with $a\in B$ such that for all $b<a$, we have $\acc(b)$.
  But then by definition (and induction on truncation), $a$ is merely accessible, and hence accessible, contradicting $a\in B$.
  Thus, $B$ is empty, so $<$ is well-founded.
\end{proof}

\begin{defn}
  A well-founded relation $<$ on a set $A$ is \textbf{extensional} if for any $a,b:A$, we have
  \[ \Big(\prd{c:A} (c<a) \leftrightarrow (c<b)\Big) \to (a=b). \]
\end{defn}

Note that since $A$ is a set, extensionality is a mere proposition.
This notion of ``extensionality'' is unrelated to function extensionality, and also unrelated to the extensionality of identity types.
Rather, it is a ``local'' counterpart of the axiom of extensionality in classical set theory.

\begin{thm}
  The type of extensional well-founded relations is a set.
\end{thm}
\begin{proof}
  By the univalence axiom, it suffices to show that if $(A,<)$ is extensional and well-founded and $f:(A,<) \cong (A,<)$, then $f=\idfunc[A]$.
  We prove by induction on $<$ that $f(a)=a$ for all $a:A$.
  The inductive hypothesis is that for all $a'<a$, we have $f(a')=a'$.

  Now since $A$ is extensional, to conclude $f(a)=a$ it is sufficient to show
  \[\prd{c:A}(c<f(a)) \leftrightarrow (c<a).\]
  However, since $f$ is an automorphism, we have $(c<a) \leftrightarrow (f(c)<f(a))$.
  But $c<a$ implies $f(c)=c$ by the induction hypothesis, so $(c<a) \to (c<f(a))$.
  On the other hand, if $c<f(a)$, then $f^{-1}(c)<a$, and so $c = f(f^{-1}(c)) = f^{-1}(c)$ by the induction hypothesis again; thus $c<a$.
  Therefore, we have $(c<a) \leftrightarrow (c<f(a))$ for any $c:A$, so $f(a)=a$.
\end{proof}

\begin{defn}\label{def:simulation}
  If $(A,<)$ and $(B,<)$ are extensional and well-founded, a \textbf{simulation} is a function $f:A\to B$ such that
  \begin{enumerate}
  \item if $a<a'$, then $f(a)<f(a')$, and\label{item:sim1}
  \item for all $a:A$ and $b:B$, if $b<f(a)$, then there merely exists an $a'<a$ with $f(a')=b$.\label{item:sim2}
  \end{enumerate}
\end{defn}

\begin{lem}
  Any simulation is injective.
\end{lem}
\begin{proof}
  We prove by double well-founded induction that for any $a,b:A$, if $f(a)=f(b)$ then $a=b$.
  The induction hypothesis for $a:A$ says that for any $a'<a$, and any $b:B$, if $f(a')=f(b)$ then $a=b$.
  The inner induction hypothesis for $b:A$ says that for any $b'<b$, if $f(a')=f(b')$ then $a'=b'$.

  Suppose $f(a)=f(b)$; we must show $a=b$.
  By extensionality, it suffices to show that for any $c:A$ we have $(c<a)\leftrightarrow (c<b)$.
  If $c<a$, then $f(c)<f(a)$ by \autoref{def:simulation}\ref{item:sim1}.
  Hence $f(c)<f(b)$, so by \autoref{def:simulation}\ref{item:sim2} there merely exists $c':A$ with $c'<b$ and $f(c)=f(c')$.
  By the induction hypothesis for $a$, we have $c=c'$, hence $c<b$.
  The dual argument is symmetrical.
\end{proof}

In particular, this implies that the word ``merely'' in \autoref{def:simulation}\ref{item:sim2} could be omitted without change of sense.

\begin{cor}
  If $f:A\to B$ is a simulation, then for all $a:A$ and $b:B$, if $b<f(a)$, there \emph{purely} exists an $a'<a$ with $f(a')=b$.
\end{cor}
\begin{proof}
  Since $f$ is injective, $\sm{a:A} (f(a)=b)$ is a mere proposition.
\end{proof}

We say that a subset $C :\mathcal{P}B$ is an \textbf{initial segment} if $c\in C$ and $b<c$ imply $b\in C$.
The image of a simulation must be an initial segment, while the inclusion of any initial segment is a simulation.
Thus, by univalence, every simulation $A\to B$ is \emph{equal} to the inclusion of some initial segment of $B$.

\begin{thm}
  For a set $A$, let $P(A)$ be the type of extensional well-founded relations on $A$.
  If $\mathord{<_A} : P(A)$ and $\mathord{<_B} : P(B)$ and $f:A\to B$, let $H_{\mathord{<_A}\mathord{<_B}}(f)$ be the mere proposition that $f$ is a simulation.
  Then $(P,H)$ is a standard notion of structure over \uset in the sense of \autoref{sec:sip}.
\end{thm}
\begin{proof}
  We leave it to the reader to verify that identities are simulations, and that composites of simulations are simulations.
  Thus, we have a notion of structure.
  For standardness, we must show that if $<$ and $\prec$ are two extensional well-founded relations on $A$, and $\idfunc[A]$ is a simulation in both directions, then $<$ and $\prec$ are equal.
  Since extensionality and well-foundedness are mere propositions, for this it suffices to have $\prd{a,b:A} (a<b) \leftrightarrow (a\prec b)$.
  But this follows from \autoref{def:simulation}\ref{item:sim1} for $\idfunc[A]$.
\end{proof}

\begin{cor}\label{thm:wfcat}
  There is a category whose objects are sets equipped with extensional well-founded relations, and whose morphisms are simulations.
\end{cor}

In fact, this category is a poset.

\begin{lem}
  For extensional and well-founded $(A,<)$ and $(B,<)$, there is at most one simulation $f:A\to B$.
\end{lem}
\begin{proof}
  Suppose $f,g:A\to B$ are simulations.
  Since being a simulation is a mere property, it suffices to show $\prd{a:A}(f(a)=g(a))$.
  By induction on $<$, we may suppose $f(a')=g(a')$ for all $a'<a$.
  And by extensionality of $B$, to have $f(a)=g(a)$ it suffices to have $\prd{b:B}(b<f(a)) \leftrightarrow (b<g(a))$.

  But since $f$ is a simulation, if $b<f(a)$, then we have $a'<a$ with $f(a')=b$.
  By the inductive hypothesis, we have also $g(a')=b$, hence $b<g(a)$.
  The dual argument is symmetrical.
\end{proof}

Thus, if $A$ and $B$ are equipped with extensional and well-founded relations, we may write $A\le B$ to mean there exists a simulation $f:A\to B$.
\autoref{thm:wfcat} implies that if $A\le B$ and $B\le A$, then $A=B$.

\begin{defn}
  An \textbf{ordinal} is a set $A$ with an extensional well-founded relation which is \emph{transitive}, i.e.\ $\prd{a,b,c:A}(a<b)\to (b<c) \to (a<c)$.
\end{defn}

\begin{eg}
  Of course, the usual strict order on \nat is transitive.
  It is easily seen to be extensional as well; thus it is an ordinal.
  As usual, we denote this ordinal by $\omega$.
\end{eg}

Let \ord denote the type of ordinals.
By the previous results, \ord is a set and has a natural partial order.
We now show that \ord also admits a well-founded relation.

If $A$ is an ordinal and $a:A$, let $\ordsl A a$ denote the initial segment $\setof{ b:A | b<a}$.
Note that if $\ordsl A a = \ordsl A b$ as ordinals, then that isomorphisms must respect their inclusions into $A$ (since simulations form a poset), and hence they are equal as subsets of $A$.
Therefore, since $A$ is extensional, $a=b$.
Thus the function $a\mapsto \ordsl A a$ is an injection $A\to \ord$.

\begin{defn}
  For ordinals $A$ and $B$, a simulation $f:A\to B$ is said to be \textbf{bounded} if there exists $b:B$ such that $A = \ordsl B b$.
\end{defn}

The remarks above imply that such a $b$ is unique when it exists, so that boundedness is a mere property.

We write $A<B$ if there exists a bounded simulation from $A$ to $B$.
Since simulations are unique, $A<B$ is also a mere proposition.

\begin{thm}\label{thm:ordord}
  $(\ord,<)$ is an ordinal.
\end{thm}

\begin{rmk}
  Note the use of universe polymorphism and typical ambiguity.
  If universe levels were made explicit, this theorem would state that the set of ordinals in one universe is an ordinal in the next higher universe.
\end{rmk}

\begin{proof}
  Let $A$ be an ordinal; we first show that $\ordsl A a$ is accessible (in \ord) for all $a:A$.
  By induction, suppose $\ordsl A b$ is accessible for all $b:A$.
  By definition of accessibility, we must show that $B$ is accessible in \ord for all $B<\ordsl A a$.
  However, if $B<\ordsl A a$ then there is some $b<a$ such that $B = \ordsl{(\ordsl A a)}{b} = \ordsl A b$, which is accessible by the inductive hypothesis.
  Thus, $\ordsl A a$ is accessible for all $a:A$.

  Now to show that $A$ is accessible in \ord, by definition we must show $B$ is accessible for all $B<A$.
  But as before, $B<A$ means $B=\ordsl A a$ for some $a:A$, which is accessible as we just proved.
  Thus, \ord is well-founded.

  For extensionality, suppose $A$ and $B$ are ordinals such that $\prd{C:\ord} (C<A) \leftrightarrow (C<B)$.
  Then for every $a:A$, since $\ordsl A a<A$, we have $\ordsl A a<B$, hence there is $b:B$ with $\ordsl A a = \ordsl B b$.
  Define $f:A\to B$ to take each $a$ to the corresponding $b$; it is straightforward to verify that $f$ is an isomorphism.
  Thus $A\cong B$, hence $A=B$ by univalence.

  Finally, it is easy to see that $<$ is transitive.
\end{proof}

Treating \ord as an ordinal is often very convenient, but it has its pitfalls as well.
For instance, consider the following lemma, for whose statement we drop briefly into \emph{explicit} universe polymorphism.

\begin{lem}\label{thm:ordsucc}
  Let \bbU be a universe.
  For any $A:\ord_\bbU$, there is a $B:\ord_\bbU$ such that $A<B$.
\end{lem}
\begin{proof}
  Let $B=A+\unit$, with the element $\star:\unit$ being greater than all elements of $A$.
  Then $B$ is an ordinal and it is easy to see that $A\cong \ordsl B \star$.
\end{proof}

This lemma illustrates a potential pitfall of typical ambiguity.
Consider the following alternative proof of it.

\begin{proof}[Another putative proof of \autoref{thm:ordsucc}]
  Note that $C<A$ if and only if $C=\ordsl A a$ for some $a:A$.
  This gives an isomorphism $A \cong \ordsl \ord A$, so that $A<\ord$.
  Thus we may take $B\defeq\ord$.
\end{proof}

The second proof would be valid if we had stated \autoref{thm:ordsucc} in a typically ambiguous style.
But the resulting lemma would be less useful, because the second proof would constrain the second ``\ord'' in the lemma statement to refer to a higher universe level than the first one.
The first proof allows both universes to be the same.

Similar remarks apply to the next lemma, which could be proved in a less useful way by observing that $A\le \ord$ for any $A:\ord$.

\begin{lem}\label{thm:ordunion}
  Let \bbU be a universe.
  For any $X:\type_\bbU$ and $F:X\to \ord_\bbU$, there exists $B:\ord_\bbU$ such that $Fx\le B$ for all $x:X$.
\end{lem}
\begin{proof}
  Let $B$ be the quotient of the equivalence relation on $\sm{x:X} Fx$ defined as follows:
  \[ (x,y) \sim (x',y')
  \;\defeq\;
  \Big(\ordsl{(Fx)}{y} \cong \ordsl{(Fx')}{y'}\Big).
  \]
  Define $(x,y)<(x',y')$ if $\ordsl{(Fx)}{y} < \ordsl{(Fx')}{y'}$.
  This clearly descends to the quotient, and can be seen to make $B$ into an ordinal.
  Moreover, for each $x:X$ the induced map $Fx\to B$ is a simulation.
\end{proof}



\section{Classical well-orderings}
\label{sec:wellorderings}

We now show the equivalence of our ordinals with the more familiar classical well-orderings.

\begin{lem}
  Assuming excluded middle, every ordinal is trichotomous:
  \[ \prd{a,b:A} (a<b) \vee (a=b) \vee (b<a). \]
\end{lem}
\begin{proof}
  By induction on $a$, we may assume that for every $a'<a$ and every $b':A$, we have $(a'<b') \vee (a'=b') \vee (b'<a')$.
  Now by induction on $b$, we may assume that for every $b'<b$, we have $(a<b') \vee (a=b') \vee (b'<a)$.

  By excluded middle, either there merely exists a $b'<b$ such that $a<b'$, or there merely exists a $b'<b$ such that $a=b'$, or for every $b'<b$ we have $b'<a$.
  In the first case, merely $a<b$ by transitivity, hence $a<b$ as it is a mere proposition.
  Similarly, in the second case, $a<b$ by transport.
  Thus, suppose $\prd{b':A}(b'<b)\to (b'<a)$.

  Now analogously, either there merely exists $a'<a$ such that $b<a'$, or there merely exists $a'<a$ such that $a'=b$, or for every $a'<a$ we have $a'<b$.
  In the first and second cases, $b<a$, so we may suppose $\prd{a':A}(a'<a)\to (a'<b)$.
  However, by extensionality, our two suppositions now imply $a=b$.
\end{proof}

\begin{lem}
  A well-founded relation contains no cycles, i.e.\
  \[ \prd{n:\mathbb{N}}{a:\mathbb{N}_n\to A} \neg\Big((a_0<a_1) \wedge \dots \wedge (a_{n-1}<a_n)\wedge (a_n<a_0)\Big). \]
\end{lem}
\begin{proof}
  We prove by induction on $a:A$ that there is no cycle containing $a$.
  Thus, suppose by induction that for all $a'<a$, there is no cycle containing $a'$.
  But in any cycle containing $a$, there is some element less than $a$ and contained in the same cycle.
\end{proof}

\begin{thm}\label{thm:wellorder}
  Assuming excluded middle, $(A,<)$ is an ordinal if and only if every nonempty subset $B\subseteq A$ has a least element.
\end{thm}
\begin{proof}
  If $A$ is an ordinal, then by \autoref{thm:wfmin} every nonempty subset merely has a minimal element.
  But trichotomy implies that any minimal element is a least element.
  Moreover, least elements are unique when they exist, so merely having one is as good as having one.

  Conversely, if every nonempty subset has a least element, then $A$ is well-founded by \autoref{thm:wfmin}.
  We also have trichotomy, since for any $a,b$ the set $\setof{a,b}$ merely has a least element, which must be either $a$ or $b$.
  This implies transitivity, since if $a<b$ and $a<c$, then either $a=c$ or $c<a$ would produce a cycle.
  Similarly, it implies extensionality, for if $\prd{c:A}(c<a)\leftrightarrow (c<b)$, then $a<b$ implies (letting $c$ be $a$) that $a<a$, which is a cycle, and similarly if $b<a$; hence $a=b$.
\end{proof}

In classical mathematics, the characterization of \autoref{thm:wellorder} is taken as the definition of a \textbf{well-ordering}, with the \emph{ordinals} being a canonical set of representatives of isomorphism classes for well-orderings.
In our context, the Structure Identity Principle means that there is no need to look for such representatives: any well-ordering is as good as any other.

We now move on to consider consequences of the axiom of choice.

\begin{thm}\label{thm:wop}
  Assuming excluded middle, the following are equivalent.
  \begin{enumerate}
  \item For every set $X$, there merely exists a function
    $ f: \mathcal{P}_+X \to X $
    such that $f(Y)\in Y$ for all $Y:\mathcal{P}X$.\label{item:wop1}
  \item Every set merely admits the structure of an ordinal.\label{item:wop2}
  \end{enumerate}
\end{thm}

Of course,~\ref{item:wop1} is a standard classical version of the axiom of choice.

\begin{proof}
  One direction is easy: suppose~\ref{item:wop2}.
  Since we aim to prove the mere proposition~\ref{item:wop1}, we may assume $A$ is an ordinal.
  But then we can define $f(B)$ to be the least element of $B$.

  Now suppose~\ref{item:wop1}.
  As before, since~\ref{item:wop2} is a mere proposition, we may assume given such an $f$.
  We extend $f$ to a function
  \[ \bar f:\mathcal{P}X \cong (\mathcal{P}_+ X) + \unit \longrightarrow X+\unit
  \]
  in the obvious way.
  Now for any ordinal $A$, we can define $g_A:A\to X+\unit$ by well-founded recursion:
  \[ g_A(a) \defeq 
    \bar f\Big(X \setminus \setof{ g_A(b) | \rule{0pt}{1em} (b<a) \wedge (g_A(b) \in X) }\Big)
  \]
  (regarding $X$ as a subset of $X+\unit$ in the obvious way).

  Let $A'$ be the preimage of $X$; then we claim the restriction $g_A':A' \to X$ is injective.
  For if $a,a':A$ with $a\neq a'$, then by trichotomy and without loss of generality, we may assume $a'<a$.
  Thus $g_A(a') \in \setof{ g_A(b) | b<a }$, so since $f(Y)\in Y$ for all $Y$ we have $g_A(a) \neq g_A(a')$.

  Moreover, $A'$ is an initial segment of $A$.
  For $g_A(a)$ lies in \unit if and only if $\setof{g_A(b)|b<a} = X$, and if this holds then it also holds for any $a'>a$.
  Thus, $A'$ is itself an ordinal.

  Finally, since \ord is an ordinal, we can take $A\defeq\ord$.
  Let $X'$ be the image of $g_\ord':\ord' \to X$; then the inverse of $g_\ord'$ yields an injection $H:X'\to \ord$.
  By \autoref{thm:ordunion}, there is an ordinal $C$ such that $Hx\le C$ for all $x:X'$.
  Then by \autoref{thm:ordsucc}, there is a further ordinal $D$ such that $C<D$, hence $Hx<D$ for all $x:X'$.
  Now we have
  \begin{align*}
    g_{\ord}(D) &= \bar f\Big( X \setminus \setof{ g_\ord(B) | \rule{0pt}{1em} B<D \wedge (g_\ord(B) \in X)} \Big)\\
    &=\bar f\Big( X \setminus \setof{ g_\ord(B) | \rule{0pt}{1em} B:\ord \wedge (g_\ord(B) \in X)} \Big)
  \end{align*}
  since if $B:\ord$ and $(g_\ord(B) \in X)$, then $B = Hx$ for some $x:X'$, hence $B<D$.
  Now if
  \[\setof{ g_\ord(B) | \rule{0pt}{1em} B:\ord \wedge (g_\ord(B) \in X)}\]
  is not all of $X$, then $g_\ord(D)$ would lie in $X$ but not in this subset, which would be a contradiction since $D$ is itself a potential value for $B$.
  So this set must be all of $X$, and hence $g_\ord'$ is surjective as well as injective.
  Thus, we can transport the ordinal structure on $\ord'$ to $X$.
\end{proof}

\begin{rmk}
  If we had given the wrong proof of \autoref{thm:ordsucc} or \autoref{thm:ordunion}, then the resulting proof of \autoref{thm:wop} would be invalid: there would be no way to consistently assign universe levels.
\end{rmk}

\begin{cor}
  Assuming the axiom of choice, the function $\ord\to\set$ (which forgets the order structure) is a surjection.
\end{cor}

Note that \ord is a set, while \set is a 1-type.
In general, there is no reason for a 1-type to admit any surjective function from a set.
Even the axiom of choice does not appear to imply that \emph{every} 1-type does so, but it readily implies that this is so for 1-types constructed out of \set, such as the types of objects of categories of structures as in \autoref{sec:sip}.
The following corollary also applies to such categories.

\begin{cor}
  Assuming AC, \uset admits a weak equivalence functor from a strict category.
\end{cor}
\begin{proof}
  Let $X_0\defeq \ord$, and for $A,B:X_0$ let $\hom_X(A,B) \defeq (A\to B)$.
  Then $X$ is a strict category, since \ord is a set, and the above surjection $X_0 \to \set$ extends to a weak equivalence functor $X\to \uset$.
\end{proof}

Now recall from \autoref{sec:cardinals} that we have a further surjection $\cd{-}:\set\to\card$, and hence a composite surjection $\ord\to\card$ which sends each ordinal to its cardinality.

\begin{thm}
  Assuming AC, the surjection $\ord\to\card$ has a section.
\end{thm}
\begin{proof}
  There is an easy and wrong proof of this: since \ord and \card are both sets, AC implies that any surjection between them \emph{merely} has a section.
  However, we actually have a canonical \emph{specified} section: because \ord is an ordinal, every nonempty subset of it has a uniquely specified least element.
  Thus, we can map each cardinal to the least element in the corresponding fiber.
\end{proof}

It is traditional in set theory to identify cardinals with their image in \ord: the least ordinal having that cardinality.

It follows that \card also canonically admits the structure of an ordinal: in fact, one isomorphic to \ord.
Specifically, we define by well-founded recursion a function $\aleph:\ord\to\ord$, such that $\aleph(A)$ is the least ordinal having cardinality greater than $\aleph({\ordsl A a})$ for all $a:A$.
Then (assuming AC) the image of $\aleph$ is exactly the image of \card.

\sectionNotes

The treatment of the category of sets in \autoref{sec:piw-pretopos} roughly follows that in~\cite{RijkeSpitters}.
The notion of a predicative topos, as a $\Pi$W pretopos, was introduced in \cite{MoerdijkPalmgren}.

The implication in \autoref{sec:surj-split} from $\mathsf{AC}$ to $\mathsf{LEM}$ is an adaptation to homotopy type theory of a theorem from topos theory due to Diaconescu~\cite{Diaconescu}; it was posed as a problem already by Bishop~\cite[Problem 2,p58]{Bishop1967}.

Voevodsky~\cite{Universe-poly} has proposed resizing rules of the kind considered in \autoref{subsec:piw}.

The idea of algebraic set theory, which informs our development in \autoref{sec:ast} of the cumulative hierarchy, is due to \cite{JoyalMoerdijk}, but it derives from earlier work by \cite{AczelCZF}.

\dots


\sectionExercises

\begin{ex}
  Show that if every surjection has a section in the category $\set$, then $\mathsf{AC}$ holds.
\end{ex}

\begin{ex}
  Prove that if $(A,<_A)$ and $(B,<_B)$ are well-founded, extensional, or ordinals, then so is $A+B$, with $<$ defined by
  \begin{alignat*}{2}
    (a<a') &\defeq (a<_A a') &\qquad\text{for }& a,a':A\\
    (b<b') &\defeq (b<_B b') &\qquad\text{for }& b,b':B\\
    (a<b) &\defeq \unit &\qquad\text{for }& (a:A),(b:B)\\
    (b<a) &\defeq \emptyset &\qquad\text{for } &(a:A),(b:B)
  \end{alignat*}
\end{ex}
% \begin{proof}
%   We first prove by induction on $<_A$ that every element of $A$ is accessible in $A+B$.
%   This is easy since the only elements less than $a:A$ in $A+B$ are also in $A$.
%   We then prove by induction on $<_B$ that every element of $B$ is accessible in $A+B$.
%   This is easy since we have already proven that every element of $A$ is accessible.
% \end{proof}

\begin{ex}
  Prove that if $(A,<_A)$ and $(B,<_B)$ are well-founded, extensional, or ordinals, then so is $A\times B$, with $<$ defined by
  \[ ((a,b) <(a',b')) \defeq (a<_A a') \vee ((a=a') \wedge (b<_B b')). \]
\end{ex}
% \begin{proof}
%   We prove by induction on $<_A$ that for every $a:A$, every element of the form $(a,b)$ is accessible in $A\times B$.
%   The induction hypothesis is that for all $a'<_A a$, every pair $(a',b)$ is accessible.
%   Inside this induction, we prove by induction on $<_B$ that for every $b:B$, the element $(a,b)$ is accessible.
%   The nested induction hypothesis is that for every $b'<_B b$, the element $(a,b')$ is accessible.
%   But now, if $(a',b')< (a,b)$, then either $a<_A a'$ in which case $(a',b')$ is accessible by the first induction hypothesis, or $a=a'$ and $b'<_B b$, in which case $(a,b')$ is accessible by the second induction hypothesis.
%   Thus, by definition of accessibility, $(a,b)$ is accessible.
%   This completes both inductions.
% \end{proof}

\begin{ex}
  Define the usual algebraic operations on ordinals, and prove that they satisfy the usual properties.
\end{ex}


% Local Variables:
% TeX-master: "main"
% End:


% Local Variables:
% TeX-master: "main"
% End:
