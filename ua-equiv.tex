% !Tex root = note.tex

\section{Equivalence Induction}

If a type universe $\bbU$ is univalent, then it admits an induction principle on
equivalences as follows:
\begin{itemize}
\item if for any two types $X,Y:\bbU$ and an equivalence $f:X\simeq Y$, we have
a type $P(X,Y,f)$, and
\item for every $X:\bbU$, we have an element $d(X):P(X,X,1_X)$,
\end{itemize}
then
\begin{itemize}
\item there exists an element $J'_d(X,Y,f):P(X,Y,f)$ for \emph{any} types
$X,Y:\bbU$ and equivalence $f:X\simeq Y$, such that $J'_d(X,X,1_X)\equiv d(X)$.
\end{itemize}
We call this principle \emph{equivalence induction}. The particular definition
of equivalence used is unimportant, as Theorem \ref{thm:equiv-iso-adj} tells us
that each implies the existence of the others.

\begin{proof}
For $X,Y:\bbU$ and $p:X=Y$, we construct a type $Q(X,Y,p)\defeq
P(X,Y,E_{X,Y}(p))$. Recall that $E_{X,Y}:(X=Y)\ra (X\simeq Y)$ is always
definable, and is surjective for a univalent $\bbU$; in particular, for each
$f:X\simeq Y$, there is some $p_f:X=Y$ such that $E_{X,Y}(p_f)=f$. 

By the induction principle for identity types, every $Q(X,Y,p)$ is inhabited so
long as there is an element of $Q(X,X,\refl X)$. But $E_{X,X}(\refl X)$ is
$1_X$, so this is satisfied by $d(X):P(X,X,1_X)$. By Id-induction, we have some
$J'_d(X,Y,f)\defeq J_{Q,d}(X,Y,p_f)$ in $Q(X,Y,p_f)\equiv P(X,Y,f)$. Moreover,
$J'_d(X,X,1_X)\equiv J_{Q,d}(X,X,\refl X)\equiv d(X)$.
\end{proof}

In particular, path induction implies that equivalent types satisfy the same
propositions.
\begin{cor}
For $X,Y:\bbU$ such that $f:X\simeq Y$, $P(X)$ implies $P(Y)$.
\end{cor}
\begin{proof}
Apply equivalence induction to $Q(A,B,e)\defeq Q(A)\to Q(B)$; it suffices to
show $Q(A,A,1_A)\equiv Q(A)\to Q(A)$, which is given by $1_{Q(A)}$. Then we have
an element of $Q(X,Y,f)\equiv P(X)\to P(Y)$ as required.
\end{proof}

Equivalence induction and the univalence axiom are in fact logically equivalent.
\begin{lem}
If equivalence induction holds, then $E_{A,B}$ is an equivalence.
\end{lem}
\begin{proof}
By Corollary \ref{cor:equivs-equiv} it suffices to show $E_{A,B}$ is an
isomorphism. We first produce a map $F_{A,B}$ which inverts $E_{A,B}$; given any
$f:A\simeq B$, we must construct an element of $A=B$. By equivalence induction, it
suffices to produce $\refl A:A=A$; because equivalence induction produces this
element on $1_A$, $F_{A,A}(1_A)\equiv \refl A$.
We check that $F_{A,B}$ is an inverse on both sides.

For any $f:A\simeq B$, $E_{A,B}(F_{A,B}(f))=f$. By equivalence induction, we
show $E_{A,A}(F_{A,A}(1_A))=1_A$, which is immediate. On the other hand, for any
$p:A=B$, $F_{A,B}(E_{A,B}(p))=p$. By Id-induction, we show
$F_{A,A}(E_{A,A}(\refl A))=\refl A$, which is again immediate.
\end{proof}

% Local Variables:
% TeX-master: "main"
% End:
