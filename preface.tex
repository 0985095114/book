\chapter*{Preface}
\label{cha:preface}
\markboth{\textsc{Preface}}{}
\addcontentsline{toc}{chapter}{Preface}

{%%%%%%% macros local to this file, discharged at the end of the file

%\newcommand{\stype}{{\;\sf type}}
%%\newcommand{\nat}{\mathbb{N}}
%\newcommand{\nat}{{\bf N}}
%\newcommand{\rec}{{\sf rec}}
%%\newcommand{\bool}{\bbB}
%\newcommand{\bool}{{\bf B}}
%\newcommand{\app}{{\sf app}}
%\newcommand{\pair}{{\sf pair}}
%\newcommand{\suc}{{\sf succ}}
%\newcommand{\inleft}{{\sf inleft}}
%\newcommand{\inright}{{\sf inright}}
%%\newcommand{\bbzero}{{0\hspace*{-4pt} 0}}
%\newcommand{\emptyt}{{\bf 0}}
%%\newcommand{\bbone}{{1\hspace*{-4pt} 1}}
%\newcommand{\unitt}{{\bf 1}}

%\newcommand{\idtypevar}{{{\sf Id}}}
%\newcommand{\eq}{{{\sf Eq}}}
%

This book was written as a collaborative effort during the 2012-13 Special Year on Univalent Foundations at the Institute for Advanced Study, School of Mathematics, organized by Steve Awodey, Thierry Coquand, and Vladimir Voevodsky.

\bigskip
%\noindent The participants in the special year were:
\centerline{\emph{Participants}}
%
\begin{multicols}{3}{
\begin{itemize}
\item[] Peter Aczel
\item[] Benedikt Ahrens
\item[] Thorsten Altenkirch
\item[] Carlo Angiuli
\item[] Steve Awodey
\item[] Bruno Barras
\item[] Andrej Bauer
\item[] Yves Bertot
\item[] Marc Bezem
\item[] Anthony Bordg
\item[] Guillaume Brunerie
\item[] Thierry Coquand
\item[] Eric Finster
\item[] Daniel Grayson
\item[] Hugo Herbelin
\item[] Andr\'e Joyal
\item[] Chris Kapulkin
\item[] Dan Licata
\item[] Peter Lumsdaine
\item[] Assia Mahboubi
\item[] Per Martin-L\"of
\item[] Sergey Melikhov
\item[] Alvaro Pelayo
\item[] Andrew Polonsky
\item[] Egbert Rijke
\item[] Michael Shulman
\item[] Kristina Sojakova
\item[] Matthieu Sozeau
\item[] Bas Spitters
\item[] Benno van den Berg
\item[] Vladimir Voevodsky
\item[] Michael Warren
\item[] Noam Zeilberger
\end{itemize}
}
\end{multicols}

%%\noindent There were also the following student participants:
%\centerline{\emph{Students}}
%%
%\begin{multicols}{3}{
%\begin{itemize}
%\item[] Carlo Angiuli
%\item[] Anthony Bordg
%\item[] Guillaume Brunerie
%\item[] Chris Kapulkin
%\item[] Egbert Rijke
%\item[] Kristina Sojakova
%\end{itemize}
%}
%\end{multicols}

%\noindent The following people visited at some point:
\centerline{\emph{Visitors}}
%
\begin{multicols}{3}{
\begin{itemize}
\item[] Jeremy Avigad
\item[] Cyril Cohen
\item[] Robert Constable
\item[] Pierre-Louis Curien
\item[] Peter Dybjer
\item[] Mart{\'\i}n Escard{\'o}
\item[] Kuen-Bang Hou
\item[] Nicola Gambino
\item[] Richard Garner
\item[] Georges Gonthier
\item[] Thomas Hales
\item[] Robert Harper
\item[] Martin Hofmann
\item[] Pieter Hofstra
\item[] Joachim Kock
\item[] Zhaohui Luo
\item[] Michael Nahas
\item[] Erik Palmgren
\item[] Dana Scott
\item[] Philip Scott
\item[] Sergei Soloviev
\end{itemize}
}
\end{multicols}

\noindent While each of the above individuals contributed something to this project --- in the form of ideas, words, or deeds --- a few of them contributed much more, and deserve to be recognized.  They are Peter Aczel, Benedikt Ahrens, Thorsten Altenkirch,  Steve Awodey, Andrej Bauer, Guillaume Brunerie, Thierry Coquand, Dan Grayson, Chris Kapulkin, Dan Licata, Peter Lumsdaine, Egbert Rijke, Kristina Sojakova, Bas Spitters --- and above all, Mike Shulman.  Let it be recorded, however, that the spirit of collaboration that prevailed during the year was truly extraordinary. 

% To get the list of contributers to github run the command:
% git log --format='%aN' | sort -u

Although this book is an experiment in \emph{informal} type theory, something must be said about the important role played by the formalization of this work in computer proof assistants.  As explained in the introduction, Univalent Foundations is closely tied to the idea of a foundation of mathematics that can be implemented in a proof assistant; although such a formalization is not part of this book, this goal has guided the development of the theory in an essential way. Indeed, in many cases, the material presented here was done first in the fully formalized setting inside a proof assistant, and only later ``unformalized" to arrive at the informal presentation you find before you --- a remarkable inversion of the usual state of affairs in formalized mathematics.   A large and growing library of code in Coq and Agda currently exists, in which much of theory presented here has been formalized.  The interested reader can find this library at {\tt homotopytypetheory.org}.

Special thanks are due to the Institute for Advanced Study, without which this book would obviously never have come to be --- but which, moreover, proved to be a stimulating and thoroughly enjoyable forge for this new field of mathematics.  We herewith commit it to the flames of public opinion; may it be sufficiently hardened and tempered to stand the test. 

\bigskip

\flushright{The Univalent Foundations Program\\
Princeton, April 2013}


}

%%%%%%%%%%%%% end of scope of local macros
% Local Variables:
% TeX-master: "main"
% End:
