\newcommand{\coh}[3]{\mathtt{coh} \; #1 \; #2 \; #3}

\chapter{Equivalences}
\label{cha:equivalences}

We now study in more detail the notion of \emph{equivalence of types} that was introduced briefly in \S\ref{sec:basics-equivalences}.
Specifically, we will give several different ways to define a type $\isequiv(f)$ having the properties mentioned there.
Recall that these were
\begin{enumerate}
\item $\qinv(f) \to \isequiv (f)$.\label{item:beb1}
\item $\isequiv (f) \to \qinv(f)$.\label{item:beb2}
\item $\isequiv(f)$ is a mere proposition.\label{item:beb3}
\end{enumerate}
Recall also that $\qinv(f)$ denotes the type
\begin{equation*}
  \sm{g:B\to A} \big((f \circ g \htpy \idfunc[B]) \times (g\circ f \htpy \idfunc[A])\big)
\end{equation*}
of quasi-inverses to $f$.
By function extensionality, it follows that $\qinv(f)$ is equivalent to the type
\begin{equation*}
  \sm{g:B\to A} \big((f \circ g = \idfunc[B]) \times (g\circ f = \idfunc[A])\big).
\end{equation*}

We will also show that all the possible definitions of $\isequiv (f)$ yield equivalent types.
Thus, for the most part, it does not really matter which one we take as ``the'' definition of ``equivalence''.
However, for purposes of the comparisons in this chapter, we will use different words for each of the definitions to disambiguate them in our proofs.


\section{Contractible fibers}
\label{sec:contrf}

Our first definition is that $f$ is an equivalence if all its fibers are contractible.
We have said previously that a type is \emph{contractible} if it is equivalent to \unit, but to use such a definition of ``contractible'' in defining ``equivalence'' would be circular.
Thus, we use instead the following definition.

\begin{defn}\label{defn:contractible}
  A type $A$ is \textbf{contractible}, or a \textbf{singleton}, if there is $a:A$, called the \textbf{center of contraction}, such that $a=x$ for all $x:A$.
\end{defn}

In other words, the type $\iscontr(A)$ is defined to be
\[ \iscontr(A) \defeq \sm{a:A} \prd{x:A}(a=x). \]
We observe some basic properties of contractible types.

\begin{lem}\label{thm:contr-paths}
  For a type $A$, the following are equivalent.
  \begin{enumerate}
  \item $A$ is contractible.
  \item $A$ is a mere proposition, and there is a point $a:A$.\label{item:contr-inhabited-prop}
  \item $A$ is equivalent to \unit.\label{item:contr-eqv-unit}
  \end{enumerate}
\end{lem}
\begin{proof}
  If $A$ is contractible, then it certainly has a point $a:A$ (the center of contraction), while for any $x,y:A$ we have $x=a=y$; thus $A$ is a mere proposition.
  Conversely, if we have $a:A$ and $A$ is a mere proposition, then for any $x:A$ we have $x=a$; thus $A$ is contractible.
  We leave the equivalence with~\ref{item:contr-eqv-unit} as an exercise.
\end{proof}

\begin{lem}\label{thm:isprop-iscontr}
  For any type $A$, the type $\iscontr(A)$ is a mere proposition.
\end{lem}
\begin{proof}
  Suppose given $c,c':\iscontr(A)$.
  We may assume $c\jdeq(a,p)$ and $c'\jdeq(a',p')$ for $a,a':A$ and $p:\prd{x:A} (a=x)$ and $p':\prd{x:A} (a'=x)$.
  By the characterization of paths in $\Sigma$-types, to show $c=c'$ it suffices to exhibit $q:a=a'$ such that $\trans{q}{p}=p'$.

  We choose $q\defeq p(a')$.
  For the other equality, by function extensionality we must show that $(\trans q p)(x)=p'(x)$ for any $x:A$.
  For this, it will suffice to show that for any $x,y:A$ and $u:x=y$ we have $u= \opp{p(x)} \ct p(y)$, since then we would have $(\trans q p)(x) = \opp{p(x)} \ct p(y) = p'(x)$.
  But now we can invoke path induction to assume that $x\jdeq y$ and $u\jdeq \refl{x}$.
  In this case our goal is to show that $\refl x = \opp{p(x)} \ct p(x)$, which is just the inversion law for paths.
\end{proof}

\begin{cor}\label{thm:contr-contr}
  If $A$ is contractible, then so is $\iscontr(A)$.
\end{cor}
\begin{proof}
  By \autoref{thm:isprop-iscontr} and \autoref{thm:contr-paths}\ref{item:contr-inhabited-prop}.
\end{proof}

\begin{lem}\label{thm:contr-forall}
  If $P:A\to\type$ is a type family such that each $P(a)$ is contractible, then $\prd{x:A} P(x)$ is contractible.
\end{lem}
\begin{proof}
  By \autoref{thm:isprop-forall}, $\prd{x:A} P(x)$ is a mere proposition since each $P(x)$ is.
  But it also has an element, namely the function sending each $x:A$ to the center of contraction of $P(x)$.
  Thus by \autoref{thm:contr-paths}\ref{item:contr-inhabited-prop}, $\prd{x:A} P(x)$ is contractible.
\end{proof}

We are now ready to define maps with contractible fibers.

\begin{defn}
  The \textbf{fiber} of a map $f:A\to B$ over a point $y:B$ is $\setof{x:A | f(x) = y}$.
\end{defn}

In homotopy theory, this is what would be called the \emph{homotopy fiber} of $f$.

\begin{defn}\label{defn:equivalence}
  A map $f:A\ra B$ is \textbf{contractible} if for all $y:B$, the fiber $\setof{x:A | f(x) = y}$ is contractible.
\end{defn}

Thus, the type $\iscontr(f)$ is defined to be
\begin{equation}
  \iscontr(f) \defeq \prd{y:B} \iscontr (\setof{x:A | f(x) = y}).\label{eq:iscontrf}
\end{equation}
This terminology follows the general homotopy-theoretic practice of saying that a map has a certain property if all of its (homotopy) fibers have that property.
In Chapter~\ref{cha:hlevels} we will learn to also call such a map \emph{$(-2)$-truncated}.

For this definition,~\ref{item:beb2} and~\ref{item:beb3} are easy to show.

\begin{lem}
  For any $f:A\to B$ we have $\iscontr(f)\to\qinv(f)$.
\end{lem}
\begin{proof}
  Suppose $c:\iscontr(f)$, and define $g:B\to A$ by $g(y)\defeq \proj1(\proj1(c(y)))$.
  Note that this makes sense, since $c(y):\iscontr (\setof{x:A | f(x) = y})$, therefore $\proj1(c(y)):\setof{x:A | f(x) = y}$, and thus $\proj1(\proj1(c(y))) : A$.
  Moreover, we have $\proj2(\proj1(c(y))) : f(g(y))=y$, yielding a homotopy $\beta:f\circ g\htpy \idfunc[B]$.
  On the other hand, for any $a:A$ we have $(a,\refl{f(a)}) : \setof{x:A | f(x) = f(a)}$, hence
  \[\proj2(c(f(a)))(a,\refl{f(a)}) : (g(f(a)),\beta(f(a))) = (a,\refl{f(a)}).\]
  Projecting to the first component of this equality, we have $g(f(a))=a$, yielding a homotopy $\alpha:g\circ f \htpy \idfunc[A]$.
\end{proof}

\begin{lem}
  For any $f$, the type $\iscontr(f)$ is a mere proposition.
\end{lem}
\begin{proof}
  By \autoref{thm:isprop-iscontr}, each type $\iscontr (\setof{x:A | f(x) = y})$ is a mere proposition.
  Thus, by \autoref{thm:isprop-forall}, so is~\eqref{eq:iscontrf}.
\end{proof}

However, it is slightly tricky to prove that even the identity map satisfies this definition.

\begin{lem}\label{lem:id-map}
For each type $A$, the identity map $\idfunc[A]\defeq \lambda_{x:A}x :A\ra A$ is contractible.
\end{lem}
\begin{proof}
  Let $a:A$ and let $\{a\}_A\defeq \setof{ x:A | x=a }$ be its fiber with respect to $\idfunc[A]$.
  We show that $\{ a\}_A$ is contractible.
  Then, discharging $a:A$, we get that $\fall{x:A} (\{ x\}_A$ is contractible); i.e.\ $\idfunc[A]$ is an equivalence.

  Let $\oa \defeq (a,\refl{a}):\{ a\}_A$.  As $(a,\refl{a}) = \oa$ we can use Id-induction \`{a} la Christine-Paulin to get
  \[\fall{x:A}{z:x=a} ((x,z)=\oa).\]
  Hence, by $\Sigma$-folding, $\fall{u:\{ a\}_A} (u=\oa)$.
  Thus $\{ a\}_A$ is contractible, as desired.
\end{proof}

To show that contractible maps satisfy condition~\ref{item:beb1}, we will go by way of a different notion of equivalence.


% %%%%%%%%%%%%%%%%%%%%%%%%%%%%%%%%%%%%%%%%%%%%
% \section{Bijections and Isomorphisms}
% %%%%%%%%%%%%%%%%%%%%%%%%%%%%%%%%%%%%%%%%%%%%


% \begin{defn} \label{defn:isos}
% Let $f:A\ra B$.
% \begin{enumerate}
% \item $f$ is a {\em bijection} if it is both injective and surjective, where
% \begin{itemize}
% \item $f$ is {\em injective} if $\forall_{x,x':A}\; \; fx=fx'\;\ra\; x=x'$,
% \item $f$ is {\em surjective} if $\forall_{y:B}\exists_{x:A}\;\; fx=y$,
% \end{itemize}
% \item $f$ is {\em an isomorphism} if it has $g:B\ra A$ that is both a left and a right
% inverse where,
% \item $f$ is a {\em weak isomorphism} if it is both left invertible and right invertible.
% \item We write $A\cong B$ if there is an isomorphism $A\ra B$, and $A\cong_w B$ if there is a weak isomorphism $A\to B$.
% \end{enumerate}
% \end{defn}

% \begin{thm}\label{thm:isos-id-and-composition} $\;$
% \begin{enumerate}
% \item $f:A\ra B$ is a surjection iff it is right-invertible.
% \item If $f:A\ra B$ is left-invertible then it is injective.
% \item $\idfunc[A]:A\ra A$ is an isomorphism.
% \item If $f:A\ra B$ and $g:B\ra C$ are isomorphisms then so is $g\circ f:A\ra C$.
% \end{enumerate}
% \end{thm}
% \begin{proof} Routine
% \end{proof}
% \begin{thm}\label{thm:bijections-isos}
% The following are logically equivalent for $f:A\ra B$.
% \begin{enumerate}
% \item $f$ is a bijection.
% \item $\forall_{y:B}\exists_{x:A}\;\; (fx=y\;\wedge \forall_{x':A}\;\; (fx'=y\;\ra\; x=x'))$.\footnote{One could consider calling this notion ``bijection`` and the other one ``weak bijection``, to match the terminology for isomorphism and weak isomorphism.}
% \item $f$ is an isomorphism.
% \item $f$ is a weak isomorphism.
% \end{enumerate}
% \end{thm}
% \begin{proof}
% The proof of this proposition in type theory can be done by proving the implications $(i)\ra (ii)\ra (iii)\ra (iv)\ra (i)$.  $(i)\ra (ii)$ and $(iii)\ra (i)$ are the same as in set theory.  The proof that $(ii)\ra (iii)$ requires, given $(ii)$, the proof of the existence of a function $g$ inverse to $f$.  In set theory this uses the fact that functions are defined to be total single-valued relations.  Instead, in type theory the proof of the existence of $g$
% uses the non-dependent version of the type theoretic axiom of choice.  This axiom holds in the propositions as types interpretation of logic because of the strong form of the existential quantifier.  For $(iv)\ra (i)$ let $f$ be both left and right invertible.  By (i) and (ii) of Theorem~\ref{thm:isos-id-and-composition}, $f$ is both injective and surjective and hence a bijection.  
% \end{proof}


%%%%%%%%%%%%%%%%%%%%%%%%%%%%%%%%%%%%%%
\section{Half Adjoint Equivalences}
%%%%%%%%%%%%%%%%%%%%%%%%%%%%%%%%%%%%%%
%\comment{%%%%%%%%%%%%%%%%%%%%%%%%%%%%%%%%%%%%%%%%%%%
%In order to define the notion of an adjoint isomorphism we need the following definition.
%\begin{defn} $\;$
%If $f:A\ra B$ define, by Id-induction, $fz:fx=fx'$ for $x,x':A, z:x=x'$ such that $f\refl{x} = \refl{fx}$ for $x:A$.
%\end{defn}
%}%%%%%%%%%%%%%%%%%%%%%%%%%%%%%%%%%%%%%%%%%%%%%%%%%%%%

The idea of this definition is that in order to make $\qinv(f)$ better behaved, we need to add additional higher ``coherence data''.
\emph{A priori}, we might expect to have to add infinitely many data at all higher levels of homotopy.
Fortunately, it turns out that a single additional datum suffices.

\begin{defn}
  A function $f:A\ra B$ is a \textbf{half adjoint equivalence} if there are $g:B\ra A$, $\eta: g \circ f = \idfunc[A]$ and $\epsilon:f \circ g = \idfunc[B]$ such that there exists a path
  \[\tau : \prd{x:A} \map{f}{\eta x} = \epsilon(fx)\]
% We write $A\cong^f_a B$ if there is an adjoint isomorphism $A\ra B$.
\end{defn}
Thus we have a type $\ishae(f)$, defined to be
\begin{equation*}
  \sm{g:B\to A}{\eta: g \circ f = \idfunc[A]}{\epsilon:f \circ g = \idfunc[B]} \prd{x:A} \map{f}{\eta x} = \epsilon(fx) .
\end{equation*}
Note that in the above definition, the coherence condition relating $\eta$ and $\epsilon$ only involves $f$.
We might consider instead an analogous coherence condition involving $g$:
\[\upsilon : \prd{y:B} \map{g}{\epsilon y} = \eta(gy)\]
% We write $A\cong^g_a B$ for the notion of adjoint isomorphism which uses the above condition instead.
Fortunately, it turns out each of the conditions implies the other one:

\begin{lem}\label{lem:coh-equiv}
For $f : A \to B$, $g:B\ra A$, $\eta: g \circ f = \idfunc[A]$ and $\epsilon:f \circ g = \idfunc[B]$, the following conditions are logically equivalent:
\begin{itemize}
\item $\prd{x:A} \map{f}{\eta x} = \epsilon(fx)$
\item $\prd{y:B} \map{g}{\epsilon y} = \eta(gy)$
\end{itemize}
\end{lem}
\begin{proof}
  It suffices to show one direction; the other one is obtained by swapping $A/B$, $f/g$, and $\eta/\epsilon$.
  Let $\tau : \prd{x:A}\;\map{f}{\eta x} = \epsilon(fx)$.
  Fix $y : B$.
  Using naturality of $\epsilon$ and applying $g$, we get the following commuting diagram :
\[\uppercurveobject{{ }}\lowercurveobject{{ }}\twocellhead{{ }}
  \xymatrix{gfgfgy \ar^-{gfg(\epsilon y)}[r] \ar_{g(\epsilon (fgy))}[d] & gfgy \ar^{g(\epsilon y)}[d] \\ gfgy \ar_{g(\epsilon y)}[r] & gy
  }\]
Using $\tau(gy)$ gives us
\[\uppercurveobject{{ }}\lowercurveobject{{ }}\twocellhead{{ }}
  \xymatrix{gfgfgy \ar^-{gfg(\epsilon y)}[r] \ar_{gf(\eta (gy))}[d] & gfgy \ar^{g(\epsilon y)}[d] \\ gfgy \ar_{g(\epsilon y)}[r] & gy
  }\]
Using the commutativity of $\eta$ with $f \circ g$ (\autoref{cor:hom-fg}), we have
\[\uppercurveobject{{ }}\lowercurveobject{{ }}\twocellhead{{ }}
  \xymatrix{gfgfgy \ar^-{gfg(\epsilon y)}[r] \ar_{\eta (gfgy)}[d] & gfgy \ar^{g(\epsilon y)}[d] \\ gfgy \ar_{g(\epsilon y)}[r] & gy
  }\]
However, by naturality of $\eta$ we also have
\[\uppercurveobject{{ }}\lowercurveobject{{ }}\twocellhead{{ }}
  \xymatrix{gfgfgy \ar^-{gfg(\epsilon y)}[r] \ar_{\eta (gfgy)}[d] & gfgy \ar^{\eta(gy)}[d] \\ gfgy \ar_{g(\epsilon y)}[r] & gy 
  }\]
Thus, canceling all but the right-hand homotopy, we have $g(\epsilon y) = \eta(g y)$ as desired.
\end{proof}

However, it is important that we do not include \emph{both} of these data in the definition of $\ishae (f)$ (whence the name ``\emph{half} adjoint equivalence'').
If we did, it would not turn out to be a mere proposition.
(On the other hand, we could recover a mere proposition by including both 2-dimensional coherence paths and also \emph{one} additional 3-dimensional coherence path, of the two possibilities at that level.
In general, homotopy theory suggests that we will get a well-behaved type if we cut off after an odd number of coherences.)

Of course, it is obvious that we have $\ishae(f) \to\qinv(f)$: simply forget the coherence datum.
The other direction is a version of a standard argument from homotopy theory and category theory.

\begin{thm}\label{thm:equiv-iso-adj}
  For any $f:A\to B$ we have $\qinv(f)\to\ishae(f)$.
\end{thm}
\begin{proof}
Suppose that $(g,\eta,\varepsilon)$ is a quasi-inverse for $f$. We have to provide
a quadruple $(g',\eta',\varepsilon',c)$ witnessing that $f$ is a half adjoint equivalence. To
define $g'$ and $\eta'$, we can just make the obvious choice by setting $g'
\defeq g$ and $\eta'\defeq \eta$. However, in the definition of $\varepsilon'$ we
need start worrying about the construction of $c$, so we cannot just follow our nose
and take $\varepsilon'$ to be $\varepsilon$. Instead, we take
\begin{equation*}
\varepsilon'(b) \defeq (\varepsilon(b)\ct \ap{f}{\eta(g(b))})\ct \opp{\varepsilon(f(g(b)))}
\end{equation*}
Now we need to find
\begin{equation*}
c(a):(\varepsilon(f(a))\ct \ap{f}{\eta(g(f(a)))})\ct \opp{\varepsilon(f(g(f(a))))}=\ap{f}{\eta(a)}
\end{equation*}
Note first that by \autoref{cor:hom-fg}, we have 
%$\eta(g(f(a)))\ct\eta(a)=\ap{g}{\ap{f}{\eta(a)}}\ct\eta(a)$ and hence it follows that
$\eta(g(f(a)))=\ap{g}{\ap{f}{\eta(a)}}$. Therefore, we can apply
\autoref{lem:htpy_natural} to compute
\begin{align*}
\varepsilon(f(a))\ct \ap{f}{\eta(g(f(a)))}
& = \varepsilon(f(a))\ct \ap{f}{\ap{g}{\ap{f}{\eta(a)}}}\\
& = \ap{f}{\eta(a)}\ct\varepsilon(f(g(f(a))))
\end{align*}
from which we get the desired path $c(a)$.
\end{proof}

Now we want to show that $\eqv{\iscontr(f)}{\ishae(f)}$.
Since $\iscontr(f)$ satisfies conditions~\ref{item:beb2} and~\ref{item:beb3} while $\ishae(f)$ satisfies conditions~\ref{item:beb1} and~\ref{item:beb3}, this will imply that in fact both satisfy all three conditions.

\begin{thm}\label{thm:equiv-contr-hae}
  For any $f:A\to B$ we have $\eqv{\iscontr(f)}{\ishae(f)}$.
\end{thm}

We give two proofs of this theorem.
The first is a high-level argument obtained by repeatedly interchanging $\Sigma$- and $\Pi$-types; the second is more computational.

\begin{proof}
  We will make heavy use of the fact that all type formers respect equivalences.
  For instance, if $P,Q:A\to\type$ and for all $x:A$ we have $\eqv{P(x)}{Q(x)}$, it follows that
  \begin{align*}
    \sm{x:A}P(x) &\;\simeq\; \sm{x:A}Q(x)\mathrlap{\qquad\text{and}}\\
    \prd{x:A}P(x) &\;\simeq\; \prd{x:A}Q(x).
  \end{align*}
  Similarly, given $P:B\to\type$ and an equivalence $f:\eqv A B$, we have
  \begin{align}
    \sm{x:A}P(fx) &\;\simeq\; \sm{y:B}P(y)\mathrlap{\qquad\text{and}}\label{eq:sigma-baseequiv}\\
    \prd{x:A}P(fx) &\;\simeq\; \prd{y:B}P(y).\label{eq:pi-baseequiv}
  \end{align}
  These facts follow immediately from function extensionality and univalence, but can also be proven more directly.
  A direct proof is perhaps to be preferred in the long run, since it has more explicit computational content, but for our purposes here this does not matter.

  We will construct a chain of equivalences whose composite yields the desired result.
  We begin by applying \autoref{thm:ttac} twice, once by way of~\eqref{eq:sigma-baseequiv}.
  \begin{align*}
    \iscontr(f)
    &\jdeq \prod_{b:B} \; \sum_{c:\sum_{a:A}(fa=b)}\; \prod_{d:\sum_{a:A}(fa=b)} (c=d)\\
    &\simeq \sum_{k:\prod_{b:B}\sum_{a:A} (fa=b)} \;\prod_{b:B} \;\prod_{d:\sum_{a:A}(fa=b)} (k(b)=d)\\
    &\simeq \sum_{k:\sum_{g:B\to A}\;\prod_{b:B} (fgb=b)} \;\prod_{b:B} \;\prod_{d:\sum_{a:A}(fa=b)} ((\proj1 k(b),\proj2 k(b))=d)\\
    &\simeq \sum_{g:B\to A} \; \sum_{\epsilon:\prod_{b:B} (fgb=b)} \;\prod_{b:B} \;\prod_{d:\sum_{a:A}(fa=b)} ((gb,\epsilon b)=d).
    % &\simeq \sum_{k:\prod_{b:B}\sum_{a:A} (fa=b)} \;\prod_{b:B}\;\prod_{a:A}\; \prod_{\gamma:fa=b} (kb = (a,\gamma))\\
    % &\simeq \sum_{k:\prod_{b:B}\sum_{a:A} (fa=b)} \;\prod_{b:B}\;\prod_{a:A}\; \prod_{\gamma:fa=b} (kb = (a,\gamma))\\
  \end{align*}
  The equivalence in the last line uses an easy-to-prove ``associativity'' of dependent pair types:
  \[\sum_{x:X} \; \sum_{y:Y(x)} Z(x,y) \;\simeq\; \sum_{w:\sum_{x:X} Y(x)} Z(w).\]
  Thus, to show that $\iscontr(f)$ is equivalent to
  \begin{equation*}
    \ishae(f) \jdeq \sum_{g:B\to A} \;\sum_{\eta: g \circ f \htpy \idfunc[A]} \; \sum_{\epsilon:f \circ g \htpy \idfunc[B]} \;
    \prod_{x:A} (\map{f}{\eta x} = \epsilon(fx)),
  \end{equation*}
  it will suffice to fix $g:B\to A$ and $\epsilon:g\circ f \htpy \idfunc[A]$, and show that
  \begin{equation}\label{eq:ech-part2}
    \prod_{b:B} \;\prod_{d:\sum_{a:A}(fa=b)} ((gb,\epsilon b)=d)
    \;\simeq \;
    \sum_{\eta: g \circ f \htpy \idfunc[A]} \;
    \prod_{x:A} (\map{f}{\eta x} = \epsilon(fx)).
  \end{equation}
  We have used an evident commutativity between unrelated $\Sigma$-types to switch the order of $\eta$ and $\epsilon$ in $\ishae(f)$.
  Now, using the universal properties~\eqref{eq:sigma-lump} and~\eqref{eq:path-lump} of $\Sigma$-types and identity types, along with a analogous commutativity for unrelated $\Pi$-types, we have
  \begin{align*}
    \prod_{b:B} \;\prod_{d:\sum_{a:A}(fa=b)} ((gb,\epsilon b)=d)
    &\simeq \prod_{b:B} \;\prod_{a:A} \;\prod_{\gamma:fa=b} ((gb,\epsilon b)= (a,\gamma))\\
    &\simeq \prod_{a:A} \;\prod_{b:B} \;\prod_{\gamma:fa=b} ((gb,\epsilon b)= (a,\gamma))\\
    &\simeq \prod_{a:A}  ((gfa,\epsilon (fa))= (a,\refl {fa})).
  \end{align*}
  The equality $((gfa,\epsilon (fa))= (a,\refl a))$ is in $\sm{x:A} (fx = fa)$.
  Thus, by the identification of paths in $\Sigma$-types, and \autoref{thm:ttac} again, we have
  \begin{align*}
    \prod_{a:A}  ((gfa,\epsilon (fa))= (a,\refl {fa}))
    &\simeq \prod_{a:A}\; \sum_{\alpha:gfa = a} (\trans{\alpha}{\epsilon(fa)} = \refl{fa})\\
    &\simeq \sum_{\eta:\prod_{a:A} (gfa = a)} \; \prod_{a:A} (\trans{\eta(a)}{\epsilon(fa)} = \refl{fa})
  \end{align*}
  Recalling our goal~\eqref{eq:ech-part2}, we see that it will suffice to fix ${\eta: g \circ f \htpy \idfunc[A]}$ and $a:A$, and show that
  \begin{equation}
    \label{eq:ech-part3}
    (\trans{(\eta a)}{\epsilon(fa)} = \refl{fa})
    \;\simeq\;
    (\map{f}{\eta a} = \epsilon(fa)).
  \end{equation}
  Now the transport along $\eta a : gfa = a$ on the left-hand side takes place with respect to the type family $x\mapsto (fx=fa)$.
  Thus, by [TODO: the pullback transport lemma] and \autoref{thm:path-paths}, we have
  \[\trans{(\eta a)}{\epsilon(fa)} = \opp{\ap f {\eta a}} \ct \epsilon(fa).\]
  But clearly $\opp{\ap f {\eta a}} \ct \epsilon(fa) = \refl{fa}$ if and only if $\map{f}{\eta a} = \epsilon(fa)$, and it is easy to show that this logical equivalence is actually an equivalence, yielding~\eqref{eq:ech-part3}.
\end{proof}

For the second proof, the following definition and lemmas will be useful.

\begin{defn}
For $f : A \to B$, $g : B \to A$, and $\eta : g \circ f \sim \idfunc[A]$, we denote
\[ \coh{f}{g}{\eta} \defeq \prd{y : B} \sm{\gamma : fgy = y} g(\gamma) = \eta y \]
\end{defn}

\begin{lem}\label{lem:hfib}
Let $f : A \to B$ and $y : B$. For any $p, p' : \mathtt{hfiber} \; f \;y$, we have
\[ \big(p = p'\big) \simeq \sm{\gamma : \proj{1}(p) = \proj{1}(p')} f\gamma \ct \proj{2}(p') = \proj{2}(p) \]
\end{lem}
\begin{proof}
We use the known interactions between $\Sigma$-types, identity types, and transports.
\end{proof}

\begin{lem}\label{lem:coheq}
Let $f : A \to B$, $g : B \to A$, and $\eta : g \circ f \sim \idfunc[A]$. For any $y : B$ and $p, p' :\coh{f}{g}{\eta} \; y$, we have
\[ \big(p = p'\big) \simeq \sm{\delta : \proj{1}(p) = \proj{1}(p')} g \delta \ct \proj{2}(p') = \proj{2}(p) \]
\end{lem}
\begin{proof}
We use the known interactions between function types, $\Sigma$-types, identity types, and transports.
\end{proof}

\begin{cor}\label{lem:higher-hom}
For any half adjoint equivalence $(f,g,\eta,\epsilon,\tau) : A \cong_a^f B$ and any $p : \id[A]{x}{x'}$ the following diagram commutes:
\begin{align*}
\xymatrix{
f(\eta x') \ar[dd]_{\tau x'} \ar[r]^{\mathit{via \;} \eta(p) \;\;\;\;\; \; \;\;\;\;\;\;} & f(\opp{gf(p)} \ct \eta x \ct p) \ar[d] \\
& \opp{fgf(p)} \ct f(\eta x) \ct f(p) \ar[d]_{\mathit{via \;} \tau x} \\
\epsilon(f x') \ar[r]_{\mathit{via \;} \epsilon(fp)\;\;\;\;\; \; \;\;\;\;\;\;\;\;\;\;}  & \opp{fgf(p)} \ct \epsilon (fx) \ct f(p)}
\end{align*}
Similarly, for any $(f,g,\eta,\epsilon,\upsilon) : A \cong_a^g B$ and any $p : \id[A]{y}{y'}$ the following diagram commutes:
\begin{align*}
\xymatrix{
g(\epsilon y') \ar[dd]_{\upsilon y'} \ar[r]^{\mathit{via \;} \epsilon(p) \;\;\;\;\; \; \;\;\;\;\;\;} & g(\opp{fg(p)} \ct \eta y \ct p) \ar[d] \\
& \opp{gfg(p)} \ct g(\epsilon y) \ct g(p) \ar[d]_{\mathit{via \;} \upsilon y} \\
\eta(g y') \ar[r]_{\mathit{via \;} \eta(gp)\;\;\;\;\; \; \;\;\;\;\;\;\;\;\;\;}  & \opp{gfg(p)} \ct \eta (gy) \ct g(p)}
\end{align*}
\end{cor}
\begin{proof}
TODO
\end{proof}

\begin{lem}\label{lem:coh-fg}
For any half adjoint equivalence $(f,g,\eta,\epsilon,\tau) : A \cong_a^f B$ the following diagram commutes for each $x : A$:

\begin{align*}
\xymatrix{
f(\eta (g f x)) \ar[d]_{\mathit{via \; } \com{\eta}{g \circ f}{x}} \ar[r]^{\tau(gfx)} &
\epsilon (fgfx) \ar[d]^{\com{\epsilon}{f\circ g}{fx}} \\
fgf (\eta x) \ar[r]_{\mathit{via \;} \tau x} &
fg(\epsilon (f x))} 
\end{align*}

Similarly, for any $(f,g,\eta,\epsilon,\upsilon) : A \cong_a^g B$ the following diagram commutes for each $y : B$:

\begin{align*}
\xymatrix{
g(\epsilon (f g y)) \ar[d]_{\mathit{via \; } \com{\epsilon}{f \circ g}{y}} \ar[r]^{\upsilon(fgy)} &
\eta (gfgy) \ar[d]^{\com{\eta}{g\circ f}{gy}} \\
gfg (\epsilon y) \ar[r]_{\mathit{via \;} \upsilon y} &
gf(\eta (g y))} 
\end{align*}
\end{lem}
\begin{proof}
TODO
\end{proof}

\begin{proof}[Second proof of \autoref{thm:equiv-contr-hae}]
Fix $f:A\to B$; we proceed in 4 steps:
\begin{enumerate}
\item Define a function $F : \iscontr(f) \to \ishae(f)$. Fix a proof $P(y) \jdeq (c(y), h(y))$ that every homotopy fiber of $f$ is contractible.
We define an inverse mapping $g : B \to A$ by mapping each $y : B$ to the center of contraction of the h-fiber at $y$:
\[ g(y) \defeq \proj{1}(c(y)) \]
We can thus define the transformation $\epsilon$ by mapping $y$ to the witness that $g(y)$ indeed belongs to the h-fiber at $y$:
\[ \epsilon(y) \defeq \proj{2}(c(y)) \]
To define $\eta$ and $\tau$, by \autoref{lem:hfib} we have a family of equivalences $E(y,p,p')$ witnessing
\[ \big( p = p' \big) \simeq \big( \sm{\gamma : \proj{1}(p) = \proj{1}(p')} f\gamma \; \ct \; \proj{2}(p') = \proj{2}(p) \big) \]
for paths in the h-fiber at $y$. A special case arises when $y \defeq fx$ and $p' \defeq (x,\refl{fx})$. In this case,
$E(fx,p,(x,\refl{fx}))$ witnesses the equivalence
\[ \big( p = (x,\refl{fx}) \big) \simeq \big( \sm{\gamma : \proj{1}(p) = x} f\gamma \; \ct \; \refl{fx} = \proj{2}(p) \big) \]
The concatenation with identity path is clearly redundant and we have
\begin{align*}
& \big( \sm{\gamma : \proj{1}(p) = x} f\gamma \; \ct \; \refl{fx} = \proj{2}(p) \big) \simeq \\ 
& \big( \sm{\gamma : \proj{1}(p) = x} f\gamma = \proj{2}(p) \big)
\end{align*}
Let $E'(p,x)$ denote the above equivalence, defined in the obvious way by concatenation.

We can now define $\eta$ and $\tau$ (roughly) by constructing a path from $(gfx,\epsilon(fx))$ to $(x,\refl{fx})$ in the h-fiber at $fx$. More precisely, we put
\begin{align*}
\eta(x) & \defeq \proj{1} (\chi(x)) \\
\tau(x) & \defeq \proj{2} (\chi(x))
\end{align*}
where
\begin{align*} \chi(x) \defeq E'(c(fx), x) \; (E(fx, c(fx), (x,\refl{fx})) \; (h(fx)(x, \refl{fx}))) \end{align*}

\item Define a function $G : \ishae(f) \to \iscontr(f)$. Fix $(g,\eta,\epsilon,\tau): \ishae(f)$. We want to show that every homotopy fiber of $f$ is contractible. Fix $y : B$. As our center of contraction for $\mathtt{hfiber} \; f \;y$ we put 
\[c(y) \defeq (gy, \epsilon y)\]
Now take any $p : \mathtt{hfiber} \; f \;y$. We want to construct a path $h(y,p) : c(y) = p$.
According to \autoref{lem:hfib}, it suffices to give a path $\gamma(y,p) : \id{gy}{\proj{1}(p)}$ for which we have a path $\theta(y,p) : f(\gamma(y,p)) \ct \proj{2}(p) = \epsilon y$. We put $\gamma(y,p) \defeq {\opp{g(\proj{2}(p))}} \ct {\eta (\proj{1}(p))}$.
Then we have 
\begin{align*}
f(\gamma(y,p)) \ct \proj{2}(p) & = \opp{fg(\proj{2}(p))} \ct {f (\eta(\proj{1}(p)))} \ct \proj{2}(p) \\
& = \opp{fg(\proj{2}(p))} \ct {\epsilon(f(\proj{1}(p)))} \ct \proj{2}(p) \\
& = \epsilon y
\end{align*}
where the second equality follows by $\tau(\proj{1}(p))$ and the third equality follows by the naturality of $\epsilon$. We let $\theta(y,p)$ be the path obtained by concatenating the above chain of equalities and put 
\begin{align*}
 h(y,p) \defeq \opp{E(y,c(y),p)} (\gamma(y,p), \theta(y,p))
\end{align*}

\item Show that $G \circ F \sim \idfunc[\iscontr(f)]$. Fix a proof $P$ that each homotopy fiber of $f$ is contractible; it suffices to prove that $P = G(F(P))$ in the type $\prd{y:B} \iscontr (\mathtt{hfiber \;} f \; y)$.
  However, this type is contractible.
  For by $P$, each $\mathtt{hfiber} \; f \; y$ is contractible, while $\iscontr$ and $\Pi$ preserve contractibility by \autoref{thm:contr-contr} and \autoref{thm:contr-forall} respectively.
  Thus we have $P = G(F(P))$, as desired.

\item Show that $F \circ G \sim \idfunc[\ishae(f)]$.
  Fix $(g,\eta,\epsilon,\tau):\ishae(f)$.
  Using definitional $\eta$-expansion for functions, we have \[F(G(g,\eta,\epsilon,\tau)) \jdeq (g, \eta', \epsilon, \tau')\] for some $\eta'$ and $\tau'$.
  It thus suffices to show that for each $x : A$, we have $(n'x,\tau' x) = (nx,\tau x)$ in the type $\sm{\gamma : gfx = x} f\gamma = \epsilon(fx)$, i.e., in $\coh{g}{f}{\eta} \; x$. Fix $x : A$. It is easy to see that
\begin{align*}
(n'x,\tau' x) = E'((gfx, \eta(fx)),x) \; (\refl{gfx} \ct \eta x, \theta)
\end{align*}
where $\theta$ is the following path, with the arrows defined in the obvious ways:
\begin{align*}
\xymatrix{
f(\refl{gfx} \ct \eta x) \ct \refl{fx} \ar[d] \\
\refl{fgfx} \ct f(\eta x) \ct \refl{fx} \ar[d]_{\mathit{via \;}\tau x} \\
\refl{fgfx} \ct \epsilon (fx) \ct \refl{fx} \ar[d] \\
\epsilon (f x)}
\end{align*}
It is easy to show by identity elimination that the following diagram commutes:
\begin{align*}
\xymatrix{
f(\refl{gfx} \ct \eta x) \ar[r] \ar[d] & f(\refl{gfx} \ct \eta x) \ct \refl{fx} \ar[d] \\
f(\eta x) \ar[r] \ar[d]_{\tau x} & \refl{fgfx} \ct f(\eta x) \ct \refl{fx} \ar[d]_{\mathit{via \;}\tau x} \\
\epsilon (f x) \ar[r] \ar[rd] & \refl{fgfx} \ct \epsilon (fx) \ct \refl{fx} \ar[d] \\
& \epsilon (f x)}
\end{align*}
Thus we have $\theta = \theta'$, where $\theta'$ is the following path:
\begin{align*}
\xymatrix{
f(\refl{gfx} \ct \eta x) \ct \refl{fx} \ar[d] \\
f(\refl{gfx} \ct \eta x) \ar[d] \\
f(\eta x) \ar[d] _{\tau x} \\
\epsilon (f x)}
\end{align*}
In other words, we have  
\begin{align*}
(n'x,\tau' x) = E'((gfx, \eta(fx)),x) \; (\refl{gfx} \ct \eta x, \theta')
\end{align*}
Now \[(\refl{gfx} \ct \eta x, \theta') = \opp{E'((gfx, \eta(fx)),x)} \; (\refl{gfx} \ct \eta x, \theta'')\] where $\theta''$ is
\begin{align*}
\xymatrix{
f(\refl{gfx} \ct \eta x) \ar[d] \\
f(\eta x) \ar[d] _{\tau x} \\
\epsilon (f x)}
\end{align*}
Thus $(n'x,\tau' x) = (\refl{gfx} \ct \eta x, \theta'')$. All that remains is to show that $(\eta x, \tau x) = (\refl{gfx} \ct \eta x, \theta'')$ in $\coh{g}{f}{\epsilon} \; x$. By \autoref{lem:coheq}, it suffices to exhibit a path $\delta : \refl{gfx} \ct \eta x = \eta x$ such that $f(\delta) \ct \tau(x) = \theta''$. It is clear that the obvious path from $\refl{gfx} \ct \eta x$ to $\eta x$ does the job. \qedhere
\end{enumerate}  
\end{proof}


\section{Bi-invertible maps}
\label{sec:biinv}

Finally, we return to the definition proposed in \S\ref{sec:basics-equivalences}.

\begin{defn}
  Let $f:A\to B$.
  \begin{itemize}
  \item $g:B\ra A$ is a \textbf{left inverse} to $f$ if $g\circ f\htpy \idfunc[A]$.
    If $f$ has a left inverse, it is \textbf{left invertible}; we have the type
    \[ \linv(f) \defeq \sm{g:B\to A} (g\circ f\htpy \idfunc[A]). \]
  \item Similarly, $g$ is a \textbf{right inverse} to $f$ if $f\circ g\htpy \idfunc[B]$, and we have
    \[ \rinv(f) \defeq \sm{g:B\to A} (f\circ g\htpy \idfunc[B]). \]
  \item We say $f$ is \textbf{bi-invertible} if it has both a left inverse and a right inverse:
    \[ \biinv (f) \defeq \linv(f) \times \rinv(f). \]
  \end{itemize}
\end{defn}

In \S\ref{sec:basics-equivalences} proved that we have $\qinv(f)\to\biinv(f)$ and $\biinv(f)\to\qinv(f)$.
We can now prove the remaining property for this definition.

\begin{lem}\label{thm:equiv-compose-equiv}
  If $f:A\to B$ has a quasi-inverse, then so do
  \begin{align*}
    (f\circ -) &: (C\to A) \to (C\to B)\\
    (-\circ f) &: (B\to C) \to (A\to C).
  \end{align*}
\end{lem}
\begin{proof}
  If $g$ is a quasi-inverse of $f$, then $(g\circ -)$ and $(-\circ g)$ are quasi-inverses of $(f\circ -)$ and $(-\circ f)$ respectively.
\end{proof}

\begin{thm}\label{thm:isprop-biinv}
  For any $f:A\to B$, the type $\biinv(f)$ is a mere proposition.
\end{thm}
\begin{proof}
  Suppose $u,v:\biinv(f)$; we want to show $u=v$.
  It will suffice to show that $\proj1u = \proj1v$ and $\proj2u=\proj2v$.
  Moreover, by function extensionality we have
  \[\eqv{\linv(f)}{\sm{g:B\to A} (g\circ f = \idfunc[A])}\]
  and similarly for $\rinv(f)$.

  Now since $u$ (or $v$) witnesses that $f$ is bi-invertible, $f$ also has a quasi-inverse.
  Thus, by \autoref{thm:equiv-compose-equiv}, $(f\circ -)$ and $(-\circ f)$ both also have quasi-inverses.
  By \autoref{thm:equiv-iso-adj} and \autoref{thm:equiv-contr-hae}, therefore, $(f\circ -)$ and $(-\circ f)$ are contractible.
  However, ${\sm{g:B\to A} (g\circ f = \idfunc[A])}$ is exactly the fiber of $(-\circ f)$ over $\idfunc[A]$, and hence is contractible.
  Thus, $\proj1u = \proj1v$ since they live in a contractible type.
  The equality $\proj2u=\proj2v$ follows similarly from $(f\circ -)$.
\end{proof}

\begin{cor}\label{thm:equiv-biinv-isequiv}
  For any $f:A\to B$ we have $\eqv{\biinv(f)}{\iscontr(f)}$, and hence also $\eqv{\biinv(f)}{\ishae(f)}$.
\end{cor}
\begin{proof}
  We have $\biinv(f) \to \qinv(f) \to \iscontr(f)$ and $\iscontr(f) \to \qinv(f) \to \biinv(f)$.
  Since both $\iscontr(f)$ and $\biinv(f)$ are mere propositions, the equivalence follows from \autoref{ex:hprop-iff-equiv}.
\end{proof}

We can also give a more explicit, calculational proof that $\eqv{\biinv(f)}{\ishae(f)}$.
[TODO: This proof needs to be switched around somehow.]

\begin{proof}[Second proof of \autoref{thm:equiv-biinv-isequiv}]
We proceed in 4 steps:
\begin{enumerate}
\item Define a function $F : (A \cong^g_a B) \to (A \cong_w B)$. This is very simple: given a half adjoint equivalence $(f,g,\eta,\epsilon,\upsilon) : A \cong^g_a B$, 
we take $(f,g,g,\eta,\epsilon)$ to be our desired weak isomorphism.



\item Define a function $G : (A \cong_w B) \to (A \cong^g_a B)$. Given a weak isomorphism $(f,g,h,\eta,\epsilon)$, our half adjoint equivalence will be $(f,g,\eta,\epsilon',\upsilon')$, for suitable $\epsilon'$ and $\upsilon'$. 
Namely, we put
\begin{align*}
\epsilon'(y) \defeq \opp{fg(\epsilon y)} \ct f(\eta (h y)) \ct \epsilon y
\end{align*}
Then for any $y : B$ we have
\begin{align*}
g(\epsilon'(y)) & = \opp{gfg(\epsilon y)} \ct gf(\eta (h y)) \ct g(\epsilon y) \\
& = \opp{gfg(\epsilon y)} \ct \eta (gfhy) \ct g(\epsilon y) \\
& = \eta(gy)
\end{align*}
where the second equality follows since $\eta$ commutes with $g \circ f$ and the third equality follows by naturality of $\eta$. We let $\upsilon'(y)$ be the above chain of equalities.



\item Show that $G \circ F \sim \idfunc[A \cong^g_a B]$. Fix a half adjoint equivalence $(f,g,\eta,\epsilon,\upsilon)$. It is easy to see that $G(F(f,g,\eta,\epsilon,\upsilon)) \equiv (f,g,\eta, \epsilon', \upsilon')$, where
\begin{align*}
\epsilon'(y) \defeq \opp{fg(\epsilon y)} \ct f(\eta (g y)) \ct \epsilon y
\end{align*}
and $\upsilon'(y)$ is the following chain of paths:

\begin{align*}
\xymatrix{
g(\opp{fg(\epsilon y)} \ct f(\eta (g y)) \ct \epsilon y) \ar[d] \\
\opp{gfg(\epsilon y)} \ct gf(\eta (g y)) \ct g(\epsilon y) \ar[d]^{\mathit{via\;}\com{\eta}{g\circ f}{gy}} \\
\opp{gfg(\epsilon y)} \ct \eta (gfgy) \ct g(\epsilon y) \ar[d]^{\mathit{via \;} \eta(g(\epsilon y))} \\
\eta(g y)}
\end{align*}
It thus suffices to show that for each $y : B$, we have $(\epsilon y,\upsilon y) = (\epsilon' y,\upsilon' y)$ in the type $\sm{\gamma : fgy = y} g(\gamma) = \eta(g y)$, i.e., in the type $\coh{f}{g}{\eta} \; y$. By \autoref{lem:coheq}, it suffices to exhibit a path $\delta : \epsilon y = \epsilon' y$ such that $g(\delta) \ct \upsilon' y = \upsilon y$.

Let $\delta : \epsilon y = \epsilon' y$ be the path 

\begin{align*}
\xymatrix{
\epsilon y \ar[d]^{\mathit{via \;} \epsilon(\epsilon y)} \\
\opp{fg(\epsilon y)} \ct \epsilon (f g y) \ct \epsilon y \ar[d]^{\mathit{via \; } \com{\epsilon}{f \circ g}{y}} \\
\opp{fg(\epsilon y)} \ct fg (\epsilon y) \ct \epsilon y \ar[d]^{\mathit{via \;} \upsilon y} \\
\opp{fg(\epsilon y)} \ct f (\eta(g y)) \ct \epsilon y}
\end{align*}
It thus remains to show that $\upsilon y$ is equal to the path

\begin{align*}
\xymatrix{
g(\epsilon y) \ar[d]^{\mathit{via \;} \epsilon(\epsilon y)} \\
g(\opp{fg(\epsilon y)} \ct \epsilon (f g y) \ct \epsilon y) \ar[d]^{\mathit{via \; } \com{\epsilon}{f \circ g}{y}} \\
g(\opp{fg(\epsilon y)} \ct fg (\epsilon y) \ct \epsilon y) \ar[d]^{\mathit{via \;} \upsilon y} \\
g(\opp{fg(\epsilon y)} \ct f (\eta(g y)) \ct \epsilon y) \ar[d] \\
\opp{gfg(\epsilon y)} \ct gf(\eta (g y)) \ct g(\epsilon y) \ar[d]^{\mathit{via\;}\com{\eta}{g\circ f}{gy}} \\
\opp{gfg(\epsilon y)} \ct \eta (gfgy) \ct g(\epsilon y) \ar[d]^{\mathit{via \;} \eta(g(\epsilon y))} \\
\eta(g y)}
\end{align*}
Now the following diagram clearly commutes:

\begin{align*}
\xymatrix{
g(\epsilon y) \ar[d]^{\mathit{via \;} \epsilon(\epsilon y)} & \\
g(\opp{fg(\epsilon y)} \ct \epsilon (f g y) \ct \epsilon y) \ar[d]^{\mathit{via \; } \com{\epsilon}{f \circ g}{y}} \ar[r] &
\opp{gfg(\epsilon y)} \ct g(\epsilon (f g y)) \ct g(\epsilon y) \ar[d]^{\mathit{via \; } \com{\epsilon}{f \circ g}{y}} \\  
g(\opp{fg(\epsilon y)} \ct fg (\epsilon y) \ct \epsilon y) \ar[d]^{\mathit{via \;} \upsilon y} \ar[r] &
\opp{gfg(\epsilon y)} \ct gfg (\epsilon y) \ct g(\epsilon y) \ar[d]^{\mathit{via \;} \upsilon y} \\
g(\opp{fg(\epsilon y)} \ct f (\eta(g y)) \ct \epsilon y) \ar[r] &
\opp{gfg(\epsilon y)} \ct gf(\eta (g y)) \ct g(\epsilon y) \ar[d]^{\mathit{via\;}\com{\eta}{g\circ f}{gy}} \\
 & \opp{gfg(\epsilon y)} \ct \eta (gfgy) \ct g(\epsilon y) \ar[d]^{\mathit{via \;} \eta(g(\epsilon y))} \\
 & \eta(g y)}
\end{align*}
This means it suffices to show that $\upsilon y$ is equal to the path

\begin{align*}
\xymatrix{
g(\epsilon y) \ar[d]^{\mathit{via \;} \epsilon(\epsilon y)} \\
g(\opp{fg(\epsilon y)} \ct \epsilon (f g y) \ct \epsilon y) \ar[d] \\
\opp{gfg(\epsilon y)} \ct g(\epsilon (f g y)) \ct g(\epsilon y) \ar[d]^{\mathit{via \; } \com{\epsilon}{f \circ g}{y}} \\
\opp{gfg(\epsilon y)} \ct gfg (\epsilon y) \ct g(\epsilon y) \ar[d]^{\mathit{via \;} \upsilon y} \\
\opp{gfg(\epsilon y)} \ct gf(\eta (g y)) \ct g(\epsilon y) \ar[d]^{\mathit{via\;}\com{\eta}{g\circ f}{gy}} \\
\opp{gfg(\epsilon y)} \ct \eta (gfgy) \ct g(\epsilon y) \ar[d]^{\mathit{via \;} \eta(g(\epsilon y))} \\
\eta(g y)}
\end{align*}
Now by \autoref{lem:coh-fg} we have

\begin{align*}
\xymatrix{
\opp{gfg(\epsilon y)} \ct g(\epsilon (f g y)) \ct g(\epsilon y) \ar[d]_{\mathit{via \; } \com{\epsilon}{f \circ g}{y}} \ar[r]^{\mathit{via \; } \upsilon(fgy)} &
\opp{gfg(\epsilon y)} \ct \eta (gfgy) \ct g(\epsilon y) \ar[d]^{\mathit{via\;}\com{\eta}{g\circ f}{gy}} \\
\opp{gfg(\epsilon y)} \ct gfg (\epsilon y) \ct g(\epsilon y) \ar[r]_{\mathit{via \;} \upsilon y} &
\opp{gfg(\epsilon y)} \ct gf(\eta (g y)) \ct g(\epsilon y)} 
\end{align*}
Thus it suffices to show that $\upsilon y$ is equal to the path

\begin{align*}
\xymatrix{
g(\epsilon y) \ar[d]^{\mathit{via \;} \epsilon(\epsilon y)} \\
g(\opp{fg(\epsilon y)} \ct \epsilon (f g y) \ct \epsilon y) \ar[d] \\
\opp{gfg(\epsilon y)} \ct g(\epsilon (f g y)) \ct g(\epsilon y) \ar[d]^{\mathit{via \; \upsilon(fgy)}}  \\
\opp{gfg(\epsilon y)} \ct \eta (gfgy) \ct g(\epsilon y) \ar[d]^{\mathit{via \;} \eta(g(\epsilon y))} \\
\eta(g y)}
\end{align*}
However, this follows at once from \autoref{lem:higher-hom}.


\item Show that $F \circ G \sim \idfunc[A \cong_w B]$. Fix a weak isomorphism $(f,g,h,\eta,\epsilon)$. It is easy to see that $F(G(f,g,h,\eta,\epsilon)) \equiv (f,g,g,\eta,\epsilon')$, where
\begin{align*}
\epsilon'(y) \defeq \opp{fg(\epsilon y)} \ct f(\eta (h y)) \ct \epsilon y
\end{align*}
Thus it suffices to show that for each $y : B$, we have $(g y, \epsilon' y) = (h y, \epsilon y)$ in the type $\sm{x : A} f x = y$, i.e., in $\mathtt{hfiber \;} f \; y$. By  \autoref{lem:hfib} it suffices to give a path $\gamma : gy = hy$ such that $f(\gamma) \ct \epsilon y = e' y$. This is obviously satisfied if we take  $\gamma \defeq \opp{g(\epsilon y)} \ct \eta(h y)$.\qedhere
\end{enumerate}

\end{proof}

\section{Concluding remarks}
\label{sec:concluding-remarks}

TODO: Prove symmetry and transitivity of equivalences, discuss definition of $\isequiv(f)$.


%%% This should go in the appendix on univalence implies funext, if anywhere.

% \section{Identity Systems on a Type Universe}
% \newcommand{\sfr}[1]{{{\sf r}_{#1}}}
% Let $\bbU$ be a type universe.

% \begin{defn} An {\em identity system $(R,\sfr{})$ on $\bbU$} consists of a type $R_{A,B}$ for $A,B:\bbU$, together with $\sfr{A}:R_{A,A}$ for $A:\bbU$, such that the following holds.  
% \begin{quote}
% If $D_{A,B}(e)$ is a type for $A,B:\bbU$ and $e:R_{A,B}$ then there is   
%   \[ J_{A,B}(e):D_{A,B}(e)\mbox{ for } A,B:\bbU \mbox{ and } e:R_{A,B},\] 
% such that
%   \[ J_{A,A}(\sfr{A})=d_A\mbox{ for } A:\bbU.\]
% \end{quote}
% \end{defn}
% \begin{eg}
% $(Id_\bbU,\refl{\bbU})$ is an identity system on $\bbU$.
% \end{eg}
% \begin{thm}
% If $(R,\sfr{})$ is an identity system on $\bbU$ then
% \begin{enumerate}
% \item $R_{A,B}\ra \eqv{A}{B}$ for $A,B:\bbU$
% \item If $f_e:A\ra B$ and $g_e:B\ra A$ for $A,B:\bbU$ and $e:R_{A,B}$ such that 
%   \[ f_{\sfr{A}} = g_{\sfr{A}}=\idfunc[A]\mbox{ for } A:\bbU\] 
% then
%   \[ g_e\circ f_e =\idfunc[A]\mbox{ and } f_e\circ g_e = \idfunc[B]
%       \mbox{ for }  A,B:\bbU \mbox{ and } e:R_{A,B} \]
% \end{enumerate}
% \end{thm}
% \newcommand{\sfequiv}[1]{{\sf reflequiv}^\bbU_{#1}}
% \begin{rmk} Once we have the notion of a univalent type universe we can get the following result. 
% \end{rmk}
% \begin{thm}
% $\bbU$ is a univalent type universe iff $(Equiv^\bbU,\sfequiv{})$ is an identity system on $\bbU$, where
%   \[ Equiv^\bbU_{A,B}\defeq (\eqv{A}{B})\mbox{ for } A,B:\bbU\]
% and, if $s_A: (\idfunc[A]\mbox{ is an equivalence }A\ra A)$,
%   \[ \sfequiv{A}\defeq (\idfunc[A],s_A):\eqv{A}{A}\mbox{ for } A:\bbU.\]
% \end{thm}


% Using the two theorems I believe (from a still mysterious coq proof of Assia and Cyril) that we can get a quick proof of function extensionality on a univalent universe.

\section*{Exercises}

\begin{ex}
  Show that a type $A$ is contractible (that is, $\iscontr(A)$ holds) if and only if $A\simeq\unit$.
\end{ex}

% Local Variables:
% TeX-master: "main"
% End:
