\section{Quasi-inverses and (coherent) equivalences}
\label{sec:basics-equivalences}

\begin{defn}
Let $f,g:\prd{x:A}{P(x)}$ be two sections of a dependent type $P:A\to\type$. A
homotopy from $f$ to $g$ is a term of type
\begin{equation*}
f\htpy g\defeq \prd{x:A}f(x)=g(x)
\end{equation*}
\end{defn}

Homotopies are always natural with respect to paths:

\begin{lem}\label{lem:htpy_natural}
Suppose $H:f\htpy g$ is a homotopy between functions $f,g:A\to B$
 and let $p:x=y$. Then there is a term of type
 \begin{equation*}
 H(y)\ct\ap{f}{p}=\ap{g}{p}\ct H(x).
 \end{equation*}
\end{lem}

\begin{defn}
Consider a function $f:A\to B$. A quasi-inverse of $f$ is a triple 
$(g,\eta,\varepsilon)$ consisting of a function $g:B\to A$ and homotopies
$\eta:\idfunc[A]\htpy g\circ f$ and $\varepsilon:f\circ g\htpy \idfunc[B]$.
\end{defn}

\begin{rm}
In other words, $f:A\to B$ has a quasi-inverse if there is a term of type
\begin{equation*}
\sm{g:B\to A}((\idfunc[A]\htpy g\circ f)\times(f\circ g\htpy\idfunc[B])).
\end{equation*}
\end{rm}

\begin{defn}
A function $f:A\to B$ is said to be a (coherent) equivalence if there is a
quadruple $(g,\eta,\varepsilon,c)$ consisting of $g:B\to A$, $\eta:
g\circ f\htpy\idfunc[A]$, $\varepsilon: f\circ g\htpy\idfunc[B]$ and
\begin{equation*}
c(a):f(\eta(a))=\varepsilon(f(a))
\end{equation*}
for every $a:A$. 
\end{defn}

\begin{rm}
In other words, $f:A\to B$ is an equivalence if there is a term of type
\begin{equation*}
\sm{g:B\to A}{\eta:g\circ f\htpy\idfunc[A]}
{\varepsilon:f\circ g\htpy \idfunc[B]}(\apfunc{f}\circ\eta\htpy\varepsilon
\circ f)
\end{equation*}
\end{rm}

\begin{thm}
A function $f:A\to B$ is an equivalence whenever it has a quasi-inverse.
\end{thm}

\begin{proof}
Suppose that $(g,\eta,\varepsilon)$ is a quasi-inverse for $f$. We have to provide
a quadruple $(g',\eta',\varepsilon',c)$ witnessing that $f$ is an equivalence. To
define $g'$ and $\eta'$, we can just make the obvious choice by setting $g'
\defeq g$ and $\eta'\defeq \eta$. However, in the definition of $\varepsilon'$ we
need start worrying about the construction of $c$, so we cannot just follow our nose
and take $\varepsilon'$ to be $\varepsilon$. Instead, we take
\begin{equation*}
\varepsilon'(b) \defeq (\varepsilon(b)\ct \ap{f}{\eta(g(b))})\ct \varepsilon(f(g(b)))^{-1}
\end{equation*}
Now we need to find
\begin{equation*}
c(a):(\varepsilon(f(a))\ct \ap{f}{\eta(g(f(a)))})\ct \varepsilon(f(g(f(a))))^{-1}=\ap{f}{\eta(a)}
\end{equation*}
Note first that by lemma \ref{lem:htpy_natural}, we have 
$\eta(g(f(a)))\ct\eta(a)=\ap{g}{\ap{f}{\eta(a)}}\ct\eta(a)$ and hence it follows
that $\eta(g(f(a)))=\ap{g}{\ap{f}{\eta(a)}}$. Therefore, we can apply lemma
\ref{lem:htpy_natural} once more to compute
\begin{align*}
\varepsilon(f(a))\ct \ap{f}{\eta(g(f(a)))}
& = \varepsilon(f(a))\ct \ap{f}{\ap{g}{\ap{f}{\eta(a)}}}\\
& = \ap{f}{\eta(a)}\ct\varepsilon(f(g(f(a))))
\end{align*}
from which we get the desired path $c(a)$.
\end{proof}


% Local Variables:
% TeX-master: "main"
% End:
