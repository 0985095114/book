\pagestyle{empty}

\cleardoublepage
\clearpage

%%%%%%%%%%%%%%%%%%%% Back cover %%%%%%%%%%%%%%%%%%%%
\pagecolor{black}
\newgeometry{margin=3cm}%
\vspace*{0.05\textheight}
{{\color{orange}\large
\raggedright
\parindent=0pt
\parskip=\baselineskip
\emph{From the Introduction:}

\emph{Homotopy type theory} (HoTT) is a new field of mathematics,
which combines aspects of several different areas in a surprising way.
It is based on a recently discovered connection between \emph{homotopy
  theory} and \emph{type theory}. Although the connections between the
two are currently the focus of intense investigation, it is
increasingly clear that they are just the beginning of a subject that
will take more time and more hard work to fully understand. It touches
on topics as seemingly distant as the homotopy groups of spheres, the
decidability of type checking algorithms, and the definition of weak
$\infty$-groupoids.

Homotopy type theory also brings new ideas into the very foundation of
mathematics. On the one hand, there is Voevodsky's subtle and
beautiful \emph{Univalence Axiom}. The univalence axiom implies, in
particular, that isomorphic structures can be identified: a principle
that mathematicians have been happily using on workdays, despite its
incompatibility with the ``official'' doctrines of conventional
foundations. On the other hand, we have \emph{Higher Inductive Types},
which provide direct, logical descriptions of some of the basic spaces
and constructions of homotopy theory: spheres, cylinders, truncations,
localizations, etc. Both ideas are impossible in classical
set-theoretic foundations, but when combined in homotopy type theory,
they permit an entirely new kind of ``logic of homotopy types''.

This suggests a new conception of foundations of mathematics, with
intrinsic homotopical content, an ``invariant'' conception of the
objects of mathematics --- and convenient machine implementations,
which can serve as a practical aid to the working mathematician. This
is the \emph{univalent foundations program}.

The present book is intended as a first systematic exposition of the
basics of univalent foundations, and a collection of examples of this
new style of reasoning --- but without requiring the reader to know or
learn any formal logic, or to use any computer proof assistant. We
contend that univalent foundations can replace set theory as the
``implicit foundation'' for the unformalized mathematics done by most
mathematicians, independently of the success of its machine
implementations.

\vfill

\begin{center}
  \emph{Get a free copy of the book at homotopytypetheory.org.}
\end{center}

}
\vspace*{0.02\textheight}

%%% Local Variables: 
%%% mode: latex
%%% TeX-master: "main"
%%% End: 
