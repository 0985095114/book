\chapter{Basics of HoTT}
\label{cha:basics}

\section{Basic properties of equality}
\label{sec:equality}


We begin by introducing identity types and the basic notions that follow from them.
For any type $A$, and any $x,y:A$, we have a type $\id[A]{x}{y}$, also written $\idtype[A]{x}{y}$ or just $x=y$.
We can think of inhabitants of $x=y$ as either
\begin{enumerate}
\item proofs that $x$ and $y$ are equal,
\item paths from $x$ to $y$, or
\item isomorphisms/equivalences from $x$ to $y$.
\end{enumerate}
The induction principle for identity types says that if
\begin{itemize}
\item for every $x,y:A$ and every $p:\id[A]xy$ we have a type $D(x,y,p)$, and
\item for every $a:A$ we have an element $d(a):D(a,a,\refl a)$, 
\end{itemize}
then there exists an element $J_{D,d}(x,y,p):D(x,y,p)$ for \emph{every} $x,y:A$ and $p:\id[A]xy$, such that $J_{D,d}(a,a,\refl a) \jdeq d(a)$.
In other words, given dependently typed functions
\begin{align*}
D & :\prod_{x,y:A}\, \prod_{p:\id{x}{y}} \,\type\\
d & :\prod_{a:A} D(a,a,\refl{a})
\end{align*}
there is a dependently typed function
\[J_{D,d}:\prod_{x,y:A} \,\prod_{p:\id{x}{y}} \,D(x,y,p)\]
such that 
\begin{equation}\label{eq:Jconv}
J_{D,d}(a,a,\refl{a})\jdeq d(a)
\end{equation}
for every $a:A$.
The notation $J$ is traditional for this function, but we will not use it very much.
Usually, every time we apply this induction rule we will either not care about the specific function being defined, or we will immediately give it a different name.

As a first application, we can show that equality is symmetric.
In topological language, this means ``paths can be reversed''.

\begin{lem}\label{lem:opp}
  For every type $A$ and every $x,y:A$ there is a function
  \begin{equation*}
    (x= y)\to(y= x)
  \end{equation*}
  denoted $p\mapsto \opp{p}$, such that $\opp{\refl{x}}\jdeq\refl{x}$ for each $x:A$.
\end{lem}
\begin{proof}[First proof]
  Let $D:\prod_{x,y:A} \prod_{p:x= y} \type$ be the dependent type defined by $D(x,y,p)\defeq (y= x)$.
  Then we have
  \begin{equation*}
    d\defeq \lambda x.\refl{x}:\prod_{x:A}\, D(x,x,\refl{x}).
  \end{equation*}
  Thus, $J$ gives us a term $J_{D,d}(x,y,p): (y= x)$ for each $p:(x= y)$.
  Thus, we can define the desired function by $\lambda p. J_{D,d}(x,y,p)$.
  The conversion rule~\eqref{eq:Jconv} gives $\opp{\refl{x}}\jdeq \refl{x}$.
\end{proof}

We have written out this proof in a very formal style, which may be helpful while the induction rule on identity types is unfamiliar.
However, eventually we prefer to use more natural language, such as in the following equivalent proof.

\begin{proof}[Second proof]
  We want to construct, for each $x,y:A$ and $p:x=y$, an element $\opp{p}:y=x$.
  By induction, it suffices to do this in the case when $y$ is $x$ and $p$ is $\refl{x}$.
  But in this case, $y=x$ and $x=y$ are both $x=x$, so we can take $\opp{\refl{x}}$ to be $\refl{x}$ again.
  The conversion rule follows by definition.
\end{proof}

We will write out the next few proofs in both styles, to help the reader become accustomed to the latter one.
Next we prove the transitivity of equality, or equivalently we ``concatenate paths''.

\begin{lem}\label{lem:concat}
  For every type $A$ and every $x,y,z:A$ there is a function
  \begin{equation*}
    {(-)}\ct{(-)}:(y= z)\to (x= y)\to (x=  z)
  \end{equation*}
  which satisfies $\refl{x}\ct \refl{x}\jdeq \refl{x}$ for any $x:A$.
\end{lem}

\begin{proof}[First proof]
  Let $D:\prd{y,z:A}{q:y=z} \type$ be the dependent type
  \begin{equation*}
    D(y,z,q)\defeq \prd{x:A}{p:x=y} (x=z).
  \end{equation*}
  In order to apply the induction principle for identity types to this $D$, we need a function of type
  \begin{equation*}
    \prd{y:A} D(y,y,\refl{y})
    \jdeq \prd{y,x:A}{p:x=y} (x=y).
  \end{equation*}
  Now let $E:\prd{x,y:A}{p:x=y}\type$ be the dependent type $E(x,y,p)\defeq (x=y)$.
  Then we have the function $e(x) \defeq \refl{x} : E(x,x,\refl{x})$.
  Thus, by the induction principle for identity types applied to $E$, we have a function
  \begin{equation*}
    d(x,y,p) : \prd{y,x:A}{p:x=y} (x=y) \jdeq \prd{y:A} D(y,y,\refl{y}).
  \end{equation*}
  Thus, we can use this function and apply the induction principle for identity types to $D$, to obtain our desired function of type
  \begin{equation*}
    \prd{x,y,z:A}{q:y=z}{p:x=y} (x=z).
  \end{equation*}
  The conversion rules for the two induction principles give us $\refl{x}\ct \refl{x}\jdeq \refl{x}$ for any $x:A$.
\end{proof}

\begin{proof}[Second proof]
  We want to construct, for every $x,y,z:A$ and every $p:x=y$ and $q:y=z$, an element of $x=z$.
  By induction on $q$, it suffices to assume that $z$ is $y$ and $q$ is $\refl{y}$.
  In this case, $x=z$ is $x=y$.
  Now by induction on $p$, it suffices to assume also that $y$ is $x$ and $p$ is $\refl{x}$.
  But in this case, we have $\refl{x}:(x=x)$.
\end{proof}

The reader may well feel that we have given an overly convoluted proof of this lemma.
In fact, we could stop after the induction on $q$, since at that point what we want to produce is an equality $x=y$, and we already have such an equality, namely $p$.
Why do we go on to do another induction on $p$?

The answer is that, as described in the introduction, we are doing \emph{proof-relevant} mathematics.
When we prove a lemma, we are defining an inhabitant of some type, and it can matter what \emph{specific} element we defined in the course of the proof, not merely the type that that element inhabits (that is, the \emph{statement} of the lemma).
\autoref{lem:concat} has three obvious proofs: we could do induction over $p$, induction over $q$, or induction over both of them.
If we proved it three different ways, we would have three different elements of the same type.
It's not hard to show that these three elements would be (provably) \emph{equal} (see \autoref{ex:basics:concat}), but there can still be reasons to prefer a particular definition over a provably equal one.

In the case of \autoref{lem:concat}, the difference hinges on the computation rule.
If we proved the lemma using a single induction over $q$, then we would end up with a computation rule of the form $\refl{y} \ct p \jdeq p$.
If we proved it with a single induction over $p$, we would have instead $q\ct\refl{x}\jdeq q$, while proving it with a double induction (as we did) gives only $\refl{x}\ct\refl{x} \jdeq \refl{x}$.
The asymmetrical computation rules can sometimes be convenient when doing \emph{formalized} mathematics, as they allow the computer to reduce some things automatically, but in informal mathematics, it is just confusing to have a concatenation operation which behaves asymmetrically and to have to remember which side is the ``special'' one.
This is why we have given the proof that we did.

Proof-relevance also means that we can't stop after proving ``symmetry'' and  ``transitivity'' of equality: we need to know that these \emph{operations} on equalities are well-behaved.
(This issue is invisible to set-level mathematics, where symmetry and transitivity are mere propositions.)
For instance, we need to know that concatenation of equalities is \emph{associative}, and that inversion provides \emph{inverses} with respect to concatenation.

A noteworthy feature of the next lemma is that these relations are themselves propositional equalities, living in the identity types of the types $x= y$.
Topologically, they are \emph{paths of paths}, and we are familiar topologically with the idea that concatenating a path with the reversed path only gives a constant path \emph{up to homotopy}, i.e.\ up to a higher path.

\begin{lem}%[The $\omega$-groupoid structure of types]
  Suppose $A$ and $B$ are types, that $x,y,z,w:A$ and that $p:x= y$, $q:y = z$ and $r:z=w$.
  We have the following:
  \begin{enumerate}
  \item $p= p\ct \refl{x}$ and $p = \refl{y} \ct p$.
  \item $\opp{p}\ct p=  \refl{x}$ and $p\ct \opp{p}= \refl{y}$.
  \item $\opp{\opp{p}}= p$.
  \item $r\ct (q\ct p)=  (r\ct q)\ct p$.
  \end{enumerate}
\end{lem}

\begin{proof}
  All the proofs use the induction principle for equalities.
  \begin{enumerate}
  \item \emph{(First proof)} Let $D:\prod_{x,y:A}\prod_{p:x=y} \type$ be the dependent type given by 
    \begin{equation*}
      D(x,y,p)\defeq (p= p\ct \refl{x}).
    \end{equation*}
    Then $D(x,x,\refl{x})$ is $\refl{x}=\refl{x}\ct\refl{x}$.
    Since $\refl{x}\ct\refl{x}\jdeq\refl{x}$, it follows that $D(x,x,\refl{x})\jdeq (\refl{x}=\refl{x})$.
    Thus, there is a term
    \begin{equation*}
      d\defeq\lambda x.\refl{\refl{x}}:\prod_{x:A} D(x,x,\refl{x}).
    \end{equation*}
    Now $J$ gives a term $J(D,d,p):(p= p\ct\refl{x})$ for each $p:x= y$.
    The other equality is proven similarly.

    \noindent
    \emph{(Second proof)} By induction, it suffices to assume that $y$ is $x$ and that $p$ is $\refl x$.
    But in this case, we have $\refl{x}\ct\refl{x}\jdeq\refl{x}$.
  \item \emph{(First proof)} Let $D:\prod_{x,y:A}\prod_{p:x=y} \type$ be the dependent type given by 
    \begin{equation*}
      D(x,y,p)\defeq (\opp{p}\ct p=  \refl{x}).
    \end{equation*}
    Then $D(x,x,\refl{x})$ is $\opp{\refl{x}}\ct\refl{x}=\refl{x}$.
    Since $\opp{\refl{x}}\jdeq\refl{x}$ and $\refl{x}\ct\refl{x}\jdeq\refl{x}$, we get that $D(x,x,\refl{x})\jdeq (\refl{x}=\refl{x})$.
    Hence we find the function
    \begin{equation*}
      d\defeq\lambda x.\refl{\refl{x}}:\prod_{x:A} D(x,x,\refl{x}).
    \end{equation*}
    Now $J$ gives a term $J(D,d,p):\opp{p}\ct p=\refl{x}$ for each $p:x= y$ in $A$.
    The other equality is similar.

    \noindent \emph{(Second proof)} By induction, it suffices to assume $p$ is $\refl x$.
    But in this case, we have $\opp{p} \ct p \jdeq \opp{\refl x} \ct \refl x \jdeq \refl x$.
  \item \emph{(First proof)} Let $D:\prod_{x,y:A} \prod_{p:x=y} \type$ be the dependent type given by
    \begin{equation*}
      D(x,y,p)\defeq (\opp{\opp{p}}= p).
    \end{equation*}
    Then $D(x,x,\refl{x})$ is $\opp{\opp{\refl x}}=\refl{x}$.
    Since $\opp{\refl{x}}\jdeq \refl{x}$ for each $x:A$, we see that $D(x,x,\refl{x})\jdeq(\refl{x}=\refl{x})$.
    Hence we find the function
    \begin{equation*}
      d\defeq\lambda x.\refl{\refl{x}}:\prod_{x:A} D(x,x,\refl{x}).
    \end{equation*}
    Now $J$ gives a term $J(D,d,p):\opp{\opp{p}}= p$ for each $p:x= y$.

    \noindent \emph{(Second proof)} By induction, it suffices to assume $p$ is $\refl x$.
    But in this case, we have $\opp{\opp{p}}\jdeq \opp{\opp{\refl x}} \jdeq \refl x$.
  \item \emph{(First proof)} Let $D:\prod_{z,w:A} \prod_{r:z=w} \type$ be the dependent type given by
    \begin{equation*}
      D(z,w,r)\defeq\prod_{x,y:A}\,\prod_{p:x= y}\,\prod_{q:y= z}\, \big(r\ct (q\ct p)=  (r\ct q)\ct p\big).
    \end{equation*}
    Then $D(z,z,\refl{z})$ is
    \begin{equation*}
      \prod_{x,y:A} \,\prod_{q:y= z} \,\prod_{p:x= y}\, \big(\refl{z}\ct(q\ct p)= (\refl{z}\ct q)\ct p\big),
    \end{equation*}
    which simplifies to
    \begin{equation*}
      \prod_{x,y:A}\,\prod_{q:y= z} \,\prod_{p:x= y}\, (q\ct p= q\ct p).
    \end{equation*}
    Therefore, we find
    \begin{equation*}
      d\defeq\lambda z.\lambda x,y,p,q.\refl{q\ct p}:\prod_{z:A} D(z,z,\refl{z}).
    \end{equation*}
    Now $J$ gives us an element $J(D,d,r,p,q):r\ct (q\ct p)= (r\ct q)\ct p$ for each $r:z= w$.

    \noindent \emph{(Second proof)} By induction, it suffices to assume $r$ is $\refl z$.
    But in this case, we have
    \[ r\ct (q\ct p) \jdeq \refl{z}\ct(q\ct p) \jdeq q\ct p \jdeq (\refl{z}\ct q)\ct p
    \jdeq (r\ct q)\ct p.\qedhere
    \]
  \end{enumerate}
\end{proof}

Now we wish to establish that functions $f:A\to B$ behave functorially on paths, or equivalently that they respect equality.
Topologically, this can be thought of as saying that every function is ``continuous''.

\begin{lem}\label{lem:map}
  Suppose that $f:A\to B$ is a function and that $p:(\id[A]xy)$. Then there is
  \begin{equation*}
    \map f p : f(x)= f(y)
  \end{equation*}
  Moreover, for each $x:A$ we have $\map {f}{\refl{x}}\jdeq \refl{f(x)}$.
\end{lem}

\begin{proof}[First proof]
  Let $D:\prod_{x,y:A} \prod_{p:x=y}$ be $D(x,y,p)\defeq (f(x)= f(y))$.
  Then we have
  \begin{equation*}
    d\defeq\lambda x.\refl{f(x)}:\prod_{x:A} D(x,x,\refl{x}).
  \end{equation*}
  Now $J$ gives us $\map f p\defeq f(x)= f(y)$ for each $p:x= y$.
  The conversion rule implies $\map f{\refl{x}}\jdeq\refl{f(x)}$ for each $x:A$.
\end{proof}

\begin{proof}[Second proof]
  By induction, it suffices to assume $p$ is $\refl{x}$, in which case we have $\refl{f(x)}:f(x)\jdeq f(x)$.
\end{proof}

Since \emph{dependently typed} functions are very important in type theory, we will also need a version of \autoref{lem:map} for these.
However, this is not quite so simple to state, because if $f:\prod_{x:A} B(x)$ and $p:x=y$, then $f(x):B(x)$ and $f(y):B(y)$ are elements of distinct types, so that \emph{a priori} we cannot even ask whether they are equal.
The missing ingredient is that $p$ itself gives us a way to relate the types $B(x)$ and $B(y)$.

\begin{lem}[Transport]\label{lem:transport}
  Suppose that $P$ is a dependent type over $A$ and that $p:\id[A]xy$.
  Then there is a function $\transf{p}:P(x)\to P(y)$.
\end{lem}

\begin{proof}[First proof]
  Let $D:\prod_{x,y:A} \prod_{p:\id{x}{y}} \type$ be $D(x,y,p)\defeq P(x)\to P(y)$.
  Then we have the function
  \begin{equation*}
    d\defeq\lambda x.\idfunc[P(x)]:\prod_{x:A} D(x,x,\refl{x}),
  \end{equation*}
  so that the induction principle gives us $J_{D,d}(x,y,p):P(x)\to P(y)$ for $p:x= y$, which we define to be $\transf p$.
\end{proof}

\begin{proof}[Second proof]
  By induction, it suffices to assume $p$ is $\refl x$.
  But in this case, we can take $\transf{(\refl x)}:P(x)\to P(x)$ to be the identity function.
\end{proof}

Recall that a dependent type $P$ over a type $A$ can be seen as a property of elements of $A$, which holds at $x$ in $A$ if $P(x)$ is inhabited.
Then the transportation lemma says that if $x$ is propositionally equal to $y$, then $P(x)$ holds if and only if $P(y)$ holds.
In fact, we will see later on that if $x=y$ then actually $P(x)$ and $P(y)$ are \emph{equivalent}.

Now we can prove the dependent version of \autoref{lem:map}.

\begin{lem}[Dependent map]\label{lem:mapdep}
  Suppose $f:\prod_{x: A} P(x)$ and $p:\id[A]xy$.
  Then we have $\mapdep fp:\id[P(y)]{\trans p{f(x)}}{f(y)}$.
\end{lem}

\begin{proof}[First proof]
  Let $D:\prod_{x,y:A}\prod_{p:x=y} \type$ be the dependent type given by
  \begin{equation*}
    D(x,y,p)\defeq \trans p {f(x)}= f(y).
  \end{equation*}
  Then $D(x,x,\refl{x})$ is $\trans{\refl{x}}{f(x)}= f(x)$.
  But since $\trans{(\refl{x})}{f(x)}\jdeq f(x)$, we get that $D(x,x,\refl{x})\jdeq (f(x)= f(x))$.
  Thus, we find the term
  \begin{equation*}
    d\defeq\lambda x.\refl{f(x)}:\prod_{x:A} D(x,x,\refl{x})
  \end{equation*}
  and now $J$ gives us $\mapdep fp:\trans p{f(x)}= f(y)$ for each $p:x= y$.
\end{proof}

\begin{proof}[Second proof]
  By induction, it suffices to assume $p$ is $\refl x$.
  But in this case, we have $\trans{(\refl{x})}{f(x)}\jdeq f(x)$.
\end{proof}

From now on, we will stick to the second, more concise, style of proof by induction on identity types.

\section*{Exercises}
\label{basics:exercises}

\begin{ex}\label{ex:basics:concat}
  Show that the three obvious proofs of \autoref{lem:concat} are pairwise equal.
\end{ex}

\begin{ex}
  Show that the three equalities of proofs constructed in the previous exercise form a commutative triangle.
\end{ex}

\begin{ex}
  Give a fourth, different, proof of \autoref{lem:concat}, and prove that it is equal to the others.
\end{ex}

% Local Variables:
% TeX-master: "main"
% End:
